\section{Traduções}
    A primeira tradução do sistema intuicionista ao sistema modal foi proposta por~\cite{Goedel} motivado pela possibilidade de leitura da necessidade como uma modalidade de construtividade. Ou seja, por meio dessa tradução, a sentença $\nec \varphi$ poderia ser lida como \textit{$\varphi$ pode ser provada construtivamente} \citep{Troelstra}. Gödel alegou --- sem apresentar provas --- a correção fraca dessa tradução e conjeiturou sua completude fraca, posteriormente provadas por~\cite{McKinsey}. As as traduções apresentadas abaixo foram retiradas de~\cite{Troelstra}.

    \begin{definition}[$\bullet^\circ$] Define-se a tradução $\bullet^\circ:\mathcal{L}\to\mathcal{L}_{\nec}$ do sistema intuicionista ao sistema modal $\mathbf{S4}$ indutivamente da seguinte maneira:
        \begin{align*}
            a^\circ&\coloneqq a\\
            \bot^\circ&\coloneqq\bot\\
            {(\varphi \wedge \psi)}^\circ & \coloneqq \varphi^\circ \wedge \psi^\circ\displaybreak[0]\\
            {(\varphi \vee \psi)}^\circ   & \coloneqq \nec \varphi^\circ \vee \nec \psi^\circ\displaybreak[0]\\
            {(\varphi \to \psi)}^\circ    & \coloneqq \nec \varphi^\circ \to \psi^\circ
            \tag*{\qed} 
        \end{align*}
    \end{definition}
    
    \begin{definition}[$\bullet^\medsquare$] Define-se a tradução $\bullet^\medsquare:\mathcal{L}\to\mathcal{L}_{\nec}$ do sistema intuicionista ao sistema modal $\mathbf{S4}$ indutivamente da seguinte maneira:
        \begin{align*}
            a^\medsquare& \coloneqq \nec a\displaybreak[0]\\
            \bot^\medsquare& \coloneqq \bot\displaybreak[0]\\
            {(\varphi \wedge \psi)}^\medsquare & \coloneqq \varphi^\medsquare \wedge \psi^\medsquare\displaybreak[0]\\
            {(\varphi \vee \psi)}^\medsquare & \coloneqq \varphi^\medsquare \vee \psi^\medsquare\displaybreak[0]\\
            {(\varphi \to \psi)}^\medsquare & \coloneqq \nec (\varphi^\medsquare \to \psi^\medsquare)
            \tag*{\qed} 
        \end{align*}
    \end{definition}
    
    Ambas as traduções providas são equivalentes, conforme demonstraremos futuramente.
    Ademais, faz-se interessante pontuar que as traduções $\bullet^\circ$ e $\bullet^\medsquare$ correspondem, respectivamente, às traduções $\bullet^\circ$ e $\bullet^*$ do sistema intuicionista ao sistema linear providas por~\cite{Girard}. A primeira tradução de Girard corresponde a uma ordem de avaliação por nome (\textit{call-by-name}) e a segunda a uma ordem de avaliação por valor (\textit{call-by-value}), conforme notam~\cite{Maraist}.

