\section{Traduções}
    A primeira tradução do sistema intuicionista ao sistema modal foi proposta por~\cite{Goedel} motivado pela possibilidade de leitura da necessidade como uma modalidade de construtividade. Ou seja, por meio dessa tradução, a sentença $\nec \varphi$ poderia ser lida como \textit{$\varphi$ pode ser provada construtivamente} \citep{Troelstra}. Gödel alegou --- sem apresentar provas --- a correção fraca dessa tradução e conjeiturou sua completude fraca, posteriormente provadas por~\cite{McKinsey}. As as traduções apresentadas abaixo foram retiradas de~\cite{Troelstra}.

    \begin{tcolorbox}[enhanced jigsaw, breakable, sharp corners, colframe=black, colback=white, boxrule=0.5pt, left=1.5mm, right=1.5mm, top=1.5mm, bottom=1.5mm]
    \begin{definition}[Traduções]\label{translations}
        Definem-se a traduções $\circ:\mathcal{L}\to\mathcal{L}_{\nec}$ do sistema $\mathfrak{B}$ ao sistema $\mathfrak{L}$ indutivamente da maneira que segue.
        \begin{align*}
            a^\circ&\mapsto a\\
            \bot^\circ&\mapsto\bot\\
            {(\varphi \wedge \psi)}^\circ & \mapsto \varphi^\circ \wedge \psi^\circ\displaybreak[0]\\
            {(\varphi \vee \psi)}^\circ   & \mapsto \nec \varphi^\circ \vee \nec \psi^\circ\displaybreak[0]\\
            {(\varphi \to \psi)}^\circ    & \mapsto \nec \varphi^\circ \to \psi^\circ
            \tag*{\qed} 
        \end{align*}
    \end{definition}
    \end{tcolorbox}
    
    \begin{definition}[$\bullet^\medsquare$]\label{translation.square}
        Define-se a tradução $\bullet^\medsquare:\mathcal{L}\to\mathcal{L}_{\nec}$ do sistema intuicionista ao sistema modal $\mathbf{S4}$ indutivamente da seguinte maneira:
        \begin{align*}
            a^\medsquare& \mapsto \nec a\displaybreak[0]\\
            \bot^\medsquare& \mapsto \bot\displaybreak[0]\\
            {(\varphi \wedge \psi)}^\medsquare & \mapsto \varphi^\medsquare \wedge \psi^\medsquare\displaybreak[0]\\
            {(\varphi \vee \psi)}^\medsquare & \mapsto \varphi^\medsquare \vee \psi^\medsquare\displaybreak[0]\\
            {(\varphi \to \psi)}^\medsquare & \mapsto \nec (\varphi^\medsquare \to \psi^\medsquare)
            \tag*{\qed} 
        \end{align*}
    \end{definition}
    
    Ambas as traduções providas são equivalentes, conforme demonstraremos futuramente.
    Ademais, faz-se interessante pontuar que as traduções $\bullet^\circ$ e $\bullet^\medsquare$ correspondem, respectivamente, às traduções $\bullet^\circ$ e $\bullet^*$ do sistema intuicionista ao sistema linear providas por~\cite{Girard}. A primeira tradução de Girard corresponde a uma ordem de avaliação por nome (\textit{call-by-name}) e a segunda a uma ordem de avaliação por valor (\textit{call-by-value}), conforme notam~\cite{Maraist}.
