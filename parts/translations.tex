\section{Traduções}
    Nesta seção, apresentaremos as duas traduções que serão abordadas neste trabalho.
    Estas traduções tem como sistema de origem o sistema $\mathfrak{I}$ e como sistema de destino o sistema $\mathfrak{M}$.
    A primeira tradução entre estes sistemas foi dada por~\cite{Gödel.1986b}, motivado pela possibilidade de leitura da necessidade como uma modalidade de \emph{construtividade}.
    Ou seja, por meio desta tradução, a sentença $\nec\alpha$ poderia ser lida como \textit{$\alpha$ pode ser provada construtivamente} \citep{Troelstra+Schwichtenberg.2000}.
    Esta tradução, que~\cite{Gödel.1986b} alegou ser correta e completa, antepõe a necessidade a alguns --- ou, numa outra formulação, a todos --- dos elementos da construção de uma dada sentença.
    As duas traduções consideradas no conseguinte foram apresentadas conforme~\cite{Troelstra+Schwichtenberg.2000}, sendo a segunda delas reminiscente da tradução de~\cite{Gödel.1986b}.

    \vspace{0.5\baselineskip}
    Ambas as traduções são fortemente corretas e equivalem --- ou seja, podem ser interderivadas ---, conforme demonstraremos futuramente.
    Ademais, pontua-se que estas traduções assemelham-se muito com duas traduções do sistema intuicionista ao sistema linear providas por~\cite{Girard.1987}.
    Nelas, sentenças da forma $\alpha\to\beta$ são mapeadas para sentenças da forma $!(\alpha\multimap\beta)$ e da forma $!\alpha\multimap\beta$.
    De fato,~\cite{Troelstra+Schwichtenberg.2000} alega ter baseado a primeira das traduções que seguem em uma das traduções de~\cite{Girard.1987}.
    De modo semelhante,~\cite{Girard.1987} alega que baseou a outra de suas traduções na segunda apresentada a seguir, a tradução de~\cite{Gödel.1986b}.
    Nomeamo-las \emph{tradução $\circ$} e \emph{tradução $\bullet$}.

    \vspace{0.5\baselineskip}
    \begin{tcolorbox}[enhanced jigsaw, breakable, sharp corners, colframe=black, colback=white, boxrule=0.5pt, left=1.5mm, right=1.5mm, top=1.5mm, bottom=1.5mm]
    \begin{definition}[Tradução $\circ$]\label{translation.circle}
        Define-se a tradução $\circ:\mathcal{L}\to\mathcal{L}_{\nec}$ do sistema $\mathfrak{I}$ ao sistema $\mathfrak{M}$ indutivamente da maneira que segue.
        Considere $a\in\mathcal{P}$.

        \begin{center}
            $a^\circ\coloneqq a\quad\quad\quad\quad\quad\quad\quad\quad\quad\bot^\circ\coloneqq\bot$\\\vspace{0.5\baselineskip}
            ${(\alpha\wedge\beta)}^\circ\coloneqq\alpha^\circ\wedge\beta^\circ\quad\quad{(\alpha \to \beta)}^\circ\coloneqq \nec \alpha^\circ \to \beta^\circ\quad\quad{(\alpha\vee\beta)}^\circ\coloneqq\nec\alpha^\circ\vee\nec\beta^\circ$
        \end{center}
    \end{definition}
    \end{tcolorbox}

    \begin{tcolorbox}[enhanced jigsaw, breakable, sharp corners, colframe=black, colback=white, boxrule=0.5pt, left=1.5mm, right=1.5mm, top=1.5mm, bottom=1.5mm]
    \begin{definition}[Tradução $\bullet$]\label{translation.square}
        Define-se a tradução $\bullet:\mathcal{L}\to\mathcal{L}_{\nec}$ do sistema $\mathfrak{I}$ ao sistema $\mathfrak{M}$ indutivamente da maneira que segue.
        Considere $a\in\mathcal{P}$.

        \begin{center}
            $a^\bullet\coloneqq\nec a\quad\quad\quad\quad\quad\quad\quad\quad\quad\bot^\bullet\coloneqq\bot$\\\vspace{0.5\baselineskip}
            ${(\alpha\wedge\beta)}^\bullet\coloneqq\alpha^\bullet\wedge\beta^\bullet\quad\quad{(\alpha \to \beta)}^\bullet\coloneqq \nec (\alpha^\bullet \to \beta^\bullet)\quad\quad{(\alpha\vee\beta)}^\bullet\coloneqq\alpha^\bullet\vee\beta^\bullet$
        \end{center}
    \end{definition}
    \end{tcolorbox}

    \vspace{0.5\baselineskip}
    As traduções do sistema intuicionista ao sistema linear de~\cite{Girard.1987} possuem relação com os conceitos duais~\citep{Wadler.2003} de \emph{avaliação por valor} e de \emph{avaliação por nome}, conforme notam~\cite{Maraist+others.1999}.
    Isto sugere que pode haver uma relação semelhante para as traduções consideradas aqui, fato confirmado por~\cite{Espírito-Santo+others.2019}.
    Deste modo, a tradução $\circ$ corresponde a uma tradução de avaliação por nome e a tradução $\bullet$ a uma tradução de avaliação por valor.
