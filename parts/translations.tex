\section{Traduções}
    A primeira tradução do sistema intuicionista ao sistema modal foi proposta por~\cite{Goedel} motivado pela possibilidade de leitura da necessidade como uma modalidade de construtividade. Ou seja, por meio dessa tradução, a sentença $\nec \varphi$ poderia ser lida como \textit{$\varphi$ pode ser provada construtivamente} \citep{Troelstra}. Gödel alegou --- sem apresentar provas --- a correção fraca dessa tradução e conjeiturou sua completude fraca, posteriormente provadas por~\cite{McKinsey}. As as traduções apresentadas abaixo foram retiradas de~\cite{Troelstra}.

    \begin{definition}[$\bullet^\circ$]\label{translation.circle}
        Define-se a tradução $\bullet^\circ:\mathcal{L}\to\mathcal{L}_{\nec}$ do sistema intuicionista ao sistema modal $\mathbf{S4}$ indutivamente da seguinte maneira:
        \begin{align*}
            a^\circ&\mapsto a\\
            \bot^\circ&\mapsto\bot\\
            {(\varphi \wedge \psi)}^\circ & \mapsto \varphi^\circ \wedge \psi^\circ\displaybreak[0]\\
            {(\varphi \vee \psi)}^\circ   & \mapsto \nec \varphi^\circ \vee \nec \psi^\circ\displaybreak[0]\\
            {(\varphi \to \psi)}^\circ    & \mapsto \nec \varphi^\circ \to \psi^\circ
            \tag*{\qed} 
        \end{align*}
    \end{definition}
    
    \begin{definition}[$\bullet^\medsquare$]\label{translation.square}
        Define-se a tradução $\bullet^\medsquare:\mathcal{L}\to\mathcal{L}_{\nec}$ do sistema intuicionista ao sistema modal $\mathbf{S4}$ indutivamente da seguinte maneira:
        \begin{align*}
            a^\medsquare& \mapsto \nec a\displaybreak[0]\\
            \bot^\medsquare& \mapsto \bot\displaybreak[0]\\
            {(\varphi \wedge \psi)}^\medsquare & \mapsto \varphi^\medsquare \wedge \psi^\medsquare\displaybreak[0]\\
            {(\varphi \vee \psi)}^\medsquare & \mapsto \varphi^\medsquare \vee \psi^\medsquare\displaybreak[0]\\
            {(\varphi \to \psi)}^\medsquare & \mapsto \nec (\varphi^\medsquare \to \psi^\medsquare)
            \tag*{\qed} 
        \end{align*}
    \end{definition}
    
    Ambas as traduções providas são equivalentes, conforme demonstraremos futuramente.
    Ademais, faz-se interessante pontuar que as traduções $\bullet^\circ$ e $\bullet^\medsquare$ correspondem, respectivamente, às traduções $\bullet^\circ$ e $\bullet^*$ do sistema intuicionista ao sistema linear providas por~\cite{Girard}. A primeira tradução de Girard corresponde a uma ordem de avaliação por nome (\textit{call-by-name}) e a segunda a uma ordem de avaliação por valor (\textit{call-by-value}), conforme notam~\cite{Maraist}.

\section{Avaliação}
    Dada uma linguagem de programação, podem haver diferentes maneiras de aplicar reduções em expressões desta linguagem --- ou seja, diferentes maneiras de performar uma computação.
    A cada uma dessas maneiras, damos o nome \emph{ordem de avaliação}.
    Como notado na seção anterior, as imersões modais do sistema intuicionista assemelham-se muito às imersões lineares do sistema intuicionista.
    Como notam~\cite{Maraist}, correspondem a uma tradução a uma ordem de avaliação por nome e a uma ordem de avaliação por valor.
    Neste sentido, pode-se dizer que, do mesmo modo, as imersões modais correspondem a estas duas ordens de avaliação, que são duais entre si (Wadler).
    Assim, esta seção busca fundamentar estas duas ordens de avaliação.

    Para tanto, consideremos expressões-lambda $e$, que consistem em valores $v$ ou em aplicações $e\ e$ de uma expressão a outra expressão.
    Os valores $v$, por sua vez, dividem-se em letras $x$ ou em funções $\lambda x.e$ que recebem $x$ e retornam uma expressão $e$.
    As letras pertencem a um conjunto de letras $\mathcal{V}$.
    Consideraremos duas beta-reduções diferentes, uma na ordem de avaliação por nome e outra por valor.
