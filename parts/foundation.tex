\chapter{Fundamentação}

\section{Sistemas}

Conforme visto, as noções de sistema variam entre diferentes autores e permanece um campo em aberto. Para as necessidades deste trabalho, usaremos a definição proposta por~\cite{Beziau}, uma vez que se trata de uma definição simples e que, portanto, não traz elementos irrelevantes aos intuitos deste trabalho.

\begin{definition}[Sistema]
    Um sistema consiste num par $\mathbf{L} = \sequence{\mathcal{L}, \vdash}$, onde $\mathcal{L}$ consiste em um conjunto e $\, \vdash \: : \wp(\mathcal{L}) \times \mathcal{L}$ em uma relação sobre o produto cartesiano do conjunto das partes de $\mathcal{L}$ e o conjunto $\mathcal{L}$ em si, sem demais condições.
    \qed{}
\end{definition}

Conforme~\cite{Beziau} aponta, a \emph{qualidade} e \emph{quantidade} dos elementos de $\mathbf{L}$ não são especificados, portanto sendo esta uma definição de grande generalidade. Deste modo, a relação $\vdash$ pode ser tanto uma relação de \emph{dedução} --- definida sintaticamente --- quanto uma relação de \emph{satisfação} --- definida semanticamente. Neste trabalho, serão abordados apenas sistemas definidos sobre relações de dedução. Cabe destacar, entretanto, que nada impede a tradução entre sistemas definidos sobre relações de satisfação, como veremos futuramente.

Ainda, com base no escopo deste trabalho, restringiremos a definição do conjunto $\mathcal{L}$ a um conjunto de \emph{sentenças proposicionais bem-formadas} --- doravante ocasionalmente chamadas apenas de sentenças. Estas sentenças, neste trabalho representadas por letras gregas em caixa-baixa, podem ser geradas indutivamente a partir de uma assinatura proposicional --- que consiste em um conjunto de letras proposicionais e um conjunto de operadores com suas aridades ---, conforme definido abaixo. Conjuntos de sentenças serão representados por letras gregas em caixa-alta.

\begin{definition}[Assinatura]
    Uma assinatura consiste num conjunto letras proposicionais --- que podem ser valoradas como verdadeiras ou falsas --- e num conjunto de operadores e suas aridades correspondentes.
    \qed{}
\end{definition}

Tendo-se a uma assinatura proposicional, podemos definir o conjunto $\mathcal{L}$ de um sistema --- chamado de linguagem --- indutivamente. Nesta definição, primeiramente definimos todas as letras proposicionais e operadores de aridade zero como pertencentes ao conjunto e, em seguida, aplicamos recursivamente as sentenças aos operadores de aridade não-nula

\begin{definition}[Sentença]
    Seja $\sequence{\mathcal{P},\mathcal{C}}$ uma assinatura proposicional. Uma sentença proposicional consiste em qualquer um dos elementos do conjunto $\mathcal{L}$, sendo este o menor conjunto induzido a partir das seguintes regras:
    \begin{align*}
        &\mathcal{P}\subseteq\mathcal{L}\\
        &\text{Se    }\bullet^0\in\mathcal{C}\text{, então }\bullet^0\in\mathcal{L}\\
        &\text{Se }\bullet^n\in\mathcal{C}\text{ e }\set{\varphi_i\mid 0<i \leq n}\subseteq\mathcal{L}\text{, então }\bullet^n\sequence{\varphi_i\mid 0<i \leq n}\subseteq\mathcal{L}.
        \tag*{\qed} 
    \end{align*}
\end{definition}

Tendo definidas as noções de assinatura e sentença, pode-se definir a noção de profundidade de uma sentença. Esta noção, em termos simples, consiste no comprimento do maior ramo da contrução de uma sentença. A definição abaixo provida consiste numa generalização da definição dada por~\cite{Troelstra}. Usaremos essa definição futuramente para provar propriedades por meio de indução sobre a profundidade da sentença.

\begin{definition}[Profundidade]
    Seja $\mathbf{A} = \sequence{\mathcal{L}, \vdash}$ um sistema, $\alpha\in\mathcal{L}$ uma sentença e $\bullet^n$ um operador de aridade $n\in\mathbb{N}$ que consta na assinatura que define $\mathcal{L}$. Pode-se definir a profundidade $|\alpha|$ de $\alpha$ recursivamente da seguinte maneira:
    \begin{align*}
        |a|&\coloneqq 0\\
        |\bullet^0|&\coloneqq 0\\
        |\bullet^n\sequence{\varphi_i\mid i\leq n}|&\coloneqq\max\set{|\varphi_i|\mid i\leq n}+1.
        \tag*{\qed} 
    \end{align*}
\end{definition}

\begin{definition}[Esquema]
    Um esquema consiste em um padrão com metavariaveis que permitem representar um conjunto, geralmente infinito, de sentenças.
    \qed{}
\end{definition}

\begin{definition}[Regra]
    Uma regra de dedução consiste num par $\sequence{\Gamma, \alpha}$, sendo $\Gamma$ um conjunto de sentenças chamadas de \textit{premissas} e $\alpha$ uma sentença chamada \textit{conclusão}.
    \qed{}
\end{definition}

\begin{definition}[Axiomatização]
    Um sistema de Hilbert para um sistema $\mathbf{L} = \sequence{\mathcal{L}, \vdash}$ consiste em um par $\mathcal{H} = \sequence{\mathcal{A}, \mathcal{R}}$, sendo $\mathcal{A}$ um conjunto de esquemas axiomas e $\mathcal{R}$ um conjunto de regras de dedução.
    \begin{alignat}{3}
        &\mathbf{A}\quad&&\text{Se }\alpha\in\mathcal{A}\text{, então }\Gamma\entails\alpha\tag*{}\\
        &\mathbf{P}\quad&&\text{Se }\alpha\in\Gamma\text{, então }\Gamma\entails\alpha\tag*{}\\
        &\mathbf{E}\quad&&\text{Se }\entails\alpha\text{, então }\Gamma\entails\alpha.\tag*{\qed}
    \end{alignat}
\end{definition}

\begin{definition}[Dedução]
    Uma dedução de $\Gamma\vdash\alpha$ consiste numa sucessão de sentenças $\sequence{\varphi_i\mid 1 \leq i\leq n}$ de modo que $\varphi_n=\alpha$ e cada sentença $\varphi_i$ foi gerada a partir da aplicação de uma regra a axiomas, premissas ou sentenças anteriores.
    \qed{}
\end{definition}

\section{Traduções}

Traduções entre sistemas consistem em funções que mapeiam sentenças de um sistema a sentenças de outro sistema garantindo certas propriedades. As propriedades a serem garantidas variam e ainda são discutidas na literatura, deixando a definição exata de tradução --- assim como houve com a definição de sistema --- varie de acordo com a predileção e as necessidades de cada autor. Nesta seção, serão abordadas historicamente noções de tradução entre sistemas, bem como serão definidos e nomeados os conceitos de tradução que serão usados no restante deste trabalho.

\begin{definition}[Tradução] 
    Uma sentença $\varphi$ de um sistema $\mathbf{A} = \langle \mathcal{L}_\mathbf{A}, \vdash_\mathbf{A}\rangle$ pode ser traduzida a uma sentença $\varphi^*$ em um sistema $\mathbf{B} = \langle \mathcal{L}_\mathbf{B}, \vdash_\mathbf{B} \rangle$ caso exista uma função $\bullet^* : \mathcal{L}_\mathbf{A} \to \mathcal{L}_\mathbf{B}$ que garanta que $\Gamma \vdash_\mathbf{A} \varphi \Leftrightarrow \Gamma^* \vdash_\mathbf{B} \varphi^*$.
    \qed{}
\end{definition}

\begin{notation}
    Seja $\Gamma\in\wp(\mathcal{L}_\mathbf{A})$ um conjunto de sentenças bem-formadas e $\bullet^*\mathrel{:}\mathcal{L}_\mathbf{A}\to\mathcal{L}_\mathbf{B}$ uma tradução. $\Gamma^*$ denota o conjunto $\set{\alpha^*\mid\alpha\in\Gamma}\in\wp(\mathcal{L}_\mathbf{B})$, ou seja, a aplicação da tradução a todos os elementos do conjunto.
    \qed{}
\end{notation}

\begin{definition}[$\bullet^\neg$] Define-se a tradução $\bullet^\neg$ indutivamente da seguinte maneira:
    \begin{align*}
        p^\neg                     & \coloneqq \neg\neg p                               \\
        \bot^\neg                  & \coloneqq \bot                                     \\
        {(\varphi \wedge \psi)}^\neg & \coloneqq \neg\neg (\varphi^\neg \wedge \psi^\neg) \\
        {(\varphi \vee \psi)}^\neg   & \coloneqq \neg\neg (\varphi^\neg \vee \psi^\neg)   \\
        {(\varphi \to \psi)}^\neg    & \coloneqq \neg\neg (\varphi^\neg \to \psi^\neg)
        \tag*{\qed} 
    \end{align*}
\end{definition}

\section{Provadores}
