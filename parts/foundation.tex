\chapter{Fundamentação}\label{foundation}

Nesta parte do trabalho, serão apresentadas definições gerais que fundamentarão as definições mais estritas que serão apresentadas futuramente. Notadamente, fundamentaremos as noções de sistemas e traduções. Ademais, discorreremos acerca da noção de provadores, que serão usados para certificar as provas apresentadas posteriormente. Antes disso, entretanto, introduziremos duas notações que serão usadas copiosamente, uma para o conjunto das partes e outra para sucessões.

\vspace{.5\baselineskip}
\begin{tcolorbox}[enhanced jigsaw, breakable, sharp corners, colframe=black, colback=white, boxrule=0.5pt, left=1.5mm, right=1.5mm, top=1.5mm, bottom=1.5mm]
\begin{notation}
    Seja $A$ um conjunto, $\mathfrak{P}(A)$ denota o conjunto $\set{X\mid X\subseteq A}$.
\end{notation}
\end{tcolorbox}

\begin{tcolorbox}[enhanced jigsaw, breakable, sharp corners, colframe=black, colback=white, boxrule=0.5pt, left=1.5mm, right=1.5mm, top=1.5mm, bottom=1.5mm]
\begin{notation}
    Seja $i\in\mathbb{N}^+$ e $n\in\mathbb{N}$, $\sequence{a_i\mid i\leq n}$ denota uma sucessão de $n$ elementos de modo que o elemento $a_i$ encontra-se na posição $i$.
\end{notation}
\end{tcolorbox}

\section{Sistemas}\label{foundation.systems}

Sistemas de dedução buscam formalizar e sistematizar o processo de razoamento.
Para tanto, são consideradas regras de dedução que permitem derivar verdades a partir de outras verdades conhecidas.
Estudos acerca deste processo datam da antiguidade, entretanto considera-se que os estudos modernos neste campo foram, dentre outras pessoas, fundados por~\cite{Frege.1967}.
Desde então, surgiram diversos sistemas com diferentes linguagens e regras de dedução.
Esta diversidade de sistemas justifica o desenvolvimento de uma teoria unificadora.

\vspace{.5\baselineskip}
\begin{tcolorbox}[enhanced jigsaw, breakable, sharp corners, colframe=black, colback=white, boxrule=0.5pt, left=1.5mm, right=1.5mm, top=1.5mm, bottom=1.5mm]
\begin{definition}[Sistema]
    Um sistema de dedução consiste num par $\mathfrak{S} = \sequence{\mathcal{L}, \vdash}$, onde $\mathcal{L}$ consiste em um conjunto e ${\vdash}\subseteq\mathfrak{P}(\mathcal{L})\times\mathcal{L}$ em uma relação sobre o produto cartesiano do conjunto das partes de $\mathcal{L}$ e o conjunto $\mathcal{L}$, sem demais condições.
\end{definition}
\end{tcolorbox}

\vspace{.5\baselineskip}
Os primeiros desenvolvimentos neste sentido foram feitos por~\cite{Tarski.1983}, que define o conceito de dedução com base num operador de fecho $\mathbf{C}\mathrel{:}\mathfrak{P}(\mathcal{L})\to\mathfrak{P}(\mathcal{L})$, sendo $\mathcal{L}$ um conjunto qualquer.
Usaremos entretanto uma definição baseada numa relação de dedução ${\vdash}\subseteq\mathfrak{P}(\mathcal{L})\times\mathcal{L}$, que se mostra mais conveniente para as intenções deste trabalho~\citep{Coniglio.2005}.
Cabe destacar que ambas as definições são equivalentes, uma vez que $\Gamma\entails\alpha$ se e somente se $\alpha\in\mathbf{C}(\Gamma)$.
Destaca-se, entretanto, que o operador de fecho requer a satisfação de postulados não requeridos aqui, motivo pelos quais sistemas que respeitam tais condições são ditos \emph{tarskianos}.

\vspace{.5\baselineskip}
A qualidade e quantidade dos elementos de um sistema $\mathfrak{S}=\sequence{\mathcal{L}, \vdash}$ não são especificados, portanto sendo esta uma definição de grande generalidade.
Neste sentido, com base no escopo deste trabalho, restringiremos a definição do conjunto $\mathcal{L}$ --- dito \emph{linguagem} --- a linguagens proposicionais.
Os elementos destas, aos quais daremos o nome de \emph{sentenças}, notabilizam-se por serem formadas por \emph{letras} --- que consistem em proposições indivisas --- e \emph{operadores} --- que podem gerar proposições maiores a partir de proposições menores.
Ao par formado por letras e operadores daremos o nome \emph{assinatura}, conforme abaixo.

\vspace{.5\baselineskip}
\begin{tcolorbox}[enhanced jigsaw, breakable, sharp corners, colframe=black, colback=white, boxrule=0.5pt, left=1.5mm, right=1.5mm, top=1.5mm, bottom=1.5mm]
\begin{definition}[Assinatura]
    Uma assinatura proposicional consiste num par $\Sigma=\langle\mathcal{P},\mathcal{C}\rangle$, onde $\mathcal{P}$ consiste num conjunto letras e $\mathcal{C}=\bigcup\set{\mathcal{C}_i\mid i\in\mathbb{N}}$ num conjunto de operadores de modo que $\mathalpha{\circ}\in\mathcal{C}_n$ se e somente se $\mathalpha{\circ}$ possuir aridade $n$.
\end{definition}
\end{tcolorbox}

\begin{tcolorbox}[enhanced jigsaw, breakable, sharp corners, colframe=black, colback=white, boxrule=0.5pt, left=1.5mm, right=1.5mm, top=1.5mm, bottom=1.5mm]
\begin{notation}
    Seja $\mathcal{C}$ um conjunto de operadores, $\mathalpha{\circ}^n$ denota um operador $\mathalpha{\circ}\in\mathcal{C}_n$.
\end{notation}
\end{tcolorbox}

\vspace{.5\baselineskip}
Podemos interpretar os conjuntos $\mathcal{P}$ e $\mathcal{C}$ de uma assinatura $\Sigma=\sequence{\mathcal{P},\mathcal{C}}$ como construtores de sentenças.
Neste sentido, o conjunto $\mathcal{C}_0$ assemelha-se mais ao conjunto $\mathcal{P}$, uma vez que seus elementos --- ditos \emph{constantes} --- não geram sentenças maiores partindo de sentenças menores.
Nota-se que uma assinatura constitui um elemento suficiente para definirmos indutivamente a linguagem de um sistema, conforme definido abaixo de maneira similar a~\cite{Franks.2025}.
Por fim, destacamos que, para todos os sistemas apresentados neste trabalho, usaremos o conjunto de letras $\mathcal{P}=\set{p_i\mid i\in\mathbb{N}}$ e letras romanas em caixa-baixa para representar seus elementos.

\vspace{.5\baselineskip}
\begin{tcolorbox}[enhanced jigsaw, breakable, sharp corners, colframe=black, colback=white, boxrule=0.5pt, left=1.5mm, right=1.5mm, top=1.5mm, bottom=1.5mm]
\begin{definition}[Linguagem]
    Seja $\Sigma=\sequence{\mathcal{P},\mathcal{C}}$ uma assinatura proposicional. Uma linguagem proposicional $\mathcal{L}$ induzida a partir de $\Sigma$ consiste no menor conjunto de sentenças bem-formadas induzido a partir das regras que seguem.
    \begin{enumerate}[label=\textbf{\emph{(\alph*)}}, left=\parindent]
        \item$\mathcal{P}\subseteq\mathcal{L}$,
        \item\text{Se }$\mathalpha{\circ}\in\mathcal{C}_n\text{ e }\set{\varphi_i\mid i\leq n}\subseteq\mathcal{L}\text{, então }\circ\sequence{\varphi_i\mid i\leq n}\in\mathcal{L}$.
    \end{enumerate}
\end{definition}
\end{tcolorbox}

\vspace{.5\baselineskip}
Neste trabalho, representaremos sentenças por letras gregas em caixa-baixa e conjuntos de sentenças por letras gregas em caixa-alta.\footnote{Desconsiderando-se o $\Sigma$, usado para representar assinaturas.}
Ademais, impõe-se definir a noção de profundidade de uma sentença.
Esta noção, em termos simples, consiste no comprimento do maior ramo da construção da dada sentença.
A definição provida abaixo consiste numa generalização para quaisquer aridades da definição dada por~\cite{Troelstra+Schwichtenberg.2000}.
Usaremos essa definição futuramente para fazer demonstrações por meio provas indutivas sobre esta propriedade.

\vspace{.5\baselineskip}
\begin{tcolorbox}[enhanced jigsaw, breakable, sharp corners, colframe=black, colback=white, boxrule=0.5pt, left=1.5mm, right=1.5mm, top=1.5mm, bottom=1.5mm]
\begin{definition}[Profundidade]
    Seja $\mathfrak{S} = \sequence{\mathcal{L}, \vdash}$ um sistema com linguagem induzida a partir de uma assinatura $\Sigma=\sequence{\mathcal{P},\mathcal{C}}$. Considerando-se uma proposição $a\in\mathcal{P}$, um operador ${\circ}\in\mathcal{C}$ e uma aridade $n>0$, definimos a profundidade $|\alpha|$ de uma sentença $\alpha\in\mathcal{L}$ indutivamente da maneira que segue.
    \begin{align*}
        |a|&\coloneqq 0\\
        |{\circ^0}|&\coloneqq 0\\
        |{\circ^n\sequence{\varphi_i\mid i\leq n}}|&\coloneqq\max\set{|\varphi_i|\mid i\leq n}+1
    \end{align*}
\end{definition}
\end{tcolorbox}

\vspace{.5\baselineskip}
Agora, apresentaremos definições relacionadas a relações de dedução, que gozam da mesma generalidade dada a liguagens.
Neste trabalho, definiremos as relações de dedução como relações de \emph{derivação} baseadas em um conjunto de regras que permitem derivar sentenças verdadeiras caso um conjunto de \emph{juízos} sejam satisfeitos.
Caso não haja nenhuma condições antecedentes para o uso das regras, dizemos que estas são \emph{axiomas}.
Adotaremos uma abordagem hilbertiana de dedução, que se distingue por conter um conjunto reduzido de regras com condições antecedentes que nunca descartam premissas~\citep{Troelstra+Schwichtenberg.2000}.

\vspace{.5\baselineskip}
\begin{tcolorbox}[enhanced jigsaw, breakable, sharp corners, colframe=black, colback=white, boxrule=0.5pt, left=1.5mm, right=1.5mm, top=1.5mm, bottom=1.5mm]
\begin{definition}[Dedução]
    Seja um sistema $\mathfrak{S} = \sequence{\mathcal{L},{\vdash}}$ com uma relação de dedução definida sobre um conjunto de regras $\mathcal{R}$ e seja um conjunto de sentenças $\Gamma\cup\set{\alpha}\subseteq\mathcal{L}$.
    A dedução $\Gamma\vdash\alpha$ vale se e somente se houver uma sucessão de sentenças $\sequence{\varphi_i\in\mathcal{L}\mid i\leq n}$ de modo que $\varphi_n=\alpha$ e de modo que cada sentença $\varphi_i$ tenha sido gerada por alguma regra $\mathbf{R}\in\mathcal{R}$ aplicada, caso preciso, a sentenças anteriores.
\end{definition}
\end{tcolorbox}

\section{Traduções}\label{foundation.translations}

Traduções entre sistemas consistem em funções que mapeiam sentenças de um sistema a sentenças de outro, garantindo certas propriedades.
As propriedades a serem garantidas variam e ainda são discutidas na literatura, podendo ser mais fortes ou mais fracas.
Neste trabalho, adotaremos uma noção forte de tradução que requer tanto a correção forte quanto a completude forte, dita \emph{conservativa}~\citep{Coniglio.2005}.
Definiremos, ainda, uma notação que nos permite aplicar sucintamente a tradução a todos os elementos de um conjunto.

\vspace{.5\baselineskip}
\begin{tcolorbox}[enhanced jigsaw, breakable, sharp corners, colframe=black, colback=white, boxrule=0.5pt, left=1.5mm, right=1.5mm, top=1.5mm, bottom=1.5mm]
\begin{definition}[Tradução] 
    Uma sentença $\alpha$ de um sistema $\mathfrak{A} = \langle\mathcal{A}, \vdash_\mathfrak{A}\rangle$ pode ser traduzida a uma sentença $\alpha^*$ em um sistema $\mathfrak{B} = \langle\mathcal{B}, \vdash_\mathfrak{B} \rangle$ caso exista uma função $\bullet^* : \mathcal{A} \to \mathcal{B}$ que garanta que $\Gamma\vdash_\mathfrak{A}\alpha$ se e somente se $\Gamma^*\vdash_\mathfrak{B}\alpha^*$.
\end{definition}
\end{tcolorbox}

\begin{tcolorbox}[enhanced jigsaw, breakable, sharp corners, colframe=black, colback=white, boxrule=0.5pt, left=1.5mm, right=1.5mm, top=1.5mm, bottom=1.5mm]
\begin{notation}
    Seja $\Gamma\in\mathfrak{P}(\mathcal{A})$ um conjunto de sentenças bem-formadas e $\bullet^*\mathrel{:}\mathcal{A}\to\mathcal{B}$ uma tradução. $\Gamma^*$ denota o conjunto $\set{\alpha^*\mid\alpha\in\Gamma}\in\mathfrak{P}(\mathcal{B})$, ou seja, a aplicação da tradução a todos os elementos do conjunto $\Gamma$.
    \qed{}
\end{notation}
\end{tcolorbox}

\vspace{.5\baselineskip}
Uma \emph{imersão} consiste numa função injetora que mapeia uma estrutura em outra de uma maneira que a estrutura de origem seja de alguma forma preservada.
Percebe-se que uma tradução, da forma como foi apresentada, pode ser vista como uma imersão caso seja injetora, com a estrutura preservada sendo a relação de dedução.
Dizemos então que um sistema foi \emph{imerso} em outro.

\vspace{.5\baselineskip}
A primeira tradução entre dois sistemas conhecida na literatura foi definida por~\cite{Kolmogorov.1967} como uma maneira de demonstrar que o uso da lei do \emph{tertium non datur} --- que valida sentenças da forma $\alpha\vee\neg\alpha$ --- não leva a contradições.
Esta mesma tradução foi descoberta de maneira independente por~\cite{Gödel.1986a} e por~\cite{Gentzen.1969}.
Essa definição consiste basicamente em prefixar uma dupla negação a cada elemento da construção de uma dada sentença \citep{Coniglio.2005}, motivo que levou esta tradução a ser chamada de \emph{tradução por negação dupla}.

\section{Provadores}\label{foundation.provers}

A primeira prova de destaque a ser realizada com grande uso de computadores foi a do teorema das quatro cores, feita por~\cite{Appel+Haken.1976}, motivado pela grande quantidade de casos a serem analisados.\footnote{Este teorema afirma que \emph{qualquer mapa planar tem uma quatro-coloração}.}
Conforme~\cite{Wilson.2021} afirma, esta prova foi, por uns, recebida com entusiasmo e por outros, devido ao uso de computadores, com cetistismo e desapontamento. Dentre aqueles que compartilharam destas visões opositoras, destaca-se~\cite{Tymoczko.1979}. Ainda segundo~\cite{Wilson.2021}, o teorema tornou-se mais aceito com o passar do tempo e foi, posteriormente, formalizado em um provador de teoremas por~\cite{Gonthier.2008}.

\vspace{.5\baselineskip}
Provadores de teoremas consistem em programas de computador que verificam a validade de teoremas. Dentre estes, podemos destacar as classes dos provadores \emph{automáticos} e dos provadores \emph{interativos}. Os primeiros buscam provar teoremas de maneira que requeira a menor quantidade de intervenção humana, enquanto os segundos --- que ganharam destaque depois das limitações dos primeiros ficarem evidentes --- delegam-se a verificar rigorosamente provas desenvolvidas por humanos em sua linguagem. Formalizaremos as provas apresentadas neste trabalho no provador de teoremas interativo \textit{Rocq}, o mesmo \emph{software} usado por~\cite{Gonthier.2008}.

\vspace{.5\baselineskip}
O \emph{Rocq} --- previamente chamado \emph{Coq} --- trata-se de um provador de teoremas interativo baseado no formalismo das \emph{construções} de~\cite{Coquand+Huet.1988}, que fica no topo do cubo-$\lambda$ de~\cite{Barendregt.1991}.
O cubo-$\lambda$ classifica sistemas-$\lambda$ em um cubo onde cada dimensão $x$, $y$ e $z$ representa uma propriedade.
O eixo $x$ representa a propriedade tipos poderem depender em termos, o eixo $y$ representa a propriedade de termos poderem depender em tipos e o eixo $z$ representa a propriedade de tipos poderem depender em tipos.
Deste modo, o \emph{Rocq} goza de todas estas capacidades.

\vspace{.5\baselineskip}
Este sistema formal fornece uma estrutura unificada para definir funções, tipos e proposições, permitindo a construção e verificação de provas dentro do mesmo formalismo. No \emph{Rocq}, entretanto, este formalismo foi estendido de modo a permitir tipos indutivos e coindutivos, criando as ditas \emph{construções indutivas e coindutivas}.
Neste, pode-se definir tipos de dados estruturados e funções e provas recursivas.
Essa fundação alinha-se com a interpretação de programas como demonstrações e de tipos como proposições, tornando o \textit{Rocq} uma ferramenta poderosa de formalização e verificação.
Para um maior aprofundamento acerca do provador de teoremas \emph{Rocq}, remetemos o leitor a~\cite{Chlipala.2013} e~\cite{Pierce.2025}.
