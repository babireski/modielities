\chapter{Fundamentação}

Nesta parte do trabalho, serão apresentadas definições gerais que fundamentarão as definições mais estritas que serão apresentadas futuramente. Notadamente, fundamentaremos as noções de sistemas e traduções. Ademais, discorreremos acerca da noção de provadores, que serão usados para certificar as provas apresentadas posteriormente. Antes disso, entretanto, introduziremos duas notações que serão usadas copiosamente, uma para o conjunto das partes e outra para sucessões.

\begin{notation}
    Seja $A$ um conjunto, $\wp(A)$ denota o conjunto $\set{X\mid X\subseteq A}$.
\end{notation}

\begin{notation}
    Seja $i\in\mathbb{N}^+$ e $n\in\mathbb{N}$, $\sequence{a_i\mid i\leq n}$ denota uma sucessão de $n$ elementos de modo que o elemento $a_i$ encontra-se na posição $i$.
\end{notation}

\section{Sistemas}

Sistemas de dedução buscam formalizar e sistematizar o processo de razoamento. Estudos acerca disso datam da antiguidade, dentre os quais destaca-se~\cite{Aristotle}. Considera-se que os estudos modernos neste campo foram, dentre outras pessoas, fundados por~\cite{Frege} e continuados por~\cite{Russel-A,Russel-B,Russel-C}. Estas investigações --- bem como outras --- levaram ao desenvolvimento do sistema hoje tido como padrão. Posteriormente a isso, viu-se o surgimento de diversos sistemas não-padrões, fato que --- conforme~\cite{Beziau-B} --- justifica uma conceituação de sistema de dedução, que apresentaremos nesta seção.

Ainda segundo~\cite{Beziau-B}, os primeiros desenvolvimentos neste sentido foram feitos por~\cite{Tarski}, que define o conceito de dedução com base num operador de fecho $C\mathrel{:}\wp(\mathcal{L})\to\wp(\mathcal{L})$, sendo $\mathcal{L}$ um conjunto qualquer. Neste trabalho entretanto usaremos a definição proposta por~\cite{Beziau} baseada numa relação de dedução ${\vdash}\subseteq\wp(\mathcal{L})\times\mathcal{L}$, uma vez que, por sua simplicidade, não traz elementos irrelevantes aos intuitos deste. Cabe destacar, conforme apontam~\cite{Font}, que ambas as definições são equivalentes\footnote{Destaca-se, entretanto, que a definição de~\cite{Tarski} requer a satisfação de postulados não requeridos por~\cite{Beziau}, sendo portanto menos geralista.}, uma vez que $\Gamma\entails\alpha$ se e somente se $\alpha\in C(\Gamma)$.

\begin{definition}[Sistema]
    Um sistema de dedução consiste num par $\mathbf{L} = \sequence{\mathcal{L}, \vdash}$, onde $\mathcal{L}$ consiste em um conjunto e ${\vdash}\subseteq\wp(\mathcal{L})\times\mathcal{L}$ em uma relação sobre o produto cartesiano do conjunto das partes de $\mathcal{L}$ e o conjunto $\mathcal{L}$, sem demais condições.
    \qed{}
\end{definition}

Conforme~\cite{Beziau} aponta, a qualidade e quantidade dos elementos de um sistema $\mathbf{L}=\sequence{\mathcal{L}, \vdash}$ não são especificados, portanto sendo esta uma definição de grande generalidade. Neste sentido, com base no escopo deste trabalho, restringiremos a definição do conjunto $\mathcal{L}$ --- dito \emph{linguagem} --- a linguagens proposicionais. Estas notabilizam-se por serem formadas por \emph{letras} --- que consistem em proposições indivisas --- e \emph{operadores} --- que podem gerar proposições maiores a partir de proposições menores. A cada uma dessas proposições daremos o nome de \emph{sentenças} e ao par formado por letras e operadores o nome de \emph{assinatura}, conforme abaixo.

\begin{definition}[Assinatura]
    Uma assinatura proposicional consiste num par $\Sigma=\sequence{\mathcal{P},\mathcal{C}}$, onde $\mathcal{P}$ consiste num conjunto letras e $\mathcal{C}=\bigcup\set{\mathcal{C}_i\mid i\in\mathbb{N}}$ num conjunto de operadores de modo que $\mathalpha{\bullet}\in\mathcal{C}_n$ se e somente se $\mathalpha{\bullet}$ possuir aridade $n$.
    \qed{}
\end{definition}

\begin{notation}
    Seja $\mathcal{C}$ um conjunto de operadores, $\mathalpha{\bullet}^n$ denota um operador $\mathalpha{\bullet}\in\mathcal{C}_n$.
\end{notation}

Podemos interpretar os conjuntos $\mathcal{P}$ e $\mathcal{C}$ de uma assinatura $\Sigma=\sequence{\mathcal{P},\mathcal{C}}$ como contrutores de sentenças.
Neste sentido, o conjunto $\mathcal{C}_0$ assemelha-se mais ao conjunto $\mathcal{P}$, uma vez que seus elementos --- ditos \emph{constantes} --- não geram sentenças maiores partindo de sentenças menores.
Nota-se que uma assinatura constitui um elemento suficiente para definirmos indutivamente a linguagem de um sistema, conforme definido abaixo de maneira similar a~\cite{Franks}.
Por fim, destacamos que, para todos os sistemas apresentados neste trabalho, usaremos $\mathcal{P}=\set{p_i\mid i\in\mathbb{N}}$ e letras romanas em caixa-baixa para representar seus elementos.

\begin{definition}[Linguagem]
    Seja $\Sigma=\sequence{\mathcal{P},\mathcal{C}}$ uma assinatura proposicional. Uma linguagem proposicional $\mathcal{L}$ consiste no menor conjunto de sentenças bem-formadas induzido a partir das seguintes regras:
    \begin{enumerate}[label=\textbf{\emph{(\alph*)}}, left=\parindent]
        \item$\mathcal{P}\subseteq\mathcal{L}$
        % \item$\text{Se }\bullet^0\in\mathcal{C}\text{, então }\bullet^0\in\mathcal{L}$
        \item\text{Se }$\mathalpha{\bullet}^n\in\mathcal{C}\text{ e }\set{\varphi_i\mid i\leq n}\subseteq\mathcal{L}\text{, então }\bullet\sequence{\varphi_i\mid i\leq n}\in\mathcal{L}$.\qed{}
    \end{enumerate}
\end{definition}

Tendo definidas as noções de assinatura e sentença, pode-se definir a noção de profundidade de uma sentença. Esta noção, em termos simples, consiste no comprimento do maior ramo da contrução da sentença. A definição abaixo provida consiste numa generalização da definição dada por~\cite{Troelstra}. Usaremos essa definição futuramente para provar propriedades por meio de indução sobre esta propriedade.

\begin{definition}[Profundidade]
    Seja $\mathbf{A} = \sequence{\mathcal{L}, \vdash}$ um sistema, $\alpha\in\mathcal{L}$ uma sentença e $\bullet^n$ um operador de aridade $n\in\mathbb{N}$ que consta na assinatura que define $\mathcal{L}$. Pode-se definir a profundidade $|\alpha|$ de $\alpha$ recursivamente da seguinte maneira:
    \begin{align*}
        |a|&\coloneqq 0\\
        |\bullet^0|&\coloneqq 0\\
        |\bullet^n\sequence{\varphi_i\mid i\leq n}|&\coloneqq\max\set{|\varphi_i|\mid i\leq n}+1.
        \tag*{\qed} 
    \end{align*}
\end{definition}

Definindo-se profundidade, encerramos as definições desta fundamentação que dizem respeito ao primeiro elemento de um sistema: a linguagem. Agora as definições fornecidas dirão respeito a relações de derivação. Neste trabalho, as relações de derivação abordadas serão baseadas em axiomatizações, ou seja, em pares de esquemas de axiomas e regras de dedução.

Deste modo, a relação $\mathalpha{\vdash}$ pode ser tanto uma relação de \emph{derivação} --- definida sintaticamente --- quanto uma relação de \emph{satisfação}\footnote{Sendo esta denotada por $\mathalpha{\vDash}$.} --- definida semanticamente. Neste trabalho, serão abordados apenas sistemas definidos sobre relações de derivação. Cabe destacar, entretanto, que nada na definição de tradução impede que esta seja feita sobre relações de satisfação, como veremos futuramente.

\begin{definition}[Esquema]
    Um esquema consiste em um padrão com metavariaveis que permitem representar um conjunto, geralmente infinito, de sentenças.
    \qed{}
\end{definition}

\begin{definition}[Regra]
    Uma regra de dedução consiste num par $\sequence{A, a}$, sendo $A$ um conjunto de asserções chamadas de \textit{premissas} e $a$ uma asserção chamada \textit{conclusão}.
    \qed{}
\end{definition}

\begin{definition}[Axiomatização]
    Um sistema de Hilbert para um sistema $\mathbf{L} = \sequence{\mathcal{L}, \vdash}$ consiste em um par $\mathcal{H} = \sequence{\mathcal{A}, \mathcal{R}}$, sendo $\mathcal{A}$ um conjunto de esquemas de axiomas e $\mathcal{R}_0\subseteq\mathcal{R}$ o conjunto de regras de dedução abaixo.
    \begin{alignat}{3}
        &\mathbf{A}\quad&&\text{Se }\alpha\in\mathcal{A}\text{, então }\Gamma\entails\alpha\tag*{}\\
        &\mathbf{P}\quad&&\text{Se }\alpha\in\Gamma\text{, então }\Gamma\entails\alpha\tag*{}\\
        &\mathbf{E}\quad&&\text{Se }\entails\alpha\text{, então }\Gamma\entails\alpha.\tag*{\qed}
    \end{alignat}
\end{definition}

\begin{definition}[Dedução]
    Uma dedução de $\Gamma\vdash\alpha$ consiste numa sucessão de sentenças $\sequence{\varphi_i\mid 1 \leq i\leq n}$ de modo que $\varphi_n=\alpha$ e cada sentença $\varphi_i$ foi gerada a partir da aplicação de uma regra a axiomas, premissas ou sentenças anteriores.
    \qed{}
\end{definition}

\section{Traduções}

Traduções entre sistemas consistem em funções que mapeiam sentenças de um sistema a sentenças de outro, garantindo certas propriedades. As propriedades a serem garantidas variam e ainda são discutidas na literatura, deixando a definição exata de tradução --- assim como houve com a definição de sistema --- varie de acordo com a predileção e as necessidades de cada autor. Nesta seção, serão abordadas historicamente noções de tradução entre sistemas, bem como serão definidos e nomeados os conceitos de tradução que serão usados no restante deste trabalho.

Neste trabalho, adotaremos uma noção forte de tradução que requer tanto a correção forte quanto a completude forte, conforme~\cite{Coniglio}. Definiremos, ainda, uma notação que nos permite aplicar sucintamente a tradução a todos os elementos de um conjunto.

\begin{definition}[Tradução] 
    Uma sentença $\varphi$ de um sistema $\mathbf{A} = \langle \mathcal{L}_\mathbf{A}, \vdash_\mathbf{A}\rangle$ pode ser traduzida a uma sentença $\varphi^*$ em um sistema $\mathbf{B} = \langle \mathcal{L}_\mathbf{B}, \vdash_\mathbf{B} \rangle$ caso exista uma função $\bullet^* : \mathcal{L}_\mathbf{A} \to \mathcal{L}_\mathbf{B}$ que garanta que $\Gamma \vdash_\mathbf{A} \varphi \Leftrightarrow \Gamma^* \vdash_\mathbf{B} \varphi^*$.
    \qed{}
\end{definition}

\begin{notation}
    Seja $\Gamma\in\wp(\mathcal{L}_\mathbf{A})$ um conjunto de sentenças bem-formadas e $\bullet^*\mathrel{:}\mathcal{L}_\mathbf{A}\to\mathcal{L}_\mathbf{B}$ uma tradução. $\Gamma^*$ denota o conjunto $\set{\alpha^*\mid\alpha\in\Gamma}\in\wp(\mathcal{L}_\mathbf{B})$, ou seja, a aplicação da tradução a todos os elementos do conjunto $\Gamma$.
    \qed{}
\end{notation}

A primeira tradução entre dois sistemas conhecida na literatura foi definida por~\cite{Kolmogorov} como uma maneira de demonstrar que o uso da \emph{lei do terço excluso}\footnote{Definido como $\entails\alpha\vee\neg\alpha$.} não leva a contradições. Essa definição consiste basicamente em dobre-negar cada elemento da construção de uma dada sentença, motivo pelo qual chamaremos essa tradução de \emph{tradução de negação dupla} \citep{Coniglio}. Essa mesma tradução foi também descoberta independentemente por Gödel e por Getzen. Curiosamente, essa tradução mostra-se relevante para o escopo deste trabalho, uma vez que consiste na contraparte da passagem por continuações segundo a interpretação prova-programa.

\begin{definition}[$\bullet^\neg$] Define-se a tradução $\bullet^\neg$ indutivamente da seguinte maneira:
    \begin{align*}
        p^\neg&\coloneqq\neg\neg p\\
        \bot^\neg&\coloneqq\neg\neg\bot\\
        {(\varphi\wedge\psi)}^\neg&\coloneqq\neg\neg(\varphi^\neg \wedge \psi^\neg)\\
        {(\varphi\vee\psi)}^\neg&\coloneqq\neg\neg (\varphi^\neg \vee \psi^\neg)\\
        {(\varphi\to\psi)}^\neg&\coloneqq\neg\neg (\varphi^\neg \to \psi^\neg)
        \tag*{\qed} 
    \end{align*}
\end{definition}

\section{Provadores}
