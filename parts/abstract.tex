\begin{titlepage}
{\customword{Abstract}}

Some modal systems, especially $\mathbf{S4}$, have interesting attributes for representing computational effects.
A language based on this system, capable of representing effects monadically, can be embedded into continuations and vice-versa.
Continuation-passing style is one of the various representations used in compilers, making these systems relevant from a compilation perspective.
Similarly, the intuitionistic system has computational significance, as the intuitionistic notion of \emph{proof construction} resembles the notion of \emph{computation}.
This work presents two translations from the intuitionistic system to the modal system $\mathbf{S4}$, along with proofs of their interderivability and correctness properties.
All definitions and assertions are implemented and verified in the interactive theorem prover \emph{Rocq}.

\vspace{.5\baselineskip}
\textit{Keywords} --- Translations between logics. Modal systems. Computational effects. Compilation. \emph{Rocq}.
\end{titlepage}