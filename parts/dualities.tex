\section{Interpretações computacionais}

    Conforme~\cite{Zach} aponta, pode-se derivar sentenças duais aos axiomas modais por meio da contraposição das implicações e da definição da possibilidade.
    No caso do sistema abordado neste trabalho, podemos derivar as sentenças $\alpha\to\pos\alpha$ e $\pos\pos\alpha\to\pos\alpha$.
    Faz-se interessante notar que estas sentenças apresentam grande similaridade com as tranformações naturais $\eta:1_C\to T$ e $\mu:T^2\to T$ de uma estrutura $\sequence{T,\eta,\mu}$, chamada \emph{mônada}.
    Esta estrutura inspirou uma metalinguagem definida por~\cite{Moggi} para a modelagem de noções de computação --- como parcialidade, não-determinismo, excessões e continuações.
    Posteriormente,~\cite{Wadler} sugeriu o uso dessas estruturas como um \emph{framework} para simular efeitos em linguagens de programação puramente funcionais.

    \begin{theorem}
        $\vdash\alpha\to\pos\alpha$.
        \begin{proof}
            Pode ser provado pela seguinte sucessão de dedução:
            \footnotesize
            \begin{fitch}
                \fb\entails\alpha\to\neg\neg\alpha&\refer{neg-intro}{L}\\
                \fa\entails\nec\neg\alpha\to\neg\alpha&$\hyperref[MB2]{\mathbf{T}}$\\
                \fa\entails(\nec\neg\alpha\to\neg\alpha)\to\neg\neg\alpha\to\pos\alpha&$\hyperref[contrapositive]{\mathbf{L_3}}$\\
                \fa\entails\neg\neg\alpha\to\pos\alpha&$\hyperref[detachment]{\mathbf{R_1}}\;\set{2,3}$\\
                \fa\entails(\alpha\to\neg\neg\alpha)\to(\neg\neg\alpha\to\pos\alpha)\to\alpha\to\pos\alpha&\refer{comp}{L}\\
                \fa\entails(\neg\neg\alpha\to\pos\alpha)\to\alpha\to\pos\alpha&$\hyperref[detachment]{\mathbf{R_1}}\;\set{1,5}$\\
                \fa\entails\alpha\to\pos\alpha&$\hyperref[detachment]{\mathbf{R_1}}\;\set{4,6}$
            \end{fitch}
            \normalsize
            Estando assim demonstrada a proposição.
        \end{proof}
    \end{theorem}

    \begin{theorem}
        $\vdash\pos\pos\alpha\to\pos\alpha$.
        \begin{proof}
            Pode ser provado pela seguinte sucessão de dedução:
            \footnotesize
            \begin{fitch}
                \fb\entails\nec\neg\alpha\to\nec\nec\neg\alpha&$\hyperref[MB3]{\mathbf{4}}$\\
                \fa\entails\nec\neg\alpha\to\neg\pos\alpha&$\hyperref[neg-intro]{\mathbf{L_{\getrefnumber{neg-intro}}}}$\\
                \fa\entails\nec(\nec\neg\alpha\to\neg\pos\alpha)&$\hyperref[necessitation]{\mathbf{R_2}}\;\set{2}$\\
                \fa\entails\nec(\neg\pos\alpha\to\nec\neg\alpha)\to\nec\nec\neg\alpha\to\nec\neg\pos\alpha&$\hyperref[MB1]{\mathbf{K}}$\\
                \fa\entails\nec\nec\neg\alpha\to\nec\neg\pos\alpha&$\hyperref[detachment]{\mathbf{R_1}}\;\set{3,4}$\\
                \fa\entails(\nec\neg\alpha\to\nec\nec\neg\alpha)\to(\nec\nec\neg\alpha\to\nec\neg\pos\alpha)\to\nec\neg\alpha\to\nec\neg\pos\alpha&\refer{comp}{L}\\
                \fa\entails(\nec\nec\neg\alpha\to\nec\neg\pos\alpha)\to\nec\neg\alpha\to\nec\neg\pos\alpha&$\hyperref[detachment]{\mathbf{R_1}}\;\set{1,6}$\\
                \fa\entails\nec\neg\alpha\to\nec\neg\pos\alpha&$\hyperref[detachment]{\mathbf{R_1}}\;\set{5,7}$\\
                \fa\entails(\nec\neg\alpha\to\nec\neg\pos\alpha)\to\pos\pos\alpha\to\pos\alpha&\refer{contrapositive}{L}\\
                \fa\entails\pos\pos\alpha\to\pos\alpha&$\hyperref[detachment]{\mathbf{R_1}}\;\set{8,9}$\\
            \end{fitch}
            \normalsize
            Estando assim demonstrada a proposição.
        \end{proof}
    \end{theorem}

    Apesar da similaridades com as transformações naturais, deve-se destacar que as noções de computação não podem ser interpretadas simplesmente como necessidade ou possibilidade, uma vez que apresentam propriedades presentes em ambas as modalidades. Neste sentido, a modalidade de \emph{laxidade} --- que combina noções de necessidade e possibilidade --- mostra-se uma melhor representação de efeitos computacionais sobre a interpretação programa-prova.
    
    Ao sistema que comporta essa modalidade --- denotada $\lax$ --- damos o nome de sistema laxo ou simplesmente $\mathbf{L}$. Este sistema foi primeiramente considerado por~\cite{Curry-A,Curry-B} e posteriormente redescoberto por~\cite{Fairtlough,Mendler} como uma tentativa de representar correção dentro de restrições na verificação formal de \emph{hardware} de computadores. Pode ser definido formalmente por meio da assinatura $\Sigma_\mathbf{L}=\sequence{\mathcal{P},\mathcal{C}_\mathbf{L}}$ e da axiomatização $\mathcal{H}=\sequence{\mathcal{A}_\mathbf{L},\mathcal{R}_\mathbf{I}}$, onde $\mathcal{C}_\mathbf{L}=\mathcal{C}_\mathbf{I}\cup\set{\lax^1}$ e $\mathcal{A}_\mathbf{L}=\mathcal{A}_\mathbf{I}\cup\set{\mathbf{C_1},\mathbf{C_2},\mathbf{C_3}}$, considerando-se os esquemas abaixo:
    \begin{alignat*}{3}
        &\mathbf{C_1}\quad&&\alpha\to\lax\alpha\\
        &\mathbf{C_2}\quad&&\lax\lax\alpha\to\lax\alpha\\
        &\mathbf{C_3}\quad&&(\alpha\to\beta)\to\lax\alpha\to\lax\beta
    \end{alignat*}

~\cite{Benton} e~\cite{Pfenning} notam a capacidade deste sistema de representar a metalinguagem de computação apresentada por~\cite{Moggi}.
    Ainda,~\cite{Pfenning} apresentam uma tradução desse sistema a um sistema $\mathbf{S_4}$ intuicionista.