\section{Interpretações computacionais}

    Conforme~\cite{Zach} aponta, pode-se derivar sentenças duais aos axiomas modais por meio da contraposição das implicações e da definição da possibilidade.
    No caso do sistema abordado neste trabalho, podemos derivar as sentenças $\alpha\to\pos\alpha$ e $\pos\pos\alpha\to\pos\alpha$.
    Faz-se interessante notar que estas sentenças apresentam grande similaridade com as tranformações naturais $\eta:1_C\to T$ e $\mu:T^2\to T$ de uma estrutura $\sequence{T,\eta,\mu}$, chamada \emph{mônada}.
    Esta estrutura inspirou uma metalinguagem definida por~\cite{Moggi} para a modelagem de noções de computação --- como parcialidade, não-determinismo, excessões e continuações.
    Posteriormente,~\cite{Wadler} sugeriu o uso dessas estruturas como um \emph{framework} para simular efeitos em linguagens de programação puramente funcionais.

    \begin{theorem}
        $\vdash\alpha\to\pos\alpha$.
        \begin{proof}
            Pode ser provado pela seguinte sucessão de dedução:
            \footnotesize
            \begin{fitch}
                \fb\entails\alpha\to\neg\neg\alpha&\refer{neg-intro}{L}\\
                \fa\entails\nec\neg\alpha\to\neg\alpha&$\hyperref[MB2]{\mathbf{B_2}}$\\
                \fa\entails(\nec\neg\alpha\to\neg\alpha)\to\neg\neg\alpha\to\pos\alpha&$\hyperref[contrapositive]{\mathbf{L_3}}$\\
                \fa\entails\neg\neg\alpha\to\pos\alpha&$\hyperref[detachment]{\mathbf{R_1}}\;\set{2,3}$\\
                \fa\entails(\alpha\to\neg\neg\alpha)\to(\neg\neg\alpha\to\pos\alpha)\to\alpha\to\pos\alpha&\refer{comp}{L}\\
                \fa\entails(\neg\neg\alpha\to\pos\alpha)\to\alpha\to\pos\alpha&$\hyperref[detachment]{\mathbf{R_1}}\;\set{1,5}$\\
                \fa\entails\alpha\to\pos\alpha&$\hyperref[detachment]{\mathbf{R_1}}\;\set{4,6}$
            \end{fitch}
            \normalsize
            Estando assim demonstrada a proposição.
        \end{proof}
    \end{theorem}

    \begin{theorem}
        $\vdash\pos\pos\alpha\to\pos\alpha$.
        \begin{proof}
            Pode ser provado pela seguinte sucessão de dedução:
            \footnotesize
            \begin{fitch}
                \fb\entails\nec\neg\alpha\to\nec\nec\neg\alpha&$\hyperref[MB3]{\mathbf{B_3}}$\\
                \fa\entails\nec\neg\alpha\to\neg\pos\alpha&$\hyperref[neg-intro]{\mathbf{L_{\getrefnumber{neg-intro}}}}$\\
                \fa\entails\nec(\nec\neg\alpha\to\neg\pos\alpha)&$\hyperref[necessitation]{\mathbf{R_2}}\;\set{2}$\\
                \fa\entails\nec(\neg\pos\alpha\to\nec\neg\alpha)\to\nec\nec\neg\alpha\to\nec\neg\pos\alpha&$\hyperref[MB1]{\mathbf{B_1}}$\\
                \fa\entails\nec\nec\neg\alpha\to\nec\neg\pos\alpha&$\hyperref[detachment]{\mathbf{R_1}}\;\set{3,4}$\\
                \fa\entails(\nec\neg\alpha\to\nec\nec\neg\alpha)\to(\nec\nec\neg\alpha\to\nec\neg\pos\alpha)\to\nec\neg\alpha\to\nec\neg\pos\alpha&\refer{comp}{L}\\
                \fa\entails(\nec\nec\neg\alpha\to\nec\neg\pos\alpha)\to\nec\neg\alpha\to\nec\neg\pos\alpha&$\hyperref[detachment]{\mathbf{R_1}}\;\set{1,6}$\\
                \fa\entails\nec\neg\alpha\to\nec\neg\pos\alpha&$\hyperref[detachment]{\mathbf{R_1}}\;\set{5,7}$\\
                \fa\entails(\nec\neg\alpha\to\nec\neg\pos\alpha)\to\pos\pos\alpha\to\pos\alpha&\refer{contrapositive}{L}\\
                \fa\entails\pos\pos\alpha\to\pos\alpha&$\hyperref[detachment]{\mathbf{R_1}}\;\set{8,9}$\\
            \end{fitch}
            \normalsize
            Estando assim demonstrada a proposição.
        \end{proof}
    \end{theorem}
