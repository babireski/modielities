\section{Efeitos}

Os \emph{efeitos computacionais}, ou simplesmente \emph{efeitos}, são todas as ações e interações efetuadas pelos computadores que vão além da simples computação.
Assim, uma função que computa a soma entre dois valores não apresenta nenhum efeito, enquanto uma função que computa a soma entre dois valores e imprime o resultado na tela produz um efeito ao fazer a impressão.
Alguns exemplos de efeitos são \emph{parcialidade}, \emph{continuações} e \emph{não-determinismo}.
Programas com efeitos podem mudar seus estados internos, bem como receber entradas externas.
Tais capacidades tornam a avaliação do comportamento do programa menos clara.

\vspace{.5\baselineskip}
Motivado pela busca de uma metalinguagem para modelar efeitos em linguagens de programação,~\cite{Moggi.1991} introduziu a linguagem $\lambda_c$.
Nesta linguagem, os termos-$\lambda$ são distinguidos entre valores de tipo $\alpha$ e computações $\mu\ \alpha$ de tipo $\alpha$ de modo que estas se comportam \emph{monadicamente}.
O construtor de tipos $\mu$ representa alguma \emph{noção de computação}, ou melhor dizendo, algum efeito.
Inspirado por este trabalho, \cite{Wadler.1993} sugere e defende o uso de \emph{mônadas} como estruturas dentro de linguagens funcionais como uma maneira de simular efeitos.

\vspace{.5\baselineskip}
Para os fins deste trabalho, consideraremos mônadas como um construtor de tipos $\mu$ munido das operações {\texttt{\textbf{lift}}}, {\texttt{\textbf{unit}}} e {\texttt{\textbf{bind}}}.\footnote{Estas operações devem respeitar certas condições, que não serão abordadas.}
A operação {\texttt{\textbf{lift}}} eleva funções a funções \emph{efeituosas}.\footnote{\emph{Effectful}.}
Do mesmo modo, a operação {\texttt{\textbf{unit}}} eleva um valor a um valor com efeitos e a operação {\texttt{\textbf{bind}}} encadeia efeitos.
Alternativamente, podemos usar a operação {\texttt{\textbf{join}}}, que une efeitos, em lugar da operação {\texttt{\textbf{bind}}}.
A seguir temos as assinaturas de tipo das operações citadas.

\vspace{.5\baselineskip}
\begin{tcolorbox}[enhanced jigsaw, breakable, sharp corners, colframe=black, colback=white, boxrule=0.5pt, left=1.5mm, right=1.5mm, top=1.5mm, bottom=1.5mm]
    \setmonofont{Fira Code}[Contextuals = Alternate]
    \texttt{\footnotesize\textbf{lift} : ∀α.∀β.∀μ.(α ⟶ β) ⟶ μα ⟶ μβ}\\
    \texttt{\footnotesize\textbf{unit} : ∀α.∀μ.α ⟶ μα}\\
    \texttt{\footnotesize\textbf{bind} : ∀α.∀β.∀μ.(α ⟶ μβ) ⟶ μα ⟶ μβ}\\
        \texttt{\footnotesize\textbf{join} : ∀α.∀μ.μμα ⟶ μα}
\end{tcolorbox}

\vspace{.5\baselineskip}
Conforme notado por~\cite{Benton+others.1998}, a linguagem $\lambda_c$ corresponde ao sistema \emph{laxo} --- doravante chamado de $\mathfrak{L}$ --- por meio da interpretação prova-programa.
Este sistema, que aumenta o sistema intuicionista $\mathfrak{I}$ com uma modalidade de \emph{laxidade}, foi inicialmente considerado por~\cite{Curry.1950, Curry.1952} e posteriormente redescoberto por~\cite{Fairtlough+Mendler.1997}.
Esta modalidade --- denotada $\bigcirc$ --- foi interpretada por estes como \emph{verdade com restrições}, motivo que justifica seu nome.
As regras que governam a modalidade laxa, conforme apresentadas abaixo, geram sentenças correspondentes aos tipos das funções {\texttt{\textbf{lift}}}, {\texttt{\textbf{unit}}} e {\texttt{\textbf{join}}}.
Do mesmo modo, podemos derivar a sentença que corresponde ao tipo da função {\texttt{\textbf{bind}}}.

\vspace{.5\baselineskip}
\begin{tcolorbox}[enhanced jigsaw, breakable, sharp corners, colframe=black, colback=white, boxrule=0.5pt, left=1.5mm, right=1.5mm, top=1.5mm, bottom=1.5mm]
\begin{definition}[$\mathcal{L}_\bigcirc$]\label{lax.language}
    A linguagem do sistema laxo, denotada $\mathcal{L}_\bigcirc$, pode ser induzida a partir da assinatura $\Sigma=\sequence{\mathcal{P},\mathcal{C}_\bigcirc}$, onde $\mathcal{C}_\bigcirc=\set{\bot^0,\bigcirc^1,\wedge^2,\vee^2,\to^2}$.
\end{definition}
\end{tcolorbox}

\begin{tcolorbox}[enhanced jigsaw, breakable, sharp corners, colframe=black, colback=white, boxrule=0.5pt, left=1.5mm, right=1.5mm, top=1.5mm, bottom=1.5mm]
\begin{definition}[$\entails_{\mathfrak{L}}$]
    Abaixo estão definidas as regras do sistema $\mathfrak{L}$.
\vspace{.5\baselineskip}
\begin{center}
    \footnotesize
    \AxiomC{}
    \RightLabel{\footnotesize$\mathbf{A_1}$}
    \UnaryInfC{$\Gamma\entails\alpha\to\beta\to\alpha$}
    \DisplayProof{}
    \quad
    \AxiomC{}
    \RightLabel{\footnotesize$\mathbf{A_2}$}
    \UnaryInfC{$\Gamma\entails(\alpha\to\beta\to\gamma)\to(\alpha\to\beta)\to\alpha\to\gamma$}
    \DisplayProof{}
\end{center}

\begin{center}
    \footnotesize
    \AxiomC{}
    \RightLabel{\footnotesize$\mathbf{A_3}$}
    \UnaryInfC{$\Gamma\entails\alpha\to\beta\to\alpha\wedge\beta$}
    \DisplayProof{}
    \quad
    \AxiomC{}
    \RightLabel{\footnotesize$\mathbf{A_4}$}
    \UnaryInfC{$\Gamma\entails\alpha\wedge\beta\to\alpha$}
    \DisplayProof{}
    \quad
    \AxiomC{}
    \RightLabel{\footnotesize$\mathbf{A_5}$}
    \UnaryInfC{$\Gamma\entails\alpha\wedge\beta\to\beta$}
    \DisplayProof{}
\end{center}

\begin{center}
    \footnotesize
    \AxiomC{}
    \RightLabel{\footnotesize$\mathbf{A_6}$}
    \UnaryInfC{$\Gamma\entails\alpha\to\alpha\vee\beta$}
    \DisplayProof{}
    \quad
    \AxiomC{}
    \RightLabel{\footnotesize$\mathbf{A_7}$}
    \UnaryInfC{$\Gamma\entails\beta\to\alpha\vee\beta$}
    \DisplayProof{}
    \quad
    \AxiomC{}
    \RightLabel{\footnotesize$\mathbf{A_8}$}
    \UnaryInfC{$\Gamma\entails(\alpha\to\gamma)\to(\beta\to\gamma)\to\alpha\vee\beta\to\gamma$}
    \DisplayProof{}
\end{center}

\begin{center}
    \footnotesize
    \AxiomC{}
    \RightLabel{\footnotesize$\mathbf{A_\bot}$}
    \UnaryInfC{$\Gamma\entails\neg\neg\alpha\to\alpha$}
    \DisplayProof{}
\end{center}

\begin{center}
    \footnotesize
    \AxiomC{}
    \RightLabel{\footnotesize$\mathbf{C_1}$}
    \UnaryInfC{$\Gamma\entails(\alpha\to\beta)\to\bigcirc\alpha\to\bigcirc\beta$}
    \DisplayProof{}
    \quad
    \AxiomC{}
    \RightLabel{\footnotesize$\mathbf{C_2}$}
    \UnaryInfC{$\Gamma\entails\alpha\to\bigcirc\alpha$}
    \DisplayProof{}
    \quad
    \AxiomC{}
    \RightLabel{\footnotesize$\mathbf{C_3}$}
    \UnaryInfC{$\Gamma\entails\bigcirc\bigcirc\alpha\to\bigcirc\alpha$}
    \DisplayProof{}
\end{center}

\begin{center}
    \footnotesize
    \AxiomC{\phantom{$\beta$}}
    \RightLabel{\footnotesize$\mathbf{R_1}$}
    \UnaryInfC{$\Gamma\cup\set{\alpha}\entails\alpha$}
    \DisplayProof{}
    \quad
    \AxiomC{$\Gamma\entails\alpha$}
    \AxiomC{$\Gamma\entails\alpha\to\beta$}
    \RightLabel{\footnotesize$\mathbf{R_2}$}
    \BinaryInfC{$\Gamma\entails\beta$}
    \DisplayProof{}
\end{center}
\end{definition}
\end{tcolorbox}

\vspace{.5\baselineskip}
Pode-se imergir sentenças do sistema $\mathfrak{L}$ em sentenças do sistema $\mathfrak{M}$ de modo que $\lax\alpha\coloneqq\pos\nec\alpha$, conforme definem~\cite{Pfenning+Davies.2001}.
Como estes apontam, isso permite com que usemos o sistema $\mathfrak{M}$ para razoar sobre o sistema $\mathfrak{L}$, uma vez que as regras deste podem ser derivadas naquele.
Outra imersão de interesse trata-se da tradução de~\cite{Fairtlough+Mendler.1997}, com sistema de origem o sistema $\mathfrak{L}$ e com sistema de destino um sistema bimodal $\langle\mathfrak{M},\mathfrak{M}\rangle$, que serve para reforçar ainda mais o caso da relação entre ambos os sistemas.

\vspace{.5\baselineskip}
A imersão apresentada acima sugere que possamos interpretar algumas das sentenças do sistema~$\mathfrak{M}$ como representando computações com efeitos.
De fato,~\cite{Pfenning+Davies.2001} alegam que $\nec\alpha$ intuitivamente representa valores de tipo $\alpha$ que sobrevivem a efeitos, enquanto $\pos\alpha$ representa computações que retornam valores de tipo $\alpha$.
Corrobora com esta visão o trabalho de~\cite{Kobayashi.1997}.
Notamos que, segundo~\cite{Zach+others.2019}, podem ser derivadas no sistema $\mathfrak{M}$ sentenças da forma $\alpha\to\pos\alpha$ e da forma $\pos\pos\alpha\to\pos\alpha$.
Do mesmo modo, podem ser derivadas sentenças da forma $\nec(\alpha\to\beta)\to\pos\alpha\to\pos\beta$.
Estas assemelham-se ao{\texttt{\textbf{lift}}}, {\texttt{\textbf{unit}}} e {\texttt{\textbf{join}}}, o que reforça o ponto apresentado aqui.

\vspace{.5\baselineskip}
Em computação, \emph{continuações} são um conceito descoberto e redescoberto diversas vezes~\citep{Reynolds.1993} que representa o restante de uma computação a partir de um determinado ponto.
O \emph{estilo de passagem por continuações} consiste num estilo de programação onde cada função, em vez de retornar um valor, recebe como argumento uma função com a sua continuação e a invoca ao fim de sua execução~\citep{Thielecke.1999}.
O estilo de passagem por continuações pode ser usado como representação dentro de compiladores~\citep{Appel.1991}, uma vez que essa inversão de controle --- ou seja, a passagem de continuações como argumento --- revela o fluxo do programa.
Uma vez que mônadas podem ser imersas em continuações e continuações podem ser imersas em mônadas \citep{Filinski.1994}, o sistema $\mathfrak{M}$ torna-se interessante do ponto de vista de compilação, motivo que justifica investigações acerca do uso de uma linguagem baseada em $\mathfrak{M}$ como representação durante a compilação.
