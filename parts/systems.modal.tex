\section{Modal}
    \babireski{\cite{Blackburn} traz uma visão da evolução dos sistemas modais.}

    \begin{definition}[$\mathcal{L}_\mathbf{M}$]
        A linguagem dos sistemas modais, denotada $\mathcal{L}_\mathbf{M}$, consiste no menor conjunto induzido a partir das seguintes regras:
        \begin{align*}
            &\bot\in\mathcal{L}_\mathbf{M} \\
            &\mathcal{P}\subseteq\mathcal{L}_\mathbf{M} \\
            &\alpha\in\mathcal{L}_\mathbf{M}\Rightarrow\nec\alpha\in\mathcal{L}_\mathbf{M} \\
            &\alpha,\beta\in\mathcal{L}_\mathbf{M}\Rightarrow\alpha\circ\beta\in\mathcal{L}_\mathbf{M}\text{, para }\circ\in\set{\wedge,\vee,\to}\text{.}
            \tag*{\qed}
        \end{align*}
    \end{definition}

    \begin{notation}
        Serão usadas as seguintes abreviações:
        \begin{align*}
            \top&\coloneqq\bot\to\bot\\
            \neg\alpha&\coloneqq\alpha\to\bot\\
            \pos\alpha&\coloneqq\neg\nec\neg\alpha\\
            \alpha\fishhook\beta&\coloneqq\nec(\alpha\to\beta)\\
            \alpha\leftrightarrow\beta&\coloneqq(\alpha\to\beta)\wedge(\beta\to\alpha)
        \end{align*}
    \end{notation}

    \begin{definition}
        A axiomatização do sistema intuicionista consiste no conjunto de esquemas de axiomas $\mathcal{A}=\set{\mathbf{A}_i\mid i\in[1,8]\vee i=\neg}\cup\set{\mathbf{B_1},\mathbf{B_2},\mathbf{B_3}}$ e no conjunto de regras $\mathcal{R}=\set{\mathbf{R_1},\mathbf{R_2}}$, definidos abaixo:
        \begin{alignat*}{3}
            & \mathbf{A_1}\quad && \alpha\to\beta\to\alpha \\
            & \mathbf{A_2}\quad && (\alpha\to\beta\to\gamma)\to(\alpha\to\beta)\to(\alpha\to\gamma) \\
            & \mathbf{A_3}\quad && \alpha\to\beta\to\alpha\wedge\beta \\
            & \mathbf{A_4}\quad && \alpha\wedge\beta\to\alpha \\
            & \mathbf{A_5}\quad && \alpha\wedge\beta\to\beta \\
            & \mathbf{A_6}\quad && \alpha\to\alpha\vee\beta \\
            & \mathbf{A_7}\quad && \beta\to\alpha\vee\beta \\
            & \mathbf{A_8}\quad && (\alpha\to\gamma)\to(\beta\to\gamma)\to(\alpha\vee\beta\to\gamma) \\
            & \mathbf{A_\neg}\quad && \neg\neg\alpha\to\alpha \\
            & \mathbf{B_1}\quad && \nec(\alpha\to\beta)\to\nec\alpha\to\nec\beta \\
            & \mathbf{B_2}\quad && \nec\alpha\to\alpha \\
            & \mathbf{B_3}\quad && \nec\alpha\to\nec\nec\alpha \\
            & \mathbf{R_1}\quad && \text{Se }\Gamma\vdash\alpha\text{ e }\Delta\vdash\alpha\to\beta\text{, então }\Gamma\cup\Delta\vdash\beta\\
            & \mathbf{R_2}\quad && \text{Se }\vdash\alpha\text{, então }\Gamma\vdash\nec\alpha\text{.} & \tag*{\qed} 
        \end{alignat*}   
    \end{definition}

    Chamaremos $\mathbf{R_1}$ de \emph{regra da separação} e $\mathbf{R_2}$ de \emph{regra da necessitação}.
