\section{Modalismo}
    \babireski{\cite{Blackburn} traz uma visão da evolução dos sistemas modais.}

    \begin{definition}[$\mathcal{L}_\mathbf{M}$]
        A linguagem dos sistemas modais, denotada $\mathcal{L}_\mathbf{M}$, consiste no menor conjunto induzido a partir das seguintes regras:
        \begin{align*}
            &\bot\in\mathcal{L}_\mathbf{M} \\
            &\mathcal{P}\subseteq\mathcal{L}_\mathbf{M} \\
            &\alpha\in\mathcal{L}_\mathbf{M}\Rightarrow\nec\alpha\in\mathcal{L}_\mathbf{M} \\
            &\alpha,\beta\in\mathcal{L}_\mathbf{M}\Rightarrow\alpha\circ\beta\in\mathcal{L}_\mathbf{M}\text{, para }\circ\in\set{\wedge,\vee,\to}\text{.}
            \tag*{\qed}
        \end{align*}
    \end{definition}

    \begin{notation}
        Serão usadas as seguintes abreviações:
        \begin{align*}
            \top&\coloneqq\bot\to\bot\\
            \neg\alpha&\coloneqq\alpha\to\bot\\
            \pos\alpha&\coloneqq\neg\nec\neg\alpha\\
            \alpha\fishhook\beta&\coloneqq\nec(\alpha\to\beta)\\
            \alpha\leftrightarrow\beta&\coloneqq(\alpha\to\beta)\wedge(\beta\to\alpha)
        \end{align*}
    \end{notation}

    \begin{notation}
        Seja $\Gamma\in\wp(\mathcal{L}_\mathbf{M})$ um conjunto de sentenças bem-formadas.
        $\nec\Gamma$ denota o conjunto $\set{\nec\alpha\mid\alpha\in\Gamma}\in\wp(\mathcal{L}_\mathbf{M})$, ou seja, a prefixação da necessitação a todos os elementos do conjunto $\Gamma$.
    \end{notation}

    \begin{definition}\label{m-axioms}
        A axiomatização do sistema modal consiste no conjunto de esquemas de axiomas $\mathcal{A}=\set{\mathbf{A}_i\mid i\in[1,8]\vee i=\neg}\cup\set{\mathbf{B_1},\mathbf{B_2},\mathbf{B_3}}$ e no conjunto de regras $\mathcal{R}=\set{\mathbf{R_1},\mathbf{R_2}}$, definidos abaixo:
        \begin{alignat}{3}
            &\mathbf{A_1}\quad&&\alpha\to\beta\to\alpha\label{MA1}\tag*{}\\
            &\mathbf{A_2}\quad&&(\alpha\to\beta\to\gamma)\to(\alpha\to\beta)\to(\alpha\to\gamma)\label{MA2}\tag*{}\\
            &\mathbf{A_3}\quad&&\alpha\to\beta\to\alpha\wedge\beta\label{MA3}\tag*{}\\
            &\mathbf{A_4}\quad&&\alpha\wedge\beta\to\alpha\label{MA4}\tag*{}\\
            &\mathbf{A_5}\quad&&\alpha\wedge\beta\to\beta\label{MA5}\tag*{}\\
            &\mathbf{A_6}\quad&&\alpha\to\alpha\vee\beta\label{MA6}\tag*{}\\
            &\mathbf{A_7}\quad&&\beta\to\alpha\vee\beta\label{MA7}\tag*{}\\
            &\mathbf{A_8}\quad&&(\alpha\to\gamma)\to(\beta\to\gamma)\to(\alpha\vee\beta\to\gamma)\label{MA8}\tag*{}\\
            &\mathbf{A_\neg}\quad&&\neg\neg\alpha\to\alpha\label{MANEG}\tag*{}\\
            &\mathbf{B_1}\quad&&\nec(\alpha\to\beta)\to\nec\alpha\to\nec\beta\label{MB1}\tag*{}\\
            &\mathbf{B_2}\quad&&\nec\alpha\to\alpha\label{MB2}\tag*{}\\
            &\mathbf{B_3}\quad&&\nec\alpha\to\nec\nec\alpha\label{MB3}\tag*{}\\
            &\mathbf{R_1}\quad&&\text{Se }\Gamma\entails\alpha\text{ e }\Gamma\entails\alpha\to\beta\text{, então }\Gamma\entails\beta\label{detachment}\tag*{}\\
            &\mathbf{R_2}\quad&&\text{Se }\entails\alpha\text{, então }\Gamma\entails\nec\alpha\text{.}\tag*{\qed}\label{necessitation} 
        \end{alignat}   
    \end{definition}

    \babireski{Falar aqui sobre como a axiomatização consiste nos esquemas clássicos mais os esquemas modais}. 

    Assim como feito para o sistema intuicionista, nomearemos os esquemas e regras acima de modo a facilitar a comunicação.
    Aos axiomas e regras que correspondem aos axiomas e regras intuicionistas receberão os mesmos nomes. Ademais, chamaremos $\mathbf{B_1}$ de axiomas da normalidade, $\mathbf{B_2}$ de axiomas da reflexividade e $\mathbf{B_3}$ de axiomas da transitividade.\footnote{Em analogia às condições relacionais impostas nos quadros.} Nomearemos $\mathbf{A_\neg}$ como chamaremos de axiomas da eliminação da negação e a $\mathbf{R_2}$ como regra da necessitação.

    A definição das regras de dedução em relação a conjuntos de sentenças baseia-se tanto em~\cite{Troelstra} como em~\cite{Hakli}. Ao decorrer do texto, ocasionalmente chamaremos $\mathbf{R_1}$ de \emph{regra da separação} e $\mathbf{R_2}$ de \emph{regra da necessitação}.
