\section{Sistema intuicionista}
    Nesta seção, definiremos os sistema intuicionista $\mathfrak{B}=\langle\mathcal{L},\vdash_\mathfrak{B}\rangle$, cuja linguagem consiste no conjunto de origem das traduções foco deste trabalho.
    Este sistema surge da rejeição da lei do \textit{tertium non datur}, ou seja $\Gamma\vdash\alpha\vee\neg\alpha$ não vale para todos os casos.

    O sistema intuicionista consiste no sistema resultante da rejeição de algumas sentenças classicamente tidas como verdadeiras, como a sentença $\alpha\vee\neg\alpha$ e a sentença $\neg\neg\alpha\to\alpha$.
    Esse sistema foi inicialmente formalizado por~\cite{Kolmogorov}, Heyting e~\cite{Glivenko-A, Glivenko-B} com inspiração nos trabalhos de~\cite{Brouwer-A, Brouwer-B} acerca do intuicionismo.
    Nesta seção, definiremos este sistema conforme~\cite{Troelstra} e traremos um breve contexto de seu uso na computação.

    \begin{tcolorbox}[enhanced jigsaw, breakable, sharp corners, colframe=black, colback=white, boxrule=0.5pt, left=1.5mm, right=1.5mm, top=1.5mm, bottom=1.5mm]
    \begin{definition}[$\mathcal{L}$]\label{intuitionistic.language}
        A linguagem do sistema intuicionista, denotada $\mathcal{L}$, pode ser induzida a partir da assinatura $\Sigma=\sequence{\mathcal{P},\mathcal{C}}$, onde $\mathcal{C}=\set{\bot^0,\wedge^2,\vee^2,\to^2}$.
        \qed{}
    \end{definition}
    \end{tcolorbox}

    \begin{tcolorbox}[enhanced jigsaw, breakable, sharp corners, colframe=black, colback=white, boxrule=0.5pt, left=1.5mm, right=1.5mm, top=1.5mm, bottom=1.5mm]
    \begin{notation}
        Seja uma sentença $\alpha\in\mathcal{L}$, $\neg\alpha$ denota a sua negação $\alpha\to\bot$.
    \end{notation}
    \end{tcolorbox}

\vspace{.5\baselineskip}
\begin{tcolorbox}[enhanced jigsaw, breakable, sharp corners, colframe=black, colback=white, boxrule=0.5pt, left=1.5mm, right=1.5mm, top=1.5mm, bottom=1.5mm]
\begin{definition}[$\vdash_{\mathfrak{B}}$]\label{intuitionistic.deduction}
    Abaixo estão definidas as regras do sistema intuicionista $\mathfrak{B}$.
\vspace{.5\baselineskip}
\begin{center}
    \footnotesize
    \AxiomC{}
    \RightLabel{\footnotesize$\mathbf{A_1}$}
    \UnaryInfC{$\Gamma\vdash\alpha\to\beta\to\alpha$}
    \DisplayProof\label{intuitionistic.axiom.1}
    \quad
    \AxiomC{}
    \RightLabel{\footnotesize$\mathbf{A_2}$}
    \UnaryInfC{$\Gamma\vdash(\alpha\to\beta\to\gamma)\to(\alpha\to\beta)\to\alpha\to\gamma$}
    \DisplayProof\label{intuitionistic.axiom.2}
\end{center}

\begin{center}
    \footnotesize
    \AxiomC{}
    \RightLabel{\footnotesize$\mathbf{A_3}$}
    \UnaryInfC{$\Gamma\vdash\alpha\to\beta\to\alpha\wedge\beta$}
    \DisplayProof\label{intuitionistic.axiom.3}
    \quad
    \AxiomC{}
    \RightLabel{\footnotesize$\mathbf{A_4}$}
    \UnaryInfC{$\Gamma\vdash\alpha\wedge\beta\to\alpha$}
    \DisplayProof\label{intuitionistic.axiom.4}
    \quad
    \AxiomC{}
    \RightLabel{\footnotesize$\mathbf{A_5}$}
    \UnaryInfC{$\Gamma\vdash\alpha\wedge\beta\to\beta$}
    \DisplayProof\label{intuitionistic.axiom.5}
\end{center}

\begin{center}
    \footnotesize
    \AxiomC{}
    \RightLabel{\footnotesize$\mathbf{A_6}$}
    \UnaryInfC{$\Gamma\vdash\alpha\to\alpha\vee\beta$}
    \DisplayProof\label{intuitionistic.axiom.6}
    \quad
    \AxiomC{}
    \RightLabel{\footnotesize$\mathbf{A_7}$}
    \UnaryInfC{$\Gamma\vdash\beta\to\alpha\vee\beta$}
    \DisplayProof\label{intuitionistic.axiom.7}
    \quad
    \AxiomC{}
    \RightLabel{\footnotesize$\mathbf{A_8}$}
    \UnaryInfC{$\Gamma\vdash(\alpha\to\gamma)\to(\beta\to\gamma)\to\alpha\vee\beta\to\gamma$}
    \DisplayProof\label{intuitionistic.axiom.8}
\end{center}

\begin{center}
    \footnotesize
    \AxiomC{}
    \RightLabel{\footnotesize$\mathbf{A_\bot}$}
    \UnaryInfC{$\Gamma\vdash\bot\to\alpha$}
    \DisplayProof\label{intuitionistic.axiom.contradiction}
\end{center}

\begin{center}
    \footnotesize
    \AxiomC{$\alpha\in\Gamma$}
    \RightLabel{\footnotesize$\mathbf{R_1}$}
    \UnaryInfC{$\Gamma\vdash\alpha$}
    \DisplayProof\label{intuitionistic.rule.1}
    \quad
    \AxiomC{$\Gamma\vdash\alpha$}
    \AxiomC{$\Gamma\vdash\alpha\to\beta$}
    \RightLabel{\footnotesize$\mathbf{R_2}$}
    \BinaryInfC{$\Gamma\vdash\beta$}\label{intuitionistic.rule.1}
    \DisplayProof
\end{center}
\end{definition}
\end{tcolorbox}

    De modo a facilitar a comunicação no decorrer deste trabalho, chamaremos $\mathbf{R_1}$ de regra da separação ou \emph{modus ponens}.

    O sistema intuicionista calca-se numa visão construtivista que fundamenta aplicações importantes na computação. O isomorfismo de Curry-Howard estabelece uma associação entre provas e programas e entre proposições e tipos.
    Enquanto isso, a interpretação de Brouwer-Heyting-Kolmogorov --- definida abaixo segundo~\cite{Troelstra} --- exige que provas sejam construtivas, garantindo a realidade efetiva dos elementos provados. Tais propriedades são largamente usadas, por exemplo, na prova de teoremas computacionalmente e na construção de compiladores robustos.

    \begin{enumerate}[label=\textbf{(\alph*)}, left=\parindent]
        \item Não existe prova de $\bot$.
        \item Uma prova de $\alpha\wedge\beta$ consiste num par $\sequence{A,B}$, sendo $A$ uma prova de $\alpha$ e $B$ uma prova de $\beta$.
        \item Uma prova de $\alpha\vee\beta$ consiste ou num par $\sequence{0,A}$, sendo $A$ uma prova de $\alpha$, ou num par $\sequence{1,B}$, sendo $B$ uma prova de $\beta$.
        \item Uma prova de $\alpha\to\beta$ consiste numa construção $C$ que transforma uma prova $A$ de $\alpha$ numa prova $B$ de $\beta$.
    \end{enumerate}
