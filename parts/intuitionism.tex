\section{Intuicionismo}
    \begin{definition}[$\mathcal{L}_\mathbf{I}$]
        A linguagem do sistema intuicionista, denotada $\mathcal{L}_\mathbf{I}$, pode ser induzida a partir da assinatura $\Sigma_\mathbf{I}=\sequence{\mathcal{P},\mathcal{C}_\mathbf{I}}$, onde $\mathcal{C}_\mathbf{I}=\set{\bot^0,\wedge^2,\vee^2,\to^2}$.
        \qed{}
    \end{definition}

    \begin{notation}
        Serão usadas as seguintes abreviações:
        \begin{align*}
            \top&\coloneqq\bot\to\bot\\
            \neg\alpha&\coloneqq\alpha\to\bot\\
            \alpha\leftrightarrow\beta&\coloneqq(\alpha\to\beta)\wedge(\beta\to\alpha)
        \end{align*}
    \end{notation}

    \begin{enumerate}[label=\textbf{(\alph*)}, left=\parindent]
        \item Não existe prova de $\bot$.
        \item Uma prova de $\alpha\wedge\beta$ consiste num par $\sequence{A,B}$, sendo $A$ uma prova de $\alpha$ e $B$ uma prova de $\beta$.
        \item Uma prova de $\alpha\vee\beta$ consiste ou num par $\sequence{0,A}$, sendo $A$ uma prova de $\alpha$, ou num par $\sequence{1,B}$, sendo $B$ uma prova de $\beta$.
        \item Uma prova de $\alpha\to\beta$ consiste numa construção $C$ que transforma uma prova $A$ de $\alpha$ numa prova $B$ de $\beta$.
    \end{enumerate}

    Definição abaixo segundo~\cite{Troelstra}.

    \begin{definition}
        A axiomatização do sistema intuicionista consiste no conjunto de esquemas de axiomas $\mathcal{A}=\set{\mathbf{A}_i\mid i\in[1,8]\vee i=\bot}$ e no conjunto de regras $\mathcal{R}=\set{\mathbf{R_1}}$, definidos abaixo:
        \begin{alignat*}{3}
            & \mathbf{A_1}\quad && \alpha\to\beta\to\alpha \\
            & \mathbf{A_2}\quad && (\alpha\to\beta\to\gamma)\to(\alpha\to\beta)\to(\alpha\to\gamma) \\
            & \mathbf{A_3}\quad && \alpha\to\beta\to\alpha\wedge\beta \\
            & \mathbf{A_4}\quad && \alpha\wedge\beta\to\alpha \\
            & \mathbf{A_5}\quad && \alpha\wedge\beta\to\beta \\
            & \mathbf{A_6}\quad && \alpha\to\alpha\vee\beta \\
            & \mathbf{A_7}\quad && \beta\to\alpha\vee\beta \\
            & \mathbf{A_8}\quad && (\alpha\to\gamma)\to(\beta\to\gamma)\to(\alpha\vee\beta\to\gamma) \\
            & \mathbf{A_\bot}\quad && \bot\to\alpha \\
            & \mathbf{R_E}\quad && \text{Se }\Gamma\entails\alpha\text{, então }\Gamma\cup\Delta\entails\alpha\\
            & \mathbf{R_1}\quad && \text{Se }\Gamma\vdash\alpha\text{ e }\Gamma\vdash\alpha\to\beta\text{, então }\Gamma\vdash\beta\text{.} & \tag*{\qed}
        \end{alignat*}   
    \end{definition}

    Daremos nomes aos esquemas e regras acima de modo a facilitar a comunicação no decorrer deste trabalho. Chamaremos o esquema $\mathbf{A_1}$ de esquema da constante e o esquema $\mathbf{A_1}$ de esquema da aplicação.\footnote{Em analogia aos combinadores $\mathbf{K}$ e $\mathbf{S}$.} Ao esquema $\mathbf{A_3}$ daremos o nome de introdução da conjunção, enquanto os esquemas $\mathbf{A_4}$ e $\mathbf{A_5}$ serão chamados de eliminação da conjunção. Analogamente, os esquemas $\mathbf{A_6}$ e $\mathbf{A_7}$ serão chamados de introdução da disjunção, enquanto ao esquema $\mathbf{A_8}$ chamaremos de eliminação da disjunção.
    Por fim, chamaremos $\mathbf{A_\bot}$ de esquema da explosão e $\mathbf{R_1}$ de regra da separação ou \emph{modus ponens}.
