\section{Sistema intuicionista}
    Nesta seção, definiremos os sistema intuicionista $\mathfrak{B}=\langle\mathcal{L},\vdash_\mathfrak{B}\rangle$, cuja linguagem consiste no conjunto de origem das traduções apresentadas deste trabalho.
    Este sistema surge da rejeição da lei do \textit{tertium non datur}, ou seja $\Gamma\vdash\alpha\vee\neg\alpha$ não vale para todos os casos.
    O intuicionismo foi primeiramente considerado e defendido por~\cite{Brouwer-A, Brouwer-B}.
    Este defendia que uma asserção poderia ser dita verdade apenas quando esta poderia ser~\emph{construída}.
    Assim, uma demonstração por contradição não valeria, uma vez que durante uma demonstração deste tipo, estar-se-ia contruindo a dupla negação da proposição em apreciação, mas não a proposição em si.
    Brouwer repudiou a tentativa formalização de seu pensamento como um sistema, todavia isso não impediu que esta fosse feita.
    A primeira formalização encontrada na literatura foi feita por~\cite{Kolmogorov}, entretanto outros também a fizeram posteriormente.
    A linguagem abaixo foi definida conforme~\cite{Troelstra}. Adotamos as ordens usuais de avaliação dos operadores.

    \vspace{\baselineskip}
    \begin{tcolorbox}[enhanced jigsaw, breakable, sharp corners, colframe=black, colback=white, boxrule=0.5pt, left=1.5mm, right=1.5mm, top=1.5mm, bottom=1.5mm]
    \begin{definition}[$\mathcal{L}$]\label{intuitionistic.language}
        A linguagem do sistema intuicionista, denotada $\mathcal{L}$, pode ser induzida a partir da assinatura $\Sigma=\sequence{\mathcal{P},\mathcal{C}}$, onde $\mathcal{C}=\set{\bot^0,\wedge^2,\vee^2,\to^2}$.
        \qed{}
    \end{definition}
    \end{tcolorbox}

    \begin{tcolorbox}[enhanced jigsaw, breakable, sharp corners, colframe=black, colback=white, boxrule=0.5pt, left=1.5mm, right=1.5mm, top=1.5mm, bottom=1.5mm]
    \begin{notation}
        Seja uma sentença $\alpha\in\mathcal{L}$, $\neg\alpha$ denota a sua negação $\alpha\to\bot$.
    \end{notation}
    \end{tcolorbox}

\vspace{.5\baselineskip}
    Consideremos agora a relação de dedução para o sistema $\mathfrak{B}$, definida por suas regras de dedução.
    Estas destacam-se pela omissão de uma regra que gera asserções do tipo $\Gamma\vdash\neg\neg\alpha$.
    Isso acontece porque não somente esta regra permite provas não-contrutivas como permite a derivação de uma regra derivada $\Gamma\vdash\alpha\vee\neg\alpha$.
    Os axiomas abaixo estão organizadas de acordo com os seus operadores principais: na primeira linha os axiomas da implicação, na segunda linha as axiomas da conjunção, na quarta linha os axiomas da disjunção e na quinta linha o axioma da contradição.
    Aaixo dos axiomas, temos as demais regras, nomeadamente a \emph{regra da assunção} e a \emph{regra da separação}.

\vspace{\baselineskip}
\begin{tcolorbox}[enhanced jigsaw, breakable, sharp corners, colframe=black, colback=white, boxrule=0.5pt, left=1.5mm, right=1.5mm, top=1.5mm, bottom=1.5mm]
\begin{definition}[$\vdash_{\mathfrak{B}}$]\label{intuitionistic.deduction}
    Abaixo estão definidas as regras do sistema intuicionista $\mathfrak{B}$.
\vspace{.5\baselineskip}
\begin{center}
    \footnotesize
    \AxiomC{}
    \RightLabel{\footnotesize$\mathbf{A_1}$}
    \UnaryInfC{$\Gamma\vdash\alpha\to\beta\to\alpha$}
    \DisplayProof\label{intuitionistic.axiom.1}
    \quad
    \AxiomC{}
    \RightLabel{\footnotesize$\mathbf{A_2}$}
    \UnaryInfC{$\Gamma\vdash(\alpha\to\beta\to\gamma)\to(\alpha\to\beta)\to\alpha\to\gamma$}
    \DisplayProof\label{intuitionistic.axiom.2}
\end{center}

\begin{center}
    \footnotesize
    \AxiomC{}
    \RightLabel{\footnotesize$\mathbf{A_3}$}
    \UnaryInfC{$\Gamma\vdash\alpha\to\beta\to\alpha\wedge\beta$}
    \DisplayProof\label{intuitionistic.axiom.3}
    \quad
    \AxiomC{}
    \RightLabel{\footnotesize$\mathbf{A_4}$}
    \UnaryInfC{$\Gamma\vdash\alpha\wedge\beta\to\alpha$}
    \DisplayProof\label{intuitionistic.axiom.4}
    \quad
    \AxiomC{}
    \RightLabel{\footnotesize$\mathbf{A_5}$}
    \UnaryInfC{$\Gamma\vdash\alpha\wedge\beta\to\beta$}
    \DisplayProof\label{intuitionistic.axiom.5}
\end{center}

\begin{center}
    \footnotesize
    \AxiomC{}
    \RightLabel{\footnotesize$\mathbf{A_6}$}
    \UnaryInfC{$\Gamma\vdash\alpha\to\alpha\vee\beta$}
    \DisplayProof\label{intuitionistic.axiom.6}
    \quad
    \AxiomC{}
    \RightLabel{\footnotesize$\mathbf{A_7}$}
    \UnaryInfC{$\Gamma\vdash\beta\to\alpha\vee\beta$}
    \DisplayProof\label{intuitionistic.axiom.7}
    \quad
    \AxiomC{}
    \RightLabel{\footnotesize$\mathbf{A_8}$}
    \UnaryInfC{$\Gamma\vdash(\alpha\to\gamma)\to(\beta\to\gamma)\to\alpha\vee\beta\to\gamma$}
    \DisplayProof\label{intuitionistic.axiom.8}
\end{center}

\begin{center}
    \footnotesize
    \AxiomC{}
    \RightLabel{\footnotesize$\mathbf{A_\bot}$}
    \UnaryInfC{$\Gamma\vdash\bot\to\alpha$}
    \DisplayProof\label{intuitionistic.axiom.contradiction}
\end{center}

\begin{center}
    \footnotesize
    \AxiomC{$\alpha\in\Gamma$}
    \RightLabel{\footnotesize$\mathbf{R_1}$}
    \UnaryInfC{$\Gamma\vdash\alpha$}
    \DisplayProof\label{intuitionistic.rule.1}
    \quad
    \AxiomC{$\Gamma\vdash\alpha$}
    \AxiomC{$\Gamma\vdash\alpha\to\beta$}
    \RightLabel{\footnotesize$\mathbf{R_2}$}
    \BinaryInfC{$\Gamma\vdash\beta$}\label{intuitionistic.rule.1}
    \DisplayProof
\end{center}
\end{definition}
\end{tcolorbox}

\vspace{.5\baselineskip}
A noção de \emph{construção} representada pelo sistema intuicionista $\mathfrak{B}$ assemelha-se muito com a noção de \emph{computação}.
Mais que isso: elas são noções \emph{isomorfas}.
Com isso queremos dizer que, dada uma linguagem de programação tipada, a asserção de que um programa dessa linguagem \emph{pertence tipo $\alpha$} implica que este programa \emph{prova a proposição $\alpha$} e vice-versa.
Em outras palavras, o problema da decisão da verdade desta proposição pode ser vista como da decisão habitação de um tipo que corresponde a ela.
Assim, podemos dizer que \emph{proposições são tipos} e \emph{provas são programas} (\textbf{CITAR} Wadler).
Os primeiros vislumbres dessa relação foram feitos por Curry e Feys (\textbf{CITAR}), no entando ela começou a ser de fato desvendada a partir do de Howard \textbf{CITAR}.
Como ilustração, abaixo as regras do fragmento implicativo do sistema intuicionista em dedução natural são postas lado a lado com as regras de tipagem do cálculo lambda simplesmente tipado.

\begin{table}[H]
    \centering
    \begin{tabular}{c c}
        \toprule
        \textbf{Regras de dedução} & \textbf{Regras de tipagem}\\
        \midrule
        \small$\Gamma\cup\{\alpha\}\vdash\alpha$ & \small$\Gamma\cup\{x:\alpha\}\vdash x:\alpha$ \\
        \small$\Gamma\cup\{\alpha\}\vdash\beta\implies\Gamma\vdash\alpha\to\beta$ & \small$\Gamma\cup\{x:\alpha\}\vdash e:\beta\implies\Gamma\vdash\lambda x.e:\alpha\to\beta$ \\
        \small$\Gamma\vdash\alpha\to\beta\implies\Gamma\vdash\alpha\implies\Gamma\vdash\beta$ & \small$\Gamma\vdash e_1:\alpha\to\beta\implies\Gamma\vdash e_2:\alpha\implies\Gamma\vdash e_1\text{ }e_2:\beta$ \\
        \bottomrule
    \end{tabular}
\end{table}
