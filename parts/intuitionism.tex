\section{Sistema intuicionista}

Nesta seção, definiremos o sistema intuicionista $\mathfrak{I}=\langle\mathcal{L},\vdash_\mathfrak{I}\rangle$, cuja linguagem consiste no conjunto de origem das traduções apresentadas deste trabalho.
Este sistema surge da rejeição da lei do \textit{tertium non datur}, ou seja $\alpha\mathrel{\vee}\neg\alpha$ não vale para todos os casos.
O intuicionismo foi primeiramente considerado e defendido por~\cite{Brouwer.1908}.
Este defendia que uma asserção poderia ser dita verdade apenas quando esta poderia ser~\emph{construída}.
Assim, uma demonstração por contradição não valeria, uma vez que durante uma demonstração deste tipo, estar-se-ia construindo a dupla negação da proposição em apreciação, mas não a proposição em si.

\vspace{0.5\baselineskip}
Brouwer opunha-se a tentativas de formalização de seu pensamento como um sistema, todavia isso não impediu que esta fosse feita.
A primeira formalização encontrada na literatura foi feita por~\cite{Kolmogorov.1967}, entretanto outros também a fizeram posteriormente.
A linguagem abaixo foi definida conforme~\cite{Troelstra+Schwichtenberg.2000}. Adotamos as ordens usuais de avaliação dos operadores.

\vspace{0.5\baselineskip}
\begin{tcolorbox}[enhanced jigsaw, breakable, sharp corners, colframe=black, colback=white, boxrule=0.5pt, left=1.5mm, right=1.5mm, top=1.5mm, bottom=1.5mm]
    \begin{definition}[$\mathcal{L}$]\label{intuitionistic.language}
        A linguagem do sistema intuicionista, denotada $\mathcal{L}$, pode ser induzida a partir da assinatura $\Sigma=\sequence{\mathcal{P},\mathcal{C}}$, onde $\mathcal{C}=\set{\bot^0,\wedge^2,\vee^2,\to^2}$.
    \end{definition}
\end{tcolorbox}

\begin{tcolorbox}[enhanced jigsaw, breakable, sharp corners, colframe=black, colback=white, boxrule=0.5pt, left=1.5mm, right=1.5mm, top=1.5mm, bottom=1.5mm]
    \begin{notation}
        Seja uma sentença $\alpha\in\mathcal{L}$, $\neg\alpha$ denota a sua negação $\alpha\to\bot$.
    \end{notation}
\end{tcolorbox}

\vspace{0.5\baselineskip}
Consideremos agora a relação de dedução para o sistema $\mathfrak{I}$, definida por suas regras de dedução.
Estas destacam-se pela omissão de uma regra que gera sentenças do tipo $\neg\neg\alpha\to\alpha$.
Isso acontece porque não somente esta regra permite provas não-construtivas como permite a derivação de sentenças do tipo $\alpha\vee\neg\alpha$.
Os axiomas abaixo estão organizados de acordo com os seus operadores principais: na primeira linha os axiomas da implicação, na segunda linha as axiomas da conjunção, na terceira linha os axiomas da disjunção e na quarta linha o axioma da contradição.
Abaixo dos axiomas, temos as demais regras, nomeadamente a \emph{regra da assunção} e a \emph{regra da separação}.

\vspace{0.5\baselineskip}
\begin{tcolorbox}[enhanced jigsaw, breakable, sharp corners, colframe=black, colback=white, boxrule=0.5pt, left=1.5mm, right=1.5mm, top=1.5mm, bottom=1.5mm]
\begin{definition}[$\vdash_{\mathfrak{I}}$]\label{intuitionistic.deduction}
    Abaixo estão definidas as regras do sistema $\mathfrak{I}$.
\vspace{.5\baselineskip}
\begin{center}
    \footnotesize
    \AxiomC{}
    \RightLabel{\footnotesize$\mathbf{A_1}$}
    \UnaryInfC{$\Gamma\vdash\alpha\to\beta\to\alpha$}
    \DisplayProof\label{intuitionistic.axiom.1}
    \quad
    \AxiomC{}
    \RightLabel{\footnotesize$\mathbf{A_2}$}
    \UnaryInfC{$\Gamma\vdash(\alpha\to\beta\to\gamma)\to(\alpha\to\beta)\to\alpha\to\gamma$}
    \DisplayProof\label{intuitionistic.axiom.2}
\end{center}

\begin{center}
    \footnotesize
    \AxiomC{}
    \RightLabel{\footnotesize$\mathbf{A_3}$}
    \UnaryInfC{$\Gamma\vdash\alpha\to\beta\to\alpha\wedge\beta$}
    \DisplayProof\label{intuitionistic.axiom.3}
    \quad
    \AxiomC{}
    \RightLabel{\footnotesize$\mathbf{A_4}$}
    \UnaryInfC{$\Gamma\vdash\alpha\wedge\beta\to\alpha$}
    \DisplayProof\label{intuitionistic.axiom.4}
    \quad
    \AxiomC{}
    \RightLabel{\footnotesize$\mathbf{A_5}$}
    \UnaryInfC{$\Gamma\vdash\alpha\wedge\beta\to\beta$}
    \DisplayProof\label{intuitionistic.axiom.5}
\end{center}

\begin{center}
    \footnotesize
    \AxiomC{}
    \RightLabel{\footnotesize$\mathbf{A_6}$}
    \UnaryInfC{$\Gamma\vdash\alpha\to\alpha\vee\beta$}
    \DisplayProof\label{intuitionistic.axiom.6}
    \quad
    \AxiomC{}
    \RightLabel{\footnotesize$\mathbf{A_7}$}
    \UnaryInfC{$\Gamma\vdash\beta\to\alpha\vee\beta$}
    \DisplayProof\label{intuitionistic.axiom.7}
    \quad
    \AxiomC{}
    \RightLabel{\footnotesize$\mathbf{A_8}$}
    \UnaryInfC{$\Gamma\vdash(\alpha\to\gamma)\to(\beta\to\gamma)\to\alpha\vee\beta\to\gamma$}
    \DisplayProof\label{intuitionistic.axiom.8}
\end{center}

\begin{center}
    \footnotesize
    \AxiomC{}
    \RightLabel{\footnotesize$\mathbf{A_\bot}$}
    \UnaryInfC{$\Gamma\vdash\bot\to\alpha$}
    \DisplayProof\label{intuitionistic.axiom.contradiction}
\end{center}

\begin{center}
    \footnotesize
    \AxiomC{\phantom{$\beta$}}
    \RightLabel{\footnotesize$\mathbf{R_1}$}
    \UnaryInfC{$\Gamma\cup\set{\alpha}\vdash\alpha$}
    \DisplayProof\label{intuitionistic.rule.1}
    \quad
    \AxiomC{$\Gamma\vdash\alpha$}
    \AxiomC{$\Gamma\vdash\alpha\to\beta$}
    \RightLabel{\footnotesize$\mathbf{R_2}$}
    \BinaryInfC{$\Gamma\vdash\beta$}\label{intuitionistic.rule.2}
    \DisplayProof
\end{center}
\end{definition}
\end{tcolorbox}

\vspace{0.5\baselineskip}
A noção de \emph{construção} representada pelo sistema intuicionista assemelha-se muito com a noção de \emph{computação}.
Mais que isso: elas são noções \emph{isomorfas}.
Com isso queremos dizer que, dado um termo-$\lambda$, a asserção de que este termo pertence ao tipo $\alpha$ implica que este termo demonstra a proposição $\alpha$.
Conversamente, a demonstração de uma proposição instuicionista $\alpha$ implica que deve haver algum termo-$\lambda$ com tipo $\alpha$.
Como aponta \cite{Wadler.2015}, podemos dizer que \emph{proposições são tipos}, \emph{derivações são programas} e \emph{normalizações de derivações são avaliações de programas}.

\counterwithout{table}{chapter}

\begin{table}[H]
    \newlength{\skipvalue}
    \setlength{\skipvalue}{\baselineskip}
    \begin{xltabular}{\linewidth}{X c X c X}
        \toprule
        &\textbf{Regras de dedução} & & \textbf{Regras de tipagem}&\\
        \midrule
            &
                \small
                \AxiomC{\phantom{$\Gamma\cup\{\alpha\}\vdash\alpha$}}
                \UnaryInfC{$\Gamma\cup\{\alpha\}\vdash\alpha$}
                \DisplayProof{}
            &
            &
                \small
                \AxiomC{\phantom{$\Gamma\cup\{\alpha\}\vdash\alpha$}}
                \UnaryInfC{$\Gamma\cup\{x:\alpha\}\vdash x:\alpha$}
                \DisplayProof{}
            &
            \\[\skipvalue]
            &
                \small
                \AxiomC{$\Gamma\cup\{\alpha\}\vdash\beta$}
                \UnaryInfC{$\Gamma\vdash\alpha\to\beta$}
                \DisplayProof{}
            &
            &
                \small
                \AxiomC{$\Gamma\cup\{x:\alpha\}\vdash e:\beta$}
                \UnaryInfC{$\Gamma\vdash\lambda x.e:\alpha\to\beta$}
                \DisplayProof{}
            &
            \\[\skipvalue]
            &
                \small
                \AxiomC{$\Gamma\vdash\alpha\to\beta$}
                \AxiomC{$\Gamma\vdash\alpha$}
                \BinaryInfC{$\Gamma\vdash\beta$}
                \DisplayProof{}
            &
            &
                \small
                \AxiomC{$\Gamma\vdash e_1:\alpha\to\beta$}
                \AxiomC{$\Gamma\vdash e_2:\alpha$}
                \BinaryInfC{$\Gamma\vdash e_1\ e_2:\beta$}
                \DisplayProof{}
            &
            \\[.5\skipvalue]
        \bottomrule
    \end{xltabular}
    \caption{Comparação entre regras de dedução intuicionistas and regras de tipagem de termos-$\lambda$. Elaborado pelo autor.}\label{comparison}
\end{table}

Os primeiros vislumbres dessa relação foram feitos por \cite{Curry+Feys.1958}, no entando ela começou a ser de fato desvendada a partir do trabalho de \cite{Howard.1980}.
Como ilustração, regras de dedução do sistema intuicionista e regras de tipagem de termos-$\lambda$ são postas lado a lado na Tabela \ref{comparison}.
