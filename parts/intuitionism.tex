\section{Sistema intuicionista}

    O sistema intuicionista consiste no sistema resultante da rejeição de algumas sentenças classicamente tidas como verdadeiras, como a sentença $\alpha\vee\neg\alpha$ e a sentença $\neg\neg\alpha\to\alpha$.
    Esse sistema foi inicialmente formalizado por~\cite{Kolmogorov}, Heyting e~\cite{Glivenko-A, Glivenko-B} com inspiração nos trabalhos de~\cite{Brouwer-A, Brouwer-B} acerca do intuicionismo.
    Nesta seção, definiremos este sistema conforme~\cite{Troelstra} e traremos um breve contexto de seu uso na computação.

    \begin{definition}[$\mathcal{L}$]
        A linguagem do sistema intuicionista, denotada $\mathcal{L}$, pode ser induzida a partir da assinatura $\Sigma=\sequence{\mathcal{P},\mathcal{C}}$, onde $\mathcal{C}=\set{\bot^0,\wedge^2,\vee^2,\to^2}$.
        \qed{}
    \end{definition}

    \begin{notation}
        Serão usadas as seguintes abreviações:
        \begin{align*}
            \top&\coloneqq\bot\to\bot\\
            \neg\alpha&\coloneqq\alpha\to\bot\\
            \alpha\leftrightarrow\beta&\coloneqq(\alpha\to\beta)\wedge(\beta\to\alpha)
        \end{align*}
    \end{notation}

    \begin{definition}[$\entails_\mathbf{I}$]
        A axiomatização do sistema intuicionista consiste no conjunto de esquemas de axiomas $\mathcal{A}=\set{\mathbf{A}_i\mid i\in[1,8]\vee i=\bot}$ e no conjunto de regras $\mathcal{R}=\set{\mathbf{R_1},\mathbf{R_2}}$, definidos abaixo:
        \begin{alignat*}{3}
            &\mathbf{A_1}\quad&&\alpha\to\beta\to\alpha\label{IA1}\tag*{}\displaybreak[0]\\
            &\mathbf{A_2}\quad&&(\alpha\to\beta\to\gamma)\to(\alpha\to\beta)\to(\alpha\to\gamma)\label{IA2}\tag*{}\displaybreak[0]\\
            &\mathbf{A_3}\quad&&\alpha\to\beta\to\alpha\wedge\beta\label{IA3}\tag*{}\displaybreak[0]\\
            &\mathbf{A_4}\quad&&\alpha\wedge\beta\to\alpha\label{IA4}\tag*{}\displaybreak[0]\\
            &\mathbf{A_5}\quad&&\alpha\wedge\beta\to\beta\label{IA5}\tag*{}\displaybreak[0]\\
            &\mathbf{A_6}\quad&&\alpha\to\alpha\vee\beta\label{IA6}\tag*{}\displaybreak[0]\\
            &\mathbf{A_7}\quad&&\beta\to\alpha\vee\beta\label{IA7}\tag*{}\displaybreak[0]\\
            &\mathbf{A_8}\quad&&(\alpha\to\gamma)\to(\beta\to\gamma)\to(\alpha\vee\beta\to\gamma)\label{IA8}\tag*{}\displaybreak[0]\\
            &\mathbf{A_\bot}\quad&&\bot\to\alpha\label{IABOT}\tag*{}\displaybreak[0]\\
            &\mathbf{R_1}\quad&&\alpha\in\Gamma\text{, então}\Gamma\entails\alpha\tag*{}\displaybreak[0]\\
            &\mathbf{R_2}\quad && \text{Se }\Gamma\vdash\alpha\text{ e }\Gamma\vdash\alpha\to\beta\text{, então }\Gamma\vdash\beta\text{.} & \tag*{\qed}
        \end{alignat*}
    \end{definition}

    De modo a facilitar a comunicação no decorrer deste trabalho, chamaremos $\mathbf{R_1}$ de regra da separação ou \emph{modus ponens}.

    O sistema intuicionista calca-se numa visão construtivista que fundamenta aplicações importantes na computação. O isomorfismo de Curry-Howard estabelece uma associação entre provas e programas e entre proposições e tipos.
    Enquanto isso, a interpretação de Brouwer-Heyting-Kolmogorov --- definida abaixo segundo~\cite{Troelstra} --- exige que provas sejam construtivas, garantindo a realidade efetiva dos elementos provados. Tais propriedades são largamente usadas, por exemplo, na prova de teoremas computacionalmente e na construção de compiladores robustos.

    \begin{enumerate}[label=\textbf{(\alph*)}, left=\parindent]
        \item Não existe prova de $\bot$.
        \item Uma prova de $\alpha\wedge\beta$ consiste num par $\sequence{A,B}$, sendo $A$ uma prova de $\alpha$ e $B$ uma prova de $\beta$.
        \item Uma prova de $\alpha\vee\beta$ consiste ou num par $\sequence{0,A}$, sendo $A$ uma prova de $\alpha$, ou num par $\sequence{1,B}$, sendo $B$ uma prova de $\beta$.
        \item Uma prova de $\alpha\to\beta$ consiste numa construção $C$ que transforma uma prova $A$ de $\alpha$ numa prova $B$ de $\beta$.
    \end{enumerate}
