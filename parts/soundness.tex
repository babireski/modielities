\section{Correção}

    Neste seção, apresentaremos demosntrações de correção para ambas as traduções.
    Antes disso, entretanto, demonstraremos a \emph{estabilidade} da tradução quadrado por indução sobre a profundidade da sentença, como sugerido por~\cite{Troelstra+Schwichtenberg.2000}.
    Usaremos este teorema, cujo nome tiramos de~\cite{Flagg+Friedman.1986}, para a demonstração da correção da tradução quadrado.

    \vspace{.5\baselineskip}
    \begin{tcolorbox}[enhanced jigsaw, breakable, sharp corners, colframe=black, colback=white, boxrule=0.5pt, left=1.5mm, right=1.5mm, top=1.5mm, bottom=1.5mm]
    \begin{theorem}[Estabilidade]\label{stability}
        $\Gamma\entails_\mathfrak{M}\alpha^\bullet\to\nec\alpha^\bullet$.
    \end{theorem}

            \emph{Demonstração.} Demonstração por indução forte sobre a profundidade da sentença \citep{Troelstra+Schwichtenberg.2000}.
            Seja $n\in\mathbb{N}^+$ a profundidade da sentença $\alpha\in\mathcal{L}$.
            Suponhamos que a asserção valha para qualquer sentença de profundidade menor que $n$ e nomeemos esta suposição $\mathbf{H}$.
            Devemos considerar cinco casos: a letra, a contradição, a conjunção, a disjunção e a implicação.

            \vspace{.5\baselineskip}
            \textsc{Caso 1.}
            Seja a sentença $\alpha$ for uma proposição $a\in\mathcal{P}$.
            Sabe-se que $a^\bullet=\nec a$ pela definição da tradução.
            Deste modo, $\Gamma\vdash\nec{a}\to\nec\nec{a}$ pode ser gerado por \hyperref[modal.axiom.modal.3]{$\mathbf{B_3}$}.

            \vspace{.5\baselineskip}
            \textsc{Caso 2.}
            Seja a sentença $\alpha$ a constante $\bot$.
            Sabe-se que $\bot^\bullet=\bot$ pela definição da tradução.
            Deste modo, $\Gamma\entails\bot\to\nec\bot$ foi provado pelo lema \hyperref[explosion]{$\mathbf{L_3}$}.

            \vspace{.5\baselineskip}
            \textsc{Caso 3.}
            Seja a sentença $\alpha$ a conjunção de duas sentenças $\alpha_1$ e $\alpha_2$.
            Sabe-se que ${(\alpha_1\wedge\alpha_2)}^\bullet=\alpha_1^\bullet\wedge\alpha_2^\bullet$ pela definição da tradução.
            A partir de $\mathbf{H}$, temos que $\Gamma\entails\alpha_1^\bullet\to\nec\alpha_1^\bullet$ e que $\Gamma\entails\alpha_2^\bullet\to\nec\alpha_2^\bullet$, ditos $\mathbf{H_1}$ e $\mathbf{H_2}$.
            Pode-se demonstrar $\Gamma\vdash{(\alpha_1\wedge\alpha_2)}^\bullet\to\nec{(\alpha_1\wedge\alpha_2)}^\bullet$ pela dedução que segue.

            \vspace{.5\baselineskip}
            \footnotesize
            \setlength{\rowskip}{.5\baselineskip}
            \begin{xltabular}{\textwidth}{r | X l l}
                \scriptsize{\phantom{1}1}\phantom{ } & $\ \Gamma\cup\set{\alpha_1^\bullet\wedge\alpha_2^\bullet}\entails\alpha_1^\bullet\to\nec\alpha_1^\bullet$                                                                 & $\mathbf{H_1}$\phantom{1}                & \phantom{$\set{00,00}$}\\[\rowskip]
                \scriptsize{\phantom{1}2}\phantom{ } & $\ \Gamma\cup\set{\alpha_1^\bullet\wedge\alpha_2^\bullet}\entails\alpha_2^\bullet\to\nec\alpha_2^\bullet$                                                                 & $\mathbf{H_2}$                                  & \\[\rowskip]
                \scriptsize{\phantom{1}3}\phantom{ } & $\ \Gamma\cup\set{\alpha_1^\bullet\wedge\alpha_2^\bullet}\entails\alpha_1^\bullet\wedge\alpha_2^\bullet$                                                                  & $\hyperref[modal.rule.1]{\mathbf{R_1}}$         & \\[\rowskip]
                \scriptsize{\phantom{1}4}\phantom{ } & $\ \Gamma\cup\set{\alpha_1^\bullet\wedge\alpha_2^\bullet}\entails\alpha_1^\bullet\wedge\alpha_2^\bullet\to\nec\alpha_1^\bullet\wedge\nec\alpha_2^\bullet$       & \refer{conjunction.exchange}{L}                 & \\[\rowskip]
                \scriptsize{\phantom{1}5}\phantom{ } & $\ \Gamma\cup\set{\alpha_1^\bullet\wedge\alpha_2^\bullet}\entails\nec\alpha_1^\bullet\wedge\nec\alpha_2^\bullet$                                                          & $\hyperref[modal.rule.2]{\mathbf{R_2}}$         & $\set{3,4}$\\[\rowskip]
                \scriptsize{\phantom{1}6}\phantom{ } & $\ \Gamma\cup\set{\alpha_1^\bullet\wedge\alpha_2^\bullet}\entails\nec\alpha_1^\bullet\wedge\nec\alpha_2^\bullet\to\nec(\alpha_1^\bullet\wedge\alpha_2^\bullet)$ & \refer{necessity.conjunction.undistribution}{L} &\\[\rowskip]
                \scriptsize{\phantom{1}7}\phantom{ } & $\ \Gamma\cup\set{\alpha_1^\bullet\wedge\alpha_2^\bullet}\entails\nec(\alpha_1^\bullet\wedge\alpha_2^\bullet)$                                                            & $\hyperref[modal.rule.2]{\mathbf{R_2}}$         & $\set{5,6}$\\[\rowskip]
                \scriptsize{\phantom{1}8}\phantom{ } & $\ \Gamma\entails\alpha_1^\bullet\wedge\alpha_2^\bullet\to\nec(\alpha_1^\bullet\wedge\alpha_2^\bullet)$                                                                   & \refer{deduction}{T} & $\set{7}$
            \end{xltabular}
            \normalsize

            \vspace{.5\baselineskip}
            \textsc{Caso 4.}
            Seja a sentença $\alpha$ a disjunção de duas sentenças $\alpha_1$ e $\alpha_2$.
            Sabe-se que ${(\alpha_1\vee\alpha_2)}^\bullet=\alpha_1^\bullet\vee\alpha_2^\bullet$ pela definição da tradução.
            A partir de $\mathbf{H}$, temos que $\Gamma\entails\alpha_1^\bullet\to\nec\alpha_1^\bullet$ e que $\Gamma\entails\alpha_2^\bullet\to\nec\alpha_2^\bullet$, ditos $\mathbf{H_1}$ e $\mathbf{H_2}$.
            Pode-se demonstrar $\Gamma\vdash{(\alpha_1\vee\alpha_2)}^\bullet\to\nec{(\alpha_1\vee\alpha_2)}^\bullet$ pela dedução que segue.

            \vspace{.5\baselineskip}
            \footnotesize
            \setlength{\rowskip}{.5\baselineskip}
            \begin{xltabular}{\textwidth}{r | X l l}
                \scriptsize{\phantom{1}1}\phantom{ } & $\ \Gamma\cup\set{\alpha_1^\bullet\vee\alpha_2^\bullet}\entails\alpha_1^\bullet\to\nec\alpha_1^\bullet$                                                                 & $\mathbf{H_1}$\phantom{1}                & \phantom{$\set{00,00}$}\\[\rowskip]
                \scriptsize{\phantom{1}2}\phantom{ } & $\ \Gamma\cup\set{\alpha_1^\bullet\vee\alpha_2^\bullet}\entails\alpha_2^\bullet\to\nec\alpha_2^\bullet$                                                                 & $\mathbf{H_2}$                                  & \\[\rowskip]\pagebreak[4]
                \scriptsize{\phantom{1}3}\phantom{ } & $\ \Gamma\cup\set{\alpha_1^\bullet\vee\alpha_2^\bullet}\entails\alpha_1^\bullet\vee\alpha_2^\bullet$                                                                    & $\hyperref[modal.rule.1]{\mathbf{R_1}}$         & \\[\rowskip]
                \scriptsize{\phantom{1}4}\phantom{ } & $\ \Gamma\cup\set{\alpha_1^\bullet\vee\alpha_2^\bullet}\entails\alpha_1^\bullet\vee\alpha_2^\bullet\to\nec\alpha_1^\bullet\vee\nec\alpha_2^\bullet$           & \refer{disjunction.exchange}{L}                 & $\set{1, 2}$\\[\rowskip]
                \scriptsize{\phantom{1}5}\phantom{ } & $\ \Gamma\cup\set{\alpha_1^\bullet\vee\alpha_2^\bullet}\entails\nec\alpha_1^\bullet\vee\nec\alpha_2^\bullet$                                                            & $\hyperref[modal.rule.2]{\mathbf{R_2}}$         & $\set{3,4}$\\[\rowskip]
                \scriptsize{\phantom{1}6}\phantom{ } & $\ \Gamma\cup\set{\alpha_1^\bullet\vee\alpha_2^\bullet}\entails\nec\alpha_1^\bullet\vee\nec\alpha_2^\bullet\to\nec(\alpha_1^\bullet\vee\alpha_2^\bullet)$     & \refer{necessity.conjunction.undistribution}{L} &\\[\rowskip]
                \scriptsize{\phantom{1}7}\phantom{ } & $\ \Gamma\cup\set{\alpha_1^\bullet\vee\alpha_2^\bullet}\entails\nec(\alpha_1^\bullet\vee\alpha_2^\bullet)$                                                              & $\hyperref[modal.rule.2]{\mathbf{R_2}}$         & $\set{5,6}$\\[\rowskip]
                \scriptsize{\phantom{1}8}\phantom{ } & $\ \Gamma\entails\alpha_1^\bullet\vee\alpha_2^\bullet\to\nec(\alpha_1^\bullet\vee\alpha_2^\bullet)$                                                                     & \refer{deduction}{T} & $\set{7}$
            \end{xltabular}
            \normalsize

            \vspace{.5\baselineskip}
            \textsc{Caso 5.}
            Seja a sentença $\alpha$ a implicação de duas sentenças $\alpha_1$ e $\alpha_2$.
            Sabe-se que ${(\alpha_1\to\alpha_2)}^\bullet=\nec(\alpha_1^\bullet\to\alpha_2^\bullet)$ pela definição da tradução.
            Deste modo, $\Gamma\vdash\nec(\alpha_1^\bullet\to\alpha_2^\bullet)\to\nec\nec(\alpha_1^\bullet\to\alpha_2^\bullet)$ pode ser gerado pela regra \hyperref[modal.axiom.modal.3]{$\mathbf{B_3}$}.

            \vspace{.5\baselineskip}
            Estando assim demonstrada a proposição.
    \end{tcolorbox}

    \vspace{.5\baselineskip}
    Uma vez demonstrado o teorema da estabilidade podemos, então, demonstrar o teorema correção da tradução quadrado por indução sobre o tamanho da sucessão de dedução, conforme~\cite{Troelstra+Schwichtenberg.2000}.

    \vspace{.5\baselineskip}
    \begin{tcolorbox}[enhanced jigsaw, breakable, sharp corners, colframe=black, colback=white, boxrule=0.5pt, left=1.5mm, right=1.5mm, top=1.5mm, bottom=1.5mm]
    \begin{theorem}\label{square.soundness}
        Se $\Gamma\entails_\mathfrak{I}\alpha$, então $\Gamma^\bullet\entails_\mathfrak{M}\alpha^\bullet$.
    \end{theorem}

        \emph{Demonstração.}
        Demonstração por indução forte sobre o tamanho da sucessão de dedução~\citep{Troelstra+Schwichtenberg.2000}.
        Seja $n\in\mathbb{N}^+$ o tamanho da sucessão de dedução que deriva $\Gamma\entails\alpha$.
        Suponhamos que a correção da tradução quadrado valha para qualquer sucessão de dedução de tamanho menor que $n$ e nomeemos esta suposição $\mathbf{H}$.
        Demos considerar onze casos: um para cada regra de dedução.

        \vspace{.5\baselineskip}
        \textsc{Caso 1.}
        Seja a linha derradeira da sucessão de dedução que prova $\Gamma\entails\alpha$ gerada pela a regra $\hyperref[intuitionistic.axiom.1]{\mathbf{A_1}}$.
        Sabe-se que $\alpha=\alpha_1\to\alpha_2\to\alpha_1$ e que $\alpha^\bullet=\alpha_1^\bullet\strictif\alpha_2^\bullet\strictif\alpha_1^\bullet$.
        Pode-se demonstrar $\Gamma^\bullet\entails\alpha^\bullet$ pela dedução que segue.

        \vspace{\baselineskip}
        \footnotesize
        \setlength{\rowskip}{.5\baselineskip}
        \begin{tabularx}{\textwidth}{r | X l l}
            \scriptsize{\phantom{0}1}\phantom{ } & $\ \vdash \alpha_1^\bullet \to \nec\alpha_1^\bullet$                                                  & \refer{stability}{T}\phantom{1}                & \phantom{$\set{00,00}$}\\[\rowskip]
            \scriptsize{\phantom{0}2}\phantom{ } & $\ \vdash \alpha_1^\bullet \to \alpha_2^\bullet \to \alpha_1^\bullet$ & $\hyperref[modal.axiom.1]{\mathbf{A_1}}$ & \\[\rowskip]
            \scriptsize{\phantom{0}3}\phantom{ } & $\ \vdash \nec(\alpha_1^\bullet \to \alpha_2^\bullet \to \alpha_1^\bullet)$ & $\hyperref[modal.rule.3]{\mathbf{R_3}}$ & $\set{2}$\\[\rowskip]
            \scriptsize{\phantom{0}4}\phantom{ } & $\ \vdash \nec(\alpha_1^\bullet \to \alpha_2^\bullet \to \alpha_1^\bullet) \to \nec\alpha_1^\bullet \to \alpha_2^\bullet \strictif \alpha_1^\bullet$ & $\hyperref[modal.axiom.modal.1]{\mathbf{B_1}}$ & \\[\rowskip]
            \scriptsize{\phantom{0}5}\phantom{ } & $\ \vdash \nec\alpha_1^\bullet \to \alpha_2^\bullet \strictif \alpha_1^\bullet$ & $\hyperref[modal.rule.2]{\mathbf{R_2}}$ & $\set{3,4}$\\[\rowskip]
            \scriptsize{\phantom{0}8}\phantom{ } & $\ \vdash \alpha_1^\bullet \to \nec\alpha_2^\bullet \strictif \alpha_1^\bullet$ & \refer{composition}{L} & $\set{1,5}$\\[\rowskip]
            \scriptsize{\phantom{0}9}\phantom{ } & $\ \Gamma^\bullet \vdash \alpha_1^\bullet \strictif \alpha_2^\bullet \strictif \alpha_1^\bullet$ & $\hyperref[modal.rule.3]{\mathbf{R_3}}$ & $\set{6}$
        \end{tabularx}
        \normalsize

        \vspace{\baselineskip}
        \textsc{Caso 2.}
        Seja a linha derradeira da sucessão de dedução que prova $\Gamma\entails\alpha$ gerada pela a regra $\hyperref[intuitionistic.axiom.2]{\mathbf{A_2}}$.
        Sabe-se que $\alpha=(\alpha_1\to\alpha_2\to\alpha_3)\to(\alpha_1\to\alpha_2)\to\alpha_1\to\alpha_3$ e que $\alpha^\bullet=(\alpha_1^\bullet\strictif\alpha_2^\bullet\strictif\alpha_3^\bullet)\strictif(\alpha_1^\bullet\strictif\alpha_2^\bullet)\strictif\alpha_1^\bullet\strictif\alpha_3^\bullet$.
        Pode-se demonstrar $\Gamma^\bullet\entails\alpha^\bullet$ pela dedução que segue.

        \vspace{\baselineskip}
        \footnotesize
        \setlength{\rowskip}{.5\baselineskip}
        \begin{tabularx}{\textwidth}{r | X l l}
            \scriptsize{\phantom{0}1}\phantom{ } & $\ \set{\alpha_1^\bullet\strictif\alpha_2^\bullet\strictif\alpha_3^\bullet,\alpha_1^\bullet\strictif\alpha_2^\bullet,\alpha_1^\bullet} \vdash \alpha_1^\bullet$ & $\hyperref[modal.rule.1]{\mathbf{R_1}}$\phantom{1}                & \phantom{$\set{00,00}$}\\[\rowskip]
            \scriptsize{\phantom{0}2}\phantom{ } & $\ \set{\alpha_1^\bullet\strictif\alpha_2^\bullet\strictif\alpha_3^\bullet,\alpha_1^\bullet\strictif\alpha_2^\bullet,\alpha_1^\bullet} \vdash \alpha_1^\bullet\strictif\alpha_2^\bullet$ & $\hyperref[modal.rule.1]{\mathbf{R_1}}$ & \\[\rowskip]
            \scriptsize{\phantom{0}3}\phantom{ } & $\ \set{\alpha_1^\bullet\strictif\alpha_2^\bullet\strictif\alpha_3^\bullet,\alpha_1^\bullet\strictif\alpha_2^\bullet,\alpha_1^\bullet} \vdash \alpha_2^\bullet$ & $\hyperref[strict.detachment]{\mathbf{L_{\getrefnumber{strict.detachment}}}}$ & $\set{1,2}$\\[\rowskip]
            \scriptsize{\phantom{0}4}\phantom{ } & $\ \set{\alpha_1^\bullet\strictif\alpha_2^\bullet\strictif\alpha_3^\bullet,\alpha_1^\bullet\strictif\alpha_2^\bullet,\alpha_1^\bullet} \vdash \alpha_1^\bullet\strictif\alpha_2^\bullet\strictif\alpha_3^\bullet$ & $\hyperref[modal.rule.1]{\mathbf{R_1}}$ & \\[\rowskip]
            \scriptsize{\phantom{0}5}\phantom{ } & $\ \set{\alpha_1^\bullet\strictif\alpha_2^\bullet\strictif\alpha_3^\bullet,\alpha_1^\bullet\strictif\alpha_2^\bullet,\alpha_1^\bullet} \vdash \alpha_2^\bullet\strictif\alpha_3^\bullet$ & $\hyperref[strict.detachment]{\mathbf{L_{\getrefnumber{strict.detachment}}}}$ & $\set{1,4}$\\[\rowskip]
            \scriptsize{\phantom{0}6}\phantom{ } & $\ \set{\alpha_1^\bullet\strictif\alpha_2^\bullet\strictif\alpha_3^\bullet,\alpha_1^\bullet\strictif\alpha_2^\bullet,\alpha_1^\bullet} \vdash \alpha_3^\bullet$ & $\hyperref[strict.detachment]{\mathbf{L_{\getrefnumber{strict.detachment}}}}$ & $\set{3,5}$\\[\rowskip]
            \scriptsize{\phantom{0}7}\phantom{ } & $\ \set{\alpha_1^\bullet\strictif\alpha_2^\bullet\strictif\alpha_3^\bullet,\alpha_1^\bullet\strictif\alpha_2^\bullet} \vdash \alpha_1^\bullet\strictif\alpha_3^\bullet$ & $\hyperref[strict.deduction]{\mathbf{L_{\getrefnumber{strict.deduction}}}}$ & $\set{6}$\\[\rowskip]
            \scriptsize{\phantom{0}8}\phantom{ } & $\ \set{\alpha_1^\bullet\strictif\alpha_2^\bullet\strictif\alpha_3^\bullet} \vdash (\alpha_1^\bullet\strictif\alpha_2^\bullet)\strictif\alpha_1^\bullet\strictif\alpha_3^\bullet$ & $\hyperref[strict.deduction]{\mathbf{L_{\getrefnumber{strict.deduction}}}}$ & $\set{7}$\\[\rowskip]
            \scriptsize{\phantom{0}9}\phantom{ } & $\ \vdash (\alpha_1^\bullet\strictif\alpha_2^\bullet\strictif\alpha_3^\bullet)\to(\alpha_1^\bullet\strictif\alpha_2^\bullet)\strictif\alpha_1^\bullet\strictif\alpha_3^\bullet$ & $\hyperref[deduction]{\mathbf{T_{\getrefnumber{deduction}}}}$ & $\set{8}$\\[\rowskip]
            \scriptsize{10}\phantom{ } & $\ \Gamma^\bullet \vdash (\alpha_1^\bullet\strictif\alpha_2^\bullet\strictif\alpha_3^\bullet)\strictif(\alpha_1^\bullet\strictif\alpha_2^\bullet)\strictif\alpha_1^\bullet\strictif\alpha_3^\bullet$ & $\hyperref[modal.rule.3]{\mathbf{R_3}}$ & $\set{9}$
        \end{tabularx}
        \normalsize

        \vspace{\baselineskip}
        \textsc{Caso 3.}
        Seja a linha derradeira da sucessão de dedução que prova $\Gamma\entails\alpha$ gerada pela a regra $\hyperref[intuitionistic.axiom.3]{\mathbf{A_3}}$.
        Sabe-se que $\alpha=\alpha_1\to\alpha_2\to\alpha_1\wedge\alpha_2$ e que $\alpha^\bullet=\alpha_1^\bullet\strictif\alpha_2^\bullet\strictif\alpha_1^\bullet\wedge\alpha_2^\bullet$.
        Pode-se demonstrar $\Gamma^\bullet\entails\alpha^\bullet$ pela dedução que segue.

        \vspace{\baselineskip}
        \footnotesize
        \setlength{\rowskip}{.5\baselineskip}
        \begin{tabularx}{\textwidth}{r | X l l}
            \scriptsize{\phantom{0}1}\phantom{ } & $\ \vdash \alpha_1^\bullet \to \nec\alpha_1^\bullet$ & \refer{stability}{T}\phantom{1}                & \phantom{$\set{00,00}$}\\[\rowskip]
            \scriptsize{\phantom{0}2}\phantom{ } & $\ \vdash \alpha_1^\bullet \to \alpha_2^\bullet \to \alpha_1^\bullet \wedge \alpha_2^\bullet$ & $\hyperref[MA3]{\mathbf{A_3}}$ & \\[\rowskip]
            \scriptsize{\phantom{0}3}\phantom{ } & $\ \vdash \nec(\alpha_1^\bullet \to \alpha_2^\bullet \to \alpha_1^\bullet \wedge \alpha_2^\bullet)$ & $\hyperref[modal.rule.3]{\mathbf{R_3}}$ & $\set{2}$\\[\rowskip]
            \scriptsize{\phantom{0}4}\phantom{ } & $\ \vdash \nec(\alpha_1^\bullet \to \alpha_2^\bullet \to \alpha_1^\bullet \wedge \alpha_2^\bullet) \to \nec\alpha_1^\bullet \to \alpha_2^\bullet \strictif \alpha_1^\bullet \wedge \alpha_2^\bullet$ & $\hyperref[modal.axiom.modal.1]{\mathbf{B_1}}$ & \\[\rowskip]
            \scriptsize{\phantom{0}5}\phantom{ } & $\ \vdash \nec\alpha_1^\bullet \to \alpha_2^\bullet \strictif \alpha_1^\bullet \wedge \alpha_2^\bullet$ & $\hyperref[modal.rule.2]{\mathbf{R_2}}$ & $\set{3,4}$\\
        \end{tabularx}
        \begin{tabularx}{\textwidth}{r | X l l}
            \scriptsize{\phantom{0}8}\phantom{ } & $\ \vdash \alpha_1^\bullet \to \alpha_2^\bullet \strictif \alpha_1^\bullet \wedge \alpha_2^\bullet$ & \refer{composition}{L} & $\set{1,6}$\\[\rowskip]
            \scriptsize{\phantom{0}9}\phantom{ } & $\ \Gamma^\bullet \vdash \alpha_1^\bullet \strictif \alpha_2^\bullet \strictif \alpha_1^\bullet \wedge \alpha_2^\bullet$ & $\hyperref[modal.rule.3]{\mathbf{R_3}}$ & $\set{6}$
        \end{tabularx}
        \normalsize

        \vspace{\baselineskip}
        \textsc{Caso 4.}
        Seja a linha derradeira da sucessão de dedução que prova $\Gamma\entails\alpha$ gerada pela a regra $\hyperref[intuitionistic.axiom.4]{\mathbf{A_4}}$.
        Sabe-se que $\alpha=\alpha_1\wedge\alpha_2\to\alpha_1$ e que $\alpha^\bullet=\alpha_1^\bullet\wedge\alpha_2^\bullet\strictif\alpha_1^\bullet$
        Pode-se demonstrar $\Gamma^\bullet\entails\alpha^\bullet$ pelo uso da regra $\hyperref[modal.axiom.4]{\mathbf{A_4}}$ seguido do uso da regra $\hyperref[modal.rule.3]{\mathbf{R_3}}$.

        \vspace{.5\baselineskip}
        \textsc{Caso 5.}
        Seja a linha derradeira da sucessão de dedução que prova $\Gamma\entails\alpha$ gerada pela a regra $\hyperref[intuitionistic.axiom.5]{\mathbf{A_5}}$.
        Sabe-se que $\alpha=\alpha_1\wedge\alpha_2\to\alpha_2$ e que $\alpha^\bullet=\alpha_1^\bullet\wedge\alpha_2^\bullet\strictif\alpha_2^\bullet$
        Pode-se demonstrar $\Gamma^\bullet\entails\alpha^\bullet$ pelo uso da regra $\hyperref[modal.axiom.5]{\mathbf{A_5}}$ seguido do uso da regra $\hyperref[modal.rule.3]{\mathbf{R_3}}$.

        \vspace{.5\baselineskip}
        \textsc{Caso 6.}
        Seja a linha derradeira da sucessão de dedução que prova $\Gamma\entails\alpha$ gerada pela a regra $\hyperref[intuitionistic.axiom.6]{\mathbf{A_6}}$.
        Sabe-se que $\alpha=\alpha_1\to\alpha_1\vee\alpha_2$ e que $\alpha^\bullet=\alpha_1^\bullet\strictif\alpha_1^\bullet\vee\alpha_2^\bullet$
        Pode-se demonstrar $\Gamma^\bullet\entails\alpha^\bullet$ pelo uso da regra $\hyperref[modal.axiom.6]{\mathbf{A_6}}$ seguido do uso da regra $\hyperref[modal.rule.3]{\mathbf{R_3}}$.

        \vspace{.5\baselineskip}
        \textsc{Caso 7.}
        Seja a linha derradeira da sucessão de dedução que prova $\Gamma\entails\alpha$ gerada pela a regra $\hyperref[intuitionistic.axiom.7]{\mathbf{A_7}}$.
        Sabe-se que $\alpha=\alpha_2\to\alpha_1\vee\alpha_2$ e que $\alpha^\bullet=\alpha_2^\bullet\strictif\alpha_1^\bullet\vee\alpha_2^\bullet$
        Pode-se demonstrar $\Gamma^\bullet\entails\alpha^\bullet$ pelo uso da regra $\hyperref[modal.axiom.7]{\mathbf{A_7}}$ seguido do uso da regra $\hyperref[modal.rule.3]{\mathbf{R_3}}$.

        \vspace{.5\baselineskip}
        \textsc{Caso 8.}
        Seja a linha derradeira da sucessão de dedução que prova $\Gamma\entails\alpha$ gerada pela a regra $\hyperref[intuitionistic.axiom.8]{\mathbf{A_8}}$.
        Sabe-se $\alpha=(\alpha_1\to\alpha_3)\to(\alpha_2\to\alpha_3)\to\alpha_1\vee\alpha_2\to\alpha_3$ e que $\alpha^\bullet=(\alpha_1^\bullet\strictif\alpha_3^\bullet)\strictif(\alpha_2^\bullet\strictif\alpha_3^\bullet)\strictif\alpha_1^\bullet\vee\alpha_2^\bullet\strictif\alpha_3^\bullet$.
        Pode-se demonstrar $\Gamma^\bullet\entails\alpha^\bullet$ pela dedução que segue, onde $\Delta=\set{\alpha_1^\bullet\strictif\alpha_3^\bullet,\alpha_2^\bullet\strictif\alpha_3^\bullet,\alpha_1^\bullet\vee\alpha_2^\bullet}$.

        \vspace{.5\baselineskip}
        \footnotesize
        \setlength{\rowskip}{.5\baselineskip}
        \begin{xltabular}{\textwidth}{r | X l l}
            \scriptsize{\phantom{0}1}\phantom{ } & $\ \Delta \vdash \alpha_1^\bullet\strictif\alpha_3^\bullet$ & $\hyperref[modal.rule.1]{\mathbf{R_1}}$\phantom{1}                & \phantom{$\set{00,00}$}\\[\rowskip]
            \scriptsize{\phantom{0}2}\phantom{ } & $\ \Delta \vdash (\alpha_1^\bullet\strictif\alpha_3^\bullet)\to\alpha_1^\bullet\to\alpha_1^\bullet$ & $\hyperref[modal.axiom.modal.2]{\mathbf{B_2}}$ & \\[\rowskip]
            \scriptsize{\phantom{0}3}\phantom{ } & $\ \Delta \vdash \alpha_1^\bullet\to\alpha_3^\bullet$ & $\hyperref[modal.rule.2]{\mathbf{R_2}}$ & $\set{1,2}$\\[\rowskip]
            \scriptsize{\phantom{0}4}\phantom{ } & $\ \Delta \vdash \alpha_2^\bullet\strictif\alpha_3^\bullet$ & $\hyperref[modal.rule.1]{\mathbf{R_1}}$ & \\[\rowskip]
            \scriptsize{\phantom{0}5}\phantom{ } & $\ \Delta \vdash (\alpha_2^\bullet\strictif\alpha_3^\bullet)\to\alpha_2^\bullet\to\alpha_3^\bullet$ & $\hyperref[modal.axiom.modal.2]{\mathbf{B_2}}$ & \\[\rowskip]
            \scriptsize{\phantom{0}6}\phantom{ } & $\ \Delta \vdash \alpha_2^\bullet\to\alpha_3^\bullet$ & $\hyperref[modal.rule.2]{\mathbf{R_2}}$ & $\set{4,5}$\\[\rowskip]
            \scriptsize{\phantom{0}7}\phantom{ } & $\ \Delta \vdash \alpha_1^\bullet\vee\alpha_2^\bullet$ & $\hyperref[modal.rule.1]{\mathbf{R_1}}$ & \\[\rowskip]
            \scriptsize{\phantom{0}8}\phantom{ } & $\ \Delta \vdash (\alpha_1^\bullet\to\alpha_3^\bullet)\to(\alpha_2^\bullet\to\alpha_3^\bullet)\to\alpha_1^\bullet\vee\alpha_2^\bullet\to\alpha_3^\bullet$ & $\hyperref[modal.axiom.8]{\mathbf{A_8}}$ & \\[\rowskip]
            \scriptsize{\phantom{0}9}\phantom{ } & $\ \Delta \vdash (\alpha_2^\bullet\to\alpha_3^\bullet)\to\alpha_1^\bullet\vee\alpha_2^\bullet\to\alpha_3^\bullet$ & $\hyperref[modal.rule.2]{\mathbf{R_2}}$ & $\set{3,8}$\\[\rowskip]
            \scriptsize{10}\phantom{ } & $\ \Delta \vdash \alpha_1^\bullet\vee\alpha_2^\bullet\to\alpha_3^\bullet$ & $\hyperref[modal.rule.2]{\mathbf{R_2}}$ & $\set{6,9}$\\[\rowskip]
            \scriptsize{11}\phantom{ } & $\ \Delta \vdash \alpha_3^\bullet$ & $\hyperref[modal.rule.2]{\mathbf{R_2}}$ & $\set{7,10}$\\[\rowskip]
            \scriptsize{12}\phantom{ } & $\ \set{\alpha_1^\bullet\strictif\alpha_3^\bullet,\alpha_2^\bullet\strictif\alpha_3^\bullet} \vdash \alpha_1^\bullet\vee\alpha_2^\bullet\strictif\alpha_3^\bullet$ & $\hyperref[strict.deduction]{\mathbf{T_{\getrefnumber{strict.deduction}}}}$ & $\set{11}$\\[\rowskip]
            \scriptsize{13}\phantom{ } & $\ \set{\alpha_1^\bullet\strictif\alpha_3^\bullet} \vdash (\alpha_2^\bullet\strictif\alpha_3^\bullet)\strictif\alpha_1^\bullet\vee\alpha_2^\bullet\strictif\alpha_3^\bullet$ & $\hyperref[strict.deduction]{\mathbf{T_{\getrefnumber{strict.deduction}}}}$ & $\set{12}$\\[\rowskip]
            \scriptsize{14}\phantom{ } & $\ \vdash (\alpha_1^\bullet\strictif\alpha_3^\bullet)\to(\alpha_2^\bullet\strictif\alpha_3^\bullet)\strictif\alpha_1^\bullet\vee\alpha_2^\bullet\strictif\alpha_3^\bullet$ & $\hyperref[deduction]{\mathbf{T_{\getrefnumber{deduction}}}}$ & $\set{13}$\\[\rowskip]
            \scriptsize{15}\phantom{ } & $\ \Gamma^\bullet \vdash (\alpha_1^\bullet\strictif\alpha_3^\bullet)\strictif(\alpha_2^\bullet\strictif\alpha_3^\bullet)\strictif\alpha_1^\bullet\vee\alpha_2^\bullet\strictif\alpha_3^\bullet$ & $\hyperref[modal.rule.3]{\mathbf{R_3}}$ & $\set{14}$
        \end{xltabular}
        \normalsize

        \vspace{.5\baselineskip}
        \textsc{Caso 9.}
        Seja a linha derradeira da sucessão de dedução que prova $\Gamma\entails\alpha$ gerada pela a regra $\hyperref[intuitionistic.axiom.contradiction]{\mathbf{A_{\bot}}}$.
        Sabe-se que $\alpha=\bot\to\alpha_1$ e que $\alpha^\bullet=\nec(\bot\to\alpha_1^\bullet)$.
        Pode-se demonstrar $\Gamma^\bullet\entails\alpha^\bullet$ pelo uso do lema \refer{explosion}{L} seguido do uso da regra $\hyperref[modal.rule.3]{\mathbf{R_3}}$.

        \vspace{.5\baselineskip}
        \textsc{Caso 10.}
        Seja a linha derradeira da sucessão de dedução que prova $\Gamma\entails\alpha$ gerada pela regra $\hyperref[intuitionistic.rule.1]{\mathbf{R_1}}$.
        Sabe-se que $\alpha\in\Gamma$ e, portanto, que $\alpha^\bullet\in\Gamma^\bullet$.
        Pode-se demonstrar que $\Gamma^\bullet\entails\alpha^\bullet$ por meio da invocação da premissa $\alpha^\bullet$ com a regra $\hyperref[modal.rule.1]{\mathbf{R_1}}$.

        \vspace{.5\baselineskip}
        \textsc{Caso 11.}
        Seja a linha derradeira da sucessão de dedução que prova $\Gamma\entails\alpha$ gerada pela regra $\hyperref[intuitionistic.rule.2]{\mathbf{R_2}}$.
        Sabe-se que $\Gamma\entails\beta$ e que $\Gamma\entails\beta\to\alpha$, para algum $\beta$.
        A partir de $\mathbf{H}$, temos que $\Gamma^\bullet\entails\beta^\bullet$ e que $\Gamma^\bullet\entails\nec(\beta^\bullet\to\alpha^\bullet)$, ditos $\mathbf{H_1}$ e $\mathbf{H_2}$.
        Pode-se demonstrar $\Gamma^\bullet\entails\alpha^\bullet$ pela dedução que segue.

        \vspace{\baselineskip}
        \footnotesize
        \setlength{\rowskip}{.5\baselineskip}
        \begin{tabularx}{\textwidth}{r | X l l}
            \scriptsize{\phantom{0}1}\phantom{ } & $\ \Gamma^\bullet \vdash \beta^\bullet$ & $\mathbf{H_2}$\phantom{1}                & \phantom{$\set{00,00}$}\\[\rowskip]
            \scriptsize{\phantom{0}2}\phantom{ } & $\ \Gamma^\bullet \vdash \nec(\beta^\bullet\to\alpha^\bullet)$ & $\mathbf{H_1}$ & \\[\rowskip]
            \scriptsize{\phantom{0}3}\phantom{ } & $\ \Gamma^\bullet \vdash \nec(\beta^\bullet\to\alpha^\bullet)\to\beta^\bullet\to\alpha^\bullet$ & $\hyperref[modal.axiom.modal.2]{\mathbf{B_2}}$ & \\[\rowskip]
            \scriptsize{\phantom{0}4}\phantom{ } & $\ \Gamma^\bullet \vdash \beta^\bullet\to\alpha^\bullet$ & $\hyperref[modal.rule.2]{\mathbf{R_2}}$ & $\set{2,3}$\\[\rowskip]
            \scriptsize{\phantom{0}5}\phantom{ } & $\ \Gamma^\bullet \vdash \alpha^\bullet$ & $\hyperref[modal.rule.2]{\mathbf{R_2}}$ & $\set{1,4}$
        \end{tabularx}
        \normalsize

        \vspace{\baselineskip}
        Estando assim demonstrada a proposição.
    \end{tcolorbox}

    \vspace{.5\baselineskip}
    A demonstração postrema deste trabalho, o teorema da correção da tradução redondo, segue de maneira simples das demonstrações de interderivação e de correção da tradução quadrado, conforme apresentado a seguir.

    \vspace{.5\baselineskip}
    \begin{tcolorbox}[enhanced jigsaw, breakable, sharp corners, colframe=black, colback=white, boxrule=0.5pt, left=1.5mm, right=1.5mm, top=1.5mm, bottom=1.5mm]
    \begin{theorem}[Correção]\label{circle.soundness}
        Se $\Gamma\entails_\mathfrak{I}\alpha$, então $\nec\Gamma^\circ\entails_\mathfrak{M}\alpha^\circ$.
    \end{theorem}
        \emph{Demonstração.} Sabe-se que $\Gamma\entails\alpha$. A partir do teorema \refer{square.soundness}{T}, temos que $\Gamma^\bullet\entails\alpha^\bullet$.
        Então, a partir do teorema \refer{interderivability}{T}, temos que $\nec\Gamma^\circ\entails\alpha^\circ$.

        \vspace{.5\baselineskip}
        Estando assim demonstrada a proposição.
    \end{tcolorbox}
