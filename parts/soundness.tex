\section{Correção}

    Neste sessão, apresentaremos uma prova de correção para a $\nec$-tradução. Antes disso, entretanto, apresentaremos uma prova de $\entails\alpha^\medsquare\to\nec\alpha^\medsquare$, que usaremos como lema para a prova da correção conforme sugerido por~\cite{Troelstra}. Provaremos o lema por indução sobre a profundidade da sentença e a correção por indução sobre o tamanho da prova.

    \begin{lemma}\label{square-nec}
        $\entails\alpha^\medsquare\to\nec\alpha^\medsquare$.

        \begin{proof}
            Prova por indução forte sobre a profundidade de $\alpha\in\mathcal{L}_\mathbf{I}$ \citep{Troelstra}.
            Assim, suponhamos que a proposição valha para qualquer sentença $\alpha$ de profundidade $n<k$.
            Demonstraremos analisando-se os casos e valendo-se da suposição acima --- doravante chamada $\mathbf{H}$ --- o passo de indução, ou seja, que a proposição vale para qualquer $\alpha$ de profundidade $n=k$.

            \begin{case}
                \textsc{Caso 1.}
                Se a sentença $\alpha$ for uma proposição $a\in\mathcal{P}$, sabe-se que $a^\medsquare=\nec a$ pela definição da tradução.
                Deste modo, $\nec{a}\to\nec\nec{a}$ constitui um axioma gerado pelo esquema \hyperref[MB3]{$\mathbf{4}$} --- sendo assim $\entails\nec{a}\to\nec\nec{a}$ provado trivialmente.
            \end{case}

            \begin{case}
                \textsc{Caso 2.}
                Se a sentença $\alpha$ for a constante $\bot$, sabe-se que $\bot^\medsquare=\bot$ pela definição da tradução.
                Deste modo, devemos provar que $\entails\bot\to\nec\bot$, o que consiste num caso particular da explosão provada pelo lema \hyperref[explosion]{$\mathbf{L_2}$}.
            \end{case}

            \begin{case}
                \textsc{Caso 3.}
                Se a sentença $\alpha$ for o resultado da conjunção de duas outras sentenças $\varphi$ e $\psi$, sabe-se que ${(\varphi\wedge\psi)}^\medsquare=\varphi^\medsquare\wedge\psi^\medsquare$ pela definição da tradução.
                A partir de $\mathbf{H}$, temos que $\mathbf{H_1}={\entails\varphi^\medsquare\to\nec\varphi^\medsquare}$ e que $\mathbf{H_2}={\entails\psi^\medsquare\to\nec\psi^\medsquare}$ por meio dos esquemas da eliminação da conjunção e da aplicação da regra do \emph{modus ponens}.
                Valendo-se do listado acima em conjunto com alguns lemas, pode-se provar que $\entails\varphi^\medsquare\wedge\psi^\medsquare\to\nec(\varphi^\medsquare\wedge\psi^\medsquare)$ pela seguinte sucessão de dedução:

                \footnotesize
                \begin{fitch}
                    \fb\set{\varphi^\medsquare\wedge\psi^\medsquare}\proves\varphi^\medsquare\wedge\psi^\medsquare&$\hyperref[premisse]{\mathbf{R_1}}$\\
                    \fa\set{\varphi^\medsquare\wedge\psi^\medsquare}\proves\varphi^\medsquare\wedge\psi^\medsquare\to\varphi^\medsquare&\hyperref[MA4]{${\mathbf{A_4}}$}\\
                    \fa\set{\varphi^\medsquare\wedge\psi^\medsquare}\proves\varphi^\medsquare&$\hyperref[detachment]{\mathbf{R_2}}\;\set{1,2}$\\
                    \fa\set{\varphi^\medsquare\wedge\psi^\medsquare}\proves\varphi^\medsquare\to\nec\varphi^\medsquare&$\mathbf{H_1}$\\
                    \fa\set{\varphi^\medsquare\wedge\psi^\medsquare}\proves\nec\varphi^\medsquare&$\hyperref[detachment]{\mathbf{R_2}}\;\set{3,4}$\\
                    \fa\set{\varphi^\medsquare\wedge\psi^\medsquare}\proves\varphi^\medsquare\wedge\psi^\medsquare\to\psi^\medsquare&\hyperref[MA5]{${\mathbf{A_5}}$}\\
                    \fa\set{\varphi^\medsquare\wedge\psi^\medsquare}\proves\psi^\medsquare&$\hyperref[detachment]{\mathbf{R_2}}\;\set{1,6}$\\
                    \fa\set{\varphi^\medsquare\wedge\psi^\medsquare}\proves\psi^\medsquare\to\nec\psi^\medsquare&$\mathbf{H_2}$\\
                    \fa\set{\varphi^\medsquare\wedge\psi^\medsquare}\proves\nec\psi^\medsquare&$\hyperref[detachment]{\mathbf{R_2}}\;\set{7,8}$\\
                    \fa\set{\varphi^\medsquare\wedge\psi^\medsquare}\proves\nec\varphi^\medsquare\to\nec\psi^\medsquare\to\nec\varphi^\medsquare\wedge\nec\psi^\medsquare&\hyperref[MA3]{${\mathbf{A_3}}$}\\
                    \fa\set{\varphi^\medsquare\wedge\psi^\medsquare}\proves\nec\psi^\medsquare\to\nec\varphi^\medsquare\wedge\nec\psi^\medsquare&$\hyperref[detachment]{\mathbf{R_2}}\;\set{5,10}$\\
                    \fa\set{\varphi^\medsquare\wedge\psi^\medsquare}\proves\nec\varphi^\medsquare\wedge\nec\psi^\medsquare&$\hyperref[detachment]{\mathbf{R_2}}\;\set{9,11}$\\
                    \fa\proves\varphi^\medsquare\wedge\psi^\medsquare\to\nec\varphi^\medsquare\wedge\nec\psi^\medsquare&$\hyperref[deduction]{\mathbf{T_\getrefnumber{deduction}}}\;\set{12}$\\
                    \fa\proves\nec\varphi^\medsquare\wedge\nec\psi^\medsquare\to\nec(\varphi^\medsquare\wedge\psi^\medsquare)&\refer{nec-distr}{L}\\
                    \fa\proves\varphi^\medsquare\wedge\psi^\medsquare\to\nec(\varphi^\medsquare\wedge\psi^\medsquare)&\refer{comp}{L}$\;\set{13, 14}$\\
                \end{fitch}
            \end{case}

            \begin{case}
                \textsc{Caso 4.}
                Se a sentença $\alpha$ for o resultado da disjunção de duas outras sentenças $\varphi$ e $\psi$, sabe-se que ${(\varphi\vee\psi)}^\medsquare=\varphi^\medsquare\vee\psi^\medsquare$ pela definição da tradução.
                A partir de $\mathbf{H}$, temos que $\mathbf{H_1}={\entails\varphi^\medsquare\to\nec\varphi^\medsquare}$ e que $\mathbf{H_2}={\entails\psi^\medsquare\to\nec\psi^\medsquare}$ por meio dos esquemas da eliminação da conjunção e da aplicação da regra do \emph{modus ponens}.
                Valendo-se do listado acima em conjunto com alguns lemas, pode-se provar que $\entails\varphi^\medsquare\vee\psi^\medsquare\to\nec(\varphi^\medsquare\vee\psi^\medsquare)$ pela seguinte sucessão de dedução:

                \footnotesize
                \begin{fitch}
                    \fb\set{\varphi^\medsquare\vee\psi^\medsquare}\entails\varphi^\medsquare\to\nec\varphi^\medsquare&$\mathbf{H_1}$\\
                    \fa\set{\varphi^\medsquare\vee\psi^\medsquare}\entails\psi^\medsquare\to\nec\psi^\medsquare&$\mathbf{H_2}$\\
                    \fa\set{\varphi^\medsquare\vee\psi^\medsquare}\entails\varphi^\medsquare\vee\psi^\medsquare&$\hyperref[premisse]{\mathbf{R_1}}$\\
                    \fa\set{\varphi^\medsquare\vee\psi^\medsquare}\entails(\varphi^\medsquare\to\nec\varphi^\medsquare)\to(\psi^\medsquare\to\nec\psi^\medsquare)\to\varphi^\medsquare\vee\psi^\medsquare\to\nec\varphi^\medsquare\vee\nec\psi^\medsquare&\refer{or-subst}{L}\\
                    \fa\set{\varphi^\medsquare\vee\psi^\medsquare}\entails(\psi^\medsquare\to\nec\psi^\medsquare)\to\varphi^\medsquare\vee\psi^\medsquare\to\nec\varphi^\medsquare\vee\nec\psi^\medsquare&$\hyperref[detachment]{\mathbf{R_1}}\;\set{1,4}$\\
                    \fa\set{\varphi^\medsquare\vee\psi^\medsquare}\entails\varphi^\medsquare\vee\psi^\medsquare\to\nec\varphi^\medsquare\vee\nec\psi^\medsquare&$\hyperref[detachment]{\mathbf{R_2}}\;\set{2,5}$\\
                    \fa\set{\varphi^\medsquare\vee\psi^\medsquare}\entails\nec\varphi^\medsquare\vee\nec\psi^\medsquare&$\hyperref[detachment]{\mathbf{R_2}}\;\set{3,6}$\\
                    \fa\set{\varphi^\medsquare\vee\psi^\medsquare}\entails\nec(\varphi^\medsquare\vee\psi^\medsquare)&\refer{or-distr}{L}$\;\set{7}$\\
                    \fa\entails\varphi^\medsquare\vee\psi^\medsquare\to\nec(\varphi^\medsquare\vee\psi^\medsquare)&$\hyperref[deduction]{\mathbf{T_\getrefnumber{deduction}}}\;\set{8}$
                \end{fitch}
            \end{case}

            \begin{case}
                \textsc{Caso 5.}
                Se a sentença $\alpha$ for o resultado da implicação de uma sentença $\varphi$ a uma sentença $\psi$, sabe-se que ${(\varphi\to\psi)}^\medsquare=\nec(\varphi^\medsquare\to\psi^\medsquare)$ pela definição da tradução.
                Deste modo, $\nec(\varphi^\medsquare\to\psi^\medsquare)\to\nec\nec(\varphi^\medsquare\to\psi^\medsquare)$ constitui um axioma gerado pelo esquema \hyperref[MB3]{$\mathbf{B_3}$} --- sendo assim $\entails\nec(\varphi^\medsquare\to\psi^\medsquare)\to\nec\nec(\varphi^\medsquare\to\psi^\medsquare)$ provado trivialmente.
            \end{case}
            \vspace{.5\baselineskip}
            Tendo-se provado todos os casos do passo de indução, podemos concluir que a propriedade vale, ou seja, que $\entails\alpha^\medsquare\to\nec\alpha^\medsquare$.
        \end{proof}
    \end{lemma}

    Uma vez provado o lema podemos, por fim, provar a correção da $\nec$-tradução.

    \begin{theorem}\label{square-correctness}
        Se $\Gamma\entails_\mathbf{I}\alpha$, então $\Gamma^\medsquare\entails_\mathbf{S4}\alpha^\medsquare$.
    \end{theorem}

    \begin{proof}
        Prova por indução forte sobre o tamanho da sucessão de dedução.
        Assim, suponhamos que a tradução seja correta para qualquer sucessão dedução de tamanho $n<k$.
        Demonstraremos, analisando-se os casos, que o a correção da tradução vale para sucessões de dedução de tamanho $n=k$.

        \begin{case}
            \textsc{Caso 1.}
            Se a linha derradeira da sucessão de dedução que prova $\Gamma\entails_\mathbf{I}\alpha$ tenha sido a evocação de alguma premissa, sabe-se que $\alpha\in\Gamma$ e, portanto, que $\alpha^\medsquare\in\Gamma^\medsquare$. Desde modo, pode-se demonstrar que $\Gamma^\medsquare\entails_\mathbf{S4}\alpha^\medsquare$ trivialmente pela evocação da premissa $\alpha^\medsquare$.
        \end{case}

        \begin{case}
            \textsc{Caso 2.}
            Se a linha derradeira da sucessão de dedução que prova $\Gamma\entails_\mathbf{I}\alpha$ tenha sido a evocação de algum axioma, sabe-se que existe algum esquema $\mathbf{A_\alpha}\in\mathcal{A}$ que gera $\alpha$. Deste modo, devemos demonstrar que para cada esquema $\mathbf{A}\in\mathcal{A}$, pode-se derivar $\Gamma^\medsquare\entails_\mathbf{S4}\mathbf{A}^\medsquare$. Nos casos abaixo, usaremos ocasionalmente a implicação estrita de modo a diminuir o espaço ocupado pelas provas.
        \end{case}

            \begin{subcase}
                \textsc{Caso 2.1.} Se a linha derradeira da sucessão de dedução que prova $\Gamma\entails_\mathbf{I}\alpha$ tenha sido a evocação de algum axioma gerado pelo esquema $\hyperref[IA1]{\mathbf{A_1}}$, sabemos que $\alpha=\varphi\to\psi\to\varphi$ e que $\alpha^\medsquare=\nec(\varphi^\medsquare\to\nec(\psi^\medsquare\to\varphi^\medsquare))$. Deste modo, podemos provar que $\Gamma^\medsquare\entails_\mathbf{S4}\nec(\varphi^\medsquare\to\nec(\psi^\medsquare\to\varphi^\medsquare))$ pela seguinte sucessão de dedução:

                \footnotesize
                \begin{fitch}
                    \fb\entails\varphi^\medsquare\to\nec\varphi^\medsquare&\refer{square-nec}{L}\\
                    \fa\entails\varphi^\medsquare\to\psi^\medsquare\to\varphi^\medsquare&\hyperref[MA1]{${\mathbf{A_1}}$}\\
                    \fa\entails\nec(\varphi^\medsquare\to\psi^\medsquare\to\varphi^\medsquare)&$\hyperref[necessitation]{\mathbf{R_3}}\;\set{2}$\\
                    \fa\entails\nec(\varphi^\medsquare\to\psi^\medsquare\to\varphi^\medsquare)\to\nec\varphi^\medsquare\to\nec(\psi^\medsquare\to\varphi^\medsquare)&\hyperref[MB1]{${\mathbf{B_1}}$}\\
                    \fa\entails\nec\varphi^\medsquare\to\nec(\psi^\medsquare\to\varphi^\medsquare)&$\hyperref[detachment]{\mathbf{R_2}}\;\set{3,4}$\\
                    \fa\entails(\varphi^\medsquare\to\nec\varphi^\medsquare)\to(\nec\varphi^\medsquare\to\nec(\psi^\medsquare\to\varphi^\medsquare))\to\varphi^\medsquare\to\nec(\psi^\medsquare\to\varphi^\medsquare)&\refer{comp}{L}\\
                    \fa\entails(\nec\varphi^\medsquare\to\nec(\psi^\medsquare\to\varphi^\medsquare))\to\varphi^\medsquare\to\nec(\psi^\medsquare\to\varphi^\medsquare)&$\hyperref[detachment]{\mathbf{R_2}}\;\set{1,6}$\\
                    \fa\entails\varphi^\medsquare\to\nec(\psi^\medsquare\to\varphi^\medsquare)&$\hyperref[detachment]{\mathbf{R_2}}\;\set{5,7}$\\
                    \fa\Gamma^\medsquare\entails\nec(\varphi^\medsquare\to\nec(\psi^\medsquare\to\varphi^\medsquare))&$\hyperref[necessitation]{\mathbf{R_3}}\;\set{8}$\\
                \end{fitch}
            \end{subcase}

            \begin{subcase}
                \textsc{Caso 2.2.} Se a linha derradeira da sucessão de dedução que prova $\Gamma\entails_\mathbf{I}\alpha$ tenha sido a evocação de algum axioma gerado pelo esquema $\hyperref[IA2]{\mathbf{A_2}}$, sabemos que $\alpha=(\varphi\to\psi\to\chi)\to(\varphi\to\psi)\to\varphi\to\chi$ e que $\alpha^\medsquare=(\varphi^\medsquare\strictif\psi^\medsquare\strictif\chi^\medsquare)\strictif(\varphi^\medsquare\strictif\psi^\medsquare)\strictif\varphi^\medsquare\strictif\chi^\medsquare$. Deste modo, podemos provar que $\Gamma^\medsquare\entails_\mathbf{S4}(\varphi^\medsquare\strictif\psi^\medsquare\strictif\chi^\medsquare)\strictif(\varphi^\medsquare\strictif\psi^\medsquare)\strictif\varphi^\medsquare\strictif\chi^\medsquare$ pela seguinte sucessão de dedução:

                \footnotesize
                \begin{fitch}
                    \fb\set{\varphi^\medsquare\strictif\psi^\medsquare\strictif\chi^\medsquare,\varphi^\medsquare\strictif\psi^\medsquare,\varphi^\medsquare}\entails\varphi^\medsquare&$\hyperref[premisse]{\mathbf{R_1}}$\\
                    \fa\set{\varphi^\medsquare\strictif\psi^\medsquare\strictif\chi^\medsquare,\varphi^\medsquare\strictif\psi^\medsquare,\varphi^\medsquare}\entails\varphi^\medsquare\strictif\psi^\medsquare&$\hyperref[premisse]{\mathbf{R_1}}$\\
                    \fa\set{\varphi^\medsquare\strictif\psi^\medsquare\strictif\chi^\medsquare,\varphi^\medsquare\strictif\psi^\medsquare,\varphi^\medsquare}\entails\psi^\medsquare&$\hyperref[strictsep]{\mathbf{T_\getrefnumber{strictsep}}}\;\set{1,2}$\\
                    \fa\set{\varphi^\medsquare\strictif\psi^\medsquare\strictif\chi^\medsquare,\varphi^\medsquare\strictif\psi^\medsquare,\varphi^\medsquare}\entails\varphi^\medsquare\strictif\psi^\medsquare\strictif\chi^\medsquare&$\hyperref[premisse]{\mathbf{R_1}}$\\
                    \fa\set{\varphi^\medsquare\strictif\psi^\medsquare\strictif\chi^\medsquare,\varphi^\medsquare\strictif\psi^\medsquare,\varphi^\medsquare}\entails\psi^\medsquare\strictif\chi^\medsquare&$\hyperref[strictsep]{\mathbf{T_\getrefnumber{strictsep}}}\;\set{1,4}$\\
                    \fa\set{\varphi^\medsquare\strictif\psi^\medsquare\strictif\chi^\medsquare,\varphi^\medsquare\strictif\psi^\medsquare,\varphi^\medsquare}\entails\chi^\medsquare&$\hyperref[strictsep]{\mathbf{T_\getrefnumber{strictsep}}}\;\set{3,5}$\\
                    \fa\set{\varphi^\medsquare\strictif\psi^\medsquare\strictif\chi^\medsquare,\varphi^\medsquare\strictif\psi^\medsquare}\entails\varphi^\medsquare\strictif\chi^\medsquare&$\hyperref[strictdeduction]{\mathbf{T_\getrefnumber{strictdeduction}}}\;\set{6}$\\
                    \fa\set{\varphi^\medsquare\strictif\psi^\medsquare\strictif\chi^\medsquare}\entails(\varphi^\medsquare\strictif\psi^\medsquare)\strictif\varphi^\medsquare\strictif\chi^\medsquare&$\hyperref[strictdeduction]{\mathbf{T_\getrefnumber{strictdeduction}}}\;\set{7}$\\
                    \fa\entails(\varphi^\medsquare\strictif\psi^\medsquare\strictif\chi^\medsquare)\to(\varphi^\medsquare\strictif\psi^\medsquare)\strictif\varphi^\medsquare\strictif\chi^\medsquare&$\hyperref[deduction]{\mathbf{T_\getrefnumber{deduction}}}\;\set{8}$\\
                    \fa\Gamma^\medsquare\entails(\varphi^\medsquare\strictif\psi^\medsquare\strictif\chi^\medsquare)\strictif(\varphi^\medsquare\strictif\psi^\medsquare)\strictif\varphi^\medsquare\strictif\chi^\medsquare&$\hyperref[necessitation]{\mathbf{R_3}}\;\set{9}$\\
                \end{fitch}
            \end{subcase}

            \begin{subcase}
                \textsc{Caso 2.3.} Se a linha derradeira da sucessão de dedução que prova $\Gamma\entails_\mathbf{I}\alpha$ tenha sido a evocação de algum axioma gerado pelo esquema $\hyperref[IA3]{\mathbf{A_3}}$, sabemos que $\alpha=\varphi\to\psi\to\varphi\wedge\psi$ e que $\alpha^\medsquare=\nec(\varphi^\medsquare\to\nec(\psi^\medsquare\to\varphi^\medsquare\wedge\psi^\medsquare))$. Deste modo, podemos provar que $\Gamma^\medsquare\entails_\mathbf{S4}\nec(\varphi^\medsquare\to\nec(\psi^\medsquare\to\varphi^\medsquare\wedge\psi^\medsquare))$ pela seguinte sucessão de dedução:

                \footnotesize
                \begin{fitch}
                    \fb\entails\varphi^\medsquare\to\nec\varphi^\medsquare&\refer{square-nec}{L}\\
                    \fa\entails\varphi^\medsquare\to\psi^\medsquare\to\varphi^\medsquare\wedge\psi^\medsquare&\hyperref[MA3]{${\mathbf{A_3}}$}\\
                    \fa\entails\nec(\varphi^\medsquare\to\psi^\medsquare\to\varphi^\medsquare\wedge\psi^\medsquare)&$\hyperref[necessitation]{\mathbf{R_3}}\;\set{2}$\\
                    \fa\entails\nec(\varphi^\medsquare\to\psi^\medsquare\to\varphi^\medsquare\wedge\psi^\medsquare)\to\nec\varphi^\medsquare\to\nec(\psi^\medsquare\to\varphi^\medsquare\wedge\psi^\medsquare)&\hyperref[MB1]{${\mathbf{B_1}}$}\\
                    \fa\entails\nec\varphi^\medsquare\to\nec(\psi^\medsquare\to\varphi^\medsquare\wedge\psi^\medsquare)&$\hyperref[detachment]{\mathbf{R_2}}\;\set{3,4}$\\
                    \fa\entails(\varphi^\medsquare\to\nec\varphi^\medsquare)\to(\nec\varphi^\medsquare\to\psi^\medsquare\strictif\varphi^\medsquare\wedge\psi^\medsquare)\to\varphi^\medsquare\to\psi^\medsquare\strictif\varphi^\medsquare\wedge\psi^\medsquare&\refer{comp}{L}\\
                    \fa\entails(\nec\varphi^\medsquare\to\nec(\psi^\medsquare\to\varphi^\medsquare\wedge\psi^\medsquare))\to\varphi^\medsquare\to\nec(\psi^\medsquare\to\varphi^\medsquare\wedge\psi^\medsquare)&$\hyperref[detachment]{\mathbf{R_2}}\;\set{1,6}$\\
                    \fa\entails\varphi^\medsquare\to\nec(\psi^\medsquare\to\varphi^\medsquare\wedge\psi^\medsquare)&$\hyperref[detachment]{\mathbf{R_2}}\;\set{5,7}$\\
                    \fa\Gamma^\medsquare\entails\nec(\varphi^\medsquare\to\nec(\psi^\medsquare\to\varphi^\medsquare\wedge\psi^\medsquare))&$\hyperref[necessitation]{\mathbf{R_3}}\;\set{8}$\\
                \end{fitch} 
            \end{subcase}

            \begin{subcase}
                \textsc{Caso 2.4.} Se a linha derradeira da sucessão de dedução que prova $\Gamma\entails_\mathbf{I}\alpha$ tenha sido a evocação de algum axioma gerado pelo esquema $\hyperref[IA4]{\mathbf{A_4}}$, sabemos que $\alpha=\varphi\wedge\psi\to\varphi$ e que $\alpha^\medsquare=\nec(\varphi^\medsquare\wedge\psi^\medsquare\to\varphi^\medsquare)$. Deste modo, podemos provar que $\Gamma^\medsquare\entails_\mathbf{S4}\nec(\varphi^\medsquare\wedge\psi^\medsquare\to\varphi^\medsquare)$ pela seguinte sucessão de dedução:

                \footnotesize
                \begin{fitch}
                    \fa\entails\varphi^\medsquare\wedge\psi^\medsquare\to\varphi^\medsquare&$\hyperref[MA4]{\mathbf{A_4}}$\\
                    \fa\Gamma^\medsquare\entails\nec(\varphi^\medsquare\wedge\psi^\medsquare\to\varphi^\medsquare)&$\hyperref[necessitation]{\mathbf{R_3}}\;\set{1}$
                \end{fitch}
            \end{subcase}

            \begin{subcase}
                \textsc{Caso 2.5.} Se a linha derradeira da sucessão de dedução que prova $\Gamma\entails_\mathbf{I}\alpha$ tenha sido a evocação de algum axioma gerado pelo esquema $\hyperref[IA5]{\mathbf{A_5}}$, sabemos que $\alpha=\varphi\wedge\psi\to\psi$ e que $\alpha^\medsquare=\nec(\varphi^\medsquare\wedge\psi^\medsquare\to\psi^\medsquare)$. Deste modo, podemos provar que $\Gamma^\medsquare\entails_\mathbf{S4}\nec(\varphi^\medsquare\wedge\psi^\medsquare\to\psi^\medsquare)$ pela seguinte sucessão de dedução:

                \footnotesize
                \begin{fitch}
                    \fb\entails\varphi^\medsquare\wedge\psi^\medsquare\to\psi^\medsquare&$\hyperref[MA5]{\mathbf{A_5}}$\\
                    \fa\Gamma^\medsquare\entails\nec(\varphi^\medsquare\wedge\psi^\medsquare\to\psi^\medsquare)&$\hyperref[necessitation]{\mathbf{R_3}}\;\set{1}$
                \end{fitch}
            \end{subcase}

            \begin{subcase}
                \textsc{Caso 2.6.} Se a linha derradeira da sucessão de dedução que prova $\Gamma\entails_\mathbf{I}\alpha$ tenha sido a evocação de algum axioma gerado pelo esquema $\hyperref[IA6]{\mathbf{A_6}}$, sabemos que $\alpha=\varphi\to\varphi\vee\psi$ e que $\alpha^\medsquare=\nec(\varphi^\medsquare\to\varphi^\medsquare\vee\psi^\medsquare)$. Deste modo, podemos provar que $\Gamma^\medsquare\entails_\mathbf{S4}\nec(\varphi^\medsquare\to\varphi^\medsquare\vee\psi^\medsquare)$ pela seguinte sucessão de dedução:

                \footnotesize
                \begin{fitch}
                    \fb\entails\varphi^\medsquare\to\varphi^\medsquare\vee\psi^\medsquare&$\hyperref[MA6]{\mathbf{A_6}}$\\
                    \fa\Gamma^\medsquare\entails\nec(\varphi^\medsquare\to\varphi^\medsquare\vee\psi^\medsquare)&$\hyperref[necessitation]{\mathbf{R_3}}\;\set{1}$
                \end{fitch}
            \end{subcase}

            \begin{subcase}
                \textsc{Caso 2.7.} Se a linha derradeira da sucessão de dedução que prova $\Gamma\entails_\mathbf{I}\alpha$ tenha sido a evocação de algum axioma gerado pelo esquema $\hyperref[IA7]{\mathbf{A_7}}$, sabemos que $\alpha=\psi\to\varphi\vee\psi$ e que $\alpha^\medsquare=\nec(\psi^\medsquare\to\varphi^\medsquare\vee\psi^\medsquare)$. Deste modo, podemos provar que $\Gamma^\medsquare\entails_\mathbf{S4}\nec(\psi^\medsquare\to\varphi^\medsquare\vee\psi^\medsquare)$ pela seguinte sucessão de dedução:

                \footnotesize
                \begin{fitch}
                    \fb\entails\psi^\medsquare\to\varphi^\medsquare\vee\psi^\medsquare&$\hyperref[MA7]{\mathbf{A_7}}$\\
                    \fa\Gamma^\medsquare\entails\nec(\psi^\medsquare\to\varphi^\medsquare\vee\psi^\medsquare)&$\hyperref[necessitation]{\mathbf{R_3}}\;\set{1}$
                \end{fitch}
            \end{subcase}

            \begin{subcase}
                \textsc{Caso 2.8.} Se a linha derradeira da sucessão de dedução que prova $\Gamma\entails_\mathbf{I}\alpha$ tenha sido a evocação de algum axioma gerado pelo esquema $\hyperref[IA8]{\mathbf{A_8}}$, sabemos que $\alpha=(\varphi\to\chi)\to(\psi\to\chi)\to\varphi\vee\psi\to\chi$ e que $\alpha^\medsquare=(\varphi^\medsquare\strictif\chi^\medsquare)\strictif(\psi^\medsquare\strictif\chi^\medsquare)\strictif\varphi^\medsquare\vee\psi^\medsquare\strictif\chi^\medsquare$. Deste modo, podemos provar que $\Gamma^\medsquare\entails_\mathbf{S4}(\varphi^\medsquare\strictif\chi^\medsquare)\strictif(\psi^\medsquare\strictif\chi^\medsquare)\strictif\varphi^\medsquare\vee\psi^\medsquare\strictif\chi^\medsquare$ pela seguinte sucessão de dedução:

                \footnotesize
                \begin{fitch}
                    \fb\set{\varphi^\medsquare\strictif\chi^\medsquare,\psi^\medsquare\strictif\chi^\medsquare,\varphi^\medsquare\vee\psi^\medsquare}\entails\varphi^\medsquare\strictif\chi^\medsquare&$\hyperref[premisse]{\mathbf{R_1}}$\\
                    \fa\set{\varphi^\medsquare\strictif\chi^\medsquare,\psi^\medsquare\strictif\chi^\medsquare,\varphi^\medsquare\vee\psi^\medsquare}\entails(\varphi^\medsquare\strictif\chi^\medsquare)\to\varphi^\medsquare\to\chi^\medsquare&\hyperref[MB2]{${\mathbf{B_2}}$}\\
                    \fa\set{\varphi^\medsquare\strictif\chi^\medsquare,\psi^\medsquare\strictif\chi^\medsquare,\varphi^\medsquare\vee\psi^\medsquare}\entails\varphi^\medsquare\to\chi^\medsquare&$\hyperref[detachment]{\mathbf{R_2}}\;\set{1,2}$\\
                    \fa\set{\varphi^\medsquare\strictif\chi^\medsquare,\psi^\medsquare\strictif\chi^\medsquare,\varphi^\medsquare\vee\psi^\medsquare}\entails\psi^\medsquare\strictif\chi^\medsquare&$\hyperref[premisse]{\mathbf{R_1}}$\\
                    \fa\set{\varphi^\medsquare\strictif\chi^\medsquare,\psi^\medsquare\strictif\chi^\medsquare,\varphi^\medsquare\vee\psi^\medsquare}\entails(\psi^\medsquare\strictif\chi^\medsquare)\to\psi^\medsquare\to\chi^\medsquare&\hyperref[MB2]{${\mathbf{B_2}}$}\\
                    \fa\set{\varphi^\medsquare\strictif\chi^\medsquare,\psi^\medsquare\strictif\chi^\medsquare,\varphi^\medsquare\vee\psi^\medsquare}\entails\psi^\medsquare\to\chi^\medsquare&$\hyperref[detachment]{\mathbf{R_2}}\;\set{4,5}$\\
                    \fa\set{\varphi^\medsquare\strictif\chi^\medsquare,\psi^\medsquare\strictif\chi^\medsquare,\varphi^\medsquare\vee\psi^\medsquare}\entails\varphi^\medsquare\vee\psi^\medsquare&$\hyperref[premisse]{\mathbf{R_1}}$\\
                    \fa\set{\varphi^\medsquare\strictif\chi^\medsquare,\psi^\medsquare\strictif\chi^\medsquare,\varphi^\medsquare\vee\psi^\medsquare}\entails(\varphi^\medsquare\to\chi^\medsquare)\to(\psi^\medsquare\to\chi^\medsquare)\to\varphi^\medsquare\vee\psi^\medsquare\to\chi^\medsquare&\hyperref[MA8]{${\mathbf{A_8}}$}\\
                    \fa\set{\varphi^\medsquare\strictif\chi^\medsquare,\psi^\medsquare\strictif\chi^\medsquare,\varphi^\medsquare\vee\psi^\medsquare}\entails(\psi^\medsquare\to\chi^\medsquare)\to\varphi^\medsquare\vee\psi^\medsquare\to\chi^\medsquare&$\hyperref[detachment]{\mathbf{R_2}}\;\set{3,8}$\\
                    \fa\set{\varphi^\medsquare\strictif\chi^\medsquare,\psi^\medsquare\strictif\chi^\medsquare,\varphi^\medsquare\vee\psi^\medsquare}\entails\varphi^\medsquare\vee\psi^\medsquare\to\chi^\medsquare&$\hyperref[detachment]{\mathbf{R_2}}\;\set{6,9}$\\
                    \fa\set{\varphi^\medsquare\strictif\chi^\medsquare,\psi^\medsquare\strictif\chi^\medsquare,\varphi^\medsquare\vee\psi^\medsquare}\entails\chi^\medsquare&$\hyperref[detachment]{\mathbf{R_2}}\;\set{7,10}$\\
                    \fa\set{\varphi^\medsquare\strictif\chi^\medsquare,\psi^\medsquare\strictif\chi^\medsquare}\entails\varphi^\medsquare\vee\psi^\medsquare\strictif\chi^\medsquare&$\hyperref[strictdeduction]{\mathbf{T_\getrefnumber{strictdeduction}}}\;\set{11}$\\
                    \fa\set{\varphi^\medsquare\strictif\chi^\medsquare}\entails(\psi^\medsquare\strictif\chi^\medsquare)\strictif\varphi^\medsquare\vee\psi^\medsquare\strictif\chi^\medsquare&$\hyperref[strictdeduction]{\mathbf{T_\getrefnumber{strictdeduction}}}\;\set{12}$\\
                    \fa\entails(\varphi^\medsquare\strictif\chi^\medsquare)\to(\psi^\medsquare\strictif\chi^\medsquare)\strictif\varphi^\medsquare\vee\psi^\medsquare\strictif\chi^\medsquare&$\hyperref[deduction]{\mathbf{T_\getrefnumber{deduction}}}\;\set{13}$\\
                    \fa\Gamma^\medsquare\entails(\varphi^\medsquare\strictif\chi^\medsquare)\strictif(\psi^\medsquare\strictif\chi^\medsquare)\strictif\varphi^\medsquare\vee\psi^\medsquare\strictif\chi^\medsquare&$\hyperref[necessitation]{\mathbf{R_3}}\;\set{14}$
                \end{fitch}
            \end{subcase}

            \begin{subcase}
                \textsc{Caso 2.9.} Se a linha derradeira da sucessão de dedução que prova $\Gamma\entails_\mathbf{I}\alpha$ tenha sido a evocação de algum axioma gerado pelo esquema $\mathbf{A_{\bot}}$, sabemos que $\alpha=\bot\to\varphi$ e que $\alpha^\medsquare=\nec(\bot\to\varphi^\medsquare)$. Deste modo, podemos provar que $\Gamma^\medsquare\entails_\mathbf{S4}\nec(\bot\to\varphi^\medsquare)$ pela seguinte sucessão de dedução:

                \footnotesize
                \begin{fitch}
                    \fb\entails\bot\to\varphi^\medsquare&\refer{explosion}{L}\\
                    \fa\Gamma^\medsquare\entails\nec(\bot\to\varphi^\medsquare)&$\hyperref[necessitation]{\mathbf{R_3}}\;\set{1}$
                \end{fitch}
            \end{subcase}

        \begin{case}
            \textsc{Caso 3.}
            Se a linha derradeira da sucessão de dedução que prova $\Gamma\entails_\mathbf{I}\alpha$ tenha sido gerada pela aplicação da regra do \emph{modus ponens} a duas sentenças $\varphi_i$ e $\varphi_j$ com $i<j<n$ pode-se assumir, sem perda de generalidade, que $\varphi_j=\varphi_i\to\alpha$.
            Assim, a partir de $\mathbf{H}$ temos que $\mathbf{H_1}=\Gamma^\medsquare\entails_\mathbf{S4}\varphi_i^\medsquare$ e que $\mathbf{H_2}=\Gamma^\medsquare\entails_\mathbf{S4}\nec(\varphi_i^\medsquare\to\alpha^\medsquare)$.
            Deste modo, podemos demonstrar que $\Gamma^\medsquare\entails_\mathbf{S4}\alpha^\medsquare$ pela seguinte sucessão de dedução:

            \footnotesize
            \begin{fitch}
                \fb\varphi_i^\medsquare&$\mathbf{H_2}$\\
                \fa\nec(\varphi_i^\medsquare\to\alpha^\medsquare)&$\mathbf{H_1}$\\
                \fa\nec(\varphi_i^\medsquare\to\alpha^\medsquare)\to\varphi_i^\medsquare\to\alpha^\medsquare&\hyperref[MB2]{${\mathbf{B_2}}$}\\
                \fa\varphi_i^\medsquare\to\alpha^\medsquare&$\hyperref[detachment]{\mathbf{R_2}}\;\set{2, 3}$\\
                \fa\alpha^\medsquare&$\hyperref[detachment]{\mathbf{R_2}}\;\set{1, 4}$
            \end{fitch}
        \end{case}
        \vspace{.5\baselineskip}
        Tendo-se provado todos os casos do passo de indução, podemos concluir que a correção da $\nec$-tradução, ou seja, que se $\Gamma\entails_\mathbf{I}\alpha$, então $\Gamma^\medsquare\entails_\mathbf{S4}\alpha^\medsquare$.
    \end{proof}
