\section{Correção}

    Neste sessão, apresentaremos uma prova de correção para a $\nec$-tradução. Antes disso, entretanto, apresentaremos uma prova de $\entails\alpha^\medsquare\to\nec\alpha^\medsquare$, que usaremos como lema para a prova da correção conforme sugerido por~\cite{Troelstra}. Provaremos o lema por indução sobre a profundidade da sentença e a correção por indução sobre o tamanho da prova.

    \begin{tcolorbox}[enhanced jigsaw, breakable, sharp corners, colframe=black, colback=white, boxrule=0.5pt, left=1.5mm, right=1.5mm, top=1.5mm, bottom=1.5mm]
    \begin{theorem}[Estabilidade]\label{square-nec}
        $\Gamma\entails_\mathfrak{L}\alpha^\medsquare\to\nec\alpha^\medsquare$.

        \begin{proof}
            Prova por indução forte sobre a profundidade da sentença \citep{Troelstra}.
            Seja $n\in\mathbb{N}^+$ a profundidade da sentença $\alpha\in\mathcal{L}$.
            Suponhamos que a asserção valha para qualquer sentença de profundidade menor que $n$ e nomeemos esta suposição $\mathbf{H}$.
            Devemos considerar cinco casos: a letra, a contradição, a conjunção, a disjunção e a implicação.

            \vspace{.5\baselineskip}
            \textsc{Caso 1.}
            Seja a sentença $\alpha$ for uma proposição $a\in\mathcal{P}$.
            Sabe-se que $a^\medsquare=\nec a$ pela definição da tradução.
            Deste modo, $\Gamma\vdash\nec{a}\to\nec\nec{a}$ pode ser gerado por \hyperref[modal.axiom.modal.3]{$\mathbf{B_3}$}.

            \vspace{.5\baselineskip}
            \textsc{Caso 2.}
            Seja a sentença $\alpha$ a constante $\bot$.
            Sabe-se que $\bot^\medsquare=\bot$ pela definição da tradução.
            Deste modo, $\Gamma\entails\bot\to\nec\bot$ foi provado pelo lema \hyperref[explosion]{$\mathbf{L_3}$}.

            \vspace{.5\baselineskip}
            \textsc{Caso 3.}
            Seja a sentença $\alpha$ a conjunção de duas sentenças $\alpha_1$ e $\alpha_2$.
            Sabe-se que ${(\alpha_1\wedge\alpha_2)}^\medsquare={\alpha_1}^\medsquare\wedge{\alpha_2}^\medsquare$ pela definição da tradução.
            A partir de $\mathbf{H}$, temos que $\Gamma\entails{\alpha_1}^\medsquare\to\nec{\alpha_1}^\medsquare$ e que $\Gamma\entails{\alpha_2}^\medsquare\to\nec{\alpha_2}^\medsquare$, ditos $\mathbf{H_1}$ e $\mathbf{H_2}$.
            Pode-se demonstrar $\Gamma\vdash{(\alpha_1\wedge\alpha_2)}^\medsquare\to\nec{(\alpha_1\wedge\alpha_2)}^\medsquare$ pela dedução que segue.

            \vspace{.5\baselineskip}
            \footnotesize
            \setlength{\rowskip}{.5\baselineskip}
            \begin{xltabular}{\textwidth}{r | X l l}
                \scriptsize{\phantom{1}1}\phantom{ } & $\ \Gamma\cup\set{{\alpha_1}^\medsquare\wedge{\alpha_2}^\medsquare}\entails{\alpha_1}^\medsquare\to\nec{\alpha_1}^\medsquare$                                                                 & $\mathbf{H_1}$                                  & \phantom{$\set{00,00}$}\\[\rowskip]
                \scriptsize{\phantom{1}2}\phantom{ } & $\ \Gamma\cup\set{{\alpha_1}^\medsquare\wedge{\alpha_2}^\medsquare}\entails{\alpha_2}^\medsquare\to\nec{\alpha_2}^\medsquare$                                                                 & $\mathbf{H_2}$                                  & \\[\rowskip]
                \scriptsize{\phantom{1}3}\phantom{ } & $\ \Gamma\cup\set{{\alpha_1}^\medsquare\wedge{\alpha_2}^\medsquare}\entails{\alpha_1}^\medsquare\wedge{\alpha_2}^\medsquare$                                                                  & $\hyperref[modal.rule.1]{\mathbf{R_1}}$         & \\[\rowskip]
                \scriptsize{\phantom{1}4}\phantom{ } & $\ \Gamma\cup\set{{\alpha_1}^\medsquare\wedge{\alpha_2}^\medsquare}\entails{\alpha_1}^\medsquare\wedge{\alpha_2}^\medsquare\to\nec{\alpha_1}^\medsquare\wedge\nec{\alpha_2}^\medsquare$       & \refer{conjunction.exchange}{L}                 & \\[\rowskip]
                \scriptsize{\phantom{1}5}\phantom{ } & $\ \Gamma\cup\set{{\alpha_1}^\medsquare\wedge{\alpha_2}^\medsquare}\entails\nec{\alpha_1}^\medsquare\wedge\nec{\alpha_2}^\medsquare$                                                          & $\hyperref[modal.rule.2]{\mathbf{R_2}}$         & $\set{3,4}$\\[\rowskip]
                \scriptsize{\phantom{1}6}\phantom{ } & $\ \Gamma\cup\set{{\alpha_1}^\medsquare\wedge{\alpha_2}^\medsquare}\entails\nec{\alpha_1}^\medsquare\wedge\nec{\alpha_2}^\medsquare\to\nec({\alpha_1}^\medsquare\wedge{\alpha_2}^\medsquare)$ & \refer{necessity.conjunction.undistribution}{L} &\\[\rowskip]
                \scriptsize{\phantom{1}7}\phantom{ } & $\ \Gamma\cup\set{{\alpha_1}^\medsquare\wedge{\alpha_2}^\medsquare}\entails\nec({\alpha_1}^\medsquare\wedge{\alpha_2}^\medsquare)$                                                            & $\hyperref[modal.rule.2]{\mathbf{R_2}}$         & $\set{5,6}$\\[\rowskip]
                \scriptsize{\phantom{1}8}\phantom{ } & $\ \Gamma\entails{\alpha_1}^\medsquare\wedge{\alpha_2}^\medsquare\to\nec({\alpha_1}^\medsquare\wedge{\alpha_2}^\medsquare)$                                                                   & \refer{deduction}{T} & $\set{7}$
            \end{xltabular}
            \normalsize

            \vspace{.5\baselineskip}
            \textsc{Caso 4.}
            Seja a sentença $\alpha$ a disjunção de duas sentenças $\alpha_1$ e $\alpha_2$.
            Sabe-se que ${(\alpha_1\vee\alpha_2)}^\medsquare={\alpha_1}^\medsquare\vee{\alpha_2}^\medsquare$ pela definição da tradução.
            A partir de $\mathbf{H}$, temos que $\Gamma\entails{\alpha_1}^\medsquare\to\nec{\alpha_1}^\medsquare$ e que $\Gamma\entails{\alpha_2}^\medsquare\to\nec{\alpha_2}^\medsquare$, ditos $\mathbf{H_1}$ e $\mathbf{H_2}$.
            Pode-se demonstrar $\Gamma\vdash{(\alpha_1\vee\alpha_2)}^\medsquare\to\nec{(\alpha_1\vee\alpha_2)}^\medsquare$ pela dedução que segue.

            \vspace{.5\baselineskip}
            \footnotesize
            \setlength{\rowskip}{.5\baselineskip}
            \begin{xltabular}{\textwidth}{r | X l l}
                \scriptsize{\phantom{1}1}\phantom{ } & $\ \Gamma\cup\set{{\alpha_1}^\medsquare\vee{\alpha_2}^\medsquare}\entails{\alpha_1}^\medsquare\to\nec{\alpha_1}^\medsquare$                                                                 & $\mathbf{H_1}$                                  & \phantom{$\set{00,00}$}\\[\rowskip]
                \scriptsize{\phantom{1}2}\phantom{ } & $\ \Gamma\cup\set{{\alpha_1}^\medsquare\vee{\alpha_2}^\medsquare}\entails{\alpha_2}^\medsquare\to\nec{\alpha_2}^\medsquare$                                                                 & $\mathbf{H_2}$                                  & \\[\rowskip]\pagebreak[4]
                \scriptsize{\phantom{1}3}\phantom{ } & $\ \Gamma\cup\set{{\alpha_1}^\medsquare\vee{\alpha_2}^\medsquare}\entails{\alpha_1}^\medsquare\vee{\alpha_2}^\medsquare$                                                                    & $\hyperref[modal.rule.1]{\mathbf{R_1}}$         & \\[\rowskip]
                \scriptsize{\phantom{1}4}\phantom{ } & $\ \Gamma\cup\set{{\alpha_1}^\medsquare\vee{\alpha_2}^\medsquare}\entails{\alpha_1}^\medsquare\vee{\alpha_2}^\medsquare\to\nec{\alpha_1}^\medsquare\vee\nec{\alpha_2}^\medsquare$           & \refer{disjunction.exchange}{L}                 & $\set{1, 2}$\\[\rowskip]
                \scriptsize{\phantom{1}5}\phantom{ } & $\ \Gamma\cup\set{{\alpha_1}^\medsquare\vee{\alpha_2}^\medsquare}\entails\nec{\alpha_1}^\medsquare\vee\nec{\alpha_2}^\medsquare$                                                            & $\hyperref[modal.rule.2]{\mathbf{R_2}}$         & $\set{3,4}$\\[\rowskip]
                \scriptsize{\phantom{1}6}\phantom{ } & $\ \Gamma\cup\set{{\alpha_1}^\medsquare\vee{\alpha_2}^\medsquare}\entails\nec{\alpha_1}^\medsquare\vee\nec{\alpha_2}^\medsquare\to\nec({\alpha_1}^\medsquare\vee{\alpha_2}^\medsquare)$     & \refer{necessity.conjunction.undistribution}{L} &\\[\rowskip]
                \scriptsize{\phantom{1}7}\phantom{ } & $\ \Gamma\cup\set{{\alpha_1}^\medsquare\vee{\alpha_2}^\medsquare}\entails\nec({\alpha_1}^\medsquare\vee{\alpha_2}^\medsquare)$                                                              & $\hyperref[modal.rule.2]{\mathbf{R_2}}$         & $\set{5,6}$\\[\rowskip]
                \scriptsize{\phantom{1}8}\phantom{ } & $\ \Gamma\entails{\alpha_1}^\medsquare\vee{\alpha_2}^\medsquare\to\nec({\alpha_1}^\medsquare\vee{\alpha_2}^\medsquare)$                                                                     & \refer{deduction}{T} & $\set{7}$
            \end{xltabular}
            \normalsize

            \vspace{.5\baselineskip}
            \textsc{Caso 5.}
            Seja a sentença $\alpha$ a implicação de duas sentenças $\alpha_1$ e $\alpha_2$.
            Sabe-se que ${(\alpha_1\to\alpha_2)}^\medsquare=\nec({\alpha_1}^\medsquare\to{\alpha_2}^\medsquare)$ pela definição da tradução.
            Deste modo, $\Gamma\vdash\nec({\alpha_1}^\medsquare\to{\alpha_2}^\medsquare)\to\nec\nec({\alpha_1}^\medsquare\to{\alpha_2}^\medsquare)$ pode ser gerado pela regra \hyperref[modal.axiom.modal.3]{$\mathbf{B_3}$}.

            \vspace{.5\baselineskip}
            Estando assim demonstrada a proposição.
        \end{proof}
    \end{theorem}
    \end{tcolorbox}

    Uma vez provado o lema podemos, por fim, provar a correção da $\nec$-tradução.

    \begin{tcolorbox}[enhanced jigsaw, breakable, sharp corners, colframe=black, colback=white, boxrule=0.5pt, left=1.5mm, right=1.5mm, top=1.5mm, bottom=1.5mm]
    \begin{theorem}\label{square.soundness}
        Se $\Gamma\entails_\mathfrak{B}\alpha$, então $\Gamma^\medsquare\entails_\mathfrak{L}\alpha^\medsquare$.
    \end{theorem}

    \begin{proof}
        Prova por indução forte sobre o tamanho da sucessão de dedução.
        Assim, suponhamos que a tradução seja correta para qualquer sucessão dedução de tamanho $n < k$.
        Demonstraremos, analisando-se os casos, que a correção da tradução vale para sucessões de dedução de tamanho $n = k$.

        \begin{case}
            \textsc{Caso 1.}
            Seja a linha derradeira da sucessão de dedução que prova $\Gamma\entails\alpha$ gerada pela invocação de alguma premissa.
            Sabe-se que $\alpha\in\Gamma$ e, portanto, que $\alpha^\medsquare\in\Gamma^\medsquare$.
            Desde modo, pode-se demonstrar que $\Gamma^\medsquare\entails\alpha^\medsquare$ trivialmente pela evocação da premissa $\alpha^\medsquare$.
        \end{case}

        \begin{case}
            \textsc{Caso 2.}
            Seja a linha derradeira da sucessão de dedução que prova $\Gamma\entails\alpha$ gerada pela invocação de algum axioma.
            Sabe-se que existe algum esquema $\mathbf{A}\in\mathcal{A}$ que gera $\alpha$. 
            Deste modo, analisaremos os casos e demonstraremos que se pode derivar $\Gamma^\medsquare\entails\alpha^\medsquare$ para cada esquema $\mathbf{A}\in\mathcal{A}$.
            Nos casos abaixo, usaremos ocasionalmente a implicação estrita de modo a diminuir o espaço ocupado pelas provas.
        \end{case}

            \begin{subcase}
                \textsc{Caso 2.1.} Se a linha derradeira da sucessão de dedução que prova $\Gamma\entails_\mathbf{I}\alpha$ tenha sido a evocação de algum axioma gerado pelo esquema $\hyperref[IA1]{\mathbf{A_1}}$, sabemos que $\alpha=\varphi\to\psi\to\varphi$ e que $\alpha^\medsquare=\nec(\varphi^\medsquare\to\nec(\psi^\medsquare\to\varphi^\medsquare))$. Deste modo, podemos provar que $\Gamma^\medsquare\entails_\mathbf{S4}\nec(\varphi^\medsquare\to\nec(\psi^\medsquare\to\varphi^\medsquare))$ pela seguinte sucessão de dedução:

                \footnotesize
                \begin{fitch}
                    \fb\entails\varphi^\medsquare\to\nec\varphi^\medsquare&\refer{square-nec}{L}\\
                    \fa\entails\varphi^\medsquare\to\psi^\medsquare\to\varphi^\medsquare&\hyperref[MA1]{${\mathbf{A_1}}$}\\
                    \fa\entails\nec(\varphi^\medsquare\to\psi^\medsquare\to\varphi^\medsquare)&$\hyperref[necessitation]{\mathbf{R_3}}\;\set{2}$\\
                    \fa\entails\nec(\varphi^\medsquare\to\psi^\medsquare\to\varphi^\medsquare)\to\nec\varphi^\medsquare\to\nec(\psi^\medsquare\to\varphi^\medsquare)&\hyperref[MB1]{${\mathbf{B_1}}$}\\
                    \fa\entails\nec\varphi^\medsquare\to\nec(\psi^\medsquare\to\varphi^\medsquare)&$\hyperref[detachment]{\mathbf{R_2}}\;\set{3,4}$\\
                    \fa\entails(\varphi^\medsquare\to\nec\varphi^\medsquare)\to(\nec\varphi^\medsquare\to\nec(\psi^\medsquare\to\varphi^\medsquare))\to\varphi^\medsquare\to\nec(\psi^\medsquare\to\varphi^\medsquare)&\refer{comp}{L}\\
                    \fa\entails(\nec\varphi^\medsquare\to\nec(\psi^\medsquare\to\varphi^\medsquare))\to\varphi^\medsquare\to\nec(\psi^\medsquare\to\varphi^\medsquare)&$\hyperref[detachment]{\mathbf{R_2}}\;\set{1,6}$\\
                    \fa\entails\varphi^\medsquare\to\nec(\psi^\medsquare\to\varphi^\medsquare)&$\hyperref[detachment]{\mathbf{R_2}}\;\set{5,7}$\\
                    \fa\Gamma^\medsquare\entails\nec(\varphi^\medsquare\to\nec(\psi^\medsquare\to\varphi^\medsquare))&$\hyperref[necessitation]{\mathbf{R_3}}\;\set{8}$\\
                \end{fitch}
            \end{subcase}

            \begin{subcase}
                \textsc{Caso 2.2.} Se a linha derradeira da sucessão de dedução que prova $\Gamma\entails_\mathbf{I}\alpha$ tenha sido a evocação de algum axioma gerado pelo esquema $\hyperref[IA2]{\mathbf{A_2}}$, sabemos que $\alpha=(\varphi\to\psi\to\chi)\to(\varphi\to\psi)\to\varphi\to\chi$ e que $\alpha^\medsquare=(\varphi^\medsquare\strictif\psi^\medsquare\strictif\chi^\medsquare)\strictif(\varphi^\medsquare\strictif\psi^\medsquare)\strictif\varphi^\medsquare\strictif\chi^\medsquare$. Deste modo, podemos provar que $\Gamma^\medsquare\entails_\mathbf{S4}(\varphi^\medsquare\strictif\psi^\medsquare\strictif\chi^\medsquare)\strictif(\varphi^\medsquare\strictif\psi^\medsquare)\strictif\varphi^\medsquare\strictif\chi^\medsquare$ pela seguinte sucessão de dedução:

                \footnotesize
                \begin{fitch}
                    \fb\set{\varphi^\medsquare\strictif\psi^\medsquare\strictif\chi^\medsquare,\varphi^\medsquare\strictif\psi^\medsquare,\varphi^\medsquare}\entails\varphi^\medsquare&$\hyperref[premisse]{\mathbf{R_1}}$\\
                    \fa\set{\varphi^\medsquare\strictif\psi^\medsquare\strictif\chi^\medsquare,\varphi^\medsquare\strictif\psi^\medsquare,\varphi^\medsquare}\entails\varphi^\medsquare\strictif\psi^\medsquare&$\hyperref[premisse]{\mathbf{R_1}}$\\
                    \fa\set{\varphi^\medsquare\strictif\psi^\medsquare\strictif\chi^\medsquare,\varphi^\medsquare\strictif\psi^\medsquare,\varphi^\medsquare}\entails\psi^\medsquare&$\hyperref[strictsep]{\mathbf{T_\getrefnumber{strictsep}}}\;\set{1,2}$\\
                    \fa\set{\varphi^\medsquare\strictif\psi^\medsquare\strictif\chi^\medsquare,\varphi^\medsquare\strictif\psi^\medsquare,\varphi^\medsquare}\entails\varphi^\medsquare\strictif\psi^\medsquare\strictif\chi^\medsquare&$\hyperref[premisse]{\mathbf{R_1}}$\\
                    \fa\set{\varphi^\medsquare\strictif\psi^\medsquare\strictif\chi^\medsquare,\varphi^\medsquare\strictif\psi^\medsquare,\varphi^\medsquare}\entails\psi^\medsquare\strictif\chi^\medsquare&$\hyperref[strictsep]{\mathbf{T_\getrefnumber{strictsep}}}\;\set{1,4}$\\
                    \fa\set{\varphi^\medsquare\strictif\psi^\medsquare\strictif\chi^\medsquare,\varphi^\medsquare\strictif\psi^\medsquare,\varphi^\medsquare}\entails\chi^\medsquare&$\hyperref[strictsep]{\mathbf{T_\getrefnumber{strictsep}}}\;\set{3,5}$\\
                    \fa\set{\varphi^\medsquare\strictif\psi^\medsquare\strictif\chi^\medsquare,\varphi^\medsquare\strictif\psi^\medsquare}\entails\varphi^\medsquare\strictif\chi^\medsquare&$\hyperref[strictdeduction]{\mathbf{T_\getrefnumber{strictdeduction}}}\;\set{6}$\\
                    \fa\set{\varphi^\medsquare\strictif\psi^\medsquare\strictif\chi^\medsquare}\entails(\varphi^\medsquare\strictif\psi^\medsquare)\strictif\varphi^\medsquare\strictif\chi^\medsquare&$\hyperref[strictdeduction]{\mathbf{T_\getrefnumber{strictdeduction}}}\;\set{7}$\\
                    \fa\entails(\varphi^\medsquare\strictif\psi^\medsquare\strictif\chi^\medsquare)\to(\varphi^\medsquare\strictif\psi^\medsquare)\strictif\varphi^\medsquare\strictif\chi^\medsquare&$\hyperref[deduction]{\mathbf{T_\getrefnumber{deduction}}}\;\set{8}$\\
                    \fa\Gamma^\medsquare\entails(\varphi^\medsquare\strictif\psi^\medsquare\strictif\chi^\medsquare)\strictif(\varphi^\medsquare\strictif\psi^\medsquare)\strictif\varphi^\medsquare\strictif\chi^\medsquare&$\hyperref[necessitation]{\mathbf{R_3}}\;\set{9}$\\
                \end{fitch}
            \end{subcase}

            \begin{subcase}
                \textsc{Caso 2.3.} Se a linha derradeira da sucessão de dedução que prova $\Gamma\entails_\mathbf{I}\alpha$ tenha sido a evocação de algum axioma gerado pelo esquema $\hyperref[IA3]{\mathbf{A_3}}$, sabemos que $\alpha=\varphi\to\psi\to\varphi\wedge\psi$ e que $\alpha^\medsquare=\nec(\varphi^\medsquare\to\nec(\psi^\medsquare\to\varphi^\medsquare\wedge\psi^\medsquare))$. Deste modo, podemos provar que $\Gamma^\medsquare\entails_\mathbf{S4}\nec(\varphi^\medsquare\to\nec(\psi^\medsquare\to\varphi^\medsquare\wedge\psi^\medsquare))$ pela seguinte sucessão de dedução:

                \footnotesize
                \begin{fitch}
                    \fb\entails\varphi^\medsquare\to\nec\varphi^\medsquare&\refer{square-nec}{L}\\
                    \fa\entails\varphi^\medsquare\to\psi^\medsquare\to\varphi^\medsquare\wedge\psi^\medsquare&\hyperref[MA3]{${\mathbf{A_3}}$}\\
                    \fa\entails\nec(\varphi^\medsquare\to\psi^\medsquare\to\varphi^\medsquare\wedge\psi^\medsquare)&$\hyperref[necessitation]{\mathbf{R_3}}\;\set{2}$\\
                    \fa\entails\nec(\varphi^\medsquare\to\psi^\medsquare\to\varphi^\medsquare\wedge\psi^\medsquare)\to\nec\varphi^\medsquare\to\nec(\psi^\medsquare\to\varphi^\medsquare\wedge\psi^\medsquare)&\hyperref[MB1]{${\mathbf{B_1}}$}\\
                    \fa\entails\nec\varphi^\medsquare\to\nec(\psi^\medsquare\to\varphi^\medsquare\wedge\psi^\medsquare)&$\hyperref[detachment]{\mathbf{R_2}}\;\set{3,4}$\\
                    \fa\entails(\varphi^\medsquare\to\nec\varphi^\medsquare)\to(\nec\varphi^\medsquare\to\psi^\medsquare\strictif\varphi^\medsquare\wedge\psi^\medsquare)\to\varphi^\medsquare\to\psi^\medsquare\strictif\varphi^\medsquare\wedge\psi^\medsquare&\refer{comp}{L}\\
                    \fa\entails(\nec\varphi^\medsquare\to\nec(\psi^\medsquare\to\varphi^\medsquare\wedge\psi^\medsquare))\to\varphi^\medsquare\to\nec(\psi^\medsquare\to\varphi^\medsquare\wedge\psi^\medsquare)&$\hyperref[detachment]{\mathbf{R_2}}\;\set{1,6}$\\
                    \fa\entails\varphi^\medsquare\to\nec(\psi^\medsquare\to\varphi^\medsquare\wedge\psi^\medsquare)&$\hyperref[detachment]{\mathbf{R_2}}\;\set{5,7}$\\
                    \fa\Gamma^\medsquare\entails\nec(\varphi^\medsquare\to\nec(\psi^\medsquare\to\varphi^\medsquare\wedge\psi^\medsquare))&$\hyperref[necessitation]{\mathbf{R_3}}\;\set{8}$\\
                \end{fitch} 
            \end{subcase}

            \begin{subcase}
                \textsc{Caso 2.4.} Se a linha derradeira da sucessão de dedução que prova $\Gamma\entails_\mathbf{I}\alpha$ tenha sido a evocação de algum axioma gerado pelo esquema $\hyperref[IA4]{\mathbf{A_4}}$, sabemos que $\alpha=\varphi\wedge\psi\to\varphi$ e que $\alpha^\medsquare=\nec(\varphi^\medsquare\wedge\psi^\medsquare\to\varphi^\medsquare)$. Deste modo, podemos provar que $\Gamma^\medsquare\entails_\mathbf{S4}\nec(\varphi^\medsquare\wedge\psi^\medsquare\to\varphi^\medsquare)$ pela seguinte sucessão de dedução:

                \footnotesize
                \begin{fitch}
                    \fa\entails\varphi^\medsquare\wedge\psi^\medsquare\to\varphi^\medsquare&$\hyperref[MA4]{\mathbf{A_4}}$\\
                    \fa\Gamma^\medsquare\entails\nec(\varphi^\medsquare\wedge\psi^\medsquare\to\varphi^\medsquare)&$\hyperref[necessitation]{\mathbf{R_3}}\;\set{1}$
                \end{fitch}
            \end{subcase}

            \begin{subcase}
                \textsc{Caso 2.5.} Se a linha derradeira da sucessão de dedução que prova $\Gamma\entails_\mathbf{I}\alpha$ tenha sido a evocação de algum axioma gerado pelo esquema $\hyperref[IA5]{\mathbf{A_5}}$, sabemos que $\alpha=\varphi\wedge\psi\to\psi$ e que $\alpha^\medsquare=\nec(\varphi^\medsquare\wedge\psi^\medsquare\to\psi^\medsquare)$. Deste modo, podemos provar que $\Gamma^\medsquare\entails_\mathbf{S4}\nec(\varphi^\medsquare\wedge\psi^\medsquare\to\psi^\medsquare)$ pela seguinte sucessão de dedução:

                \footnotesize
                \begin{fitch}
                    \fb\entails\varphi^\medsquare\wedge\psi^\medsquare\to\psi^\medsquare&$\hyperref[MA5]{\mathbf{A_5}}$\\
                    \fa\Gamma^\medsquare\entails\nec(\varphi^\medsquare\wedge\psi^\medsquare\to\psi^\medsquare)&$\hyperref[necessitation]{\mathbf{R_3}}\;\set{1}$
                \end{fitch}
            \end{subcase}

            \begin{subcase}
                \textsc{Caso 2.6.} Se a linha derradeira da sucessão de dedução que prova $\Gamma\entails_\mathbf{I}\alpha$ tenha sido a evocação de algum axioma gerado pelo esquema $\hyperref[IA6]{\mathbf{A_6}}$, sabemos que $\alpha=\varphi\to\varphi\vee\psi$ e que $\alpha^\medsquare=\nec(\varphi^\medsquare\to\varphi^\medsquare\vee\psi^\medsquare)$. Deste modo, podemos provar que $\Gamma^\medsquare\entails_\mathbf{S4}\nec(\varphi^\medsquare\to\varphi^\medsquare\vee\psi^\medsquare)$ pela seguinte sucessão de dedução:

                \footnotesize
                \begin{fitch}
                    \fb\entails\varphi^\medsquare\to\varphi^\medsquare\vee\psi^\medsquare&$\hyperref[MA6]{\mathbf{A_6}}$\\
                    \fa\Gamma^\medsquare\entails\nec(\varphi^\medsquare\to\varphi^\medsquare\vee\psi^\medsquare)&$\hyperref[necessitation]{\mathbf{R_3}}\;\set{1}$
                \end{fitch}
            \end{subcase}

            \begin{subcase}
                \textsc{Caso 2.7.} Se a linha derradeira da sucessão de dedução que prova $\Gamma\entails_\mathbf{I}\alpha$ tenha sido a evocação de algum axioma gerado pelo esquema $\hyperref[IA7]{\mathbf{A_7}}$, sabemos que $\alpha=\psi\to\varphi\vee\psi$ e que $\alpha^\medsquare=\nec(\psi^\medsquare\to\varphi^\medsquare\vee\psi^\medsquare)$. Deste modo, podemos provar que $\Gamma^\medsquare\entails_\mathbf{S4}\nec(\psi^\medsquare\to\varphi^\medsquare\vee\psi^\medsquare)$ pela seguinte sucessão de dedução:

                \footnotesize
                \begin{fitch}
                    \fb\entails\psi^\medsquare\to\varphi^\medsquare\vee\psi^\medsquare&$\hyperref[MA7]{\mathbf{A_7}}$\\
                    \fa\Gamma^\medsquare\entails\nec(\psi^\medsquare\to\varphi^\medsquare\vee\psi^\medsquare)&$\hyperref[necessitation]{\mathbf{R_3}}\;\set{1}$
                \end{fitch}
            \end{subcase}

            \begin{subcase}
                \textsc{Caso 2.8.} Se a linha derradeira da sucessão de dedução que prova $\Gamma\entails_\mathbf{I}\alpha$ tenha sido a evocação de algum axioma gerado pelo esquema $\hyperref[IA8]{\mathbf{A_8}}$, sabemos que $\alpha=(\varphi\to\chi)\to(\psi\to\chi)\to\varphi\vee\psi\to\chi$ e que $\alpha^\medsquare=(\varphi^\medsquare\strictif\chi^\medsquare)\strictif(\psi^\medsquare\strictif\chi^\medsquare)\strictif\varphi^\medsquare\vee\psi^\medsquare\strictif\chi^\medsquare$. Deste modo, podemos provar que $\Gamma^\medsquare\entails_\mathbf{S4}(\varphi^\medsquare\strictif\chi^\medsquare)\strictif(\psi^\medsquare\strictif\chi^\medsquare)\strictif\varphi^\medsquare\vee\psi^\medsquare\strictif\chi^\medsquare$ pela seguinte sucessão de dedução:

                \footnotesize
                \begin{fitch}
                    \fb\set{\varphi^\medsquare\strictif\chi^\medsquare,\psi^\medsquare\strictif\chi^\medsquare,\varphi^\medsquare\vee\psi^\medsquare}\entails\varphi^\medsquare\strictif\chi^\medsquare&$\hyperref[premisse]{\mathbf{R_1}}$\\
                    \fa\set{\varphi^\medsquare\strictif\chi^\medsquare,\psi^\medsquare\strictif\chi^\medsquare,\varphi^\medsquare\vee\psi^\medsquare}\entails(\varphi^\medsquare\strictif\chi^\medsquare)\to\varphi^\medsquare\to\chi^\medsquare&\hyperref[MB2]{${\mathbf{B_2}}$}\\
                    \fa\set{\varphi^\medsquare\strictif\chi^\medsquare,\psi^\medsquare\strictif\chi^\medsquare,\varphi^\medsquare\vee\psi^\medsquare}\entails\varphi^\medsquare\to\chi^\medsquare&$\hyperref[detachment]{\mathbf{R_2}}\;\set{1,2}$\\
                    \fa\set{\varphi^\medsquare\strictif\chi^\medsquare,\psi^\medsquare\strictif\chi^\medsquare,\varphi^\medsquare\vee\psi^\medsquare}\entails\psi^\medsquare\strictif\chi^\medsquare&$\hyperref[premisse]{\mathbf{R_1}}$\\
                    \fa\set{\varphi^\medsquare\strictif\chi^\medsquare,\psi^\medsquare\strictif\chi^\medsquare,\varphi^\medsquare\vee\psi^\medsquare}\entails(\psi^\medsquare\strictif\chi^\medsquare)\to\psi^\medsquare\to\chi^\medsquare&\hyperref[MB2]{${\mathbf{B_2}}$}\\
                    \fa\set{\varphi^\medsquare\strictif\chi^\medsquare,\psi^\medsquare\strictif\chi^\medsquare,\varphi^\medsquare\vee\psi^\medsquare}\entails\psi^\medsquare\to\chi^\medsquare&$\hyperref[detachment]{\mathbf{R_2}}\;\set{4,5}$\\
                    \fa\set{\varphi^\medsquare\strictif\chi^\medsquare,\psi^\medsquare\strictif\chi^\medsquare,\varphi^\medsquare\vee\psi^\medsquare}\entails\varphi^\medsquare\vee\psi^\medsquare&$\hyperref[premisse]{\mathbf{R_1}}$\\
                    \fa\set{\varphi^\medsquare\strictif\chi^\medsquare,\psi^\medsquare\strictif\chi^\medsquare,\varphi^\medsquare\vee\psi^\medsquare}\entails(\varphi^\medsquare\to\chi^\medsquare)\to(\psi^\medsquare\to\chi^\medsquare)\to\varphi^\medsquare\vee\psi^\medsquare\to\chi^\medsquare&\hyperref[MA8]{${\mathbf{A_8}}$}\\
                    \fa\set{\varphi^\medsquare\strictif\chi^\medsquare,\psi^\medsquare\strictif\chi^\medsquare,\varphi^\medsquare\vee\psi^\medsquare}\entails(\psi^\medsquare\to\chi^\medsquare)\to\varphi^\medsquare\vee\psi^\medsquare\to\chi^\medsquare&$\hyperref[detachment]{\mathbf{R_2}}\;\set{3,8}$\\
                    \fa\set{\varphi^\medsquare\strictif\chi^\medsquare,\psi^\medsquare\strictif\chi^\medsquare,\varphi^\medsquare\vee\psi^\medsquare}\entails\varphi^\medsquare\vee\psi^\medsquare\to\chi^\medsquare&$\hyperref[detachment]{\mathbf{R_2}}\;\set{6,9}$\\
                    \fa\set{\varphi^\medsquare\strictif\chi^\medsquare,\psi^\medsquare\strictif\chi^\medsquare,\varphi^\medsquare\vee\psi^\medsquare}\entails\chi^\medsquare&$\hyperref[detachment]{\mathbf{R_2}}\;\set{7,10}$\\
                    \fa\set{\varphi^\medsquare\strictif\chi^\medsquare,\psi^\medsquare\strictif\chi^\medsquare}\entails\varphi^\medsquare\vee\psi^\medsquare\strictif\chi^\medsquare&$\hyperref[strictdeduction]{\mathbf{T_\getrefnumber{strictdeduction}}}\;\set{11}$\\
                    \fa\set{\varphi^\medsquare\strictif\chi^\medsquare}\entails(\psi^\medsquare\strictif\chi^\medsquare)\strictif\varphi^\medsquare\vee\psi^\medsquare\strictif\chi^\medsquare&$\hyperref[strictdeduction]{\mathbf{T_\getrefnumber{strictdeduction}}}\;\set{12}$\\
                    \fa\entails(\varphi^\medsquare\strictif\chi^\medsquare)\to(\psi^\medsquare\strictif\chi^\medsquare)\strictif\varphi^\medsquare\vee\psi^\medsquare\strictif\chi^\medsquare&$\hyperref[deduction]{\mathbf{T_\getrefnumber{deduction}}}\;\set{13}$\\
                    \fa\Gamma^\medsquare\entails(\varphi^\medsquare\strictif\chi^\medsquare)\strictif(\psi^\medsquare\strictif\chi^\medsquare)\strictif\varphi^\medsquare\vee\psi^\medsquare\strictif\chi^\medsquare&$\hyperref[necessitation]{\mathbf{R_3}}\;\set{14}$
                \end{fitch}
            \end{subcase}

            \begin{subcase}
                \textsc{Caso 2.9.} Se a linha derradeira da sucessão de dedução que prova $\Gamma\entails_\mathbf{I}\alpha$ tenha sido a evocação de algum axioma gerado pelo esquema $\mathbf{A_{\bot}}$, sabemos que $\alpha=\bot\to\varphi$ e que $\alpha^\medsquare=\nec(\bot\to\varphi^\medsquare)$. Deste modo, podemos provar que $\Gamma^\medsquare\entails_\mathbf{S4}\nec(\bot\to\varphi^\medsquare)$ pela seguinte sucessão de dedução:

                \footnotesize
                \begin{fitch}
                    \fb\entails\bot\to\varphi^\medsquare&\refer{explosion}{L}\\
                    \fa\Gamma^\medsquare\entails\nec(\bot\to\varphi^\medsquare)&$\hyperref[necessitation]{\mathbf{R_3}}\;\set{1}$
                \end{fitch}
            \end{subcase}

        \begin{case}
            \textsc{Caso 3.}
            Se a linha derradeira da sucessão de dedução que prova $\Gamma\entails_\mathbf{I}\alpha$ tenha sido gerada pela aplicação da regra do \emph{modus ponens} a duas sentenças $\varphi_i$ e $\varphi_j$ com $i<j<n$ pode-se assumir, sem perda de generalidade, que $\varphi_j=\varphi_i\to\alpha$.
            Assim, a partir de $\mathbf{H}$ temos que $\mathbf{H_1}=\Gamma^\medsquare\entails_\mathbf{S4}\varphi_i^\medsquare$ e que $\mathbf{H_2}=\Gamma^\medsquare\entails_\mathbf{S4}\nec(\varphi_i^\medsquare\to\alpha^\medsquare)$.
            Deste modo, podemos demonstrar que $\Gamma^\medsquare\entails_\mathbf{S4}\alpha^\medsquare$ pela seguinte sucessão de dedução:

            \footnotesize
            \begin{fitch}
                \fb\varphi_i^\medsquare&$\mathbf{H_2}$\\
                \fa\nec(\varphi_i^\medsquare\to\alpha^\medsquare)&$\mathbf{H_1}$\\
                \fa\nec(\varphi_i^\medsquare\to\alpha^\medsquare)\to\varphi_i^\medsquare\to\alpha^\medsquare&\hyperref[MB2]{${\mathbf{B_2}}$}\\
                \fa\varphi_i^\medsquare\to\alpha^\medsquare&$\hyperref[detachment]{\mathbf{R_2}}\;\set{2, 3}$\\
                \fa\alpha^\medsquare&$\hyperref[detachment]{\mathbf{R_2}}\;\set{1, 4}$
            \end{fitch}
        \end{case}
        \vspace{.5\baselineskip}
        Estando assim demonstrada a proposição.
    \end{proof}
    \end{tcolorbox}

    \begin{tcolorbox}[enhanced jigsaw, breakable, sharp corners, colframe=black, colback=white, boxrule=0.5pt, left=1.5mm, right=1.5mm, top=1.5mm, bottom=1.5mm]
    \begin{theorem}[Correção]\label{circle.soundness}
        Se $\Gamma\entails_\mathfrak{B}\alpha$, então $\nec\Gamma^\circ\entails_\mathfrak{L}\alpha^\circ$.
    \end{theorem}
    \begin{proof}
        Partindo-se do teorema de correção da outra tradução \refer{square.soundness}{T}, temos que $\Gamma^\nec\entails\alpha^\nec$.
        Então, por meio do teorema da interderivabilidade \refer{interderivability}{T}, temos que $\nec\Gamma^\circ\entails\alpha^\circ$.
        Estando assim demonstrada a proposição.
    \end{proof}
    \end{tcolorbox}
