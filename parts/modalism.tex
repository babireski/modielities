\section{Sistemas modais}

Os sistemas modais surgem a partir da investigação acerca dos \emph{modos} em que uma proposição pode ser verdadeira, ou seja, das \emph{modalidades}.
As modalidades costumeiramente definidas são as noções duais de \emph{necessidade} e \emph{possibilidade}.
Assim, estão presentes nas linguagens destes sistemas sentenças da forma $\nec\alpha$ e da forma $\pos\alpha$, lidas \emph{necessariamente} $\alpha$ e \emph{possivelmente} $\alpha$.
Intuitivamente, uma necessidade deve ser verdade em todos os casos, enquanto uma possibilidade deve ser verdade em algum caso.
Em despeito dos seus nomes e de suas interpretações informais, estas modalidades podem ser usadas para modelar diferentes noções, como \emph{conhecimento} ou \emph{obrigação}.
Como veremos adiante, podemos relacionar a possibilidade como efeitos computationais, o que a torna interessante do ponto de vistada computação.

\vspace{0.5\baselineskip}
As primeiras investigações acerca das modalidades foram feitos na antiguidade.
Entretanto, os estudos modais modernos foram fundados por~\cite{Lewis}.
Lewis motivou-se a criar um sistema onde os condicionais tivessem uma interpretação mais perto daquela das linguagens naturais.
Neste sistema, não valeriam para todos os casos sentenças da forma $\alpha\to\beta\to\alpha$, por exemplo.
Estes estudos culminaram no trabalho conjunto de~\cite{Langford}.
Este trabalho definiu cinco sistemas --- nomeados de $\mathbf{S1}$ a $\mathbf{S5}$ --- que contavam com a dita \emph{implicação estrita} $\alpha\strictif\beta$ definida em termos da possibilidade e que não permitia a demonstração de sentenças consideradas indesejadas.

\vspace{0.5\baselineskip}
\cite{Goedel} apresentou uma definição equivalente do sistema $\mathbf{S4}$, tendo a necessidade como operador primitivo e a dedução estrita podendo ser definida em termos desta e da implicação, como $\nec(\alpha\to\beta)$.
Este sistema apresentou a vantagem de separar as regras proposicionais das regras modais, propriedade que não estava presente nas definições anteriores, e tornou-se o padrão a partir de então.
Nesta sessão, apresentaremos este sistema, doravante chamado $\mathfrak{M}=\langle\mathcal{L}_\nec,\vdash_\mathfrak{M}\rangle$.
A linguagem abaixo foi definida conforme~\cite{Troelstra}.
Foram definidas notações costumeiras, das quais destacamos a dedução estrita.
    Ainda, definiu-se uma notação para conjuntos com sentenças necessariamente verdadeiras.

\vspace{0.5\baselineskip}
\begin{tcolorbox}[enhanced jigsaw, breakable, sharp corners, colframe=black, colback=white, boxrule=0.5pt, left=1.5mm, right=1.5mm, top=1.5mm, bottom=1.5mm]
\begin{definition}[$\mathcal{L}_{\nec}$]\label{modal.language}
    A linguagem dos sistemas modais, denotada $\mathcal{L}_{\nec}$, pode ser induzida a partir da assinatura $\Sigma_{\nec}=\sequence{\mathcal{P},\mathcal{C}_{\nec}}$, onde $\mathcal{C}_{\nec}=\set{\bot^0,\nec^1,\wedge^2,\vee^2,\to^2}$.
\end{definition}
\end{tcolorbox}

\begin{tcolorbox}[enhanced jigsaw, breakable, sharp corners, colframe=black, colback=white, boxrule=0.5pt, left=1.5mm, right=1.5mm, top=1.5mm, bottom=1.5mm]
\begin{notation}
    Sejam duas sentenças $\alpha,\beta\in\mathcal{L}$.
    Denotamos por $\neg\alpha$ a negação $\alpha\to\bot$.
    Denotamos por $\pos\alpha$ a possibilidade $\neg\nec\neg\alpha$.
    Denotamos por $\alpha\strictif\beta$ a implicação estrita $\nec(\alpha\to\beta)$.
    E denotamos por $\alpha\leftrightarrow\beta$ a bi-implicação $(\alpha\to\beta)\wedge(\beta\to\alpha)$.
\end{notation}
\end{tcolorbox}

\begin{tcolorbox}[enhanced jigsaw, breakable, sharp corners, colframe=black, colback=white, boxrule=0.5pt, left=1.5mm, right=1.5mm, top=1.5mm, bottom=1.5mm]
\begin{notation}
    Seja $\Gamma\in\wp(\mathcal{L}_{\nec})$ um conjunto de sentenças bem-formadas.
    $\nec\Gamma$ denota o conjunto $\set{\nec\alpha\mid\alpha\in\Gamma}\in\wp(\mathcal{L}_{\nec})$, ou seja, a prefixação da necessitação a todos os elementos do conjunto $\Gamma$.
\end{notation}
\end{tcolorbox}

\vspace{0.5\baselineskip}
Consideremos agora a relação de dedução para o sistema $\mathfrak{M}$, definida por suas regras de dedução.
Nomeamos as regras de maneira com que as regras comuns ao sistema $\mathfrak{I}$ e ao sistema $\mathfrak{M}$ tenham nomes comuns, enquanto as regras diferentes tenham nomes diferentes, bem como as arranjamos de forma semelhante
A presença da regra $\mathbf{A_\neg}$ permite a derivação do \emph{tertium non datur}, em oposição ao sistema intuicionista.
Assim sendo, esta relação consiste no aumento das regras proposicionais com regras para a necessidade.
A definição que segue baseia-se tanto em~\cite{Troelstra} como em~\cite{Hakli}.

\vspace{0.5\baselineskip}
\begin{tcolorbox}[enhanced jigsaw, breakable, sharp corners, colframe=black, colback=white, boxrule=0.5pt, left=1.5mm, right=1.5mm, top=1.5mm, bottom=1.5mm]
\begin{definition}[$\vdash_{\mathfrak{M}}$]\label{modal.deduction}
    Abaixo estão definidas as regras do sistema $\mathfrak{M}$.
\vspace{.5\baselineskip}
\begin{center}
    \footnotesize
    \AxiomC{}
    \RightLabel{\footnotesize$\mathbf{A_1}$}
    \UnaryInfC{$\Gamma\vdash\alpha\to\beta\to\alpha$}
    \DisplayProof\label{modal.axiom.1}
    \quad
    \AxiomC{}
    \RightLabel{\footnotesize$\mathbf{A_2}$}
    \UnaryInfC{$\Gamma\vdash(\alpha\to\beta\to\gamma)\to(\alpha\to\beta)\to\alpha\to\gamma$}
    \DisplayProof\label{modal.axiom.2}
\end{center}

\begin{center}
    \footnotesize
    \AxiomC{}
    \RightLabel{\footnotesize$\mathbf{A_3}$}
    \UnaryInfC{$\Gamma\vdash\alpha\to\beta\to\alpha\wedge\beta$}
    \DisplayProof\label{modal.axiom.3}
    \quad
    \AxiomC{}
    \RightLabel{\footnotesize$\mathbf{A_4}$}
    \UnaryInfC{$\Gamma\vdash\alpha\wedge\beta\to\alpha$}
    \DisplayProof\label{modal.axiom.4}
    \quad
    \AxiomC{}
    \RightLabel{\footnotesize$\mathbf{A_5}$}
    \UnaryInfC{$\Gamma\vdash\alpha\wedge\beta\to\beta$}
    \DisplayProof\label{modal.axiom.5}
\end{center}

\begin{center}
    \footnotesize
    \AxiomC{}
    \RightLabel{\footnotesize$\mathbf{A_6}$}
    \UnaryInfC{$\Gamma\vdash\alpha\to\alpha\vee\beta$}
    \DisplayProof\label{modal.axiom.6}
    \quad
    \AxiomC{}
    \RightLabel{\footnotesize$\mathbf{A_7}$}
    \UnaryInfC{$\Gamma\vdash\beta\to\alpha\vee\beta$}
    \DisplayProof\label{modal.axiom.7}
    \quad
    \AxiomC{}
    \RightLabel{\footnotesize$\mathbf{A_8}$}
    \UnaryInfC{$\Gamma\vdash(\alpha\to\gamma)\to(\beta\to\gamma)\to\alpha\vee\beta\to\gamma$}
    \DisplayProof\label{modal.axiom.8}
\end{center}

\begin{center}
    \footnotesize
    \AxiomC{}
    \RightLabel{\footnotesize$\mathbf{A_\neg}$}
    \UnaryInfC{$\Gamma\vdash\neg\neg\alpha\to\alpha$}
    \DisplayProof\label{modal.axiom.negation}
\end{center}

\begin{center}
    \footnotesize
    \AxiomC{}
    \RightLabel{\footnotesize$\mathbf{B_1}$}
    \UnaryInfC{$\Gamma\vdash\Box(\alpha\to\beta)\to\Box\alpha\to\Box\beta$}
    \DisplayProof\label{modal.axiom.modal.1}
    \quad
    \AxiomC{}
    \RightLabel{\footnotesize$\mathbf{B_2}$}
    \UnaryInfC{$\Gamma\vdash\Box\alpha\to\alpha$}
    \DisplayProof\label{modal.axiom.modal.2}
    \quad
    \AxiomC{}
    \RightLabel{\footnotesize$\mathbf{B_3}$}
    \UnaryInfC{$\Gamma\vdash\Box\alpha\to\Box\Box\alpha$}
    \DisplayProof\label{modal.axiom.modal.3}
\end{center}

\begin{center}
    \footnotesize
    \AxiomC{$\alpha\in\Gamma$}
    \RightLabel{\footnotesize$\mathbf{R_1}$}
    \UnaryInfC{$\Gamma\vdash\alpha$}
    \DisplayProof\label{modal.rule.1}
    \quad
    \AxiomC{$\Gamma\vdash\alpha$}
    \AxiomC{$\Gamma\vdash\alpha\to\beta$}
    \RightLabel{\footnotesize$\mathbf{R_2}$}
    \BinaryInfC{$\Gamma\vdash\beta$}
    \DisplayProof\label{modal.rule.2}
    \quad
    \AxiomC{$\vdash\alpha$}
    \RightLabel{\footnotesize$\mathbf{R_3}$}
    \UnaryInfC{$\Gamma\vdash\Box\alpha$}
    \DisplayProof\label{modal.rule.3}
\end{center}
\end{definition}
\end{tcolorbox}

\vspace{0.5\baselineskip}
A definição da regra $\mathbf{R_3}$, dita \emph{regra da necessitação}, foi feita de maneira cuidadosa de modo a permitir a correta derivação do teorema da dedução.
Este teorema mostra-se importante para tornar as demonstações feitas neste trabalho mais breves e claras.
Para tanto, foi preciso restringir a condição antecedente da regra a conjuntos vazios.
Em outras palavras, esta regra permite tornar apenas teoremas necessidades.
Para uma discussão mais aprofudada acerca da validade do teorema da dedução em sistemas modais, remetemos o leitor a~\cite{Hakli}.
