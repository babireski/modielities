\begin{titlepage}
{\customword{Resumo}}

Alguns dos sistemas modais, sobretudo o $\mathbf{S4}$, possuem atributos interessantes para a representação de efeitos computacionais.
Uma linguagem baseada neste sistema, podendo representar efeitos monadicamente, pode ser imersa em continuações e vice-versa.
O estilo de passagem por continuações trata-se de uma das diversas representações usadas em compiladores, o que torna estes sistemas relevantes do ponto de vista da compilação.
O sistema intuicionista, do mesmo modo, possui interesses computacionais, uma vez que a noção intuicionista de \emph{construção} assemelha-se com a noção de \emph{computação}.
Este trabalho apresenta duas traduções do sistema intuicionista ao sistema modal $\mathbf{S4}$ em conjunto com as demonstrações das propriedades de interderivação e correção.
Todas as definições e asserções são implementadas e verificadas pela provador interativo de teoremas \emph{Rocq}.

\vspace{.5\baselineskip}
\textit{Palavras-chave} --- Traduções entre lógicas. Sistemas modais. Efeitos computacionais. Compilação. \emph{Rocq}.
\end{titlepage}