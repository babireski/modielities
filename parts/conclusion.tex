\chapter{Conclusão}
    Os sistemas modais possuem atributos interessantes para a representação de efeitos computacionais, sobretudo o sistema $\mathfrak{M}$ definido neste trabalho.
    Uma linguagem baseada nesse sistema, podendo representar efeitos monadicamente, imerge e pode ser imersa em continuações~\citep{Filinski.1994}.
    O estilo de passagem por continuações trata-se de uma das diversas representações usadas em compiladores.
    Deste modo, este sistema possui interesse no ponto de vista de compilação.
    Motivados por esta relação formalizamos duas traduções do sistema intuicionista $\mathfrak{I}$ ao sistema $\mathfrak{M}$ encontradas em~\cite{Troelstra+Schwichtenberg.2000} no provador de teoremas \emph{Rocq}.
    Esperamos que estes desenvolvimentos fundamentem investigações futuras acerca do uso de uma linguagem baseada no sistema $\mathfrak{M}$ durante a compilação.

    \vspace{.5\baselineskip}
    Apresentamos neste trabalho apresentou noções gerais sobre sistemas formais e sobre traduções entre tais sistemas.
    Com base nisso, foram definidos os sistemas intuicionista e modais, bem como duas traduções da linguagem do primeiro à linguagem do segundo.
    Foram apresentadas demonstrações da correção de ambas as traduções e de sua interderivabilidade.
    Para tanto, foram usados um conjunto de lemas e teoremas.
    Todas as demonstrações foram formalizadas no provador de teoremas \emph{Rocq}.
    Deste modo, cumpriram-se as metas definidas na introdução.

    \vspace{.5\baselineskip}
    Sugerimos diversos trabalhos futuros. Primeiramente, a demonstração da completude das traduções apresentadas neste trabalho e sua formalização assistida por computador.
    Em segundo lugar, sugerimos que o mesmo seja feito para a tradução do sistema laxo $\mathfrak{L}$ ao sistema $\mathfrak{M}$, conforme apresentada por~\cite{Pfenning+Davies.2001}. 
    Outra tradução de interesse trata-se da tradução de~\cite{Fairtlough+Mendler.1997} do sistema $\mathfrak{L}$ a um sistema bimodal $\langle\mathfrak{M},\mathfrak{M}\rangle$, uma vez que os desenvolvimentos de~\cite{Nunes+Roggia+Torrens.2025} permitem a fusões entre sistemas modais.
    Por fim, sugerimos investigações acerca do uso de uma de linguagem de representação baseada no sistema $\mathfrak{M}$ para uso em compiladores.

    \vspace{.5\baselineskip}
    Para as demonstrações de completude, sugerimos que estas deixem de se basear nas demonstrações de~\cite{Troelstra+Schwichtenberg.2000} e passem a se basear na demonstração de~\cite{Flagg+Friedman.1986}.
    Este abandono deve-se ao uso de propriedades de sequentes na demonstração, que deveriam ser acomodadas ao sistema de dedução usado neste trabalho.
    A sugestão desta demonstração, que se baseia na definição de uma contratradução a uma das traduções abordadas neste trabalho, acontece por esta ser feita construtivamente.
    Com isso, os autores reduzem o problema de demonstrar a completude da tradução ao problema de demonstrar a correção da contratradução, feito por indução sobre o tamanho da demonstração.

    \vspace{.5\baselineskip}
    Este trabalho foi apresentado como \emph{poster} durante a \emph{XXII Brazilian Logic Conference}, ocorrido entre os dias doze e dezesseis de maio do corrente ano em Serra Negra --- São Paulo, e foi parcialmente apoiado pela \emph{Fundação de Amparo à Pesquisa e Inovação do Estado de Santa Catarina} --- Fapesc.
