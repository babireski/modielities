\section{Correção}
    \begin{theorem}
        $\text{Se }\Gamma\entails_\mathbf{I}\alpha\text{, então }\Gamma^\medsquare\entails_\mathbf{4}\alpha^\medsquare$.
    \end{theorem}

    \begin{proof}
        Prova por indução forte sobre o tamanho da sucessão de dedução.
        Assim, suponhamos que a tradução seja correta para qualquer sucessão dedução de tamanho $n<k$.
        Demonstraremos, analisando-se os casos, que o a correção da tradução vale para sucessões de dedução de tamanho $n=k+1$.

        \begin{case}
            \textsc{Caso 1.}
            Se a linha derradeira da sucessão de dedução que prova $\Gamma\entails\alpha$ tenha sido a evocação de alguma premissa, sabe-se que $\alpha\in\Gamma$ e, portanto, que $\alpha^\medsquare\in\Gamma^\medsquare$. Desde modo, pode-se demonstrar que $\Gamma^\medsquare\entails\alpha^\medsquare$ trivialmente pela evocação da premissa $\alpha^\medsquare$.
        \end{case}

        \begin{case}
            \textsc{Caso 2.}
            Se a linha derradeira da sucessão de dedução que prova $\Gamma\entails\alpha$ tenha sido a evocação de algum axioma, sabe-se que existe algum esquema $\mathbf{A_\alpha}\in\mathcal{A}$ que gera $\alpha$. Deste modo, devemos demonstrar que para cada esquema $\mathbf{A}\in\mathcal{A}$, pode-se derivar $\Gamma^\medsquare\entails_\mathbf{4}\mathbf{A}^\medsquare$. Nos casos abaixo, usaremos ocasionalmente a implicação estrita de modo a diminuir o espaço ocupado pelas provas.
        \end{case}

            \begin{subcase}
                \textsc{Caso 2.1.} Se a linha derradeira da sucessão de dedução que prova $\Gamma\entails\alpha$ tenha sido a evocação de algum axioma gerado pelo esquema $\mathbf{A_1}$, sabemos que $\alpha=\alpha\to\beta\to\alpha$ e que $\alpha^\medsquare=\alpha\strictif\beta\strictif\alpha$. Deste modo, podemos provar que $\Gamma^\medsquare\entails\alpha\strictif\beta\strictif\alpha$ pela seguinte sucessão de dedução:

                \begin{fitch}
                    \fa\Gamma^\medsquare\cup\set{\nec{}a,\nec{}b}\entails\nec{}a&$\mathbf{P_1}$\\
                    \fa\Gamma^\medsquare\cup\set{\nec{}a}\entails\nec{}b\strictif\nec{}a&\refer{strictdeduction}{T}$\;\set{1}$\\
                    \fa\Gamma^\medsquare\entails\nec{}a\strictif\nec{}b\strictif\nec{}a&\refer{strictdeduction}{T}$\;\set{2}$.
                \end{fitch}
            \end{subcase}

            \begin{subcase}
                \textsc{Caso 2.2.} Se a linha derradeira da sucessão de dedução que prova $\Gamma\entails\alpha$ tenha sido a evocação de algum axioma gerado pelo esquema $\mathbf{A_2}$, sabemos que $\alpha=(\alpha\to\beta\to\gamma)\to(\alpha\to\beta)\to\alpha\to\gamma$ e que $\alpha^\medsquare=(\alpha\strictif\beta\strictif\gamma)\strictif(\alpha\strictif\beta)\strictif\alpha\strictif\gamma$. Deste modo, podemos provar que $\Gamma^\medsquare\entails(\alpha\strictif\beta\strictif\gamma)\strictif(\alpha\strictif\beta)\strictif\alpha\strictif\gamma$ pela seguinte sucessão de dedução:

                \begin{fitch}
                    \fa\Gamma^\medsquare\cup\set{\nec{a}\strictif\nec{b}\strictif\nec{c},\nec{a}\strictif\nec{b},\nec{a}}\entails\nec{}a\\
                    \fa\Gamma^\medsquare\cup\set{\nec{a}\strictif\nec{b}\strictif\nec{c},\nec{a}\strictif\nec{b},\nec{a}}\entails\nec{}a\strictif\nec{}b\\
                    \fa\Gamma^\medsquare\cup\set{\nec{a}\strictif\nec{b}\strictif\nec{c},\nec{a}\strictif\nec{b},\nec{a}}\entails\nec{}b\\
                    \fa\Gamma^\medsquare\cup\set{\nec{a}\strictif\nec{b}\strictif\nec{c},\nec{a}\strictif\nec{b},\nec{a}}\entails\nec{a}\strictif\nec{b}\strictif\nec{c}\\
                    \fa\Gamma^\medsquare\cup\set{\nec{a}\strictif\nec{b}\strictif\nec{c},\nec{a}\strictif\nec{b},\nec{a}}\entails\nec{b}\strictif\nec{c}\\
                    \fa\Gamma^\medsquare\cup\set{\nec{a}\strictif\nec{b}\strictif\nec{c},\nec{a}\strictif\nec{b},\nec{a}}\entails\nec{c}\\
                    \fa\Gamma^\medsquare\cup\set{\nec{a}\strictif\nec{b}\strictif\nec{c},\nec{a}\strictif\nec{b}}\entails\nec{a}\strictif\nec{c}\\
                    \fa\Gamma^\medsquare\cup\set{\nec{a}\strictif\nec{b}\strictif\nec{c}}\entails(\nec{a}\strictif\nec{b})\strictif\nec{a}\strictif\nec{c}\\
                    \fa\Gamma^\medsquare\entails(\nec{a}\strictif\nec{b}\strictif\nec{c})\strictif(\nec{a}\strictif\nec{b})\strictif\nec{a}\strictif\nec{c}\\

                \end{fitch}
            \end{subcase}

            \begin{subcase}
                \textsc{Caso 2.3.} Se a linha derradeira da sucessão de dedução que prova $\Gamma\entails\alpha$ tenha sido a evocação de algum axioma gerado pelo esquema $\mathbf{A_3}$, sabemos que $\alpha=\alpha\to\beta\to\alpha\wedge\beta$ e que $\alpha^\medsquare=\alpha\strictif\beta\strictif\alpha\wedge\beta$. Deste modo, podemos provar que $\Gamma^\medsquare\entails\alpha\strictif\beta\strictif\alpha\wedge\beta$ pela seguinte sucessão de dedução:

                \begin{fitch}
                    \fa\Gamma^\medsquare\cup\set{\nec{a},\nec{b}}\entails\nec{a}\\
                    \fa\Gamma^\medsquare\cup\set{\nec{a},\nec{b}}\entails\nec{b}\\
                    \fa\Gamma^\medsquare\cup\set{\nec{a},\nec{b}}\entails\nec{a}\to\nec{b}\to\nec{a}\wedge\nec{b}\\
                    \fa\Gamma^\medsquare\cup\set{\nec{a},\nec{b}}\entails\nec{b}\to\nec{a}\wedge\nec{b}\\
                    \fa\Gamma^\medsquare\cup\set{\nec{a},\nec{b}}\entails\nec{a}\wedge\nec{b}\\
                    \fa\Gamma^\medsquare\cup\set{\nec{a},\nec{b}}\entails\nec{a}\wedge\nec{b}\\
                    \fa\Gamma^\medsquare\cup\set{\nec{a}}\entails\nec{b}\strictif\nec{a}\wedge\nec{b}\\
                    \fa\Gamma^\medsquare\entails\nec{a}\strictif\nec{b}\strictif\nec{a}\wedge\nec{b}\\
                \end{fitch} 
            \end{subcase}

            \begin{subcase}
                \textsc{Caso 2.4.} Se a linha derradeira da sucessão de dedução que prova $\Gamma\entails\alpha$ tenha sido a evocação de algum axioma gerado pelo esquema $\mathbf{A_4}$, sabemos que $\alpha=\alpha\wedge\beta\to\alpha$ e que $\alpha^\medsquare=\nec(\alpha\wedge\beta\to\alpha)$. Deste modo, podemos provar que $\Gamma^\medsquare\entails\nec(\alpha\wedge\beta\to\alpha)$ pela seguinte sucessão de dedução:

                \begin{fitch}
                    \fa\Gamma^\medsquare\entails\alpha\wedge\beta\to\alpha&$\hyperref[MA4]{\mathbf{A_4}}$\\
                    \fa\Gamma^\medsquare\entails\nec(\alpha\wedge\beta\to\alpha)&$\hyperref[necessitation]{\mathbf{R_2}}\;\set{1}$.
                \end{fitch}
            \end{subcase}

            \begin{subcase}
                \textsc{Caso 2.5.} Se a linha derradeira da sucessão de dedução que prova $\Gamma\entails\alpha$ tenha sido a evocação de algum axioma gerado pelo esquema $\mathbf{A_5}$, sabemos que $\alpha=\alpha\wedge\beta\to\beta$ e que $\alpha^\medsquare=\nec(\alpha\wedge\beta\to\beta)$. Deste modo, podemos provar que $\Gamma^\medsquare\entails\nec(\alpha\wedge\beta\to\beta)$ pela seguinte sucessão de dedução:

                \begin{fitch}
                    \fa\Gamma^\medsquare\entails\alpha\wedge\beta\to\beta&$\hyperref[MA5]{\mathbf{A_5}}$\\
                    \fa\Gamma^\medsquare\entails\nec(\alpha\wedge\beta\to\beta)&$\hyperref[necessitation]{\mathbf{R_2}}\;\set{1}$.
                \end{fitch}
            \end{subcase}

            \begin{subcase}
                \textsc{Caso 2.6.} Se a linha derradeira da sucessão de dedução que prova $\Gamma\entails\alpha$ tenha sido a evocação de algum axioma gerado pelo esquema $\mathbf{A_6}$, sabemos que $\alpha=\alpha\to\alpha\vee\beta$ e que $\alpha^\medsquare=\nec(\alpha\to\alpha\vee\beta)$. Deste modo, podemos provar que $\Gamma^\medsquare\entails\nec(\alpha\to\alpha\vee\beta)$ pela seguinte sucessão de dedução:

                \begin{fitch}
                    \fa\Gamma^\medsquare\entails\alpha\to\alpha\vee\beta&$\hyperref[MA6]{\mathbf{A_6}}$\\
                    \fa\Gamma^\medsquare\entails\nec(\alpha\to\alpha\vee\beta)&$\hyperref[necessitation]{\mathbf{R_2}}\;\set{1}$.
                \end{fitch}
            \end{subcase}

            \begin{subcase}
                \textsc{Caso 2.7.} Se a linha derradeira da sucessão de dedução que prova $\Gamma\entails\alpha$ tenha sido a evocação de algum axioma gerado pelo esquema $\mathbf{A_7}$, sabemos que $\alpha=\beta\to\alpha\vee\beta$ e que $\alpha^\medsquare=\nec(\beta\to\alpha\vee\beta)$. Deste modo, podemos provar que $\Gamma^\medsquare\entails\nec(\beta\to\alpha\vee\beta)$ pela seguinte sucessão de dedução:

                \begin{fitch}
                    \fa\Gamma^\medsquare\entails\beta\to\alpha\vee\beta&$\hyperref[MA7]{\mathbf{A_7}}$\\
                    \fa\Gamma^\medsquare\entails\nec(\beta\to\alpha\vee\beta)&$\hyperref[necessitation]{\mathbf{R_2}}\;\set{1}$.
                \end{fitch}
            \end{subcase}

            \begin{subcase}
                \textsc{Caso 2.8.} ($\mathbf{A_8}$).
                
                \footnotesize
                \begin{fitch}
                    \fa\Gamma^\medsquare\cup\set{\nec{a}\strictif\nec{c},\nec{b}\strictif\nec{c},\nec{a}\vee\nec{b}}\entails\nec{a}\strictif\nec{c}\\
                    \fa\Gamma^\medsquare\cup\set{\nec{a}\strictif\nec{c},\nec{b}\strictif\nec{c},\nec{a}\vee\nec{b}}\entails(\nec{a}\strictif\nec{c})\to\nec{a}\to\nec{c}\\
                    \fa\Gamma^\medsquare\cup\set{\nec{a}\strictif\nec{c},\nec{b}\strictif\nec{c},\nec{a}\vee\nec{b}}\entails\nec{a}\to\nec{c}\\
                    \fa\Gamma^\medsquare\cup\set{\nec{a}\strictif\nec{c},\nec{b}\strictif\nec{c},\nec{a}\vee\nec{b}}\entails\nec{b}\strictif\nec{c}\\
                    \fa\Gamma^\medsquare\cup\set{\nec{a}\strictif\nec{c},\nec{b}\strictif\nec{c},\nec{a}\vee\nec{b}}\entails(\nec{b}\strictif\nec{c})\to\nec{b}\to\nec{c}\\
                    \fa\Gamma^\medsquare\cup\set{\nec{a}\strictif\nec{c},\nec{b}\strictif\nec{c},\nec{a}\vee\nec{b}}\entails\nec{b}\to\nec{c}\\
                    \fa\Gamma^\medsquare\cup\set{\nec{a}\strictif\nec{c},\nec{b}\strictif\nec{c},\nec{a}\vee\nec{b}}\entails\nec{a}\vee\nec{b}\\
                    \fa\Gamma^\medsquare\cup\set{\nec{a}\strictif\nec{c},\nec{b}\strictif\nec{c},\nec{a}\vee\nec{b}}\entails(\nec{a}\to\nec{c})\to(\nec{b}\to\nec{c})\to\nec{a}\vee\nec{b}\to\nec{c}\\
                    \fa\Gamma^\medsquare\cup\set{\nec{a}\strictif\nec{c},\nec{b}\strictif\nec{c},\nec{a}\vee\nec{b}}\entails(\nec{b}\to\nec{c})\to\nec{a}\vee\nec{b}\to\nec{c}\\
                    \fa\Gamma^\medsquare\cup\set{\nec{a}\strictif\nec{c},\nec{b}\strictif\nec{c},\nec{a}\vee\nec{b}}\entails\nec{a}\vee\nec{b}\to\nec{c}\\
                    \fa\Gamma^\medsquare\cup\set{\nec{a}\strictif\nec{c},\nec{b}\strictif\nec{c}}\entails\nec{a}\vee\nec{b}\strictif\nec{c}\\
                    \fa\Gamma^\medsquare\cup\set{\nec{a}\strictif\nec{c}}\entails(\nec{b}\strictif\nec{c})\strictif\nec{a}\vee\nec{b}\strictif\nec{c}\\
                    \fa\Gamma^\medsquare\entails(\nec{a}\strictif\nec{c})\strictif(\nec{b}\strictif\nec{c})\strictif\nec{a}\vee\nec{b}\strictif\nec{c}\\
                \end{fitch}
            \end{subcase}

            \begin{subcase}
                \textsc{Caso 2.9.} Se a linha derradeira da sucessão de dedução que prova $\Gamma\entails\alpha$ tenha sido a evocação de algum axioma gerado pelo esquema $\mathbf{A_{\bot}}$, sabemos que $\alpha=\bot\to\alpha$ e que $\alpha^\medsquare=\nec(\bot\to\alpha)$. Deste modo, podemos provar que $\Gamma^\medsquare\entails\nec(\bot\to\alpha)$ pela seguinte sucessão de dedução:

                \begin{fitch}
                    \fa\Gamma^\medsquare\entails\bot\to\alpha&\refer{explosion}{L}\\
                    \fa\Gamma^\medsquare\entails\nec(\bot\to\alpha)&$\hyperref[necessitation]{\mathbf{R_2}}\;\set{1}$.\\
                \end{fitch}
            \end{subcase}

        \begin{case}
            \textsc{Caso 3.}
            Deve-se demonstrar que, se $\entails\nec(\alpha^\medsquare\to\beta^\medsquare)$ ($\mathbf{H_1}$) e $\entails\alpha^\medsquare$ ($\mathbf{H_2}$), então $\beta^\medsquare$.
            Isso pode ser feito pela seguinte sucessão de dedução:

            \begin{fitch}
                \fa\nec(\alpha^\medsquare\to\beta^\medsquare)\to\alpha^\medsquare\to\beta^\medsquare&$\mathbf{B_2}$\\
                \fa\nec(\alpha^\medsquare\to\beta^\medsquare)&$\mathbf{H_1}$\\
                \fa\alpha^\medsquare\to\beta^\medsquare&$\mathbf{R_1}\;\sequence{1, 2}$\\
                \fa\alpha^\medsquare&$\mathbf{H_2}$\\
                \fa\beta^\medsquare&$\mathbf{R_1}\;\sequence{3, 4}$.
            \end{fitch}
        \end{case}
    \end{proof}