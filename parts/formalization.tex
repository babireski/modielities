\chapter{Formalização}
    \section{Sistemas}
        \begin{definition}[$\mathcal{L}_\mathbf{I}$]
            A linguagem do sistema intuicionista, denotada $\mathcal{L}_\mathbf{I}$, consiste no menor conjunto induzido a partir das seguintes regras:
            \begin{align*}
                &\bot\in\mathcal{L}_\mathbf{I} \\
                &\mathcal{P}\subseteq\mathcal{L}_\mathbf{I} \\
                &\alpha,\beta\in\mathcal{L}_\mathbf{I}\Rightarrow\alpha\circ\beta\in\mathcal{L}_\mathbf{I}\text{, para }\circ\in\set{\wedge,\vee,\to}\text{.}
                \tag*{\qed}
            \end{align*}
        \end{definition}

        \begin{notation}
            Serão usadas as seguintes abreviações:
            \begin{align*}
                \top&\coloneqq\bot\to\bot\\
                \neg\alpha&\coloneqq\alpha\to\bot\\
                \alpha\leftrightarrow\beta&\coloneqq(\alpha\to\beta)\wedge\beta\to\alpha
            \end{align*}
        \end{notation}

        \begin{definition}
            A axiomatização do sistema intuicionista consiste nos seguintes esquemas e regras:
            \begin{alignat*}{3}
                & \mathbf{A_1}\quad && \alpha\to\beta\to\alpha \\
                & \mathbf{A_2}\quad && (\alpha\to\beta\to\gamma)\to(\alpha\to\beta)\to(\alpha\to\gamma) \\
                & \mathbf{A_3}\quad && \alpha\to\beta\to\alpha\wedge\beta \\
                & \mathbf{A_4}\quad && \alpha\wedge\beta\to\alpha \\
                & \mathbf{A_5}\quad && \alpha\wedge\beta\to\beta \\
                & \mathbf{A_6}\quad && \alpha\to\alpha\vee\beta \\
                & \mathbf{A_7}\quad && \beta\to\alpha\vee\beta \\
                & \mathbf{A_8}\quad && (\alpha\to\gamma)\to(\beta\to\gamma)\to(\alpha\vee\beta\to\gamma) \\
                & \mathbf{A_\bot}\quad && \bot\to\alpha \\
                & \mathbf{R_1}\quad && \text{Se }\vdash\alpha\text{ e }\vdash\alpha\to\beta\text{, então }\vdash\beta\text{.} & \tag*{\qed}
            \end{alignat*}   
        \end{definition}

        \begin{definition}[$\mathcal{L}_\mathbf{M}$]
            A linguagem do sistemas modais, denotada $\mathcal{L}_\mathbf{M}$, consiste no menor conjunto induzido a partir das seguintes regras:
            \begin{align*}
                &\bot\in\mathcal{L}_\mathbf{M} \\
                &\mathcal{P}\subseteq\mathcal{L}_\mathbf{M} \\
                &\alpha\in\mathcal{L}_\mathbf{M}\Rightarrow\nec\alpha\in\mathcal{L}_\mathbf{M} \\
                &\alpha,\beta\in\mathcal{L}_\mathbf{M}\Rightarrow\alpha\circ\beta\in\mathcal{L}_\mathbf{M}\text{, para }\circ\in\set{\wedge,\vee,\to}\text{.}
                \tag*{\qed}
            \end{align*}
        \end{definition}

        \begin{notation}
            Serão usadas as seguintes abreviações:
            \begin{align*}
                \top&\coloneqq\bot\to\bot\\
                \neg\alpha&\coloneqq\alpha\to\bot\\
                \pos\alpha&\coloneqq\neg\nec\neg\alpha\\
                \alpha\leftrightarrow\beta&\coloneqq(\alpha\to\beta)\wedge\beta\to\alpha
            \end{align*}
        \end{notation}

        \begin{definition}
            A axiomatização do sistema modal consiste nos seguintes esquemas e regras:
            \begin{alignat*}{3}
                & \mathbf{A_1}\quad && \alpha\to\beta\to\alpha \\
                & \mathbf{A_2}\quad && (\alpha\to\beta\to\gamma)\to(\alpha\to\beta)\to(\alpha\to\gamma) \\
                % & \mathbf{A_3}\quad && (\neg\alpha\to\neg\beta)\to\alpha\to\beta \\
                & \mathbf{A_3}\quad && \alpha\to\beta\to\alpha\wedge\beta \\
                & \mathbf{A_4}\quad && \alpha\wedge\beta\to\alpha \\
                & \mathbf{A_5}\quad && \alpha\wedge\beta\to\beta \\
                & \mathbf{A_6}\quad && \alpha\to\alpha\vee\beta \\
                & \mathbf{A_7}\quad && \beta\to\alpha\vee\beta \\
                & \mathbf{A_8}\quad && (\alpha\to\gamma)\to(\beta\to\gamma)\to(\alpha\vee\beta\to\gamma) \\
                & \mathbf{A_\neg}\quad && \neg\neg\alpha\to\alpha \\
                & \mathbf{B_1}\quad && \nec(\alpha\to\beta)\to\nec\alpha\to\nec\beta \\
                % & \mathbf{B_B}\quad && \pos(\alpha\vee\beta)\to\pos\alpha\vee\pos\beta \\
                & \mathbf{B_2}\quad && \nec\alpha\to\alpha \\
                & \mathbf{B_3}\quad && \nec\alpha\to\nec\nec\alpha \\
                & \mathbf{R_1}\quad && \text{Se }\vdash\alpha\text{ e }\vdash\alpha\to\beta\text{, então }\vdash\beta \\
                & \mathbf{R_2}\quad && \text{Se }\vdash\alpha\text{, então }\vdash\nec\alpha\text{.} & \tag*{\qed} 
            \end{alignat*}   
        \end{definition}

        Provaremos, para o sistema modal apresentado, uma variação do teorema da dedução, que permite simplificar muitas das demonstrações apresentadas futuramente neste trabalho.
        \babireski{\citet{Hakli} apresentam uma discussão sobre a validade do metateorema da dedução e suas variantes nos sistemas modais. Quem sabe valha a pena escrever um pouco sobre isso, pois se trata de uma leitura interessante.}

        \begin{theorem}
            $\forall\,\Gamma\cup\set{\alpha,\beta}\in\wp(\mathcal{L}_\mathbf{M})\point\Gamma\cup\set{\alpha}\vdash\beta\Rightarrow  \Gamma\vdash\nec(\alpha\to\beta)$.
        \end{theorem}

        \begin{proof}
            Seja $\Gamma$ um conjunto de sentenças e sejam $\alpha$ e $\beta$ sentenças de modo que $\Gamma\cup\set{\alpha}\vdash\beta$, deve-se provar que $\Gamma\vdash\nec(\alpha\to\beta)$. Como $\Gamma\cup\set{\alpha}\vdash\beta$, existe uma prova de $\beta$ a partir de $\Gamma\cup\set{\alpha}$. A prova baseia-se numa indução sobre o tamanho $n$ da prova.

            \begin{case}
                \textbf{Caso 1} (Base)\textbf{.}
                Para a base requer-se considerar os seguintes casos: 
                    \textbf{(1)} $\beta$ consiste num axioma,
                    \textbf{(2)} $\beta\in\Gamma$ e
                    \textbf{(3)} $\beta=\alpha$.

                \begin{case}
                    \textbf{Caso 1.1} ($\beta\in\mathcal{A}$)\textbf{.}

                    \begin{fitch}
                        \fa\beta\to\alpha\to\beta&$\mathbf{A_1}$\\
                        \fa\beta&$\mathbf{A_\beta}$\\
                        \fa\alpha\to\beta&$\mathbf{R_1}\;\sequence{1, 2}$\\
                        \fa\nec(\alpha\to\beta)&$\mathbf{R_2}\;\sequence{3}$
                    \end{fitch}
                \end{case}

                \begin{case}
                    \textbf{Caso 1.2} ($\beta\in\Gamma$)\textbf{.}

                    \begin{fitch}
                        \fa\beta\to\alpha\to\beta&$\mathbf{A_1}$\\
                        \fa\beta&$\mathbf{P}$\\
                        \fa\alpha\to\beta&$\mathbf{R_1}\;\sequence{1, 2}$\\
                        \fa\nec(\alpha\to\beta)&$\mathbf{R_2}\;\sequence{3}$
                    \end{fitch}
                \end{case}

                \begin{case}
                    \textbf{Caso 1.3} ($\beta=\alpha$)\textbf{.}
                    
                    \begin{fitch}
                        \fa\alpha\to\alpha&$\mathbf{L_1}$\\
                        \fa\nec(\alpha\to\alpha)&$\mathbf{R_2}\;\sequence{1}$
                    \end{fitch}
                \end{case}
            \end{case}

            \begin{case}
                \textbf{Caso 2} (Passo)\textbf{.}
                Supõe-se que, para qualquer prova de $\Gamma\cup\set{\alpha}\vdash\beta$ com tamanho $k$, tem-se que $\Gamma\vdash\nec(\alpha\to\beta)$.
                Deve-se mostrar que a proposição segue verdadeira caso a prova tenha tamanho $k+1$. 
                Assim, sendo $\sequence{\varphi_i\mid1\leq i\leq k+1}$ uma sucessão de dedução com $\varphi_{k+1}=\beta$, requer-se considerar os seguintes casos:

                \begin{case}
                    \textbf{Caso 2.1} ($\beta\in\mathcal{A}$)\textbf{.} \textit{Vide} caso $\mathbf{C_{1.1}}$.
                \end{case}

                \begin{case}
                    \textbf{Caso 2.2} ($\beta\in\Gamma$)\textbf{.} \textit{Vide} caso $\mathbf{C_{1.2}}$.
                \end{case}

                \begin{case}
                    \textbf{Caso 2.3} ($\beta=\alpha$)\textbf{.} \textit{Vide} caso $\mathbf{C_{1.3}}$.
                \end{case}

                \begin{case}
                    \textbf{Caso 2.4} ($\mathbf{R_1}$)\textbf{.}
                \end{case}

                \begin{case}
                    \textbf{Caso 2.5} ($\mathbf{R_2}$)\textbf{.}
                \end{case}
            \end{case}
        \end{proof}

    \section{Traduções}

        A primeira tradução do sistema intuicionista ao sistema modal foi proposta por Gödel \cite{Goedel} motivado pela possibilidade de leitura da necessidade como uma modalidade de construtividade. Ou seja, por meio dessa tradução, a sentença $\nec \varphi$ poderia ser lida como \textit{$\varphi$ pode ser provada construtivamente} \cite{Troelstra}. Gödel conjeiturou a corretude fraca dessa tradução, que foi posteriormente provada por McKinsey e Tarski \cite{McKinsey} em conjunto com sua completude fraca.

        \begin{definition}[$\bullet^\circ$] Define-se a tradução $\bullet^\circ$ indutivamente da seguinte maneira:
            \begin{align*}
                p^\circ                     & \coloneqq p                                       \\
                \bot^\circ                  & \coloneqq \bot                                    \\
                (\varphi \wedge \psi)^\circ & \coloneqq \varphi^\circ \wedge \psi^\circ         \\
                (\varphi \vee \psi)^\circ   & \coloneqq \nec \varphi^\circ \vee \nec \psi^\circ \\
                (\varphi \to \psi)^\circ    & \coloneqq \nec \varphi^\circ \to \psi^\circ
                \tag*{\qed} 
            \end{align*}
        \end{definition}
        
        \begin{definition}[$\bullet^\nec$] Define-se a tradução $\bullet^\nec$ indutivamente da seguinte maneira:
            \begin{align*}
                p^\nec                     & \coloneqq \nec p                                        \\
                \bot^\nec                  & \coloneqq \bot                                          \\
                (\varphi \wedge \psi)^\nec & \coloneqq \varphi^\nec \wedge \psi^\nec     \\
                (\varphi \vee \psi)^\nec   & \coloneqq \varphi^\nec \vee \psi^\nec       \\
                (\varphi \to \psi)^\nec    & \coloneqq \nec (\varphi^\nec \to \psi^\nec)
                \tag*{\qed} 
            \end{align*}
        \end{definition}
        
        Faz-se interessante pontuar que as traduções $\bullet^\circ$ e $\bullet^\nec$ correspondem, respectivamente, às traduções $\bullet^\circ$ e $\bullet^*$ do sistema intuicionista ao sistema linear providas por Girard \cite{Girard}, sendo as primeiras correspondentes a uma ordem de avaliação por nome (\textit{call-by-name}) e as segundas a uma ordem de avaliação por valor (\textit{call-by-value}). 
        Ademais, as duas traduções providas são equivalentes, conforme demonstrado pelo teorema $\mathbf{T_2}$.

        \begin{theorem}
            $\forall\alpha\in\mathcal{L}_\mathbf{I}\point\nec\alpha^\circ\leftrightarrow\alpha^\nec$.
        \end{theorem}

        \begin{proof}
            Prova por indução na profundidade de $\alpha$.

            \begin{case}
                \textbf{Caso 1} (Base)\textbf{.}
                    Para $|\alpha| = 0$, existem dois casos a serem considerados.

                    \begin{case}
                    \textbf{Caso 1.1} ($\alpha = a$)\textbf{.}
                        $a^\circ = a$ e $a^\nec = \nec a$, assim $\nec a^\circ = a^\nec$ e, portanto, $\nec a^\circ \leftrightarrow a^\nec$.
                    \end{case}
                    \begin{case}
                    \textbf{Caso 2.1} ($\alpha = \bot$)\textbf{.}
                        $\bot^\circ = \bot$ e $\bot^\nec = \bot$. A ida $\nec\bot\to\bot$ consiste em um axioma, sendo, portando provada trivialmente pela sucessão de dedução $\sequence{\nec\bot\to\bot}$.
                        A volta $\bot\to\nec\bot$ equivale a provar, por meio do teorema da dedução, que $\set{\bot}\vdash_\mathbf{M}\nec\bot$, o que pode ser provado trivialmente pela sucessão de dedução $\sequence{\bot, \nec\bot}$, que consiste na invocação da premissa e aplicação da regra da necessitação, nessa ordem.
                    \end{case}
            \end{case}

            \begin{case}
                \textbf{Caso 2} (Passo)\textbf{.} No passo, deve-se demonstrar que, caso $\nec\alpha^\circ\leftrightarrow\alpha^\nec$ para $|\alpha| = n$, 
                então $\nec\alpha^\circ\leftrightarrow\alpha^\nec$ para $|\alpha| = n + 1$, onde $n \in \mathbb{N}$. Assim, seja $\nec\alpha^\circ\leftrightarrow\alpha^\nec$ uma proposição verdadeira para $|\alpha| = k$, onde $k \in \mathbb{N}$. Existem os seguintes casos a serem considerados para $|\alpha| = k + 1$.

                \begin{case}
                    \textbf{Caso 2.1} ($\alpha = \alpha_1\wedge\alpha_2$)\textbf{.}
                \end{case}

                \begin{case}
                    \textbf{Caso 2.2} ($\alpha = \alpha_1\vee\alpha_2$)\textbf{.}
                \end{case}

                \begin{case}
                    \textbf{Caso 2.3} ($\alpha = \alpha_1\to\alpha_2$)\textbf{.}
                \end{case}
            \end{case}
        \end{proof}

    \section{Corretude}

        \begin{theorem}
            $\forall \alpha \in \mathcal{L}_\mathbf{I} \point \Gamma \vdash_\mathbf{I} \alpha \Rightarrow \Gamma^\nec \vdash_\mathbf{M} \alpha^\nec$
        \end{theorem}

        \begin{proof}
            Como $\Gamma \vdash_\mathbf{I} \alpha$, sabe-se que existe uma prova $\sequence{\varphi_i\mid 1 \leq i \leq n}$ tal que $\varphi_n = \alpha$. A demonstração deste teorema será feita por indução no tamanho $n$ da prova.

            \begin{case}
                \textbf{Passo} ($n = 1$)\textbf{.} A prova, caso possua tamanho $n = 1$, tem obrigatoriamente a forma $\sequence{\alpha}$. Deste modo, existem duas casos a serem considerados: $\alpha$ ser um axioma ou $\alpha$ ser uma premissa.
            \end{case}

                \begin{casee}
                    \textbf{Caso 1} ($\alpha\in\Gamma$)\textbf{.} Como $\alpha\in\Gamma$, sabe-se que $\alpha^\nec\in\Gamma^\nec$, uma vez que $\Gamma^\nec = \set{\varphi^\nec\mid\varphi\in\Gamma}$. Desta forma, $\sequence{\alpha^\nec}$ constitui uma prova para $\Gamma^\nec\vdash\alpha^\nec$.
                \end{casee}

                \begin{casee}
                    \textbf{Caso 2} ($\alpha\in\mathcal{A}$)\textbf{.}
                \end{casee}

                    \begin{caseee}
                        \textbf{Caso 2.1} ($\mathbf{A_1}$)\textbf{.}
                        Deve-se demonstrar que $\vdash\nec(\alpha^\nec\to\nec(\beta^\nec\to\alpha^\nec))$.
                        Pelo teorema $\mathbf{T_1}$, basta provar que $\set{\alpha^\nec,\beta^\nec}\vdash\alpha^\nec$, o que pode ser feito pela seguinte sucessão de dedução:

                        \begin{fitch}
                            \fa\alpha^\nec&$\mathbf{P}$
                        \end{fitch}
                    \end{caseee}

                    \begin{caseee}
                        \textbf{Caso 2.2} ($\mathbf{A_2}$)\textbf{.}

                        \begin{fitch}
                            \fa\nec(\alpha^\nec\to\beta^\nec)\to\alpha^\nec\to\beta^\nec\\
                            \fa\nec(\alpha^\nec\to\beta^\nec)\\
                            \fa\alpha^\nec\to\beta^\nec\\
                            \fa\alpha^\nec\\
                            \fa\beta^\nec\\
                            \fa\nec(\alpha^\nec\to\nec(\beta^\nec\to\alpha^\nec))\to\alpha^\nec\to\nec(\beta^\nec\to\alpha^\nec)\\
                            \fa\nec(\alpha^\nec\to\nec(\beta^\nec\to\alpha^\nec))\\
                            \fa\alpha^\nec\to\nec(\beta^\nec\to\alpha^\nec)\\
                            \fa\nec(\beta^\nec\to\alpha^\nec)\\
                            \fa\nec(\beta^\nec\to\alpha^\nec)\to\beta^\nec\to\gamma^\nec\\
                            \fa\beta^\nec\to\gamma^\nec\\
                            \fa\gamma^\nec
                        \end{fitch}
                    \end{caseee}

                    \begin{caseee}
                        \textbf{Caso 2.3} ($\mathbf{A_3}$)\textbf{.}

                        \begin{fitch}
                            \fa\alpha^\nec\to\beta^\nec\to\alpha^\nec\wedge\beta^\nec&$\mathbf{A_3}$\\
                            \fa\alpha^\nec&$\mathbf{P}$\\
                            \fa\beta^\nec\to\alpha^\nec\wedge\beta^\nec&$\mathbf{R_1}\;\sequence{1,2}$\\
                            \fa\beta^\nec&$\mathbf{P}$\\
                            \fa\alpha^\nec\wedge\beta^\nec&$\mathbf{R_1}\;\sequence{3,4}$
                        \end{fitch} 
                    \end{caseee}

                    \begin{caseee}
                        \textbf{Caso 2.4} ($\mathbf{A_4}$)\textbf{.}

                        \begin{fitch}
                            \fa \alpha^\nec\wedge\beta^\nec\to\alpha^\nec & $\mathbf{A_4}$ \\
                            \fa \nec(\alpha^\nec \wedge \beta^\nec \to \alpha^\nec) & $\mathbf{R_2}\;\sequence{1}$
                        \end{fitch}
                    \end{caseee}

                    \begin{caseee}
                        \textbf{Caso 2.5} ($\mathbf{A_5}$)\textbf{.}

                        \begin{fitch}
                            \fa \alpha^\nec\wedge\beta^\nec\to\beta^\nec & $\mathbf{A_5}$ \\
                            \fa \nec(\alpha^\nec \wedge \beta^\nec \to \beta^\nec) & $\mathbf{R_2}\;\sequence{1}$
                        \end{fitch}
                    \end{caseee}

                    \begin{caseee}
                        \textbf{Caso 2.6} ($\mathbf{A_6}$)\textbf{.}

                        \begin{fitch}
                            \fa \alpha^\nec\to\alpha^\nec\vee\beta^\nec & $\mathbf{A_6}$ \\
                            \fa \nec(\alpha^\nec\to\alpha^\nec\vee\beta^\nec) & $\mathbf{R_2}\;\sequence{1}$
                        \end{fitch}
                    \end{caseee}

                    \begin{caseee}
                        \textbf{Caso 2.7} ($\mathbf{A_7}$)\textbf{.}

                        \begin{fitch}
                            \fa \beta^\nec\to\alpha^\nec\vee\beta^\nec & $\mathbf{A_7}$ \\
                            \fa \nec(\beta^\nec\to\alpha^\nec\vee\beta^\nec) & $\mathbf{R_2}\;\sequence{1}$
                        \end{fitch}
                    \end{caseee}

                    \begin{caseee}
                        \textbf{Caso 2.8} ($\mathbf{A_8}$)\textbf{.}

                        \begin{fitch}
                            \fa(\alpha^\nec\to\gamma^\nec)\to(\beta^\nec\to\gamma^\nec)\to\alpha^\nec\vee\beta^\nec\to\gamma^\nec\\
                            \fa\nec(\alpha^\nec\to\gamma^\nec)\to\alpha^\nec\to\gamma^\nec\\
                            \fa\nec(\alpha^\nec\to\gamma^\nec)\\
                            \fa\alpha^\nec\to\gamma^\nec\\
                            \fa(\beta^\nec\to\gamma^\nec)\to\alpha^\nec\vee\beta^\nec\to\gamma^\nec\\
                            \fa\nec(\beta^\nec\to\gamma^\nec)\to\beta^\nec\to\gamma^\nec\\
                            \fa\nec(\beta^\nec\to\gamma^\nec)\\
                            \fa\beta^\nec\to\gamma^\nec\\
                            \fa\alpha^\nec\vee\beta^\nec\to\gamma^\nec\\
                            \fa\alpha^\nec\vee\beta^\nec\\
                            \fa\gamma^\nec
                        \end{fitch}
                    \end{caseee}

                    \begin{caseee}
                        \textbf{Caso 2.9} ($\mathbf{A_\bot}$)\textbf{.}
                        Deve-se demonstrar que $\vdash\nec(\bot\to\alpha^\nec)$.
                        Pelo teorema $\mathbf{T_1}$, basta provar que $\set{\bot}\vdash\alpha^\nec$, o que pode ser feito pela seguinte sucessão de dedução:

                        \begin{fitch}
                            \fa((\alpha^\nec\to\bot)\to\bot)\to\alpha^\nec&$\mathbf{A_\neg}$\\
                            \fa\bot\to(\alpha^\nec\to\bot)\to\bot&$\mathbf{A_1}$\\
                            \fa\bot&$\mathbf{P}$\\
                            \fa(\alpha^\nec\to\bot)\to\bot&$\mathbf{R_1}\;\sequence{2, 3}$\\
                            \fa\alpha^\nec&$\mathbf{R_1}\;\sequence{1, 4}$.
                        \end{fitch}
                    \end{caseee}

                    \begin{caseee}
                        \textbf{Caso 2.9} ($\mathbf{R_1}$)\textbf{.}
                        Deve-se demonstrar que, se $\vdash\nec(\alpha^\nec\to\beta^\nec)$ ($\mathbf{H_1}$) e $\vdash\alpha^\nec$ ($\mathbf{H_2}$), então $\beta^\nec$.
                        Isso pode ser feito pela seguinte sucessão de dedução:

                        \begin{fitch}
                            \fa\nec(\alpha^\nec\to\beta^\nec)\to\alpha^\nec\to\beta^\nec&$\mathbf{B_2}$\\
                            \fa\nec(\alpha^\nec\to\beta^\nec)&$\mathbf{H_1}$\\
                            \fa\alpha^\nec\to\beta^\nec&$\mathbf{R_1}\;\sequence{1, 2}$\\
                            \fa\alpha^\nec&$\mathbf{H_2}$\\
                            \fa\beta^\nec&$\mathbf{R_1}\;\sequence{3, 4}$.
                        \end{fitch}
                    \end{caseee}
        \end{proof}

    \section{Completude}
