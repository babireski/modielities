\section{Dualidades}

    \babireski{Ver~\cite{Zach} acerca dos axiomas duais e suas derivações.}

    \begin{theorem}
        $\vdash\nec(\alpha\to\beta)\to\pos\alpha\to\pos\beta$.

        \begin{proof}
            Pode ser provado pela seguinte sucessão de dedução:
            \begin{fitch}
                \fa\set{\nec(\alpha\to\beta),\pos\alpha}\vdash\pos\beta\\
                \fa\set{\nec(\alpha\to\beta)}\vdash\pos\alpha\to\pos\beta\\
                \fa\vdash\nec(\alpha\to\beta)\to\pos\alpha\to\pos\beta\\
            \end{fitch}
        \end{proof}
    \end{theorem}

    \begin{theorem}
        $\vdash\alpha\to\pos\alpha$.
        \begin{proof}
            Pode ser provado pela seguinte sucessão de dedução:

            \begin{fitch}
                \fa(\nec\neg\alpha\to\neg\alpha)\to\neg\neg\alpha\to\neg\nec\neg\alpha\\
                \fa\nec\neg\alpha\to\neg\alpha\\
                \fa\neg\neg\alpha\to\neg\nec\neg\alpha{}
            \end{fitch}
        \end{proof}
    \end{theorem}

    \begin{theorem}
        $\vdash\pos\pos\alpha\to\pos\alpha$.
        \begin{proof}
            Isso equivale, por reescrita, a demonstrar $\vdash\neg\nec\neg\neg\nec\neg\alpha\to\neg\nec\neg\alpha$, o que pode ser feito pela seguinte sucessão de dedução:

            \begin{fitch}
                \fa(\nec\neg\alpha\to\neg\alpha)\to\neg\neg\alpha\to\neg\nec\neg\alpha\\
                \fa\nec\neg\alpha\to\neg\alpha\\
                \fa\neg\neg\alpha\to\neg\nec\neg\alpha{}
            \end{fitch}
        \end{proof}
    \end{theorem}