\section{Dualidades}

    \babireski{Ver~\cite{Zach} acerca dos axiomas duais e suas derivações.}

    % \begin{theorem}
    %     $\vdash\nec(\alpha\to\beta)\to\pos\alpha\to\pos\beta$.

    %     \begin{proof}
    %         Pode ser provado pela seguinte sucessão de dedução:

    %         \begin{fitch}
    %             \fa\set{\nec(\alpha\to\beta),\pos\alpha}\vdash\pos\beta\\
    %             \fa\set{\nec(\alpha\to\beta)}\vdash\pos\alpha\to\pos\beta\\
    %             \fa\vdash\nec(\alpha\to\beta)\to\pos\alpha\to\pos\beta\\
    %         \end{fitch}
    %         \vspace*{-18pt-0.7em}
    %         \qedhere
    %     \end{proof}
    % \end{theorem}

    \begin{theorem}
        $\vdash\alpha\to\pos\alpha$.
        \begin{proof}
            Pode ser provado pela seguinte sucessão de dedução:

            \begin{fitch}
                \fa\entails\nec\neg\alpha\to\neg\alpha&$\hyperref[MB2]{\mathbf{B_2}}$\\
                \fa\entails(\nec\neg\alpha\to\neg\alpha)\to\neg\neg\alpha\to\pos\alpha&$\hyperref[contrapositive]{\mathbf{L_3}}$\\
                \fa\entails\neg\neg\alpha\to\pos\alpha&$\hyperref[detachment]{\mathbf{R_1}}\;\set{1,2}$\\
                \fa\entails(\neg\neg\alpha\to\neg\nec\neg\alpha)\to\alpha\to(\neg\neg\alpha\to\pos\alpha)&$\hyperref[MA1]{\mathbf{A_1}}$\\
                \fa\entails\alpha\to\neg\neg\alpha\to\pos\alpha&$\hyperref[detachment]{\mathbf{R_1}}\;\set{3,4}$\\
                \fa\entails(\alpha\to\neg\neg\alpha\to\pos\alpha)\to(\alpha\to\neg\neg\alpha)\to(\alpha\to\pos\alpha)&$\hyperref[MA2]{\mathbf{A_2}}$\\
                \fa\entails\alpha\to\neg\neg\alpha&\babireski{Provar.}\\
                \fa\entails(\alpha\to\neg\neg\alpha)\to(\alpha\to\pos\alpha)&$\hyperref[detachment]{\mathbf{R_1}}\;\set{5,6}$\\
                \fa\entails\alpha\to\pos\alpha&$\hyperref[detachment]{\mathbf{R_1}}\;\set{7,8}$.
            \end{fitch}
            \vspace*{-18pt-0.7em}
            \qedhere
        \end{proof}
    \end{theorem}

    \begin{theorem}
        $\vdash\pos\pos\alpha\to\pos\alpha$.
        \begin{proof}
            Pode ser provado pela seguinte sucessão de dedução:

            \begin{fitch}
                
                \fa\entails\neg\nec\alpha\to\neg\nec\pos\alpha\\
                \fa\entails\pos\pos\alpha\to\pos\alpha\\
            \end{fitch}
            \vspace*{-18pt-0.7em}
            \qedhere
        \end{proof}
    \end{theorem}

    Apesar da similaridades com as transformações naturais, deve-se destacar que as noções de computação não podem ser interpretadas simplesmente como necessidade ou possibilidade, uma vez que apresenta propriedades presente em ambas as modalidades. Neste sentido, a modalidade de \emph{laxidade} --- que combina noções de necessidade e possibilidade --- mostra-se uma melhor representação de efeitos computacionais sobre a interpretação programa-prova.
    
    Ao sistema que comporta a modalidade de laxidade damos o nome de sistema laxo. Este sistema consiste numa extensão do sistema intuicionista com a adição da modalidade $\lax^1$ definida por~\cite{Fairtlough,Mendler}. A sentença $\lax\alpha$ --- lida como \emph{laxamente} $\alpha$ --- pode ser axiomatizada pelos seguintes esquemas:
    \begin{alignat*}{3}
        &\mathbf{C_1}\quad&&\alpha\to\lax\alpha\\
        &\mathbf{C_2}\quad&&\lax\lax\alpha\to\lax\alpha\\
        &\mathbf{C_3}\quad&&(\alpha\to\beta)\to\lax\alpha\to\lax\beta
    \end{alignat*}

    Esse sistema, entretanto, pode ser interpretado modalmente por meio da seguinte tradução \citep{Pfenning}:

    \begin{definition}[$\bullet^+$] A tradução $\bullet^+:\mathcal{L}_\mathbf{M}\to\mathcal{L}_\mathbf{L}$ do sistema $\mathbf{S_4}$ intuicionista ao sistema $\mathbf{L}$ pode ser definida indutivamente da seguinte maneira:
        \begin{align*}
            a^+&\coloneq a\\
            \bot^+&\coloneq\bot\\
            {(\lax\alpha)}^+&\coloneq\pos\nec\alpha^+\\
            {(\alpha\to\beta)}^+&\coloneq\nec\alpha^+\to\beta^+
            \tag*{\qed} 
        \end{align*}
    \end{definition}