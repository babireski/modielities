\section{Corretude}
    \begin{theorem}
        $\text{Se }\Gamma\vdash_\mathbf{I}\alpha\text{, então }\Gamma^\nec\vdash_\mathbf{M}\alpha^\nec$.
    \end{theorem}

    \begin{proof}
        Prova por indução forte sobre o tamanho da sucessão de dedução.
        Assim, suponhamos que a tradução seja correta para qualquer sucessão dedução de tamanho $n<k$.
        Demonstraremos, analisando-se os casos, que o a corretude da tradução vale para sucessões de dedução de tamanho $n=k+1$.

        \begin{case}
            \textsc{Caso 1.}
            Como $\alpha\in\Gamma$, sabe-se que $\alpha^\nec\in\Gamma^\nec$, uma vez que $\Gamma^\nec = \set{\varphi^\nec\mid\varphi\in\Gamma}$.
            Desta forma, $\sequence{\alpha^\nec}$ constitui uma prova para $\Gamma^\nec\vdash\alpha^\nec$.
        \end{case}

        \begin{case}
            \textsc{Caso 2.}
        \end{case}

            \begin{subcase}
                \textsc{Caso 2.1} ($\mathbf{A_1}$).
                Deve-se demonstrar que $\vdash\nec(\alpha^\nec\to\nec(\beta^\nec\to\alpha^\nec))$.
                Pelo teorema $\mathbf{T_1}$, basta provar que $\set{\alpha^\nec,\beta^\nec}\vdash\alpha^\nec$, o que pode ser feito pela seguinte sucessão de dedução:

                \begin{fitch}
                    \fa\alpha^\nec&$\mathbf{P}$
                \end{fitch}
            \end{subcase}

            \begin{subcase}
                \textsc{Caso 2.2} ($\mathbf{A_2}$).
                Deve-se demonstrar que $\vdash\nec(\nec(\alpha^\nec\to\nec(\beta^\nec\to\gamma^\nec))\to\nec(\nec(\alpha^\nec\to\beta^\nec)\to\nec(\alpha^\nec\to\gamma^\nec)))$.

                \begin{fitch}
                    \fa\nec(\alpha^\nec\to\beta^\nec)\to\alpha^\nec\to\beta^\nec&$\mathbf{B_2}$\\
                    \fa\nec(\alpha^\nec\to\beta^\nec)&$\mathbf{P}$\\
                    \fa\alpha^\nec\to\beta^\nec&$\mathbf{R_1}\;\sequence{1,2}$\\
                    \fa\alpha^\nec&$\mathbf{P}$\\
                    \fa\beta^\nec&$\mathbf{R_1}\;\sequence{3,4}$\\
                    \fa\nec(\alpha^\nec\to\nec(\beta^\nec\to\alpha^\nec))\to\alpha^\nec\to\nec(\beta^\nec\to\alpha^\nec)&$\mathbf{B_2}$\\
                    \fa\nec(\alpha^\nec\to\nec(\beta^\nec\to\alpha^\nec))&$\mathbf{P}$\\
                    \fa\alpha^\nec\to\nec(\beta^\nec\to\alpha^\nec)&$\mathbf{R_1}\;\sequence{6,7}$\\
                    \fa\nec(\beta^\nec\to\alpha^\nec)&$\mathbf{R_1}\;\sequence{4,8}$\\
                    \fa\nec(\beta^\nec\to\alpha^\nec)\to\beta^\nec\to\gamma^\nec&$\mathbf{B_2}$\\
                    \fa\beta^\nec\to\gamma^\nec&$\mathbf{R_1}\;\sequence{10,9}$\\
                    \fa\gamma^\nec&$\mathbf{R_1}\;\sequence{11,5}${}
                \end{fitch}
            \end{subcase}

            \begin{subcase}
                \textsc{Caso 2.3} ($\mathbf{A_3}$).

                \begin{fitch}
                    \fa\alpha^\nec\to\beta^\nec\to\alpha^\nec\wedge\beta^\nec&$\mathbf{A_3}$\\
                    \fa\alpha^\nec&$\mathbf{P}$\\
                    \fa\beta^\nec\to\alpha^\nec\wedge\beta^\nec&$\mathbf{R_1}\;\sequence{1,2}$\\
                    \fa\beta^\nec&$\mathbf{P}$\\
                    \fa\alpha^\nec\wedge\beta^\nec&$\mathbf{R_1}\;\sequence{3,4}$
                \end{fitch} 
            \end{subcase}

            \begin{subcase}
                \textsc{Caso 2.4} ($\mathbf{A_4}$).
                Deve-se demonstrar que $\vdash\nec(\alpha^\nec\wedge\beta^\nec\to\alpha^\nec)$, o que pode ser feito ela seguinte sucessão de dedução:

                \begin{fitch}
                    \fa\alpha^\nec\wedge\beta^\nec\to\alpha^\nec&$\mathbf{A_4}$\\
                    \fa\nec(\alpha^\nec\wedge\beta^\nec\to\alpha^\nec)&$\mathbf{R_2}\;\sequence{1}$
                \end{fitch}
            \end{subcase}

            \begin{subcase}
                \textsc{Caso 2.5} ($\mathbf{A_5}$).
                Deve-se demonstrar que $\vdash\nec(\alpha^\nec\wedge\beta^\nec\to\beta^\nec)$, o que pode ser feito ela seguinte sucessão de dedução:

                \begin{fitch}
                    \fa\alpha^\nec\wedge\beta^\nec\to\beta^\nec&$\mathbf{A_5}$\\
                    \fa\nec(\alpha^\nec\wedge\beta^\nec\to\beta^\nec)&$\mathbf{R_2}\;\sequence{1}$
                \end{fitch}
            \end{subcase}

            \begin{subcase}
                \textsc{Caso 2.6} ($\mathbf{A_6}$).
                Deve-se demonstrar que $\vdash\nec(\alpha^\nec\to\alpha^\nec\vee\beta^\nec)$, o que pode ser feito ela seguinte sucessão de dedução:

                \begin{fitch}
                    \fa\alpha^\nec\to\alpha^\nec\vee\beta^\nec&$\mathbf{A_6}$ \\
                    \fa\nec(\alpha^\nec\to\alpha^\nec\vee\beta^\nec)&$\mathbf{R_2}\;\sequence{1}$
                \end{fitch}
            \end{subcase}

            \begin{subcase}
                \textsc{Caso 2.7} ($\mathbf{A_7}$).
                Deve-se demonstrar que $\vdash\nec(\beta^\nec\to\alpha^\nec\vee\beta^\nec)$, o que pode ser feito ela seguinte sucessão de dedução:

                \begin{fitch}
                    \fa\beta^\nec\to\alpha^\nec\vee\beta^\nec&$\mathbf{A_7}$\\
                    \fa\nec(\beta^\nec\to\alpha^\nec\vee\beta^\nec)&$\mathbf{R_2}\;\sequence{1}$
                \end{fitch}
            \end{subcase}

            \begin{subcase}
                \textsc{Caso 2.8} ($\mathbf{A_8}$).
                Deve-se demonstrar que $\vdash(\alpha^\nec\fishhook\gamma^\nec)\fishhook(\beta^\nec\fishhook\gamma^\nec)\fishhook\alpha^\nec\vee\beta^\nec\fishhook\gamma^\nec$.
                Pelo teorema $\mathbf{T_1}$, basta provar que $\set{\alpha\fishhook\gamma,\beta\fishhook\gamma,\alpha\vee\gamma}\vdash\alpha^\nec$, o que pode ser feito pela seguinte sucessão de dedução:
                
                % $\vdash\nec(\nec(\alpha^\nec\to\gamma^\nec)\to\nec(\nec(\beta^\nec\to\gamma)\to\nec(\alpha^\nec\vee\beta^\nec\to\gamma^\nec)))$.
                
                \begin{fitch}
                    \fa(\alpha^\nec\to\gamma^\nec)\to(\beta^\nec\to\gamma^\nec)\to\alpha^\nec\vee\beta^\nec\to\gamma^\nec&$\mathbf{A_8}$\\
                    \fa\nec(\alpha^\nec\to\gamma^\nec)\to\alpha^\nec\to\gamma^\nec&$\mathbf{B_2}$\\
                    \fa\nec(\alpha^\nec\to\gamma^\nec)&$\mathbf{P}$\\
                    \fa\alpha^\nec\to\gamma^\nec&$\mathbf{R_1}\;\sequence{2,3}$\\
                    \fa(\beta^\nec\to\gamma^\nec)\to\alpha^\nec\vee\beta^\nec\to\gamma^\nec&$\mathbf{R_1}\;\sequence{1,4}$\\
                    \fa\nec(\beta^\nec\to\gamma^\nec)\to\beta^\nec\to\gamma^\nec&$\mathbf{B_2}$\\
                    \fa\nec(\beta^\nec\to\gamma^\nec)&$\mathbf{P}$\\
                    \fa\beta^\nec\to\gamma^\nec&$\mathbf{R_1}\;\sequence{6,7}$\\
                    \fa\alpha^\nec\vee\beta^\nec\to\gamma^\nec&$\mathbf{R_1}\;\sequence{5,8}$\\
                    \fa\alpha^\nec\vee\beta^\nec&$\mathbf{P}$\\
                    \fa\gamma^\nec&$\mathbf{R_1}\;\sequence{9,10}${}
                \end{fitch}
            \end{subcase}

            \begin{subcase}
                \textsc{Caso 2.9} ($\mathbf{A_\bot}$).
                Deve-se demonstrar que $\vdash\nec(\bot\to\alpha^\nec)$.
                Pelo teorema $\mathbf{T_1}$, basta provar que $\set{\bot}\vdash\alpha^\nec$, o que pode ser feito pela seguinte sucessão de dedução:

                \begin{fitch}
                    \fa((\alpha^\nec\to\bot)\to\bot)\to\alpha^\nec&$\mathbf{A_\neg}$\\
                    \fa\bot\to(\alpha^\nec\to\bot)\to\bot&$\mathbf{A_1}$\\
                    \fa\bot&$\mathbf{P}$\\
                    \fa(\alpha^\nec\to\bot)\to\bot&$\mathbf{R_1}\;\sequence{2, 3}$\\
                    \fa\alpha^\nec&$\mathbf{R_1}\;\sequence{1, 4}$.
                \end{fitch}
            \end{subcase}

        \begin{case}
            \textsc{Caso 3.}
            Deve-se demonstrar que, se $\vdash\nec(\alpha^\nec\to\beta^\nec)$ ($\mathbf{H_1}$) e $\vdash\alpha^\nec$ ($\mathbf{H_2}$), então $\beta^\nec$.
            Isso pode ser feito pela seguinte sucessão de dedução:

            \begin{fitch}
                \fa\nec(\alpha^\nec\to\beta^\nec)\to\alpha^\nec\to\beta^\nec&$\mathbf{B_2}$\\
                \fa\nec(\alpha^\nec\to\beta^\nec)&$\mathbf{H_1}$\\
                \fa\alpha^\nec\to\beta^\nec&$\mathbf{R_1}\;\sequence{1, 2}$\\
                \fa\alpha^\nec&$\mathbf{H_2}$\\
                \fa\beta^\nec&$\mathbf{R_1}\;\sequence{3, 4}$.
            \end{fitch}
        \end{case}
    \end{proof}