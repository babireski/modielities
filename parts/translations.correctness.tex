\section{Corretude}
    \begin{theorem}
        $\forall \alpha \in \mathcal{L}_\mathbf{I} \point \Gamma \vdash_\mathbf{I} \alpha \Rightarrow \Gamma^\nec \vdash_\mathbf{M} \alpha^\nec$
    \end{theorem}

    \begin{proof}
        Como $\Gamma \vdash_\mathbf{I} \alpha$, sabe-se que existe uma prova $\sequence{\varphi_i\mid 1 \leq i \leq n}$ tal que $\varphi_n = \alpha$. A demonstração deste teorema será feita por indução no tamanho $n$ da prova.

        \begin{case}
            \textbf{Passo} ($n = 1$)\textbf{.} A prova, caso possua tamanho $n = 1$, tem obrigatoriamente a forma $\sequence{\alpha}$. Deste modo, existem duas casos a serem considerados: $\alpha$ ser um axioma ou $\alpha$ ser uma premissa.
        \end{case}

            \begin{subcase}
                \textbf{Caso 1} ($\alpha\in\Gamma$)\textbf{.} Como $\alpha\in\Gamma$, sabe-se que $\alpha^\nec\in\Gamma^\nec$, uma vez que $\Gamma^\nec = \set{\varphi^\nec\mid\varphi\in\Gamma}$. Desta forma, $\sequence{\alpha^\nec}$ constitui uma prova para $\Gamma^\nec\vdash\alpha^\nec$.
            \end{subcase}

            \begin{subcase}
                \textbf{Caso 2} ($\alpha\in\mathcal{A}$)\textbf{.}
            \end{subcase}

                \begin{caseee}
                    \textbf{Caso 2.1} ($\mathbf{A_1}$)\textbf{.}
                    Deve-se demonstrar que $\vdash\nec(\alpha^\nec\to\nec(\beta^\nec\to\alpha^\nec))$.
                    Pelo teorema $\mathbf{T_1}$, basta provar que $\set{\alpha^\nec,\beta^\nec}\vdash\alpha^\nec$, o que pode ser feito pela seguinte sucessão de dedução:

                    \begin{fitch}
                        \fa\alpha^\nec&$\mathbf{P}$
                    \end{fitch}
                \end{caseee}

                \begin{caseee}
                    \textbf{Caso 2.2} ($\mathbf{A_2}$)\textbf{.}
                    Deve-se demonstrar que $\vdash\nec(\nec(\alpha^\nec\to\nec(\beta^\nec\to\gamma^\nec))\to\nec(\nec(\alpha^\nec\to\beta^\nec)\to\nec(\alpha^\nec\to\gamma^\nec)))$.

                    \begin{fitch}
                        \fa\nec(\alpha^\nec\to\beta^\nec)\to\alpha^\nec\to\beta^\nec&$\mathbf{B_2}$\\
                        \fa\nec(\alpha^\nec\to\beta^\nec)&$\mathbf{P}$\\
                        \fa\alpha^\nec\to\beta^\nec&$\mathbf{R_1}\;\sequence{1,2}$\\
                        \fa\alpha^\nec&$\mathbf{P}$\\
                        \fa\beta^\nec&$\mathbf{R_1}\;\sequence{3,4}$\\
                        \fa\nec(\alpha^\nec\to\nec(\beta^\nec\to\alpha^\nec))\to\alpha^\nec\to\nec(\beta^\nec\to\alpha^\nec)&$\mathbf{B_2}$\\
                        \fa\nec(\alpha^\nec\to\nec(\beta^\nec\to\alpha^\nec))&$\mathbf{P}$\\
                        \fa\alpha^\nec\to\nec(\beta^\nec\to\alpha^\nec)&$\mathbf{R_1}\;\sequence{6,7}$\\
                        \fa\nec(\beta^\nec\to\alpha^\nec)&$\mathbf{R_1}\;\sequence{4,8}$\\
                        \fa\nec(\beta^\nec\to\alpha^\nec)\to\beta^\nec\to\gamma^\nec&$\mathbf{B_2}$\\
                        \fa\beta^\nec\to\gamma^\nec&$\mathbf{R_1}\;\sequence{10,9}$\\
                        \fa\gamma^\nec&$\mathbf{R_1}\;\sequence{11,5}${}
                    \end{fitch}
                \end{caseee}

                \begin{caseee}
                    \textbf{Caso 2.3} ($\mathbf{A_3}$)\textbf{.}

                    \begin{fitch}
                        \fa\alpha^\nec\to\beta^\nec\to\alpha^\nec\wedge\beta^\nec&$\mathbf{A_3}$\\
                        \fa\alpha^\nec&$\mathbf{P}$\\
                        \fa\beta^\nec\to\alpha^\nec\wedge\beta^\nec&$\mathbf{R_1}\;\sequence{1,2}$\\
                        \fa\beta^\nec&$\mathbf{P}$\\
                        \fa\alpha^\nec\wedge\beta^\nec&$\mathbf{R_1}\;\sequence{3,4}$
                    \end{fitch} 
                \end{caseee}

                \begin{caseee}
                    \textbf{Caso 2.4} ($\mathbf{A_4}$)\textbf{.}
                    Deve-se demonstrar que $\vdash\nec(\alpha^\nec\wedge\beta^\nec\to\alpha^\nec)$, o que pode ser feito ela seguinte sucessão de dedução:

                    \begin{fitch}
                        \fa\alpha^\nec\wedge\beta^\nec\to\alpha^\nec&$\mathbf{A_4}$\\
                        \fa\nec(\alpha^\nec\wedge\beta^\nec\to\alpha^\nec)&$\mathbf{R_2}\;\sequence{1}$
                    \end{fitch}
                \end{caseee}

                \begin{caseee}
                    \textbf{Caso 2.5} ($\mathbf{A_5}$)\textbf{.}
                    Deve-se demonstrar que $\vdash\nec(\alpha^\nec\wedge\beta^\nec\to\beta^\nec)$, o que pode ser feito ela seguinte sucessão de dedução:

                    \begin{fitch}
                        \fa\alpha^\nec\wedge\beta^\nec\to\beta^\nec&$\mathbf{A_5}$\\
                        \fa\nec(\alpha^\nec\wedge\beta^\nec\to\beta^\nec)&$\mathbf{R_2}\;\sequence{1}$
                    \end{fitch}
                \end{caseee}

                \begin{caseee}
                    \textbf{Caso 2.6} ($\mathbf{A_6}$)\textbf{.}
                    Deve-se demonstrar que $\vdash\nec(\alpha^\nec\to\alpha^\nec\vee\beta^\nec)$, o que pode ser feito ela seguinte sucessão de dedução:

                    \begin{fitch}
                        \fa\alpha^\nec\to\alpha^\nec\vee\beta^\nec&$\mathbf{A_6}$ \\
                        \fa\nec(\alpha^\nec\to\alpha^\nec\vee\beta^\nec)&$\mathbf{R_2}\;\sequence{1}$
                    \end{fitch}
                \end{caseee}

                \begin{caseee}
                    \textbf{Caso 2.7} ($\mathbf{A_7}$)\textbf{.}
                    Deve-se demonstrar que $\vdash\nec(\beta^\nec\to\alpha^\nec\vee\beta^\nec)$, o que pode ser feito ela seguinte sucessão de dedução:

                    \begin{fitch}
                        \fa\beta^\nec\to\alpha^\nec\vee\beta^\nec&$\mathbf{A_7}$\\
                        \fa\nec(\beta^\nec\to\alpha^\nec\vee\beta^\nec)&$\mathbf{R_2}\;\sequence{1}$
                    \end{fitch}
                \end{caseee}

                \begin{caseee}
                    \textbf{Caso 2.8} ($\mathbf{A_8}$)\textbf{.}
                    Deve-se demonstrar que $\vdash(\alpha^\nec\fishhook\gamma^\nec)\fishhook(\beta^\nec\fishhook\gamma^\nec)\fishhook\alpha^\nec\vee\beta^\nec\fishhook\gamma^\nec$.
                    Pelo teorema $\mathbf{T_1}$, basta provar que $\set{\alpha\fishhook\gamma,\beta\fishhook\gamma,\alpha\vee\gamma}\vdash\alpha^\nec$, o que pode ser feito pela seguinte sucessão de dedução:
                    
                    % $\vdash\nec(\nec(\alpha^\nec\to\gamma^\nec)\to\nec(\nec(\beta^\nec\to\gamma)\to\nec(\alpha^\nec\vee\beta^\nec\to\gamma^\nec)))$.
                    
                    \begin{fitch}
                        \fa(\alpha^\nec\to\gamma^\nec)\to(\beta^\nec\to\gamma^\nec)\to\alpha^\nec\vee\beta^\nec\to\gamma^\nec&$\mathbf{A_8}$\\
                        \fa\nec(\alpha^\nec\to\gamma^\nec)\to\alpha^\nec\to\gamma^\nec&$\mathbf{B_2}$\\
                        \fa\nec(\alpha^\nec\to\gamma^\nec)&$\mathbf{P}$\\
                        \fa\alpha^\nec\to\gamma^\nec&$\mathbf{R_1}\;\sequence{2,3}$\\
                        \fa(\beta^\nec\to\gamma^\nec)\to\alpha^\nec\vee\beta^\nec\to\gamma^\nec&$\mathbf{R_1}\;\sequence{1,4}$\\
                        \fa\nec(\beta^\nec\to\gamma^\nec)\to\beta^\nec\to\gamma^\nec&$\mathbf{B_2}$\\
                        \fa\nec(\beta^\nec\to\gamma^\nec)&$\mathbf{P}$\\
                        \fa\beta^\nec\to\gamma^\nec&$\mathbf{R_1}\;\sequence{6,7}$\\
                        \fa\alpha^\nec\vee\beta^\nec\to\gamma^\nec&$\mathbf{R_1}\;\sequence{5,8}$\\
                        \fa\alpha^\nec\vee\beta^\nec&$\mathbf{P}$\\
                        \fa\gamma^\nec&$\mathbf{R_1}\;\sequence{9,10}${}
                    \end{fitch}
                \end{caseee}

                \begin{caseee}
                    \textbf{Caso 2.9} ($\mathbf{A_\bot}$)\textbf{.}
                    Deve-se demonstrar que $\vdash\nec(\bot\to\alpha^\nec)$.
                    Pelo teorema $\mathbf{T_1}$, basta provar que $\set{\bot}\vdash\alpha^\nec$, o que pode ser feito pela seguinte sucessão de dedução:

                    \begin{fitch}
                        \fa((\alpha^\nec\to\bot)\to\bot)\to\alpha^\nec&$\mathbf{A_\neg}$\\
                        \fa\bot\to(\alpha^\nec\to\bot)\to\bot&$\mathbf{A_1}$\\
                        \fa\bot&$\mathbf{P}$\\
                        \fa(\alpha^\nec\to\bot)\to\bot&$\mathbf{R_1}\;\sequence{2, 3}$\\
                        \fa\alpha^\nec&$\mathbf{R_1}\;\sequence{1, 4}$.
                    \end{fitch}
                \end{caseee}

                \begin{caseee}
                    \textbf{Caso 2.9} ($\mathbf{R_1}$)\textbf{.}
                    Deve-se demonstrar que, se $\vdash\nec(\alpha^\nec\to\beta^\nec)$ ($\mathbf{H_1}$) e $\vdash\alpha^\nec$ ($\mathbf{H_2}$), então $\beta^\nec$.
                    Isso pode ser feito pela seguinte sucessão de dedução:

                    \begin{fitch}
                        \fa\nec(\alpha^\nec\to\beta^\nec)\to\alpha^\nec\to\beta^\nec&$\mathbf{B_2}$\\
                        \fa\nec(\alpha^\nec\to\beta^\nec)&$\mathbf{H_1}$\\
                        \fa\alpha^\nec\to\beta^\nec&$\mathbf{R_1}\;\sequence{1, 2}$\\
                        \fa\alpha^\nec&$\mathbf{H_2}$\\
                        \fa\beta^\nec&$\mathbf{R_1}\;\sequence{3, 4}$.
                    \end{fitch}
                \end{caseee}
    \end{proof}