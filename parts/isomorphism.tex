\section{Isomorfismo entre as traduções}

Conforme afirmado anteriormente, ambas as traduções apresentadas neste trabalho equivalem --- ou seja, são isomorfas --- na forma $\entails\nec\alpha^\circ\leftrightarrow\alpha^\medsquare$. Nesta seção, provaremos este isomorfismo que, não somente constitui puramente um resultado de interesse, como permite tornar a prova de propriedades de uma tradução triviais caso tais propriedades valham para a outra tradução.

\begin{theorem}\label{isomorphism}
    $\entails\nec\alpha^\circ\leftrightarrow\alpha^\medsquare$.

    \begin{proof}
        Prova por indução forte sobre a profundidade de $\alpha$ \citep{Troelstra}.
        Assim, suponhamos que as traduções equivalham para qualquer $\alpha$ de profundidade $n<k$.
        Demonstraremos analisando-se os casos e valendo-se da suposição acima --- doravante chamada $\mathbf{H}$ --- o passo de indução, ou seja, que as traduções equivalem para qualquer $\alpha$ de profundidade $n=k$.

        \begin{case}
            \textsc{Caso 1.}
            Se a sentença $\alpha$ for uma proposição $a\in\mathcal{P}$, sabe-se que $\nec a^\circ=\nec a$ e que $a^\medsquare=\nec a$ pelas definições das traduções.
            Deste modo, tanto a ida quanto a volta possuem a forma $\nec a\to\nec a$ e podem ser provadas pelo lema \hyperref[identity]{$\mathbf{L_\getrefnumber{identity}}$}.
            Ambas as implicações posteriormente podem ser unidas em uma bi-implicação por meio do esquema \hyperref[MA3]{$\mathbf{A_3}$}.
        \end{case}

        \begin{case}
            \textsc{Caso 2.}
            Se a sentença $\alpha$ for a constante $\bot$, sabe-se que $\nec\bot^\circ=\nec\bot$ e que $\bot^\medsquare=\bot$ pelas definições das traduções.
            Deste modo, a ida $\nec\bot\to\bot$ constitui um axioma gerado pelo esquema \hyperref[MB2]{$\mathbf{B_2}$} --- sendo assim provada trivialmente --- e a volta $\bot\to\nec\bot$ pode ser provada pelo lema \hyperref[explosion]{$\mathbf{L_2}$}.
            Ambas as implicações posteriormente podem ser unidas em uma bi-implicação por meio do esquema \hyperref[MA3]{$\mathbf{A_3}$}.
        \end{case}

        \begin{case}
            \textsc{Caso 3.}
            Se a sentença $\alpha$ for o resultado da conjunção de duas outras sentenças $\varphi$ e $\psi$, sabe-se que $\nec{(\varphi\wedge\psi)}^\circ=\nec(\varphi^\circ\wedge\psi^\circ)$ e que ${(\varphi\wedge\psi)}^\medsquare=\varphi^\medsquare\wedge\psi^\medsquare$ pelas definições das traduções.
            Separaremos a prova em dois casos: um para a ida $\nec(\varphi^\circ\wedge\psi^\circ)\to\varphi^\medsquare\wedge\psi^\medsquare$ e outro para a volta $\varphi^\medsquare\wedge\psi^\medsquare\to\nec(\varphi^\circ\wedge\psi^\circ)$. Ambas as implicações posteriormente podem ser unidas em uma bi-implicação por meio do esquema da introdução da conjunção \hyperref[MA3]{$\mathbf{A_3}$}.
        \end{case}

            \begin{subcase}
                \textsc{Caso 3.1.}
                A partir de $\mathbf{H}$, temos que $\mathbf{H_1}={\entails\nec\varphi^\circ\to\varphi^\medsquare}$ e que $\mathbf{H_2}={\entails\nec\psi^\circ\to\psi^\medsquare}$ por meio dos esquemas da eliminação da conjunção e da aplicação da regra da separação.
                Valendo-se do listado acima em conjunto com alguns lemas, pode-se provar que $\entails\nec(\varphi^\circ\wedge\psi^\circ)\to\varphi^\medsquare\wedge\psi^\medsquare$ pela seguinte sucessão de dedução:

                \footnotesize
                \begin{fitch}
                    \fb\set{\nec(\varphi^\circ\wedge\psi^\circ)}\proves\nec(\varphi^\circ\wedge\psi^\circ)&$\mathbf{P_1}$\\
                    \fa\set{\nec(\varphi^\circ\wedge\psi^\circ)}\proves\nec(\varphi^\circ\wedge\psi^\circ)\to\nec\varphi^\circ\wedge\nec\psi^\circ&$\mathbf{L_5}$\\
                    \fa\set{\nec(\varphi^\circ\wedge\psi^\circ)}\proves\nec\varphi^\circ\wedge\nec\psi^\circ&$\mathbf{R_1}\;\set{1,2}$\\
                    \fa\set{\nec(\varphi^\circ\wedge\psi^\circ)}\proves\nec\varphi^\circ\wedge\nec\psi^\circ\to\nec\varphi^\circ&$\mathbf{A_4}$\\
                    \fa\set{\nec(\varphi^\circ\wedge\psi^\circ)}\proves\nec\varphi^\circ&$\mathbf{R_1}\;\set{3,4}$\\
                    \fa\set{\nec(\varphi^\circ\wedge\psi^\circ)}\proves\nec\varphi^\circ\to\varphi^\medsquare&$\mathbf{H_1}$\\
                    \fa\set{\nec(\varphi^\circ\wedge\psi^\circ)}\proves\varphi^\medsquare&$\mathbf{R_1}\;\set{5,6}$\\
                    \fa\set{\nec(\varphi^\circ\wedge\psi^\circ)}\proves\nec\varphi^\circ\wedge\nec\psi^\circ\to\nec\psi^\circ&$\mathbf{A_4}$\\
                    \fa\set{\nec(\varphi^\circ\wedge\psi^\circ)}\proves\nec\psi^\circ&$\mathbf{R_1}\;\set{3,8}$\\
                    \fa\set{\nec(\varphi^\circ\wedge\psi^\circ)}\proves\nec\psi^\circ\to\psi^\medsquare&$\mathbf{H_2}$\\
                    \fa\set{\nec(\varphi^\circ\wedge\psi^\circ)}\proves\psi^\medsquare&$\mathbf{R_1}\;\set{9,10}$\\
                    \fa\set{\nec(\varphi^\circ\wedge\psi^\circ)}\proves\varphi^\medsquare\to\psi^\medsquare\to\varphi^\medsquare\wedge\psi^\medsquare&$\mathbf{A_3}$\\
                    \fa\set{\nec(\varphi^\circ\wedge\psi^\circ)}\proves\psi^\medsquare\to\varphi^\medsquare\wedge\psi^\medsquare&$\mathbf{R_1}\;\set{7,12}$\\
                    \fa\set{\nec(\varphi^\circ\wedge\psi^\circ)}\proves\varphi^\medsquare\wedge\psi^\medsquare&$\mathbf{R_1}\;\set{9,13}$\\
                    \fa\proves\nec(\varphi^\circ\wedge\psi^\circ)\to\varphi^\medsquare\wedge\psi^\medsquare&$\mathbf{T_1}\;\set{14}$\\
                \end{fitch}
            \end{subcase}

            \begin{subcase}
                \textsc{Caso 3.2.}
                A partir de $\mathbf{H}$, temos que $\mathbf{H_1}={\entails\varphi^\medsquare\to\nec\varphi^\circ}$ e que $\mathbf{H_2}={\entails\psi^\medsquare\to\nec\psi^\circ}$ por meio dos esquemas da eliminação da conjunção e da aplicação regra da separação.
                Valendo-se do listado acima em conjunto com alguns lemas, pode-se provar que $\entails\varphi^\medsquare\wedge\psi^\medsquare\to\nec(\varphi^\circ\wedge\psi^\circ)$ pela seguinte sucessão de dedução:

                \footnotesize
                \begin{fitch}
                    \fb\set{\varphi^\medsquare\wedge\psi^\medsquare}\proves\varphi^\medsquare\wedge\psi^\medsquare&$\mathbf{P}$\\
                    \fa\set{\varphi^\medsquare\wedge\psi^\medsquare}\proves\varphi^\medsquare\wedge\psi^\medsquare\to\varphi^\medsquare&$\mathbf{P}$\\
                    \fa\set{\varphi^\medsquare\wedge\psi^\medsquare}\proves\varphi^\medsquare&$\mathbf{P}$\\
                    \fa\set{\varphi^\medsquare\wedge\psi^\medsquare}\proves\varphi^\medsquare\to\nec\varphi^\circ&$\mathbf{P}$\\
                    \fa\set{\varphi^\medsquare\wedge\psi^\medsquare}\proves\nec\varphi^\circ&$\mathbf{P}$\\
                    \fa\set{\varphi^\medsquare\wedge\psi^\medsquare}\proves\varphi^\medsquare\wedge\psi^\medsquare\to\psi^\medsquare&$\mathbf{P}$\\
                    \fa\set{\varphi^\medsquare\wedge\psi^\medsquare}\proves\psi^\medsquare&$\mathbf{P}$\\
                    \fa\set{\varphi^\medsquare\wedge\psi^\medsquare}\proves\psi^\medsquare\to\nec\psi^\circ&$\mathbf{P}$\\
                    \fa\set{\varphi^\medsquare\wedge\psi^\medsquare}\proves\nec\psi^\circ&$\mathbf{P}$\\
                    \fa\set{\varphi^\medsquare\wedge\psi^\medsquare}\proves\nec\varphi^\circ\to\nec\psi^\circ\to\nec\varphi^\circ\wedge\nec\psi^\circ&$\mathbf{P}$\\
                    \fa\set{\varphi^\medsquare\wedge\psi^\medsquare}\proves\nec\psi^\circ\to\nec\varphi^\circ\wedge\nec\psi^\circ&$\mathbf{P}$\\
                    \fa\set{\varphi^\medsquare\wedge\psi^\medsquare}\proves\nec\varphi^\circ\wedge\nec\psi^\circ&$\mathbf{P}$\\
                    \fa\set{\varphi^\medsquare\wedge\psi^\medsquare}\proves\nec\varphi^\circ\wedge\nec\psi^\circ\to\nec(\varphi^\circ\wedge\psi^\circ)&$\mathbf{P}$\\
                    \fa\set{\varphi^\medsquare\wedge\psi^\medsquare}\proves\nec(\varphi^\circ\wedge\psi^\circ)&$\mathbf{P}$\\
                    \fa\proves\varphi^\medsquare\wedge\psi^\medsquare\to\nec(\varphi^\circ\wedge\psi^\circ)&$\mathbf{P}$\\
                \end{fitch}
            \end{subcase}

        \begin{case}
            \textsc{Caso 4.}
            Se a sentença $\alpha$ for o resultado da disjunção de duas outras sentenças $\varphi$ e $\psi$, sabe-se que $\nec{(\varphi\vee\psi)}^\circ=\nec(\nec\varphi^\circ\vee\nec\psi^\circ)$ e que ${(\varphi\vee\psi)}^\medsquare=\varphi^\medsquare\vee\psi^\medsquare$ pelas definições das traduções.
            Separaremos a prova em dois casos: um para a ida $\nec(\nec\varphi^\circ\vee\nec\psi^\circ)\to\varphi^\medsquare\vee\psi^\medsquare$ e outro para a volta $\varphi^\medsquare\vee\psi^\medsquare\to\nec(\nec\varphi^\circ\vee\nec\psi^\circ)$.
            Ambas as implicações, então, podem ser unidas em uma bi-implicação por meio do esquema da introdução da conjunção \hyperref[MA3]{$\mathbf{A_3}$}.
        \end{case}

        \begin{subcase}
            \textsc{Caso 4.1.}
            A partir de $\mathbf{H}$, temos que $\mathbf{H_1}={\entails\nec\varphi^\circ\to\varphi^\medsquare}$ e que $\mathbf{H_2}={\entails\nec\psi^\circ\to\psi^\medsquare}$ por meio dos esquemas da eliminação da conjunção e da aplicação da regra da separação.
            Valendo-se do listado acima em conjunto com alguns lemas, pode-se provar que $\entails\nec(\nec\varphi^\circ\vee\nec\psi^\circ)\to\varphi^\medsquare\vee\psi^\medsquare$ pela seguinte sucessão de dedução:
            \footnotesize
            \begin{fitch}
                \fb\set{\nec(\nec\varphi^\circ\vee\nec\psi^\circ)}\entails\nec\varphi^\circ\to\varphi^\medsquare\\
                \fa\set{\nec(\nec\varphi^\circ\vee\nec\psi^\circ)}\entails(\nec\varphi^\circ\to\varphi^\medsquare)\to\nec\varphi^\circ\to\varphi^\medsquare\vee\psi^\medsquare\\
                \fa\set{\nec(\nec\varphi^\circ\vee\nec\psi^\circ)}\entails\nec\varphi^\circ\to\varphi^\medsquare\vee\psi^\medsquare\\

                \fa\set{\nec(\nec\varphi^\circ\vee\nec\psi^\circ)}\entails\nec\psi^\circ\to\psi^\medsquare\\
                \fa\set{\nec(\nec\varphi^\circ\vee\nec\psi^\circ)}\entails(\nec\psi^\circ\to\psi^\medsquare)\to\nec\psi^\circ\to\varphi^\medsquare\vee\psi^\medsquare\\
                \fa\set{\nec(\nec\varphi^\circ\vee\nec\psi^\circ)}\entails\nec\psi^\circ\to\varphi^\medsquare\vee\psi^\medsquare\\

                \fa\set{\nec(\nec\varphi^\circ\vee\nec\psi^\circ)}\entails\nec(\nec\varphi^\circ\vee\nec\psi^\circ)\\
                \fa\set{\nec(\nec\varphi^\circ\vee\nec\psi^\circ)}\entails\nec(\nec\varphi^\circ\vee\nec\psi^\circ)\to\nec\varphi^\circ\vee\nec\psi^\circ\\
                \fa\set{\nec(\nec\varphi^\circ\vee\nec\psi^\circ)}\entails\nec\varphi^\circ\vee\nec\psi^\circ\\
                \fa\set{\nec(\nec\varphi^\circ\vee\nec\psi^\circ)}\entails(\nec\varphi^\circ\to\varphi^\medsquare\vee\psi^\medsquare)\to(\nec\psi^\circ\to\varphi^\medsquare\vee\psi^\medsquare)\to\nec\varphi^\circ\vee\nec\psi^\circ\to\varphi^\medsquare\vee\psi^\medsquare\\
                \fa\set{\nec(\nec\varphi^\circ\vee\nec\psi^\circ)}\entails(\nec\psi^\circ\to\varphi^\medsquare\vee\psi^\medsquare)\to\nec\varphi^\circ\vee\nec\psi^\circ\to\varphi^\medsquare\vee\psi^\medsquare\\
                \fa\set{\nec(\nec\varphi^\circ\vee\nec\psi^\circ)}\entails\nec\varphi^\circ\vee\nec\psi^\circ\to\varphi^\medsquare\vee\psi^\medsquare\\
                \fa\set{\nec(\nec\varphi^\circ\vee\nec\psi^\circ)}\entails\varphi^\medsquare\vee\psi^\medsquare\\
                \fa\entails\nec(\nec\varphi^\circ\vee\nec\psi^\circ)\to\varphi^\medsquare\vee\psi^\medsquare\\
            \end{fitch}
        \end{subcase}

        \begin{subcase}
            \textsc{Caso 4.2.}
            A partir de $\mathbf{H}$, temos que $\mathbf{H_1}={\entails\varphi^\medsquare\to\nec\varphi^\circ}$ e que $\mathbf{H_2}={\entails\psi^\medsquare\to\nec\psi^\circ}$ por meio dos esquemas da eliminação da conjunção e da aplicação regra da separação.
            Valendo-se do listado acima em conjunto com alguns lemas, pode-se provar que $\entails\varphi^\medsquare\vee\psi^\medsquare\to\nec(\nec\varphi^\circ\vee\nec\psi^\circ)$ pela seguinte sucessão de dedução:
            \footnotesize
            \begin{fitch}
                \fb\set{\varphi^\medsquare\vee\psi^\medsquare}\entails\varphi^\medsquare\to\nec\varphi^\circ\\
                \fa\set{\varphi^\medsquare\vee\psi^\medsquare}\entails\nec\varphi^\circ\to\nec\nec\varphi^\circ\\
                \fa\set{\varphi^\medsquare\vee\psi^\medsquare}\entails(\varphi^\medsquare\to\nec\varphi^\circ)\to(\nec\varphi^\circ\to\nec\nec\varphi^\circ)\to\varphi^\medsquare\to\nec\nec\varphi^\circ\\
                \fa\set{\varphi^\medsquare\vee\psi^\medsquare}\entails(\nec\varphi^\circ\to\nec\nec\varphi^\circ)\to\varphi^\medsquare\to\nec\nec\varphi^\circ\\
                \fa\set{\varphi^\medsquare\vee\psi^\medsquare}\entails\varphi^\medsquare\to\nec\nec\varphi^\circ\\
                \fa\set{\varphi^\medsquare\vee\psi^\medsquare}\entails(\varphi^\medsquare\to\nec\nec\varphi^\circ)\to\varphi^\medsquare\to\nec\nec\varphi^\circ\vee\nec\nec\psi^\circ\\
                \fa\set{\varphi^\medsquare\vee\psi^\medsquare}\entails\varphi^\medsquare\to\nec\nec\varphi^\circ\vee\nec\nec\psi^\circ\\
                
                \fa\set{\varphi^\medsquare\vee\psi^\medsquare}\entails\psi^\medsquare\to\nec\psi^\circ\\
                \fa\set{\varphi^\medsquare\vee\psi^\medsquare}\entails\nec\psi^\circ\to\nec\nec\psi^\circ\\
                \fa\set{\varphi^\medsquare\vee\psi^\medsquare}\entails(\psi^\medsquare\to\nec\psi^\circ)\to(\nec\psi^\circ\to\nec\nec\psi^\circ)\to\psi^\medsquare\to\nec\nec\psi^\circ\\
                \fa\set{\varphi^\medsquare\vee\psi^\medsquare}\entails(\nec\psi^\circ\to\nec\nec\psi^\circ)\to\psi^\medsquare\to\nec\nec\psi^\circ\\
                \fa\set{\varphi^\medsquare\vee\psi^\medsquare}\entails\psi^\medsquare\to\nec\nec\psi^\circ\\
                \fa\set{\varphi^\medsquare\vee\psi^\medsquare}\entails(\psi^\medsquare\to\nec\nec\psi^\circ)\to\psi^\medsquare\to\nec\nec\varphi^\circ\vee\nec\nec\psi^\circ\\
                \fa\set{\varphi^\medsquare\vee\psi^\medsquare}\entails\psi^\medsquare\to\nec\nec\varphi^\circ\vee\nec\nec\psi^\circ\\

                \fa\set{\varphi^\medsquare\vee\psi^\medsquare}\entails\varphi^\medsquare\vee\psi^\medsquare\\
                \fa\set{\varphi^\medsquare\vee\psi^\medsquare}\entails(\varphi^\medsquare\to\nec\nec\varphi^\circ\vee\nec\nec\psi^\circ)\to(\psi^\medsquare\to\nec\nec\varphi^\circ\vee\nec\nec\psi^\circ)\to\varphi^\medsquare\vee\psi^\medsquare\to\nec\nec\varphi^\circ\vee\nec\nec\psi^\circ\\
                \fa\set{\varphi^\medsquare\vee\psi^\medsquare}\entails(\psi^\medsquare\to\nec\nec\varphi^\circ\vee\nec\nec\psi^\circ)\to\varphi^\medsquare\vee\psi^\medsquare\to\nec\nec\varphi^\circ\vee\nec\nec\psi^\circ\\
                \fa\set{\varphi^\medsquare\vee\psi^\medsquare}\entails\varphi^\medsquare\vee\psi^\medsquare\to\nec\nec\varphi^\circ\vee\nec\nec\psi^\circ\\
                \fa\set{\varphi^\medsquare\vee\psi^\medsquare}\entails\nec\nec\varphi^\circ\vee\nec\nec\psi^\circ\\
                \fa\set{\varphi^\medsquare\vee\psi^\medsquare}\entails\nec\nec\varphi^\circ\vee\nec\nec\psi^\circ\to\nec(\nec\varphi^\circ\vee\nec\psi^\circ)\\
                \fa\set{\varphi^\medsquare\vee\psi^\medsquare}\entails\nec(\nec\varphi^\circ\vee\nec\psi^\circ)\\
                \fa\entails\varphi^\medsquare\vee\psi^\medsquare\to\nec(\nec\varphi^\circ\vee\nec\psi^\circ)\\
            \end{fitch}
        \end{subcase}

        \begin{case}
            \textsc{Caso 5.}
            Se a sentença $\alpha$ for o resultado da implicação de uma sentença $\varphi$ a uma sentença $\psi$, sabe-se que $\nec{(\varphi\to\psi)}^\circ=\nec(\nec\varphi^\circ\to\psi^\circ)$ e que ${(\varphi\to\psi)}^\medsquare=\nec(\varphi^\medsquare\to\psi^\medsquare)$ pelas definições das traduções.
            Separaremos a prova em dois casos: um para a ida $\nec(\nec\varphi^\circ\to\psi^\circ)\to\nec(\varphi^\medsquare\to\psi^\medsquare)$ e outro para a volta $\nec(\varphi^\medsquare\to\psi^\medsquare)\to\nec(\nec\varphi^\circ\to\psi^\circ)$.
            Ambas as implicações, então, podem ser unidas em uma bi-implicação por meio do esquema \hyperref[MA3]{$\mathbf{A_3}$}.
        \end{case}

            \begin{subcase}
                \textsc{Caso 5.1.}
                A partir de $\mathbf{H}$, temos que $\mathbf{H_1}={\entails\varphi^\medsquare\to\nec\varphi^\circ}$ e que $\mathbf{H_2}={\entails\psi^\medsquare\to\nec\psi^\circ}$ por meio dos esquemas da eliminação da conjunção e da aplicação regra da separação.
                Valendo-se do listado acima em conjunto com alguns lemas, pode-se provar que $\entails\nec(\nec\varphi^\circ\to\psi^\circ)\to\nec(\psi^\medsquare\to\psi^\medsquare)$ pela seguinte sucessão de dedução:

                \footnotesize
                \begin{fitch}
                    \fb\set{\nec(\nec\varphi^\circ\to\psi^\circ)}\entails\nec(\nec\varphi^\circ\to\psi^\circ)\\
                    \fa\set{\nec(\nec\varphi^\circ\to\psi^\circ)}\entails\nec(\nec\varphi^\circ\to\psi^\circ)\to\nec\nec\varphi^\circ\to\nec\psi^\circ\\
                    \fa\set{\nec(\nec\varphi^\circ\to\psi^\circ)}\entails\nec\varphi^\circ\to\nec\nec\varphi^\circ\\
                    \fa\set{\nec(\nec\varphi^\circ\to\psi^\circ)}\entails\nec\nec\varphi^\circ\to\nec\psi^\circ\\
                    \fa\set{\nec(\nec\varphi^\circ\to\psi^\circ)}\entails(\nec\varphi^\circ\to\nec\nec\varphi^\circ)\to(\nec\nec\varphi^\circ\to\nec\psi^\circ)\to\nec\varphi^\circ\to\nec\psi^\circ\\
                    \fa\set{\nec(\nec\varphi^\circ\to\psi^\circ)}\entails(\nec\nec\varphi^\circ\to\nec\psi^\circ)\to\nec\varphi^\circ\to\nec\psi^\circ\\
                    \fa\set{\nec(\nec\varphi^\circ\to\psi^\circ)}\entails\varphi^\medsquare\to\nec\varphi^\circ\\
                    \fa\set{\nec(\nec\varphi^\circ\to\psi^\circ)}\entails\nec\varphi^\circ\to\nec\psi^\circ\\
                    \fa\set{\nec(\nec\varphi^\circ\to\psi^\circ)}\entails(\varphi^\medsquare\to\nec\varphi^\circ)\to(\nec\varphi^\circ\to\nec\psi^\circ)\to\varphi^\medsquare\to\nec\psi^\circ\\
                    \fa\set{\nec(\nec\varphi^\circ\to\psi^\circ)}\entails(\nec\varphi^\circ\to\nec\psi^\circ)\to\varphi^\medsquare\to\nec\psi^\circ\\
                    \fa\set{\nec(\nec\varphi^\circ\to\psi^\circ)}\entails\varphi^\medsquare\to\nec\psi^\circ\\
                    \fa\set{\nec(\nec\varphi^\circ\to\psi^\circ)}\entails\nec\psi^\circ\to\psi^\medsquare\\
                    \fa\set{\nec(\nec\varphi^\circ\to\psi^\circ)}\entails(\varphi^\medsquare\to\nec\psi^\circ)\to(\nec\psi^\circ\to\psi^\medsquare)\to\psi^\medsquare\to\psi^\medsquare\\
                    \fa\set{\nec(\nec\varphi^\circ\to\psi^\circ)}\entails(\nec\psi^\circ\to\psi^\medsquare)\to\psi^\medsquare\to\psi^\medsquare\\
                    \fa\set{\nec(\nec\varphi^\circ\to\psi^\circ)}\entails\psi^\medsquare\to\psi^\medsquare\\
                    \fa\set{\nec(\nec\varphi^\circ\to\psi^\circ)}\entails\nec(\psi^\medsquare\to\psi^\medsquare)\\
                    \fa\entails\nec(\nec\varphi^\circ\to\psi^\circ)\to\nec(\psi^\medsquare\to\psi^\medsquare)\\
                \end{fitch}
            \end{subcase}

            \begin{subcase}
                \textsc{Caso 5.2.}
                A partir de $\mathbf{H}$, temos que $\mathbf{H_1}=\nec\varphi^\circ\to\varphi^\medsquare$ e que $\mathbf{H_2}=\psi^\medsquare\to\nec\psi^\circ$ por meio dos esquemas da eliminação da conjunção e da aplicação regra da separação.
                Valendo-se do listado acima em conjunto com alguns lemas, pode-se provar que $\entails\nec(\psi^\medsquare\to\psi^\medsquare)\to\nec(\nec\varphi^\circ\to\psi^\circ)$ pela seguinte sucessão de dedução:

                \footnotesize
                \begin{fitch}
                    \fb\set{\nec(\varphi^\medsquare\to\psi^\medsquare),\nec\varphi^\circ}\entails\nec\varphi^\circ\to\varphi^\medsquare\\
                    \fa\set{\nec(\varphi^\medsquare\to\psi^\medsquare),\nec\varphi^\circ}\entails\nec(\varphi^\medsquare\to\psi^\medsquare)\\
                    \fa\set{\nec(\varphi^\medsquare\to\psi^\medsquare),\nec\varphi^\circ}\entails\nec(\varphi^\medsquare\to\psi^\medsquare)\to\varphi^\medsquare\to\psi^\medsquare\\
                    \fa\set{\nec(\varphi^\medsquare\to\psi^\medsquare),\nec\varphi^\circ}\entails\varphi^\medsquare\to\psi^\medsquare\\
                    \fa\set{\nec(\varphi^\medsquare\to\psi^\medsquare),\nec\varphi^\circ}\entails(\nec\varphi^\circ\to\varphi^\medsquare)\to(\varphi^\medsquare\to\psi^\medsquare)\to\nec\varphi^\circ\to\psi^\medsquare\\
                    \fa\set{\nec(\varphi^\medsquare\to\psi^\medsquare),\nec\varphi^\circ}\entails(\varphi^\medsquare\to\psi^\medsquare)\to\nec\varphi^\circ\to\psi^\medsquare\\
                    \fa\set{\nec(\varphi^\medsquare\to\psi^\medsquare),\nec\varphi^\circ}\entails\nec\varphi^\circ\to\psi^\medsquare\\
                    \fa\set{\nec(\varphi^\medsquare\to\psi^\medsquare),\nec\varphi^\circ}\entails\psi^\medsquare\\
                    \fa\set{\nec(\varphi^\medsquare\to\psi^\medsquare),\nec\varphi^\circ}\entails\psi^\medsquare\to\nec\psi^\circ\\
                    \fa\set{\nec(\varphi^\medsquare\to\psi^\medsquare),\nec\varphi^\circ}\entails\nec\psi^\circ\\
                    \fa\set{\nec(\varphi^\medsquare\to\psi^\medsquare),\nec\varphi^\circ}\entails\nec\psi^\circ\to\psi^\circ\\
                    \fa\set{\nec(\varphi^\medsquare\to\psi^\medsquare),\nec\varphi^\circ}\entails\psi^\circ\\
                    \fa\set{\nec(\varphi^\medsquare\to\psi^\medsquare)}\entails\nec(\nec\varphi^\circ\to\psi^\circ)\\
                    \fa\entails\nec(\varphi^\medsquare\to\psi^\medsquare)\to\nec(\nec\varphi^\circ\to\psi^\circ)\\
                \end{fitch}
            \end{subcase}
        \vspace{.5\baselineskip}
        Tendo-se provado todos os casos do passo de indução, podemos concluir que ambas as traduções apresentadas equivalem, ou seja, que $\entails\nec\alpha^\circ\leftrightarrow\alpha^\medsquare$.
    \end{proof}
\end{theorem}