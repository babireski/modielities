\section{Interderivabilidade}

Conforme afirmado anteriormente, ambas as traduções apresentadas gozam de propriedade da interderivação.
Ou seja, a derivação de uma sentença traduzida por uma das traduções implica que se pode derivar esta mesma sentença traduzida pela outra tradução, e vice-versa.
Esta seção busca demonstrar essa interderivabilidade de duas maneiras: tanto como uma bi-implicação dentro do sistema $\mathfrak{L}$ quanto como uma bi-implicação na metalinguagem.
Para tanto, precisaremos demonstrar uma quantidade de lemas.
Nomearemos o primeiro deles \emph{explosão}, conforme abaixo.

\vspace{.5\baselineskip}
\begin{tcolorbox}[enhanced jigsaw, breakable, sharp corners, colframe=black, colback=white, boxrule=0.5pt, left=1.5mm, right=1.5mm, top=1.5mm, bottom=1.5mm]
\begin{lemma}[Explosão]\label{explosion}
    $\Gamma\entails_\mathfrak{L}\bot\to\alpha$.
    \begin{proof}
        Pode ser demonstrado pela dedução que segue.

        \vspace{0.5\baselineskip}
        \footnotesize
        \setlength{\rowskip}{0.5\baselineskip}
        \begin{xltabular}{\textwidth}{r | X l l}
            \scriptsize{\phantom{0}1}\phantom{ } & $\ \Gamma\set{\bot}\entails\bot$                              & $\hyperref[modal.rule.1]{\mathbf{R_1}}$\phantom{1} & \phantom{$\set{00,00}$}\\[\rowskip]
            \scriptsize{\phantom{0}2}\phantom{ } & $\ \Gamma\set{\bot}\entails\bot\to(\alpha\to\bot)\to\bot$     & $\hyperref[modal.axiom.1]{\mathbf{A_1}}$           & \\[\rowskip]
            \scriptsize{\phantom{0}3}\phantom{ } & $\ \Gamma\set{\bot}\entails(\alpha\to\bot)\to\bot$            & $\hyperref[modal.rule.2]{\mathbf{R_2}}$            & $\set{1,2}$\\[\rowskip]
            \scriptsize{\phantom{0}4}\phantom{ } & $\ \Gamma\set{\bot}\entails((\alpha\to\bot)\to\bot)\to\alpha$ & $\hyperref[modal.axiom.negation]{\mathbf{A_\neg}}$ & \\[\rowskip]
            \scriptsize{\phantom{0}5}\phantom{ } & $\ \Gamma\set{\bot}\entails\alpha$                            & $\hyperref[modal.rule.2]{\mathbf{R_2}}$            & $\set{3,4}$\\[\rowskip]
            \scriptsize{\phantom{0}6}\phantom{ } & $\ \Gamma\entails\bot\to\alpha$                               & \refer{deduction}{T}                               & $\set{5}$\\[\rowskip]
        \end{xltabular}
        \normalsize

        \vspace{0.5\baselineskip}
        Estando assim demonstrada a proposição.
    \end{proof}
\end{lemma}
\end{tcolorbox}

\vspace{.5\baselineskip}
Em seguida, demonstraremos um lema que combina duas implicações com uma conjunção dos antecedentes de modo a inferir uma conjunção dos consequentes.
Para tanto, foram usados os axiomas de introdução e eliminação da negação em conjunto com a regra da separação.
A demontração deste lema busca simplificar a prova de interderivabilidade, uma vez que a faremos por indução sobre a profundidade da sentença, sendo um dos seus casos a conjunção.

\vspace{.5\baselineskip}
\begin{tcolorbox}[enhanced jigsaw, breakable, sharp corners, colframe=black, colback=white, boxrule=0.5pt, left=1.5mm, right=1.5mm, top=1.5mm, bottom=1.5mm]
    \begin{lemma}\label{conjunction.exchange}
        Se $\Gamma\entails_\mathfrak{L}\alpha\to\gamma$ e $\Gamma\entails_\mathfrak{L}\beta\to\delta$, então $\Gamma\entails_\mathfrak{L}\alpha\wedge\beta\to\gamma\wedge\delta$.
        \begin{proof}
        Pode ser demonstrado pela dedução que segue.

        \vspace{0.5\baselineskip}
        \footnotesize
        \setlength{\rowskip}{0.5\baselineskip}
        \begin{xltabular}{\textwidth}{r | X l l}
            \scriptsize{\phantom{0}1}\phantom{ } & $\ \Gamma\cup\set{\alpha\wedge\beta}\vdash\alpha\wedge\beta$                     & $\hyperref[modal.rule.1]{\mathbf{R_1}}$\phantom{1} & \phantom{$\set{00,00}$}\\[\rowskip]
            \scriptsize{\phantom{0}2}\phantom{ } & $\ \Gamma\cup\set{\alpha\wedge\beta}\vdash\alpha\wedge\beta\to\alpha$            & $\hyperref[modal.axiom.4]{\mathbf{A_4}}$           & \\[\rowskip]
            \scriptsize{\phantom{0}3}\phantom{ } & $\ \Gamma\cup\set{\alpha\wedge\beta}\vdash\alpha$                                & $\hyperref[modal.rule.2]{\mathbf{R_2}}$            & $\set{1,2}$\\[\rowskip]
            \scriptsize{\phantom{0}4}\phantom{ } & $\ \Gamma\cup\set{\alpha\wedge\beta}\vdash\alpha\wedge\beta\to\beta$             & $\hyperref[modal.axiom.4]{\mathbf{A_4}}$           & \\[\rowskip]
            \scriptsize{\phantom{0}5}\phantom{ } & $\ \Gamma\cup\set{\alpha\wedge\beta}\vdash\beta$                                 & $\hyperref[modal.rule.2]{\mathbf{R_2}}$            & $\set{1,4}$\\[\rowskip]
            \scriptsize{\phantom{0}6}\phantom{ } & $\ \Gamma\cup\set{\alpha\wedge\beta}\vdash\alpha\to\gamma$                       & $\mathbf{H_1}$                                     & \\[\rowskip]
            \scriptsize{\phantom{0}7}\phantom{ } & $\ \Gamma\cup\set{\alpha\wedge\beta}\vdash\beta\to\delta$                        & $\mathbf{H_2}$                                     & \\[\rowskip]
            \scriptsize{\phantom{0}8}\phantom{ } & $\ \Gamma\cup\set{\alpha\wedge\beta}\vdash\gamma$                                & $\hyperref[modal.rule.2]{\mathbf{R_2}}$            & $\set{3,6}$\\[\rowskip]
            \scriptsize{\phantom{0}9}\phantom{ } & $\ \Gamma\cup\set{\alpha\wedge\beta}\vdash\delta$                                & $\hyperref[modal.rule.2]{\mathbf{R_2}}$            & $\set{5,7}$\\[\rowskip]
            \scriptsize{10}\phantom{ }           & $\ \Gamma\cup\set{\alpha\wedge\beta}\vdash\gamma\to\delta\to\gamma\wedge\delta$  & $\hyperref[modal.axiom.3]{\mathbf{A_4}}$           & \\[\rowskip]
            \scriptsize{11}\phantom{ }           & $\ \Gamma\cup\set{\alpha\wedge\beta}\vdash\delta\to\gamma\wedge\delta$           & $\hyperref[modal.rule.2]{\mathbf{R_2}}$            & $\set{8,10}$\\[\rowskip]
            \scriptsize{12}\phantom{ }           & $\ \Gamma\cup\set{\alpha\wedge\beta}\vdash\gamma\wedge\delta$                    & $\hyperref[modal.rule.2]{\mathbf{R_2}}$            & $\set{9,11}$\\[\rowskip]
            \scriptsize{13}\phantom{ }           & $\ \Gamma\vdash\alpha\wedge\beta\to\gamma\wedge\delta$                           & \refer{deduction}{T}                               & $\set{12}$
        \end{xltabular}
        \normalsize

        \vspace{0.5\baselineskip}
        Estando assim demonstrada a proposição.
        \end{proof}
    \end{lemma}
\end{tcolorbox}

\vspace{.5\baselineskip}
Agora, demonstraremos a \emph{distribução da necessidade sobre a conjunção}.
Este lema afirma que, caso tenhamos necessidade de uma conjunção, então temos a conjunção da necessidade dos conjuntos.
Assim como no lema anterior, foram usados os axiomas de introdução e eliminação da negação em conjunto com a regra da separação.
Neste lema, entretanto, o teorema da generalização da necessitação e o teorema do enfraqueciemnto desempenham um funções importantes.
Usaremos esta asseção para a demonstração da interderivabilidade.

\vspace{.5\baselineskip}
\begin{tcolorbox}[enhanced jigsaw, breakable, sharp corners, colframe=black, colback=white, boxrule=0.5pt, left=1.5mm, right=1.5mm, top=1.5mm, bottom=1.5mm]
    \begin{lemma}\label{necessity.conjunction.distribution}
        $\Gamma\entails_\mathfrak{L}\nec(\alpha\wedge\beta)\to\nec\alpha\wedge\nec\beta$.
        \begin{proof}
        Pode ser demonstrado pela dedução que segue.

        \vspace{0.5\baselineskip}
        \footnotesize
        \setlength{\rowskip}{0.5\baselineskip}
        \begin{xltabular}{\textwidth}{r | X l l}
            \scriptsize{\phantom{0}1}\phantom{ } & $\ \set{\nec(\alpha\wedge\beta)}\vdash\nec(\alpha\wedge\beta)$                            & $\hyperref[modal.rule.1]{\mathbf{R_1}}$        & \phantom{$\set{00,00}$}\\[\rowskip]
            \scriptsize{\phantom{0}2}\phantom{ } & $\ \set{\nec(\alpha\wedge\beta)}\vdash\nec(\alpha\wedge\beta)\to\alpha\wedge\beta$        & $\hyperref[modal.axiom.modal.2]{\mathbf{B_2}}$ & \\[\rowskip]
            \scriptsize{\phantom{0}3}\phantom{ } & $\ \set{\nec(\alpha\wedge\beta)}\vdash\alpha\wedge\beta$                                  & $\hyperref[modal.rule.2]{\mathbf{R_2}}$        & $\set{1,2}$\\[\rowskip]
            \scriptsize{\phantom{0}4}\phantom{ } & $\ \set{\nec(\alpha\wedge\beta)}\vdash\alpha\wedge\beta\to\alpha$                         & $\hyperref[modal.axiom.4]{\mathbf{A_4}}$       & \\[\rowskip]
            \scriptsize{\phantom{0}5}\phantom{ } & $\ \set{\nec(\alpha\wedge\beta)}\vdash\alpha$                                             & $\hyperref[modal.rule.2]{\mathbf{R_2}}$        & $\set{3,4}$\\[\rowskip]
            \scriptsize{\phantom{0}6}\phantom{ } & $\ \set{\nec(\alpha\wedge\beta)}\vdash\nec\alpha$                                         & \refer{generalization}{T}\phantom{1}           & $\set{5}$\\[\rowskip]
            \scriptsize{\phantom{0}7}\phantom{ } & $\ \set{\nec(\alpha\wedge\beta)}\vdash\alpha\wedge\beta\to\beta$                          & $\hyperref[modal.axiom.5]{\mathbf{A_5}}$       & \\[\rowskip]
            \scriptsize{\phantom{0}8}\phantom{ } & $\ \set{\nec(\alpha\wedge\beta)}\vdash\beta$                                              & $\hyperref[modal.rule.2]{\mathbf{R_2}}$        & $\set{3,7}$\\[\rowskip]
            \scriptsize{\phantom{0}9}\phantom{ } & $\ \set{\nec(\alpha\wedge\beta)}\vdash\nec\beta$                                          & \refer{generalization}{T}                      & $\set{8}$\\[\rowskip]
            \scriptsize{10}\phantom{ }           & $\ \set{\nec(\alpha\wedge\beta)}\vdash\nec\alpha\to\nec\beta\to\nec\alpha\wedge\nec\beta$ & $\hyperref[modal.axiom.3]{\mathbf{A_3}}$       & \\[\rowskip]
            \scriptsize{11}\phantom{ }           & $\ \set{\nec(\alpha\wedge\beta)}\vdash\nec\beta\to\nec\alpha\wedge\nec\beta$              & $\hyperref[modal.rule.2]{\mathbf{R_2}}$        & $\set{6,10}$\\[\rowskip]
            \scriptsize{12}\phantom{ }           & $\ \set{\nec(\alpha\wedge\beta)}\vdash\nec\alpha\wedge\nec\beta$                          & $\hyperref[modal.rule.2]{\mathbf{R_2}}$        & $\set{9,11}$\\[\rowskip]
            \scriptsize{13}\phantom{ }           & $\ \vdash\nec(\alpha\wedge\beta)\to\nec\alpha\wedge\nec\beta$                             & \refer{deduction}{T}                           & $\set{12}$\\[\rowskip]
            \scriptsize{14}\phantom{ }           & $\ \Gamma\vdash\nec(\alpha\wedge\beta)\to\nec\alpha\wedge\nec\beta$                       & \refer{weakening}{T}                           & $\set{13}$
        \end{xltabular}
        \normalsize

        \vspace{0.5\baselineskip}
        Estando assim demonstrada a proposição.
        \end{proof}
    \end{lemma}
\end{tcolorbox}

\vspace{.5\baselineskip}
Consideremos agora um novo lema.
Este lema afirma que, dada uma implicação dupla, podemos as antecedentes destas duas implicações podem ser conjuntos em um antecedente de uma implicação apenas.
Na computação, este corresponte ao tipo da função de \emph{descurrificação}.
Aqui entretanto referiremos a ele como \emph{importação}.
Usaremos esta asseção para a demonstração da do lema que o segue.

\vspace{.5\baselineskip}
\begin{tcolorbox}[enhanced jigsaw, breakable, sharp corners, colframe=black, colback=white, boxrule=0.5pt, left=1.5mm, right=1.5mm, top=1.5mm, bottom=1.5mm]
    \begin{lemma}[Importação]\label{importation}
        Se $\Gamma\entails_\mathfrak{L}\alpha\to\beta\to\gamma$ então $\Gamma\entails_\mathfrak{L}\alpha\wedge\beta\to\gamma$.
        \begin{proof}
        Pode ser demonstrado pela dedução que segue.

        \vspace{0.5\baselineskip}
        \footnotesize
        \setlength{\rowskip}{0.5\baselineskip}
        \begin{xltabular}{\textwidth}{r | X l l}
            \scriptsize{\phantom{0}1}\phantom{ } & $\ \Gamma\vdash\alpha\to\beta\to\gamma$                               & $\mathbf{H_1}$\phantom{1}                & \phantom{$\set{00,00}$}\\[\rowskip]
            \scriptsize{\phantom{0}2}\phantom{ } & $\ \Gamma\cup\set{\alpha\wedge\beta}\vdash\alpha\wedge\beta$          & $\hyperref[modal.rule.1]{\mathbf{R_1}}$  & \\[\rowskip]
            \scriptsize{\phantom{0}3}\phantom{ } & $\ \Gamma\cup\set{\alpha\wedge\beta}\vdash\alpha\wedge\beta\to\alpha$ & $\hyperref[modal.axiom.4]{\mathbf{A_4}}$ & \\[\rowskip]
            \scriptsize{\phantom{0}4}\phantom{ } & $\ \Gamma\cup\set{\alpha\wedge\beta}\vdash\alpha$                     & $\hyperref[modal.rule.2]{\mathbf{R_2}}$  & $\set{2,3}$\\[\rowskip]
            \scriptsize{\phantom{0}5}\phantom{ } & $\ \Gamma\cup\set{\alpha\wedge\beta}\vdash\alpha\wedge\beta\to\beta$  & $\hyperref[modal.axiom.5]{\mathbf{A_5}}$ & \\[\rowskip]
            \scriptsize{\phantom{0}6}\phantom{ } & $\ \Gamma\cup\set{\alpha\wedge\beta}\vdash\beta$                      & $\hyperref[modal.rule.2]{\mathbf{R_2}}$  & $\set{2,5}$\\[\rowskip]
            \scriptsize{\phantom{0}7}\phantom{ } & $\ \Gamma\cup\set{\alpha\wedge\beta}\vdash\alpha\to\beta\to\gamma$    & \refer{weakening}{T}                     & $\set{1}$\\[\rowskip]
            \scriptsize{\phantom{0}8}\phantom{ } & $\ \Gamma\cup\set{\alpha\wedge\beta}\vdash\beta\to\gamma$             & $\hyperref[modal.rule.2]{\mathbf{R_2}}$  & $\set{4,7}$\\[\rowskip]
            \scriptsize{\phantom{0}9}\phantom{ } & $\ \Gamma\cup\set{\alpha\wedge\beta}\vdash\gamma$                     & $\hyperref[modal.rule.2]{\mathbf{R_2}}$  & $\set{6,8}$\\[\rowskip]
            \scriptsize{10}\phantom{ }           & $\ \Gamma\vdash\alpha\wedge\beta\to\gamma$                            & \refer{deduction}{T}                     & $\set{9}$
        \end{xltabular}
        \normalsize

        \vspace{0.5\baselineskip}
        Estando assim demonstrada a proposição.
        \end{proof}
    \end{lemma}
\end{tcolorbox}
\vspace{.5\baselineskip}
\begin{tcolorbox}[enhanced jigsaw, breakable, sharp corners, colframe=black, colback=white, boxrule=0.5pt, left=1.5mm, right=1.5mm, top=1.5mm, bottom=1.5mm]
    \begin{lemma}\label{necessity.conjunction.undistribution}
        $\Gamma\entails_\mathfrak{L}\nec\alpha\wedge\nec\beta\to\nec(\alpha\wedge\beta)$.
        \begin{proof}
        Pode ser demonstrado pela dedução que segue.

        \vspace{0.5\baselineskip}
        \footnotesize
        \setlength{\rowskip}{0.5\baselineskip}
        \begin{xltabular}{\textwidth}{r | X l l}
            \scriptsize{\phantom{0}1}\phantom{ } & $\ \set{\nec\alpha,\nec\beta}\vdash\nec\alpha$                         & $\hyperref[modal.rule.1]{\mathbf{R_1}}$        & \phantom{$\set{00,00}$}\\[\rowskip]
            \scriptsize{\phantom{0}2}\phantom{ } & $\ \set{\nec\alpha,\nec\beta}\vdash\nec\alpha\to\alpha$                & $\hyperref[modal.axiom.modal.2]{\mathbf{B_2}}$ & \\[\rowskip]
            \scriptsize{\phantom{0}3}\phantom{ } & $\ \set{\nec\alpha,\nec\beta}\vdash\alpha$                             & $\hyperref[modal.rule.2]{\mathbf{R_2}}$        & $\set{1,2}$\\[\rowskip]
            \scriptsize{\phantom{0}4}\phantom{ } & $\ \set{\nec\alpha,\nec\beta}\vdash\nec\beta$                          & $\hyperref[modal.rule.1]{\mathbf{R_1}}$        & \\[\rowskip]
            \scriptsize{\phantom{0}5}\phantom{ } & $\ \set{\nec\alpha,\nec\beta}\vdash\nec\beta\to\beta$                  & $\hyperref[modal.axiom.modal.2]{\mathbf{B_2}}$ & \\[\rowskip]
            \scriptsize{\phantom{0}6}\phantom{ } & $\ \set{\nec\alpha,\nec\beta}\vdash\beta$                              & $\hyperref[modal.rule.2]{\mathbf{R_2}}$        & $\set{4,5}$\\[\rowskip]
            \scriptsize{\phantom{0}7}\phantom{ } & $\ \set{\nec\alpha,\nec\beta}\vdash\alpha\to\beta\to\alpha\wedge\beta$ & $\hyperref[modal.axiom.3]{\mathbf{A_3}}$       & \\[\rowskip]
            \scriptsize{\phantom{0}8}\phantom{ } & $\ \set{\nec\alpha,\nec\beta}\vdash\beta\to\alpha\wedge\beta$          & $\hyperref[modal.rule.2]{\mathbf{R_2}}$        & $\set{3,7}$\\[\rowskip]
            \scriptsize{\phantom{0}9}\phantom{ } & $\ \set{\nec\alpha,\nec\beta}\vdash\alpha\wedge\beta$                  & $\hyperref[modal.rule.2]{\mathbf{R_2}}$        & $\set{6,8}$\\[\rowskip]
            \scriptsize{10}\phantom{ }           & $\ \set{\nec\alpha,\nec\beta}\vdash\nec(\alpha\wedge\beta)$            & \refer{generalization}{T}\phantom{1}           & $\set{9}$\\[\rowskip]
            \scriptsize{11}\phantom{ }           & $\ \set{\nec\alpha}\vdash\nec\beta\to\nec(\alpha\wedge\beta)$          & \refer{deduction}{T}                           & $\set{10}$\\[\rowskip]
            \scriptsize{12}\phantom{ }           & $\ \vdash\nec\alpha\to\nec\beta\to\nec(\alpha\wedge\beta)$             & \refer{deduction}{T}                           & $\set{11}$\\[\rowskip]
            \scriptsize{13}\phantom{ }           & $\ \vdash\nec\alpha\wedge\nec\beta\to\nec(\alpha\wedge\beta)$          & \refer{importation}{L}                         & $\set{12}$\\[\rowskip]
            \scriptsize{14}\phantom{ }           & $\ \Gamma\vdash\nec\alpha\wedge\nec\beta\to\nec(\alpha\wedge\beta)$    & \refer{weakening}{T}                           & $\set{13}$
        \end{xltabular}
        \normalsize

        \vspace{0.5\baselineskip}
        Estando assim demonstrada a proposição.
        \end{proof}
    \end{lemma}
\end{tcolorbox}
\vspace{.5\baselineskip}
\begin{tcolorbox}[enhanced jigsaw, breakable, sharp corners, colframe=black, colback=white, boxrule=0.5pt, left=1.5mm, right=1.5mm, top=1.5mm, bottom=1.5mm]
    \begin{lemma}\label{disjunction.exchange}
        Se $\Gamma\entails_\mathfrak{L}\alpha\to\gamma$ e $\Gamma\entails_\mathfrak{L}\beta\to\delta$, então $\Gamma\entails_\mathfrak{L}\alpha\vee\beta\to\gamma\vee\delta$.
        \begin{proof}
        Pode ser demonstrado pela dedução que segue.

        \vspace{0.5\baselineskip}
        \footnotesize
        \setlength{\rowskip}{0.5\baselineskip}
        \begin{xltabular}{\textwidth}{r | X l l}
            \scriptsize{\phantom{0}1}\phantom{ } & $\ \Gamma\cup\set{\alpha\vee\beta,\alpha\to\gamma,\alpha}\vdash\alpha$                                                                  & $\hyperref[modal.rule.1]{\mathbf{R_1}}$\phantom{1} & \phantom{$\set{00,00}$}\\[\rowskip]
            \scriptsize{\phantom{0}2}\phantom{ } & $\ \Gamma\cup\set{\alpha\vee\beta,\alpha\to\gamma,\alpha}\vdash\alpha\to\gamma$                                                         & $\hyperref[modal.rule.1]{\mathbf{R_1}}$            & \\[\rowskip]
            \scriptsize{\phantom{0}3}\phantom{ } & $\ \Gamma\cup\set{\alpha\vee\beta,\alpha\to\gamma,\alpha}\vdash\gamma$                                                                  & $\hyperref[modal.rule.2]{\mathbf{R_2}}$            & $\set{1,2}$\\[\rowskip]
            \scriptsize{\phantom{0}4}\phantom{ } & $\ \Gamma\cup\set{\alpha\vee\beta,\alpha\to\gamma,\alpha}\vdash\gamma\to\gamma\vee\delta$                                               & $\hyperref[modal.axiom.6]{\mathbf{A_6}}$           & \\[\rowskip]
            \scriptsize{\phantom{0}5}\phantom{ } & $\ \Gamma\cup\set{\alpha\vee\beta,\alpha\to\gamma,\alpha}\vdash\gamma\vee\delta$                                                        & $\hyperref[modal.rule.2]{\mathbf{R_2}}$            & $\set{3,4}$\\[\rowskip]
            \scriptsize{\phantom{0}6}\phantom{ } & $\ \Gamma\cup\set{\alpha\vee\beta,\beta\to\delta,\beta}\vdash\beta$                                                                     & $\hyperref[modal.rule.1]{\mathbf{R_1}}$            & \\[\rowskip]
            \scriptsize{\phantom{0}7}\phantom{ } & $\ \Gamma\cup\set{\alpha\vee\beta,\beta\to\delta,\beta}\vdash\beta\to\delta$                                                            & $\hyperref[modal.rule.1]{\mathbf{R_1}}$            & \\[\rowskip]
            \scriptsize{\phantom{0}8}\phantom{ } & $\ \Gamma\cup\set{\alpha\vee\beta,\beta\to\delta,\beta}\vdash\delta$                                                                    & $\hyperref[modal.rule.2]{\mathbf{R_2}}$            & $\set{6,7}$\\[\rowskip]
            \scriptsize{\phantom{0}9}\phantom{ } & $\ \Gamma\cup\set{\alpha\vee\beta,\beta\to\delta,\beta}\vdash\delta\to\gamma\vee\delta$                                                 & $\hyperref[modal.axiom.6]{\mathbf{A_7}}$           & \\[\rowskip]
            \scriptsize{10}\phantom{ }           & $\ \Gamma\cup\set{\alpha\vee\beta,\beta\to\delta,\beta}\vdash\gamma\vee\delta$                                                          & $\hyperref[modal.rule.2]{\mathbf{R_2}}$            & $\set{8,9}$\\[\rowskip]
            \scriptsize{11}\phantom{ }           & $\ \Gamma\cup\set{\alpha\vee\beta,\alpha\to\gamma}\vdash\alpha\to\gamma\vee\delta$                                                      & \refer{deduction}{T}                               & $\set{4}$\\[\rowskip]
            \scriptsize{12}\phantom{ }           & $\ \Gamma\cup\set{\alpha\vee\beta,\beta\to\delta}\vdash\beta\to\gamma\vee\delta$                                                        & \refer{deduction}{T}                               & $\set{10}$\\[\rowskip]
            \scriptsize{13}\phantom{ }           & $\ \Gamma\cup\set{\alpha\vee\beta}\vdash\alpha\to\gamma$                                                                                & $\mathbf{H_1}$                                     & \\[\rowskip]
            \scriptsize{14}\phantom{ }           & $\ \Gamma\cup\set{\alpha\vee\beta}\vdash(\alpha\to\gamma)\to\alpha\to\gamma\vee\delta$                                                  & \refer{deduction}{T}                               & $\set{11}$\\[\rowskip]
            \scriptsize{15}\phantom{ }           & $\ \Gamma\cup\set{\alpha\vee\beta}\vdash\alpha\to\gamma\vee\delta$                                                                      & $\hyperref[modal.rule.2]{\mathbf{R_2}}$            & $\set{13,14}$\\[\rowskip]
            \scriptsize{16}\phantom{ }           & $\ \Gamma\cup\set{\alpha\vee\beta}\vdash\beta\to\delta$                                                                                 & $\mathbf{H_2}$                                     & \\[\rowskip]
            \scriptsize{17}\phantom{ }           & $\ \Gamma\cup\set{\alpha\vee\beta}\vdash(\beta\to\delta)\to\beta\to\gamma\vee\delta$                                                    & \refer{deduction}{T}                               & $\set{12}$\\[\rowskip]
            \scriptsize{18}\phantom{ }           & $\ \Gamma\cup\set{\alpha\vee\beta}\vdash\beta\to\gamma\vee\delta$                                                                       & $\hyperref[modal.rule.2]{\mathbf{R_2}}$            & $\set{16,17}$\\[\rowskip]
            \scriptsize{19}\phantom{ }           & $\ \Gamma\cup\set{\alpha\vee\beta}\vdash\alpha\vee\beta$                                                                                & $\hyperref[modal.rule.1]{\mathbf{R_1}}$            & \\[\rowskip]
            \scriptsize{20}\phantom{ }           & $\ \Gamma\cup\set{\alpha\vee\beta}\vdash(\alpha\to\gamma\vee\delta)\to(\beta\to\gamma\vee\delta)\to\alpha\vee\beta\to\gamma\vee\delta$  & $\hyperref[modal.axiom.8]{\mathbf{A_8}}$           & \\[\rowskip]
            \scriptsize{21}\phantom{ }           & $\ \Gamma\cup\set{\alpha\vee\beta}\vdash(\beta\to\gamma\vee\delta)\to\alpha\vee\beta\to\gamma\vee\delta$                                & $\hyperref[modal.rule.2]{\mathbf{R_2}}$            & $\set{15,20}$\\[\rowskip]
            \scriptsize{22}\phantom{ }           & $\ \Gamma\cup\set{\alpha\vee\beta}\vdash\alpha\vee\beta\to\gamma\vee\delta$                                                             & $\hyperref[modal.rule.2]{\mathbf{R_2}}$            & $\set{18,21}$\\[\rowskip]
            \scriptsize{23}\phantom{ }           & $\ \Gamma\cup\set{\alpha\vee\beta}\vdash\gamma\vee\delta$                                                                               & $\hyperref[modal.rule.2]{\mathbf{R_2}}$            & $\set{19,22}$\\[\rowskip]
            \scriptsize{24}\phantom{ }           & $\ \Gamma\vdash\alpha\vee\beta\to\gamma\vee\delta$                                                                                      & \refer{deduction}{T}                               & $\set{23}$
        \end{xltabular}
        \normalsize

        \vspace{0.5\baselineskip}
        Estando assim demonstrada a proposição.
        \end{proof}
    \end{lemma}
\end{tcolorbox}
\vspace{.5\baselineskip}
\begin{tcolorbox}[enhanced jigsaw, breakable, sharp corners, colframe=black, colback=white, boxrule=0.5pt, left=1.5mm, right=1.5mm, top=1.5mm, bottom=1.5mm]
    \begin{lemma}\label{necessity.disjunction.undistribution}
        $\Gamma\entails_\mathfrak{L}\nec\alpha\vee\nec\beta\to\nec(\alpha\vee\beta)$.
        \begin{proof}
        Pode ser demonstrado pela dedução que segue.

        \vspace{0.5\baselineskip}
        \footnotesize
        \setlength{\rowskip}{0.5\baselineskip}
        \begin{xltabular}{\textwidth}{r | X l l}
            \scriptsize{\phantom{0}1}\phantom{ } & $\ \set{\nec\alpha}\vdash\nec\alpha$                                                                                                   & $\hyperref[modal.rule.1]{\mathbf{R_1}}$        & \phantom{$\set{00,00}$}\\[\rowskip]
            \scriptsize{\phantom{0}2}\phantom{ } & $\ \set{\nec\alpha}\vdash\nec\alpha\to\alpha$                                                                                          & $\hyperref[modal.axiom.modal.2]{\mathbf{B_2}}$ & \\[\rowskip]
            \scriptsize{\phantom{0}3}\phantom{ } & $\ \set{\nec\alpha}\vdash\alpha$                                                                                                       & $\hyperref[modal.rule.2]{\mathbf{R_2}}$        & $\set{1,2}$\\[\rowskip]
            \scriptsize{\phantom{0}4}\phantom{ } & $\ \set{\nec\alpha}\vdash\alpha\to\alpha\vee\beta$                                                                                     & $\hyperref[modal.axiom.6]{\mathbf{A_6}}$       & \\[\rowskip]
            \scriptsize{\phantom{0}5}\phantom{ } & $\ \set{\nec\alpha}\vdash\alpha\vee\beta$                                                                                              & $\hyperref[modal.rule.2]{\mathbf{R_2}}$        & $\set{3,4}$\\[\rowskip]
            \scriptsize{\phantom{0}6}\phantom{ } & $\ \set{\nec\alpha}\vdash\nec(\alpha\vee\beta)$                                                                                        & \refer{generalization}{T}                      & $\set{5}$\\[\rowskip]
            \scriptsize{\phantom{0}7}\phantom{ } & $\ \set{\nec\beta}\vdash\nec\beta$                                                                                                     & $\hyperref[modal.rule.1]{\mathbf{R_1}}$        & \\[\rowskip]
            \scriptsize{\phantom{0}8}\phantom{ } & $\ \set{\nec\beta}\vdash\nec\beta \to \beta$                                                                                           & $\hyperref[modal.axiom.modal.2]{\mathbf{B_2}}$ & \\[\rowskip]
            \scriptsize{\phantom{0}9}\phantom{ } & $\ \set{\nec\beta}\vdash\beta$                                                                                                         & $\hyperref[modal.rule.2]{\mathbf{R_2}}$        & $\set{7,8}$\\[\rowskip]
            \scriptsize{10}\phantom{ }           & $\ \set{\nec\beta}\vdash\beta\to\alpha\vee\beta$                                                                                       & $\hyperref[modal.axiom.7]{\mathbf{A_7}}$       & \\[\rowskip]
            \scriptsize{11}\phantom{ }           & $\ \set{\nec\beta}\vdash\alpha\vee\beta$                                                                                               & $\hyperref[modal.rule.2]{\mathbf{R_2}}$        & $\set{9,10}$\\[\rowskip]
            \scriptsize{12}\phantom{ }           & $\ \set{\nec\beta}\vdash\nec(\alpha\vee\beta)$                                                                                         & \refer{generalization}{T}                      & $\set{11}$\\[\rowskip]
            \scriptsize{13}\phantom{ }           & $\ \vdash\nec\alpha\to\nec(\alpha\vee\beta)$                                                                                           & \refer{deduction}{T}                           & $\set{6}$\\[\rowskip]
            \scriptsize{14}\phantom{ }           & $\ \vdash\nec\beta\to\nec(\alpha\vee\beta)$                                                                                            & \refer{deduction}{T}                           & $\set{12}$\\[\rowskip]
            \scriptsize{15}\phantom{ }           & $\ \vdash(\nec\alpha\to\nec(\alpha\vee\beta))\to(\nec\beta\to\nec(\alpha\vee\beta))\to\nec\alpha\vee\nec\beta\to\nec(\alpha\vee\beta)$ & $\hyperref[modal.axiom.8]{\mathbf{A_8}}$       & \\[\rowskip]
            \scriptsize{16}\phantom{ }           & $\ \vdash(\nec\beta\to\nec(\alpha\vee\beta))\to\nec\alpha\vee\nec\beta\to\nec(\alpha\vee\beta)$                                        & $\hyperref[modal.rule.2]{\mathbf{R_2}}$        & $\set{13,15}$\\[\rowskip]
            \scriptsize{17}\phantom{ }           & $\ \vdash\nec\alpha\vee\nec\beta\to\nec(\alpha\vee\beta)$                                                                              & $\hyperref[modal.rule.2]{\mathbf{R_2}}$        & $\set{14,16}$\\[\rowskip]
            \scriptsize{18}\phantom{ }           & $\ \Gamma\vdash\nec\alpha\vee\nec\beta\to\nec(\alpha\vee\beta)$                                                                        & \refer{weakening}{T}\phantom{1}                & $\set{17}$
        \end{xltabular}
        \normalsize

        \vspace{0.5\baselineskip}
        Estando assim demonstrada a proposição.
        \end{proof}
    \end{lemma}
\end{tcolorbox}
\vspace{.5\baselineskip}
\begin{tcolorbox}[enhanced jigsaw, breakable, sharp corners, colframe=black, colback=white, boxrule=0.5pt, left=1.5mm, right=1.5mm, top=1.5mm, bottom=1.5mm]
\begin{theorem}\label{isomorphism}
    $\Gamma\entails_\mathfrak{L}\nec\alpha^\circ\leftrightarrow\alpha^\medsquare$.

    \begin{proof}
        Prova por indução forte sobre a profundidade da sentença \citep{Troelstra}.
        Seja $n\in\mathbb{N}^+$ a profundidade da sentença $\alpha\in\mathcal{L}$.
        Suponhamos que a asserção valha para qualquer sentença de profundidade menor que $n$ e nomeemos esta suposição $\mathbf{H}$.
        Devemos considerar cinco casos: a letra, a contradição, a conjunção, a disjunção e a implicação.

        \vspace{.5\baselineskip}
        \textsc{Caso 1.}
        Se a sentença $\alpha$ for uma proposição $a\in\mathcal{P}$, sabe-se que $\nec a^\circ=\nec a$ e que $a^\medsquare=\nec a$ pelas definições das traduções.
        Deste modo, tanto a ida quanto a volta da bi-implicação possuem a forma $\nec a\to\nec a$ e podem ser provadas pelo lema \hyperref[identity]{$\mathbf{L_\getrefnumber{identity}}$}.
        Ambas as implicações posteriormente podem ser unidas em uma bi-implicação por meio das regras \hyperref[MA3]{$\mathbf{A_3}$} e \hyperref[detachment]{$\mathbf{R_2}$}.

        \vspace{.5\baselineskip}
        \textsc{Caso 2.}
        Se a sentença $\alpha$ for a constante $\bot$, sabe-se que $\nec\bot^\circ=\nec\bot$ e que $\bot^\medsquare=\bot$ pelas definições das traduções.
        Deste modo, a ida $\nec\bot\to\bot$ da bi-implicação constitui um axioma gerado pela regra \hyperref[MB2]{$\mathbf{B_2}$}.
        A volta $\bot\to\nec\bot$ da bi-implicação pode ser provada pelo lema \hyperref[explosion]{$\mathbf{L_3}$}.
        Ambas as implicações posteriormente podem ser unidas em uma bi-implicação por meio das regras \hyperref[MA3]{$\mathbf{A_3}$} e \hyperref[detachment]{$\mathbf{R_2}$}.

        \vspace{.5\baselineskip}
        \textsc{Caso 3.}
        Seja a sentença $\alpha$ a conjunção de duas sentenças ${\alpha_1}$ e ${\alpha_2}$.
        Sabe-se que $\nec{({\alpha_1}\wedge{\alpha_2})}^\circ=\nec({\alpha_1}^\circ\wedge{\alpha_2}^\circ)$ e que ${({\alpha_1}\wedge{\alpha_2})}^\medsquare={\alpha_1}^\medsquare\wedge{\alpha_2}^\medsquare$ pelas definições das traduções.
        A partir de $\mathbf{H}$, temos que $\Gamma\entails\nec{\alpha_1}^\circ\leftrightarrow{\alpha_1}^\medsquare$ e que $\Gamma\entails\nec{\alpha_2}^\circ\leftrightarrow{\alpha_2}^\medsquare$, ditos $\mathbf{H_1}$ e $\mathbf{H_2}$.
        Pode-se demonstrar $\Gamma\entails\nec\alpha^\circ\leftrightarrow\alpha^\medsquare$ por meio da dedução que segue.

        \vspace{.5\baselineskip}
        \footnotesize
        \setlength{\rowskip}{.5\baselineskip}
        \begin{xltabular}{\textwidth}{r | X l l}
            \scriptsize{\phantom{0}1}\phantom{ } & $\ \Gamma\vdash\nec{\alpha_1}^\circ\leftrightarrow{\alpha_1}^\medsquare$                                                                                                                                                      & $\mathbf{H_1}$\phantom{1}                & \phantom{$\set{00,00}$}\\[\rowskip]
            \scriptsize{\phantom{0}2}\phantom{ } & $\ \Gamma\vdash\nec{\alpha_2}^\circ\leftrightarrow{\alpha_2}^\medsquare$                                                                                                                                                      & $\mathbf{H_2}$                           & \\[\rowskip]
            \scriptsize{\phantom{0}3}\phantom{ } & $\ \Gamma\vdash(\nec{\alpha_1}^\circ\leftrightarrow{\alpha_1}^\medsquare)\to\nec{\alpha_1}^\circ\to{\alpha_1}^\medsquare$                                                                                                     & $\hyperref[modal.axiom.4]{\mathbf{A_4}}$ & \\[\rowskip]
            \scriptsize{\phantom{0}4}\phantom{ } & $\ \Gamma\vdash(\nec{\alpha_2}^\circ\leftrightarrow{\alpha_2}^\medsquare)\to\nec{\alpha_2}^\circ\to{\alpha_2}^\medsquare$                                                                                                     & $\hyperref[modal.axiom.4]{\mathbf{A_4}}$ & \\[\rowskip]
            \scriptsize{\phantom{0}5}\phantom{ } & $\ \Gamma\vdash(\nec{\alpha_1}^\circ\leftrightarrow{\alpha_1}^\medsquare)\to{\alpha_1}^\medsquare \to \nec{\alpha_1}^\circ$                                                                                                   & $\hyperref[modal.axiom.5]{\mathbf{A_5}}$ & \\[\rowskip]
            \scriptsize{\phantom{0}6}\phantom{ } & $\ \Gamma\vdash(\nec{\alpha_2}^\circ\leftrightarrow{\alpha_2}^\medsquare)\to{\alpha_2}^\medsquare \to \nec{\alpha_2}^\circ$                                                                                                   & $\hyperref[modal.axiom.5]{\mathbf{A_5}}$ & \\[\rowskip]
            \scriptsize{\phantom{0}7}\phantom{ } & $\ \Gamma\vdash\nec{\alpha_1}^\circ\to{\alpha_1}^\medsquare$                                                                                                                                                                  & $\hyperref[modal.rule.2]{\mathbf{R_2}}$  & $\set{1,3}$\\[\rowskip]
            \scriptsize{\phantom{0}8}\phantom{ } & $\ \Gamma\vdash\nec{\alpha_2}^\circ\to{\alpha_2}^\medsquare$                                                                                                                                                                  & $\hyperref[modal.rule.2]{\mathbf{R_2}}$  & $\set{2,4}$\\[\rowskip]
            \scriptsize{\phantom{0}9}\phantom{ } & $\ \Gamma\vdash{\alpha_1}^\medsquare\to\nec{\alpha_1}^\circ$                                                                                                                                                                  & $\hyperref[modal.rule.2]{\mathbf{R_2}}$  & $\set{1,5}$\\[\rowskip]
            \scriptsize{10}\phantom{ }           & $\ \Gamma\vdash{\alpha_2}^\medsquare\to\nec{\alpha_2}^\circ$                                                                                                                                                                  & $\hyperref[modal.rule.2]{\mathbf{R_2}}$  & $\set{2,6}$\\[\rowskip]
            \scriptsize{11}\phantom{ }           & $\ \Gamma\cup\set{\nec({\alpha_1}^\circ\wedge{\alpha_2}^\circ)} \vdash \nec{\alpha_1}^\circ \to {\alpha_1}^\medsquare$                                                                                                        & \refer{weakening}{T}                     & $\set{7}$\\[\rowskip]
            \scriptsize{12}\phantom{ }           & $\ \Gamma\cup\set{\nec({\alpha_1}^\circ\wedge{\alpha_2}^\circ)} \vdash \nec{\alpha_2}^\circ \to {\alpha_2}^\medsquare$                                                                                                        & \refer{weakening}{T}                     & $\set{8}$\\[\rowskip]
            \scriptsize{13}\phantom{ }           & $\ \Gamma\cup\set{\nec({\alpha_1}^\circ\wedge{\alpha_2}^\circ)} \vdash \nec({\alpha_1}^\circ \wedge {\alpha_2}^\circ)$                                                                                                        & $\hyperref[modal.rule.1]{\mathbf{R_1}}$  & \\[\rowskip]
            \scriptsize{14}\phantom{ }           & $\ \Gamma\cup\set{\nec({\alpha_1}^\circ\wedge{\alpha_2}^\circ)} \vdash \nec({\alpha_1}^\circ \wedge {\alpha_2}^\circ) \to \nec{\alpha_1}^\circ \wedge \nec{\alpha_2}^\circ$                                                   & \refer{necessity.conjunction.distribution}{L} & \\[\rowskip]
            \scriptsize{15}\phantom{ }           & $\ \Gamma\cup\set{\nec({\alpha_1}^\circ\wedge{\alpha_2}^\circ)} \vdash \nec{\alpha_1}^\circ \wedge \nec{\alpha_2}^\circ$                                                                                                      & $\hyperref[modal.rule.2]{\mathbf{R_2}}$ & $\set{13,14}$\\[\rowskip]
            \scriptsize{16}\phantom{ }           & $\ \Gamma\cup\set{\nec({\alpha_1}^\circ\wedge{\alpha_2}^\circ)} \vdash \nec{\alpha_1}^\circ \wedge \nec{\alpha_2}^\circ \to {\alpha_1}^\medsquare \wedge {\alpha_2}^\medsquare$                                               & \refer{conjunction.exchange}{L} & $\set{11,12}$\\[\rowskip]
            \scriptsize{17}\phantom{ }           & $\ \Gamma\cup\set{\nec({\alpha_1}^\circ\wedge{\alpha_2}^\circ)} \vdash {\alpha_1}^\medsquare \wedge {\alpha_2}^\medsquare$                                                                                                    & $\hyperref[modal.rule.2]{\mathbf{R_2}}$ & $\set{15,16}$\\[\rowskip]
            \scriptsize{18}\phantom{ }           & $\ \Gamma\cup\set{{\alpha_1}^\medsquare\wedge{\alpha_2}^\medsquare} \vdash {\alpha_1}^\medsquare \to \nec{\alpha_1}^\circ$                                                                                                    & \refer{weakening}{T} & $\set{9}$\\[\rowskip]
            \scriptsize{19}\phantom{ }           & $\ \Gamma\cup\set{{\alpha_1}^\medsquare\wedge{\alpha_2}^\medsquare} \vdash {\alpha_2}^\medsquare \to \nec{\alpha_2}^\circ$                                                                                                    & \refer{weakening}{T} & $\set{10}$\\[\rowskip]
            \scriptsize{20}\phantom{ }           & $\ \Gamma\cup\set{{\alpha_1}^\medsquare\wedge{\alpha_2}^\medsquare} \vdash {\alpha_1}^\medsquare \wedge {\alpha_2}^\medsquare$                                                                                                & $\hyperref[modal.rule.1]{\mathbf{R_1}}$        & \\[\rowskip]
            \scriptsize{21}\phantom{ }           & $\ \Gamma\cup\set{{\alpha_1}^\medsquare\wedge{\alpha_2}^\medsquare} \vdash {\alpha_1}^\medsquare \wedge {\alpha_2}^\medsquare \to \nec{\alpha_1}^\circ \wedge \nec{\alpha_2}^\circ$                                           & \refer{conjunction.exchange}{L} & $\set{18,19}$\\[\rowskip]
            \scriptsize{22}\phantom{ }           & $\ \Gamma\cup\set{{\alpha_1}^\medsquare\wedge{\alpha_2}^\medsquare} \vdash \nec{\alpha_1}^\circ \wedge \nec{\alpha_2}^\circ$                                                                                                  & $\hyperref[modal.rule.2]{\mathbf{R_2}}$ & $\set{20,21}$\\[\rowskip]
            \scriptsize{23}\phantom{ }           & $\ \Gamma\cup\set{{\alpha_1}^\medsquare\wedge{\alpha_2}^\medsquare} \vdash \nec{\alpha_1}^\circ \wedge \nec{\alpha_2}^\circ \to \nec({\alpha_1}^\circ \wedge {\alpha_2}^\circ)$                                               & \refer{necessity.conjunction.undistribution}{L} & \\[\rowskip]
            \scriptsize{24}\phantom{ }           & $\ \Gamma\cup\set{{\alpha_1}^\medsquare\wedge{\alpha_2}^\medsquare} \vdash \nec({\alpha_1}^\circ \wedge {\alpha_2}^\circ)$                                                                                                    & $\hyperref[modal.rule.2]{\mathbf{R_2}}$ & $\set{22,23}$\\[\rowskip]
            \scriptsize{25}\phantom{ }           & $\ \Gamma\vdash\nec({\alpha_1}^\circ \wedge {\alpha_2}^\circ) \to {\alpha_1}^\medsquare \wedge {\alpha_2}^\medsquare$                                                                                                         & \refer{deduction}{T} & $\set{17}$\\[\rowskip]
            \scriptsize{26}\phantom{ }           & $\ \Gamma\vdash{\alpha_1}^\medsquare \wedge {\alpha_2}^\medsquare \to \nec({\alpha_1}^\circ \wedge {\alpha_2}^\circ)$                                                                                                         & \refer{deduction}{T} & $\set{24}$\\[\rowskip]
            \scriptsize{27}\phantom{ }           & $\ \Gamma\vdash(\nec\alpha^\circ \to \alpha^\medsquare) \to (\alpha^\medsquare \to \nec\alpha^\circ) \to (\nec({\alpha_1}^\circ \wedge {\alpha_2}^\circ) \leftrightarrow {\alpha_1}^\medsquare \wedge {\alpha_2}^\medsquare)$ & $\hyperref[modal.axiom.4]{\mathbf{A_4}}$ & \\[\rowskip]
            \scriptsize{28}\phantom{ }           & $\ \Gamma\vdash(\alpha^\medsquare \to \nec\alpha^\circ) \to (\nec({\alpha_1}^\circ \wedge {\alpha_2}^\circ) \leftrightarrow {\alpha_1}^\medsquare \wedge {\alpha_2}^\medsquare)$                                              & $\hyperref[modal.rule.2]{\mathbf{R_2}}$ & $\set{25,27}$\\[\rowskip]
            \scriptsize{29}\phantom{ }           & $\ \Gamma\vdash\nec({\alpha_1}^\circ \wedge {\alpha_2}^\circ) \leftrightarrow {\alpha_1}^\medsquare \wedge {\alpha_2}^\medsquare$                                                                                             & $\hyperref[modal.rule.2]{\mathbf{R_2}}$ & $\set{26,28}$
        \end{xltabular}
        \normalsize

        \vspace{.5\baselineskip}
        \textsc{Caso 4.}
        Seja a sentença $\alpha$ a disjunção de duas sentenças ${\alpha_1}$ e ${\alpha_2}$.
        Sabe-se que $\nec{({\alpha_1}\vee\psi)}^\circ=\nec(\nec{\alpha_1}^\circ\vee\nec\psi^\circ)$ e que ${({\alpha_1}\vee\psi)}^\medsquare={\alpha_1}^\medsquare\vee\psi^\medsquare$ pelas definições das traduções.
        A partir de $\mathbf{H}$, temos que $\Gamma\entails\nec{\alpha_1}^\circ\leftrightarrow{\alpha_1}^\medsquare$ e que $\Gamma\entails\nec{\alpha_2}^\circ\leftrightarrow{\alpha_2}^\medsquare$, ditos $\mathbf{H_1}$ e $\mathbf{H_2}$.
        Pode-se demonstrar $\Gamma\entails\nec\alpha^\circ\leftrightarrow\alpha^\medsquare$ por meio da dedução que segue.

        \vspace{.5\baselineskip}
        \footnotesize
        \setlength{\rowskip}{.5\baselineskip}
        \begin{xltabular}{\textwidth}{r | X l l}
            \scriptsize{\phantom{0}1}\phantom{ } & $\ \Gamma \vdash \nec{\alpha_1}^\circ \leftrightarrow {\alpha_1}^\medsquare$ & $\mathbf{H_1}$ & \\[\rowskip]
            \scriptsize{\phantom{0}2}\phantom{ } & $\ \Gamma \vdash \nec{\alpha_2}^\circ \leftrightarrow {\alpha_2}^\medsquare$ & $\mathbf{H_2}$ & \\[\rowskip]
            \scriptsize{\phantom{0}3}\phantom{ } & $\ \Gamma \vdash (\nec{\alpha_1}^\circ \leftrightarrow {\alpha_1}^\medsquare) \to \nec{\alpha_1}^\circ \to {\alpha_1}^\medsquare$ & $\hyperref[modal.axiom.4]{\mathbf{A_4}}$ & \\[\rowskip]
            \scriptsize{\phantom{0}4}\phantom{ } & $\ \Gamma \vdash (\nec{\alpha_2}^\circ \leftrightarrow {\alpha_2}^\medsquare) \to \nec{\alpha_2}^\circ \to {\alpha_2}^\medsquare$ & $\hyperref[modal.axiom.4]{\mathbf{A_4}}$ & \\[\rowskip]
            \scriptsize{\phantom{0}5}\phantom{ } & $\ \Gamma \vdash (\nec{\alpha_1}^\circ \leftrightarrow {\alpha_1}^\medsquare) \to {\alpha_1}^\medsquare \to \nec{\alpha_1}^\circ$ & $\hyperref[modal.axiom.5]{\mathbf{A_5}}$ & \\[\rowskip]
            \scriptsize{\phantom{0}6}\phantom{ } & $\ \Gamma \vdash (\nec{\alpha_2}^\circ \leftrightarrow {\alpha_2}^\medsquare) \to {\alpha_2}^\medsquare \to \nec{\alpha_2}^\circ$ & $\hyperref[modal.axiom.5]{\mathbf{A_5}}$ & \\[\rowskip]
            \scriptsize{\phantom{0}7}\phantom{ } & $\ \Gamma \vdash \nec{\alpha_1}^\circ \to {\alpha_1}^\medsquare$ & $\hyperref[modal.rule.2]{\mathbf{R_2}}$ & $\set{1,3}$\\[\rowskip]
            \scriptsize{\phantom{0}8}\phantom{ } & $\ \Gamma \vdash \nec{\alpha_2}^\circ \to {\alpha_2}^\medsquare$ & $\hyperref[modal.rule.2]{\mathbf{R_2}}$ & $\set{2,4}$\\[\rowskip]
            \scriptsize{\phantom{0}9}\phantom{ } & $\ \Gamma \vdash {\alpha_1}^\medsquare \to \nec{\alpha_1}^\circ$ & $\hyperref[modal.rule.2]{\mathbf{R_2}}$ & $\set{1,5}$\\[\rowskip]
            \scriptsize{10}\phantom{ } & $\ \Gamma \vdash {\alpha_2}^\medsquare \to \nec{\alpha_2}^\circ$ & $\hyperref[modal.rule.2]{\mathbf{R_2}}$ & $\set{2,6}$\\[\rowskip]
            \scriptsize{11}\phantom{ } & $\ \Gamma \cup \set{\nec(\nec{\alpha_1}^\circ \vee \nec{\alpha_2}^\circ)} \vdash \nec{\alpha_1}^\circ \to {\alpha_1}^\medsquare$ & \refer{weakening}{T} & $\set{7}$\\[\rowskip]
            \scriptsize{12}\phantom{ } & $\ \Gamma \cup \set{\nec(\nec{\alpha_1}^\circ \vee \nec{\alpha_2}^\circ)} \vdash \nec{\alpha_2}^\circ \to {\alpha_2}^\medsquare$ & \refer{weakening}{T} & $\set{8}$\\[\rowskip]
            \scriptsize{13}\phantom{ } & $\ \Gamma \cup \set{\nec(\nec{\alpha_1}^\circ \vee \nec{\alpha_2}^\circ)} \vdash \nec(\nec{\alpha_1}^\circ \vee \nec{\alpha_2}^\circ)$ & $\hyperref[modal.rule.1]{\mathbf{R_1}}$ & \\[\rowskip]
            \scriptsize{14}\phantom{ } & $\ \Gamma \cup \set{\nec(\nec{\alpha_1}^\circ \vee \nec{\alpha_2}^\circ)} \vdash \nec(\nec{\alpha_1}^\circ \vee \nec{\alpha_2}^\circ) \to \nec{\alpha_1}^\circ \vee \nec{\alpha_2}^\circ$ & $\hyperref[modal.axiom.modal.2]{\mathbf{B_2}}$ & \\[\rowskip]
            \scriptsize{15}\phantom{ } & $\ \Gamma \cup \set{\nec(\nec{\alpha_1}^\circ \vee \nec{\alpha_2}^\circ)} \vdash \nec{\alpha_1}^\circ \vee \nec{\alpha_2}^\circ$ & $\hyperref[modal.rule.2]{\mathbf{R_2}}$ & $\set{13,14}$\\[\rowskip]
            \scriptsize{16}\phantom{ } & $\ \Gamma \cup \set{\nec(\nec{\alpha_1}^\circ \vee \nec{\alpha_2}^\circ)} \vdash \nec{\alpha_1}^\circ \vee \nec{\alpha_2}^\circ \to {\alpha_1}^\medsquare \vee {\alpha_2}^\medsquare$ & \refer{disjunction.exchange}{L} & $\set{11,12}$\\[\rowskip]
            \scriptsize{17}\phantom{ } & $\ \Gamma \cup \set{\nec(\nec{\alpha_1}^\circ \vee \nec{\alpha_2}^\circ)} \vdash {\alpha_1}^\medsquare \vee {\alpha_2}^\medsquare$ & $\hyperref[modal.rule.2]{\mathbf{R_2}}$ & $\set{15,16}$\\[\rowskip]
            \scriptsize{19}\phantom{ } & $\ \Gamma \cup \set{{\alpha_1}^\medsquare \vee {\alpha_2}^\medsquare} \vdash {\alpha_1}^\medsquare \to \nec{\alpha_1}^\circ$ & \refer{weakening}{T} & $\set{9}$\\[\rowskip]
            \scriptsize{20}\phantom{ } & $\ \Gamma \cup \set{{\alpha_1}^\medsquare \vee {\alpha_2}^\medsquare} \vdash {\alpha_2}^\medsquare \to \nec{\alpha_2}^\circ$ & \refer{weakening}{T} & $\set{10}$\\[\rowskip]
            \scriptsize{21}\phantom{ } & $\ \Gamma \cup \set{{\alpha_1}^\medsquare \vee {\alpha_2}^\medsquare} \vdash \nec{\alpha_1}^\circ \to \nec\nec{\alpha_1}^\circ$ & $\hyperref[modal.axiom.modal.3]{\mathbf{B_3}}$ & \\[\rowskip]
            \scriptsize{22}\phantom{ } & $\ \Gamma \cup \set{{\alpha_1}^\medsquare \vee {\alpha_2}^\medsquare} \vdash \nec{\alpha_2}^\circ \to \nec\nec{\alpha_2}^\circ$ & $\hyperref[modal.axiom.modal.3]{\mathbf{B_3}}$ & \\[\rowskip]
            \scriptsize{23}\phantom{ } & $\ \Gamma \cup \set{{\alpha_1}^\medsquare \vee {\alpha_2}^\medsquare} \vdash {\alpha_1}^\medsquare \to \nec\nec{\alpha_1}^\circ$ & \refer{composition}{L} & $\set{18,20}$\\[\rowskip]
            \scriptsize{24}\phantom{ } & $\ \Gamma \cup \set{{\alpha_1}^\medsquare \vee {\alpha_2}^\medsquare} \vdash {\alpha_2}^\medsquare \to \nec\nec{\alpha_2}^\circ$ & \refer{composition}{L} & $\set{19,21}$\\[\rowskip]
            \scriptsize{25}\phantom{ } & $\ \Gamma \cup \set{{\alpha_1}^\medsquare \vee {\alpha_2}^\medsquare} \vdash {\alpha_1}^\medsquare \vee {\alpha_2}^\medsquare$ & $\hyperref[modal.rule.1]{\mathbf{R_1}}$ & \\[\rowskip]
            \scriptsize{26}\phantom{ } & $\ \Gamma \cup \set{{\alpha_1}^\medsquare \vee {\alpha_2}^\medsquare} \vdash {\alpha_1}^\medsquare \vee {\alpha_2}^\medsquare \to \nec\nec{\alpha_1}^\circ \vee \nec\nec{\alpha_2}^\circ$ & \refer{disjunction.exchange}{L} & $\set{22,23}$\\[\rowskip]
            \scriptsize{27}\phantom{ } & $\ \Gamma \cup \set{{\alpha_1}^\medsquare \vee {\alpha_2}^\medsquare} \vdash \nec\nec{\alpha_1}^\circ \vee \nec\nec{\alpha_2}^\circ$ & $\hyperref[modal.rule.2]{\mathbf{R_2}}$ & $\set{24,25}$\\[\rowskip]
            \scriptsize{28}\phantom{ } & $\ \Gamma \cup \set{{\alpha_1}^\medsquare \vee {\alpha_2}^\medsquare} \vdash \nec\nec{\alpha_1}^\circ \vee \nec\nec{\alpha_2}^\circ \to \nec(\nec{\alpha_1}^\circ \vee \nec{\alpha_2}^\circ)$ & \refer{necessity.disjunction.undistribution}{L} & \\[\rowskip]
            \scriptsize{29}\phantom{ } & $\ \Gamma \cup \set{{\alpha_1}^\medsquare \vee {\alpha_2}^\medsquare} \vdash \nec(\nec{\alpha_1}^\circ \vee \nec{\alpha_2}^\circ)$ & $\hyperref[modal.rule.2]{\mathbf{R_2}}$ & $\set{26,27}$\\[\rowskip]
            \scriptsize{30}\phantom{ } & $\ \Gamma \vdash \nec(\nec{\alpha_1}^\circ \vee \nec{\alpha_2}^\circ) \to {\alpha_1}^\medsquare \vee {\alpha_2}^\medsquare$ & \refer{deduction}{T} & $\set{17}$\\[\rowskip]
            \scriptsize{31}\phantom{ } & $\ \Gamma \vdash {\alpha_1}^\medsquare \vee {\alpha_2}^\medsquare \to \nec(\nec{\alpha_1}^\circ \vee \nec{\alpha_2}^\circ)$ & \refer{deduction}{T} & $\set{28}$\\[\rowskip]
            \scriptsize{32}\phantom{ } & $\ \Gamma \vdash (\nec\alpha^\circ \to \alpha^\medsquare) \to (\alpha^\medsquare \to \nec\alpha^\circ) \to (\nec(\nec{\alpha_1}^\circ \vee \nec{\alpha_2}^\circ) \leftrightarrow {\alpha_1}^\medsquare \vee {\alpha_2}^\medsquare)$ & $\hyperref[modal.axiom.4]{\mathbf{A_4}}$ & \\[\rowskip]
            \scriptsize{33}\phantom{ } & $\ \Gamma \vdash (\alpha^\medsquare \to \nec\alpha^\circ) \to (\nec(\nec{\alpha_1}^\circ \vee \nec{\alpha_2}^\circ) \leftrightarrow {\alpha_1}^\medsquare \vee {\alpha_2}^\medsquare)$ & $\hyperref[modal.rule.2]{\mathbf{R_2}}$ & $\set{29,31}$\\[\rowskip]
            \scriptsize{34}\phantom{ } & $\ \Gamma \vdash \nec(\nec{\alpha_1}^\circ \vee \nec{\alpha_2}^\circ) \leftrightarrow {\alpha_1}^\medsquare \vee {\alpha_2}^\medsquare$ & $\hyperref[modal.rule.2]{\mathbf{R_2}}$ & $\set{30,32}$\
        \end{xltabular}
        \normalsize

        \vspace{.5\baselineskip}
            \textsc{Caso 5.}
            Seja a sentença $\alpha$ a implicação de duas sentenças ${\alpha_1}$ e ${\alpha_2}$.
            Sabe-se que $\nec{({\alpha_1}\to{\alpha_2})}^\circ=\nec(\nec{\alpha_1}^\circ\to{\alpha_2}^\circ)$ e que ${({\alpha_1}\to{\alpha_2})}^\medsquare=\nec({\alpha_1}^\medsquare\to{\alpha_2}^\medsquare)$ pelas definições das traduções.
            A partir de $\mathbf{H}$, temos que $\Gamma\entails\nec{\alpha_1}^\circ\leftrightarrow{\alpha_1}^\medsquare$ e que $\Gamma\entails\nec{\alpha_2}^\circ\leftrightarrow{\alpha_2}^\medsquare$, ditos $\mathbf{H_1}$ e $\mathbf{H_2}$.
            Pode-se demonstrar $\Gamma\entails\nec\alpha^\circ\leftrightarrow\alpha^\medsquare$ por meio da dedução que segue.

                \footnotesize
                \begin{fitch}
                    \fb\set{\nec(\nec{\alpha_1}^\circ\to{\alpha_2}^\circ),{\alpha_1}^\medsquare}\entails{\alpha_1}^\medsquare&$\hyperref[premisse]{\mathbf{R_1}}$\\
                    \fa\set{\nec(\nec{\alpha_1}^\circ\to{\alpha_2}^\circ),{\alpha_1}^\medsquare}\entails{\alpha_1}^\medsquare\to\nec{\alpha_1}^\circ&$\mathbf{H_1}$\\
                    \fa\set{\nec(\nec{\alpha_1}^\circ\to{\alpha_2}^\circ),\nec{\alpha_1}^\circ}\entails\nec{\alpha_1}^\circ&$\hyperref[premisse]{\mathbf{R_1}}$\\
                    \fa\set{\nec(\nec{\alpha_1}^\circ\to{\alpha_2}^\circ),\nec{\alpha_1}^\circ}\entails\nec(\nec{\alpha_1}^\circ\to{\alpha_2}^\circ)&$\hyperref[premisse]{\mathbf{R_1}}$\\
                    \fa\set{\nec(\nec{\alpha_1}^\circ\to{\alpha_2}^\circ),\nec{\alpha_1}^\circ}\entails\nec(\nec{\alpha_1}^\circ\to{\alpha_2}^\circ)\to\nec{\alpha_1}^\circ\to{\alpha_2}^\circ&\hyperref[MB2]{${\mathbf{B_2}}$}\\
                    \fa\set{\nec(\nec{\alpha_1}^\circ\to{\alpha_2}^\circ),\nec{\alpha_1}^\circ}\entails\nec{\alpha_1}^\circ\to{\alpha_2}^\circ&$\hyperref[detachment]{\mathbf{R_2}}\;\set{4,5}$\\
                    \fa\set{\nec(\nec{\alpha_1}^\circ\to{\alpha_2}^\circ),\nec{\alpha_1}^\circ}\entails{\alpha_2}^\circ&$\hyperref[detachment]{\mathbf{R_2}}\;\set{3,6}$\\
                    \fa\set{\nec(\nec{\alpha_1}^\circ\to{\alpha_2}^\circ),\nec{\alpha_1}^\circ}\entails\nec{\alpha_2}^\circ&$\hyperref[gen-nec]{\mathbf{T_\getrefnumber{gen-nec}}}\;\set{7}$\\
                    \fa\set{\nec(\nec{\alpha_1}^\circ\to{\alpha_2}^\circ)}\entails\nec{\alpha_1}^\circ\to\nec{\alpha_2}^\circ&$\hyperref[deduction]{\mathbf{T_\getrefnumber{deduction}}}\;\set{8}$\\
                    \fa\set{\nec(\nec{\alpha_1}^\circ\to{\alpha_2}^\circ),{\alpha_1}^\medsquare}\entails\nec{\alpha_1}^\circ\to\nec{\alpha_2}^\circ&$\hyperref[weakening]{\mathbf{T_\getrefnumber{weakening}}}\;\set{9}$\\
                    \fa\set{\nec(\nec{\alpha_1}^\circ\to{\alpha_2}^\circ),{\alpha_1}^\medsquare}\entails{\alpha_1}^\medsquare\to\nec{\alpha_2}^\circ&\refer{comp}{L}$\;\set{2,10}$\\
                    \fa\set{\nec(\nec{\alpha_1}^\circ\to{\alpha_2}^\circ),{\alpha_1}^\medsquare}\entails\nec{\alpha_2}^\circ\to{\alpha_2}^\medsquare&$\mathbf{H_2}$\\
                    \fa\set{\nec(\nec{\alpha_1}^\circ\to{\alpha_2}^\circ),{\alpha_1}^\medsquare}\entails{\alpha_1}^\medsquare\to{\alpha_2}^\medsquare&\refer{comp}{L}$\;\set{11,12}$\\
                    \fa\set{\nec(\nec{\alpha_1}^\circ\to{\alpha_2}^\circ),{\alpha_1}^\medsquare}\entails{\alpha_2}^\medsquare&$\hyperref[detachment]{\mathbf{R_2}}\;\set{1,13}$\\
                    \fa\set{\nec(\nec{\alpha_1}^\circ\to{\alpha_2}^\circ)}\entails\nec({\alpha_1}^\medsquare\to{\alpha_2}^\medsquare)&$\hyperref[gen-nec]{\mathbf{T_{\getrefnumber{gen-nec}}}}\;\set{14}$\\
                    \fa\entails\nec(\nec{\alpha_1}^\circ\to{\alpha_2}^\circ)\to\nec({\alpha_1}^\medsquare\to{\alpha_2}^\medsquare)&$\hyperref[deduction]{\mathbf{T_\getrefnumber{deduction}}}\;\set{15}$
                \end{fitch}

                \footnotesize
                \begin{fitch}
                    \fb\set{\nec({\alpha_1}^\medsquare\to{\alpha_2}^\medsquare),\nec{\alpha_1}^\circ}\entails\nec{\alpha_1}^\circ&$\hyperref[premisse]{\mathbf{R_1}}$\\
                    \fa\set{\nec({\alpha_1}^\medsquare\to{\alpha_2}^\medsquare),\nec{\alpha_1}^\circ}\entails\nec{\alpha_1}^\circ\to{\alpha_1}^\medsquare&$\mathbf{H_1}$\\
                    \fa\set{\nec({\alpha_1}^\medsquare\to{\alpha_2}^\medsquare),\nec{\alpha_1}^\circ}\entails\nec({\alpha_1}^\medsquare\to{\alpha_2}^\medsquare)&$\hyperref[premisse]{\mathbf{R_1}}$\\
                    \fa\set{\nec({\alpha_1}^\medsquare\to{\alpha_2}^\medsquare),\nec{\alpha_1}^\circ}\entails\nec({\alpha_1}^\medsquare\to{\alpha_2}^\medsquare)\to{\alpha_1}^\medsquare\to{\alpha_2}^\medsquare&\hyperref[MB2]{${\mathbf{B_2}}$}\\
                    \fa\set{\nec({\alpha_1}^\medsquare\to{\alpha_2}^\medsquare),\nec{\alpha_1}^\circ}\entails{\alpha_1}^\medsquare\to{\alpha_2}^\medsquare&$\hyperref[detachment]{\mathbf{R_2}}\;\set{2,3}$\\
                    \fa\set{\nec({\alpha_1}^\medsquare\to{\alpha_2}^\medsquare),\nec{\alpha_1}^\circ}\entails\nec{\alpha_1}^\circ\to{\alpha_2}^\medsquare&\refer{comp}{L}$\;\set{2,5}$\\
                    \fa\set{\nec({\alpha_1}^\medsquare\to{\alpha_2}^\medsquare),\nec{\alpha_1}^\circ}\entails{\alpha_2}^\medsquare&$\hyperref[detachment]{\mathbf{R_2}}\;\set{1,6}$\\
                    \fa\set{\nec({\alpha_1}^\medsquare\to{\alpha_2}^\medsquare),\nec{\alpha_1}^\circ}\entails{\alpha_2}^\medsquare\to\nec{\alpha_2}^\circ&$\mathbf{H_2}$\\
                    \fa\set{\nec({\alpha_1}^\medsquare\to{\alpha_2}^\medsquare),\nec{\alpha_1}^\circ}\entails\nec{\alpha_2}^\circ&$\hyperref[detachment]{\mathbf{R_2}}\;\set{7,8}$\\
                    \fa\set{\nec({\alpha_1}^\medsquare\to{\alpha_2}^\medsquare),\nec{\alpha_1}^\circ}\entails\nec{\alpha_2}^\circ\to{\alpha_2}^\circ&\hyperref[MB2]{${\mathbf{B_2}}$}\\
                    \fa\set{\nec({\alpha_1}^\medsquare\to{\alpha_2}^\medsquare),\nec{\alpha_1}^\circ}\entails{\alpha_2}^\circ&$\hyperref[detachment]{\mathbf{R_2}}\;\set{9,10}$\\
                    \fa\set{\nec({\alpha_1}^\medsquare\to{\alpha_2}^\medsquare)}\entails\nec(\nec{\alpha_1}^\circ\to{\alpha_2}^\circ)&$\hyperref[strictdeduction]{\mathbf{T_\getrefnumber{strictdeduction}}}\;\set{11}$\\
                    \fa\entails\nec({\alpha_1}^\medsquare\to{\alpha_2}^\medsquare)\to\nec(\nec{\alpha_1}^\circ\to{\alpha_2}^\circ)&$\hyperref[deduction]{\mathbf{T_\getrefnumber{deduction}}}\;\set{12}$
                \end{fitch}

        \vspace{.5\baselineskip}
        Estando assim demonstrada a proposição.
    \end{proof}
\end{theorem}
\end{tcolorbox}

\begin{tcolorbox}[enhanced jigsaw, breakable, sharp corners, colframe=black, colback=white, boxrule=0.5pt, left=1.5mm, right=1.5mm, top=1.5mm, bottom=1.5mm]
\begin{theorem}[Interderivabilidade]\label{interderivability}
    $\nec\Gamma^\circ\entails_\mathfrak{L}\alpha^\circ$ se e somente se $\Gamma^\nec\entails_\mathfrak{L}\alpha^\nec$.
\end{theorem}
\end{tcolorbox}
