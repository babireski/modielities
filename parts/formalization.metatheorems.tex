\section{Metateoremas}
    Provaremos o metateorema da dedução para os sistemas modais, conforme apresentado por~\cite{Hakli}.

    \begin{theorem}
        $\text{Se }\Gamma\cup\set{\alpha}\vdash\beta\text{, então }\Gamma\vdash\alpha\to\beta$.

        \begin{proof}
            Prova por indução forte sobre o tamanho da sucessão de dedução.
            Assim, suponhamos que o teorema da dedução valha para qualquer sucessão dedução de tamanho $n\leq k$.
            Demonstraremos, analisando-se os casos, que o teorema da dedução vale para sucessões de deduções de tamanho $n=k+1$.

            \begin{case}
                \textbf{Caso 1.} Caso $\beta\in\Gamma\cup\set{\alpha}$.
            \end{case}

            \begin{case}
                \textbf{Caso 2.} Caso $\beta\in\mathcal{A}$.
            \end{case}

            \begin{case}
                \textbf{Caso 3.} Caso $\beta$ tenha sido gerado pela regra da necessitação.

                \begin{fitch}
                    \fa\vdash\varphi&$\mathbf{H}$\\
                    \fa\Gamma\vdash\nec\varphi&$\mathbf{R_2}\;\set{1}$\\
                    \fa\vdash\nec\varphi\to\alpha\to\nec\varphi&$\mathbf{A_1}$\\
                    \fa\Gamma\vdash\alpha\to\nec\varphi&$\mathbf{R_1}\;\set{2,3}$.
                \end{fitch}
            \end{case}

            \begin{case}
                \textbf{Caso 4.} Caso $\beta$ tenha sido gerado pela regra da separação.
            \end{case}
        \end{proof}
    \end{theorem}

    \begin{theorem}
        $\text{Se }\nec\Gamma\cup\set{\nec\alpha}\vdash\beta\text{, então }\nec\Gamma\vdash\nec(\nec\alpha\to\beta)$.

        \begin{proof}
            Prova em~\cite{Marcus}.
        \end{proof}
    \end{theorem}

    % \begin{theorem}
    %     $\forall\,\Gamma\cup\set{\alpha,\beta}\in\wp(\mathcal{L}_\mathbf{M})\point\Gamma\cup\set{\alpha}\vdash\beta\Rightarrow  \Gamma\vdash\nec(\alpha\to\beta)$.
    % \end{theorem}

    % \begin{proof}
    %     Seja $\Gamma$ um conjunto de sentenças e sejam $\alpha$ e $\beta$ sentenças de modo que $\Gamma\cup\set{\alpha}\vdash\beta$, deve-se provar que $\Gamma\vdash\nec(\alpha\to\beta)$. Como $\Gamma\cup\set{\alpha}\vdash\beta$, existe uma prova de $\beta$ a partir de $\Gamma\cup\set{\alpha}$. A prova baseia-se numa indução sobre o tamanho $n$ da prova.

    %     \begin{case}
    %         \textbf{Caso 1} (Base)\textbf{.}
    %         Para a base requer-se considerar os seguintes casos: 
    %             \textbf{(1)} $\beta$ consiste num axioma,
    %             \textbf{(2)} $\beta\in\Gamma$ e
    %             \textbf{(3)} $\beta=\alpha$.

    %         \begin{case}
    %             \textbf{Caso 1.1} ($\beta\in\mathcal{A}$)\textbf{.}
                
    %             \begin{fitch}
    %                 \fa\beta\to\alpha\to\beta&$\mathbf{A_1}$\\
    %                 \fa\beta&$\mathbf{A_\beta}$\\
    %                 \fa\alpha\to\beta&$\mathbf{R_1}\;\sequence{1, 2}$\\
    %                 \fa\nec(\alpha\to\beta)&$\mathbf{R_2}\;\sequence{3}$
    %             \end{fitch}
    %         \end{case}

    %         \begin{case}
    %             \textbf{Caso 1.2} ($\beta\in\Gamma$)\textbf{.}

    %             \begin{fitch}
    %                 \fa\beta\to\alpha\to\beta&$\mathbf{A_1}$\\
    %                 \fa\beta&$\mathbf{P}$\\
    %                 \fa\alpha\to\beta&$\mathbf{R_1}\;\sequence{1, 2}$\\
    %                 \fa\nec(\alpha\to\beta)&$\mathbf{R_2}\;\sequence{3}$
    %             \end{fitch}
    %         \end{case}

    %         \begin{case}
    %             \textbf{Caso 1.3} ($\beta=\alpha$)\textbf{.}
                
    %             \begin{fitch}
    %                 \fa\alpha\to\alpha&$\mathbf{L_1}$\\
    %                 \fa\nec(\alpha\to\alpha)&$\mathbf{R_2}\;\sequence{1}$
    %             \end{fitch}
    %         \end{case}
    %     \end{case}

    %     \begin{case}
    %         \textbf{Caso 2} (Passo)\textbf{.}
    %         Supõe-se que, para qualquer prova de $\Gamma\cup\set{\alpha}\vdash\beta$ com tamanho $k$, tem-se que $\Gamma\vdash\nec(\alpha\to\beta)$.
    %         Deve-se mostrar que a proposição segue verdadeira caso a prova tenha tamanho $k+1$. 
    %         Assim, sendo $\sequence{\varphi_i\mid1\leq i\leq k+1}$ uma sucessão de dedução com $\varphi_{k+1}=\beta$, requer-se considerar os seguintes casos:

    %         \begin{case}
    %             \textbf{Caso 2.1} ($\beta\in\mathcal{A}$)\textbf{.} \textit{Vide} caso $\mathbf{C_{1.1}}$.
    %         \end{case}

    %         \begin{case}
    %             \textbf{Caso 2.2} ($\beta\in\Gamma$)\textbf{.} \textit{Vide} caso $\mathbf{C_{1.2}}$.
    %         \end{case}

    %         \begin{case}
    %             \textbf{Caso 2.3} ($\beta=\alpha$)\textbf{.} \textit{Vide} caso $\mathbf{C_{1.3}}$.
    %         \end{case}

    %         \begin{case}
    %             \textbf{Caso 2.4} ($\mathbf{R_1}$)\textbf{.} \babireski{Aqui basta transcrever a prova da aula.}
    %         \end{case}

    %         \begin{case}
    %             \textbf{Caso 2.5} ($\mathbf{R_2}$)\textbf{.} \babireski{Estou sofrendo nesse caso.}
    %         \end{case}
    %     \end{case}
    % \end{proof}