\section{Derivações}
    Nesta seção apresentaremos alguns lemas e teoremas para os sistemas modais que permitirão simplificar muito as provas apresentadas no decorrer deste trabalho.
    Primeiramente, provaremos que, dada uma sentença qualquer, esta sempre implica a si mesma. A este lema daremos o nome de identidade\footnote{Em analogia ao combinador $\mathbf{I}$.} e, em seguida, usaremo-no para a prova da regra da dedução.

    \begin{lemma}\label{identity}
        $\entails\alpha\to\alpha$.
        \begin{proof}
            Pode ser provado pela seguinte sucessão de dedução:
            \footnotesize
            \begin{fitch}   
                \fb\vdash\alpha\to\alpha\to\alpha&\hyperref[MA1]{$\mathbf{A_1}$}\\
                \fa\vdash\alpha\to(\alpha\to\alpha)\to\alpha&\hyperref[MA1]{$\mathbf{A_1}$}\\
                \fa\vdash(\alpha\to(\alpha\to\alpha)\to\alpha)\to(\alpha\to\alpha\to\alpha)\to\alpha\to\alpha&\hyperref[MA2]{$\mathbf{A_2}$}\\
                \fa\vdash(\alpha\to\alpha\to\alpha)\to\alpha\to\alpha&$\hyperref[detachment]{\mathbf{R_1}}\;\set{2,3}$\\
                \fa\vdash\alpha\to\alpha&$\hyperref[detachment]{\mathbf{R_1}}\;\set{1,4}$
            \end{fitch}
            \normalsize
            Estando assim demonstrada a proposição.
        \end{proof}
    \end{lemma}

    Tendo-se provado o lema da identidade, agora provaremos a regra da dedução para os sistemas modais com base na prova apresentada por~\cite{Hakli}.
    Pequenas alterações foram feitas de modo a garantir a adequação da prova com a axiomatização provida na definição \refer{m-axioms}{D}.

    \begin{theorem}\label{deduction}
        $\text{Se }\Gamma\cup\set{\alpha}\vdash\beta\text{, então }\Gamma\vdash\alpha\to\beta$.

        \begin{proof}
            Prova por indução forte sobre o tamanho da sucessão de dedução.\footnote{Nota-se que, para a indução forte, não se faz preciso provar nenhuma base \citep{Velleman}.}
            Assim, suponhamos que o teorema da dedução valha para qualquer sucessão dedução de tamanho $n<k$.
            Demonstraremos analisando-se os casos e valendo-se da suposição acima --- doravante chamada $\mathbf{H}$ --- o passo de indução, ou seja, que o teorema da dedução vale para sucessões de dedução de tamanho $n=k+1$.

            \begin{case}
                \textsc{Caso 1.}
                Se a linha derradeira da sucessão de dedução que prova $\Gamma\cup\set{\alpha}\vdash\beta$ tenha sido a evocação de alguma premissa, sabe-se que $\beta\in\Gamma\cup\set{\alpha}$.
                Deste modo, existem outros dois casos a serem analisados.
            \end{case}

            \begin{subcase}
                \textsc{Caso 1.1.}
                Se a linha derradeira da sucessão de dedução que prova $\Gamma\cup\set{\alpha}\vdash\beta$ tenha sido a evocação de alguma premissa do conjunto $\Gamma$, sabe-se que $\beta\in\Gamma$. Deste modo, podemos demonstrar que $\Gamma\vdash\alpha\to\beta$ pela seguinte sucessão de dedução:

                \footnotesize
                \begin{fitch}
                    \fb\Gamma\vdash\beta&$\mathbf{P_\beta}$\\
                    \fa\Gamma\vdash\beta\to\alpha\to\beta&$\hyperref[MA1]{\mathbf{A_1}}$\\
                    \fa\Gamma\vdash\alpha\to\beta&$\hyperref[detachment]{\mathbf{R_1}}\;\set{1,2}$.
                \end{fitch}
            \end{subcase}

            \begin{subcase}
                \textsc{Caso 1.2.}
                Se a linha derradeira da sucessão de dedução que prova $\Gamma\cup\set{\alpha}\vdash\beta$ tenha sido a evocação da premissa $\alpha$, sabe-se que $\beta=\alpha$.
                Deste modo, basta demonstrar que $\Gamma\vdash\alpha\to\alpha$, que consiste num enfraquecimento do lema \refer{identity}{L}.
            \end{subcase}

            \begin{case}
                \textsc{Caso 2.}
                Se a linha derradeira da sucessão de dedução que prova $\Gamma\cup\set{\alpha}\vdash\beta$ tenha sido a evocação de algum axioma, sabe-se que existe algum esquema $\mathbf{A_\beta}\in\mathcal{A}$ que instancia $\beta$.
                Deste modo, podemos demonstrar que $\Gamma\vdash\alpha\to\beta$ pela seguinte sucessão de dedução:

                \footnotesize
                \begin{fitch}
                    \fb\Gamma\vdash\beta&$\mathbf{A_\beta}$\\
                    \fa\Gamma\vdash\beta\to\alpha\to\beta&$\hyperref[MA1]{\mathbf{A_1}}$\\
                    \fa\Gamma\vdash\alpha\to\beta&$\hyperref[detachment]{\mathbf{R_1}}\;\set{1,2}$.
                \end{fitch}
            \end{case}

            \begin{case}
                \textsc{Caso 3.}
                Se a linha derradeira da sucessão de dedução que prova $\Gamma\cup\set{\alpha}\vdash\beta$ tenha sido gerada pela aplicação da regra da necessitação a uma linha anterior, sabe-se que $\beta=\nec\varphi$ e que $\mathbf{H_1}={\entails\varphi}$.
                Deste modo, podemos demonstrar que $\Gamma\vdash\alpha\to\nec\varphi$ pela seguinte sucessão de dedução:

                \footnotesize
                \begin{fitch}
                    \fb\vdash\varphi&$\mathbf{H_1}$\\
                    \fa\Gamma\vdash\nec\varphi&$\hyperref[necessitation]{\mathbf{R_2}}\;\set{1}$\\
                    \fa\Gamma\vdash\nec\varphi\to\alpha\to\nec\varphi&$\hyperref[MA1]{\mathbf{A_1}}$\\
                    \fa\Gamma\vdash\alpha\to\nec\varphi&$\hyperref[detachment]{\mathbf{R_1}}\;\set{2,3}$.
                \end{fitch}
            \end{case}

            \begin{case}
                \textsc{Caso 4.} Seja a sentença $\varphi_n=\beta$ gerada pela aplicação da regra da separação a duas sentenças $\varphi_i$ e $\varphi_j$ com $i<j<n$. Assumiremos, sem perda de generalidade, que $\varphi_j=\varphi_i\to\varphi_n$.
                Assim, a partir de $\mathbf{H}$ temos que $\mathbf{H_1}=\Gamma\entails\alpha\to\varphi_i$ e que $\mathbf{H_2}=\Gamma\entails\alpha\to\varphi_i\to\varphi_n$.
                Deste modo, podemos demonstrar que $\Gamma\vdash\alpha\to\nec\varphi$ pela seguinte sucessão de dedução:

                \footnotesize
                \begin{fitch}
                    \fb\Gamma\entails\alpha\to\varphi_j&$\mathbf{H_1}$\\
                    \fa\Gamma\entails\alpha\to\varphi_j\to\beta&$\mathbf{H_2}$\\
                    \fa\Gamma\entails(\alpha\to\varphi_j\to\beta)\to(\alpha\to\varphi_j)\to(\alpha\to\beta)&$\hyperref[MA2]{\mathbf{A_2}}$\\
                    \fa\Gamma\entails(\alpha\to\varphi_j)\to(\alpha\to\beta)&$\hyperref[detachment]{\mathbf{R_1}}\;\set{2,3}$\\
                    \fa\Gamma\entails\alpha\to\beta&$\hyperref[detachment]{\mathbf{R_1}}\;\set{1,4}$.
                \end{fitch}
            \end{case}
            Uma vez provada a propriedade para todos os casos do passo de indução, provamos que o teorema da dedução vale para o sistema $\mathbf{S_4}$.
        \end{proof}
    \end{theorem}

    \begin{theorem}\label{gen-nec}
        Se $\nec\Gamma\entails\alpha$, então $\nec\Gamma\entails\nec\alpha$.

        \begin{proof}
            Prova por indução fraca sobre o tamanho $n$ do conjunto $\Gamma$ \citep{Troelstra}. A prova consiste em dois casos: um para a base da indução e outro para o passo da indução, considerando-se $\mathbf{H_1}=\nec\Gamma\entails\alpha$.

            \begin{case}
                \textsc{Caso 1.} Para a base, consideraremos que $\Gamma=\varnothing$.
                Assim, sabemos que o conjunto possui tamanho nulo e que $\entails\alpha$. Portanto, sabe-se que existe uma sucessão de dedução $\sequence{\varphi_i\mid 0\leq i\leq n}$ com $\varphi_n=\alpha$
                Deste modo, pode-se demonstrar que $\entails\nec\alpha$ trivialmente pela aplicação da regra da necessitação \hyperref[necessitation]{$\mathbf{R_2}$} sobre a sentença $\varphi_n$.
            \end{case}

            \begin{case}
                \textsc{Caso 2.} 
                Para o passo, suponhamos que a generalização da regra da necessitação valha para qualquer conjunto $\Gamma$ de tamanho $n=k$.
                Demonstraremos, valendo-se da suposição acima --- doravante chamada $\mathbf{H_2}$ --- e pela sucessão de dedução apresentada abaixo, que a generalização da regra da necessitação vale para conjuntos $\Gamma$ de tamanho $n=k+1$.
                \footnotesize
                \begin{fitch}
                    \fb\nec\Gamma\cup\set{\nec\alpha}\entails\beta&$\mathbf{H_1}$\\
                    \fa\nec\Gamma\entails\nec\alpha\to\beta&$\hyperref[deduction]{\mathbf{T_1}}\;\set{1}$\\
                    \fa\nec\Gamma\entails\nec(\nec\alpha\to\beta)&$\mathbf{H_2}\;\set{2}$\\
                    \fa\nec\Gamma\entails\nec(\nec\alpha\to\beta)\to\nec\nec\alpha\to\nec\beta&$\hyperref[MB1]{\mathbf{B_1}}$\\
                    \fa\nec\Gamma\entails\nec\nec\alpha\to\nec\beta&$\hyperref[detachment]{\mathbf{R_1}}\;\set{3,4}$\\
                    \fa\nec\Gamma\entails\nec\alpha\to\nec\nec\alpha&$\hyperref[MB3]{\mathbf{B_3}}$\\
                    \fa\nec\Gamma\entails(\nec\alpha\to\nec\nec\alpha)\to(\nec\nec\alpha\to\nec\beta)\to\nec\alpha\to\nec\beta&$\hyperref[MA2]{\mathbf{A_2}}$\\
                    \fa\nec\Gamma\entails(\nec\nec\alpha\to\nec\beta)\to\nec\alpha\to\nec\beta&$\hyperref[detachment]{\mathbf{R_1}}\;\set{6,7}$\\
                    \fa\nec\Gamma\entails\nec\alpha\to\nec\beta&$\hyperref[detachment]{\mathbf{R_1}}\;\set{5,8}$\\
                    \fa\nec\Gamma\cup\set{\nec\alpha}\entails\nec\alpha&$\mathbf{P_1}$\\
                    \fa\nec\Gamma\cup\set{\nec\alpha}\entails\nec\alpha\to\nec\beta&$\mathbf{E_1}\;\set{9}$\\
                    \fa\nec\Gamma\cup\set{\nec\alpha}\entails\nec\beta&$\hyperref[detachment]{\mathbf{R_1}}\;\set{10,11}$.
                \end{fitch}
            \end{case}
            \vspace{.5\baselineskip}
            Tendo-se provado a base e o passo de indução, podemos concluir que generalização da regra da necessitação vale, ou seja, que se $\nec\Gamma\entails\alpha$, então $\nec\Gamma\entails\nec\alpha$.
        \end{proof}
    \end{theorem}

    Uma vez provada a generalização da regra da implicação, a prova da regra da dedução estrita --- conforme descrito por~\cite{Barcan, Marcus} --- torna-se trivial, como pode ser visto abaixo. Esta regra derivada permite simplificar as provas de corretude das traduções, uma vez que uma das traduções que serão apresentadas mapeia implicações materiais do sistema intuicionista em implicações estritas.

    \begin{theorem}\label{strictdeduction}
        $\text{Se }\nec\Gamma\cup\set{\alpha}\entails\beta\text{, então }\nec\Gamma\entails\nec(\alpha\to\beta)$.

        \begin{proof}
            Pode ser provado pela seguinte sucessão de dedução:

            \footnotesize
            \begin{fitch}
                \fb\nec\Gamma\cup\set{\alpha}\entails\beta&$\mathbf{H_1}$\\
                \fa\nec\Gamma\entails\alpha\to\beta&\refer{deduction}{T}$\;\set{1}$\\
                \fa\nec\Gamma\entails\nec(\alpha\to\beta)&\refer{gen-nec}{T}$\;\set{2}$
            \end{fitch}
            \normalsize
            Estando assim demonstrada a proposição.
        \end{proof}
    \end{theorem}

    \begin{theorem}\label{strictsep}
        Se $\Gamma\entails\alpha$ e $\Gamma\entails\nec(\alpha\to\beta)$, então $\Gamma\entails\beta$.
        \begin{proof}
            Pode ser provado pela seguinte sucessão de dedução:
            \footnotesize
            \begin{fitch}
                \fb\Gamma\entails\alpha&$\mathbf{H_1}$\\
                \fa\Gamma\entails\nec(\alpha\to\beta)&$\mathbf{H_2}$\\
                \fa\Gamma\entails\nec(\alpha\to\beta)\to\alpha\to\beta&$\hyperref[MB2]{\mathbf{B_2}}$\\
                \fa\Gamma\entails\alpha\to\beta&$\hyperref[detachment]{\mathbf{R_1}}\;\set{2,3}$\\
                \fa\Gamma\entails\beta&$\hyperref[detachment]{\mathbf{R_1}}\;\set{1,4}$
            \end{fitch}
            \normalsize
            Estando assim demonstrada a proposição.
        \end{proof}
    \end{theorem}

    \begin{theorem}\label{conjunctiondeduction}
        Se $\set{\alpha,\beta}\entails\gamma$, então $\set{\alpha\wedge\beta}\entails\gamma$.
        \begin{proof}
            Seja $\mathbf{H_1}=\set{\alpha,\beta}\entails\gamma$. A proposição pode ser provada pela seguinte sucessão de dedução:
            \footnotesize
            \begin{fitch}
                \fb\set{\alpha,\beta}\entails\gamma&$\mathbf{H_1}$\\
                \fa\set{\alpha}\entails\beta\to\gamma&$\hyperref[deduction]{\mathbf{T_1}}\;\set{1}$\\
                \fa\entails\alpha\to\beta\to\gamma&$\hyperref[deduction]{\mathbf{T_1}}\;\set{2}$\\
                \fa\set{\alpha\wedge\beta}\entails\alpha\wedge\beta&$\mathbf{P_1}$\\
                \fa\set{\alpha\wedge\beta}\entails\alpha\wedge\beta\to\alpha&$\hyperref[MA4]{\mathbf{A_4}}$\\
                \fa\set{\alpha\wedge\beta}\entails\alpha&$\hyperref[detachment]{\mathbf{R_1}}\;\set{4,5}$\\
                \fa\set{\alpha\wedge\beta}\entails\alpha\wedge\beta\to\beta&$\hyperref[MA5]{\mathbf{A_5}}$\\
                \fa\set{\alpha\wedge\beta}\entails\beta&$\hyperref[detachment]{\mathbf{R_1}}\;\set{4,7}$\\
                \fa\set{\alpha\wedge\beta}\entails\alpha\to\beta\to\gamma&${\mathbf{E_1}}\;\set{3}$\\
                \fa\set{\alpha\wedge\beta}\entails\beta\to\gamma&$\hyperref[detachment]{\mathbf{R_1}}\;\set{6,9}$\\
                \fa\set{\alpha\wedge\beta}\entails\gamma&$\hyperref[detachment]{\mathbf{R_1}}\;\set{8,10}$
            \end{fitch}
            \normalsize
            Estando assim demonstrada a proposição.
        \end{proof}
    \end{theorem}

    \begin{theorem}\label{disjunctiondeduction}
        Se $\set{\alpha}\entails \gamma$ e $\set{\beta}\entails \gamma$, então $\set{\alpha \vee \beta}\entails \gamma$.
        \begin{proof}
            Seja $\mathbf{H_1}=\set{\alpha}\entails\gamma$ e $\mathbf{H_2}=\set{\beta}\entails\gamma$. A proposição pode ser provada pela seguinte sucessão de dedução:
            \footnotesize
            \begin{fitch}
                \fb\set{\alpha}\entails\gamma&$\mathbf{H_1}$\\
                \fa\set{\beta}\entails\gamma&$\mathbf{H_2}$\\
                \fa\entails\alpha\to\gamma&$\hyperref[deduction]{\mathbf{T_1}}\;\set{1}$\\
                \fa\entails\beta\to\gamma&$\hyperref[deduction]{\mathbf{T_1}}\;\set{2}$\\
                \fa\set{\alpha\vee\beta}\entails\alpha\to\gamma&${\mathbf{E_1}}\;\set{3}$\\
                \fa\set{\alpha\vee\beta}\entails\beta\to\gamma&${\mathbf{E_1}}\;\set{4}$\\
                \fa\set{\alpha\vee\beta}\entails\alpha\vee\beta&$\mathbf{P_1}$\\
                \fa\set{\alpha\vee\beta}\entails(\alpha\to\gamma)\to(\beta\to\gamma)\to\alpha\vee\beta\to\gamma&$\hyperref[MA8]{\mathbf{A_8}}$\\
                \fa\set{\alpha\vee\beta}\entails(\beta\to\gamma)\to\alpha\vee\beta\to\gamma&$\hyperref[detachment]{\mathbf{R_1}}\;\set{5,8}$\\
                \fa\set{\alpha\vee\beta}\entails\alpha\vee\beta\to\gamma&$\hyperref[detachment]{\mathbf{R_1}}\;\set{4,9}$\\
                \fa\set{\alpha\vee\beta}\entails\gamma&$\hyperref[detachment]{\mathbf{R_1}}\;\set{7,10}$
            \end{fitch}
            \normalsize
            Estando assim demonstrada a proposição.
        \end{proof}
    \end{theorem}

    \begin{lemma}\label{explosion}
        $\entails\bot\to\alpha$.
        \begin{proof}
            Pode ser provado pela seguinte sucessão de dedução:
            \footnotesize
            \begin{fitch}
                \fb\set{\bot}\entails\bot&$\mathbf{P_1}$\\
                \fa\set{\bot}\entails\bot\to(\alpha\to\bot)\to\bot&$\hyperref[MA1]{\mathbf{A_1}}$\\
                \fa\set{\bot}\entails\neg\neg\alpha&$\hyperref[detachment]{\mathbf{R_1}}\;\set{1,2}$\\
                \fa\set{\bot}\entails\neg\neg\alpha\to\alpha&$\hyperref[MANEG]{\mathbf{A_\neg}}$\\
                \fa\set{\bot}\entails\alpha&$\hyperref[detachment]{\mathbf{R_1}}\;\set{3,4}$\\
                \fa\entails\bot\to\alpha&$\hyperref[deduction]{\mathbf{T_1}}\;\set{5}$.
            \end{fitch}
            \normalsize
            Estando assim demonstrada a proposição.
        \end{proof}
    \end{lemma}

    \begin{lemma}\label{contrapositive}
        $\entails(\alpha\to\beta)\to(\neg\beta\to\neg\alpha)$.
        \begin{proof}
            Pode ser provado pela seguinte sucessão de dedução:
            \footnotesize
            \begin{fitch}
                \fb\set{\alpha\to\beta,\neg\beta}\entails\beta\to\bot&$\mathbf{P_1}$\\
                \fa\set{\alpha\to\beta,\neg\beta}\entails(\beta\to\bot)\to\alpha\to(\beta\to\bot)&\hyperref[MA1]{$\mathbf{A_1}$}\\
                \fa\set{\alpha\to\beta,\neg\beta}\entails\alpha\to\beta\to\bot&$\hyperref[detachment]{\mathbf{R_1}}\;\set{1,2}$\\
                \fa\set{\alpha\to\beta,\neg\beta}\entails(\alpha\to\beta\to\bot)\to(\alpha\to\beta)\to(\alpha\to\bot)&\hyperref[MA2]{$\mathbf{A_2}$}\\
                \fa\set{\alpha\to\beta,\neg\beta}\entails\alpha\to\beta&$\mathbf{P_2}$\\
                \fa\set{\alpha\to\beta,\neg\beta}\entails(\alpha\to\beta)\to(\alpha\to\bot)&$\hyperref[detachment]{\mathbf{R_1}}\;\set{3,4}$\\
                \fa\set{\alpha\to\beta,\neg\beta}\entails\neg\alpha&$\hyperref[detachment]{\mathbf{R_1}}\;\set{5,6}$\\
                \fa\set{\alpha\to\beta}\entails\neg\beta\to\neg\alpha&\refer{deduction}{T}$\;\set{7}$\\
                \fa\entails(\alpha\to\beta)\to(\neg\beta\to\neg\alpha)&\refer{deduction}{T}$\;\set{8}$.
            \end{fitch}
            \normalsize
            Estando assim demonstrada a proposição.
        \end{proof}
    \end{lemma}

    \begin{lemma}\label{and-intro}
        $\vdash(\alpha\to\beta)\to(\alpha\to\gamma)\to\alpha\to\beta\wedge\gamma$.
        \begin{proof}
            Pode ser provado pela seguinte sucessão de dedução:
            \footnotesize
            \begin{fitch}
                \fb\set{\alpha\to\beta,\alpha\to\gamma,\alpha}\vdash\alpha&$\mathbf{P_1}$\\
                \fa\set{\alpha\to\beta,\alpha\to\gamma,\alpha}\vdash\alpha\to\beta&$\mathbf{P_3}$\\
                \fa\set{\alpha\to\beta,\alpha\to\gamma,\alpha}\vdash\beta&$\hyperref[detachment]{\mathbf{R_1}}\;\set{1, 2}$\\
                \fa\set{\alpha\to\beta,\alpha\to\gamma,\alpha}\vdash\alpha\to\gamma&$\mathbf{P_2}$\\
                \fa\set{\alpha\to\beta,\alpha\to\gamma,\alpha}\vdash\gamma&$\hyperref[detachment]{\mathbf{R_1}}\;\set{1, 4}$\\
                \fa\set{\alpha\to\beta,\alpha\to\gamma,\alpha}\vdash\beta\to\gamma\to\beta\wedge\gamma&\hyperref[MA3]{$\mathbf{A_3}$}\\
                \fa\set{\alpha\to\beta,\alpha\to\gamma,\alpha}\vdash\gamma\to\beta\wedge\gamma&$\hyperref[detachment]{\mathbf{R_1}}\;\set{3, 6}$\\
                \fa\set{\alpha\to\beta,\alpha\to\gamma,\alpha}\vdash\beta\wedge\gamma&$\hyperref[detachment]{\mathbf{R_1}}\;\set{5, 7}$\\
                \fa\set{\alpha\to\beta,\alpha\to\gamma}\vdash\alpha\to\beta\wedge\gamma&\refer{deduction}{T}$\;\set{8}$\\
                \fa\set{\alpha\to\beta}\vdash(\alpha\to\gamma)\to\alpha\to\beta\wedge\gamma&\refer{deduction}{T}$\;\set{9}$\\
                \fa\vdash(\alpha\to\beta)\to(\alpha\to\gamma)\to\alpha\to\beta\wedge\gamma&\refer{deduction}{T}$\;\set{10}$.
            \end{fitch}
            \normalsize
            Estando assim demonstrada a proposição.
        \end{proof}
    \end{lemma}

    \begin{lemma}\label{nec-distr}
        $\vdash\nec(\alpha\wedge\beta)\to\nec\alpha\wedge\nec\beta$.
        \begin{proof}
            Pode ser provado pela seguinte sucessão de dedução:
            \footnotesize
            \begin{fitch}
                \fb\entails\alpha\wedge\beta\to\alpha&\hyperref[MA4]{$\mathbf{A_4}$}\\
                \fa\entails\nec(\alpha\wedge\beta\to\alpha)&$\hyperref[necessitation]{\mathbf{R_2}}\;\set{1}$\\
                \fa\entails\nec(\alpha\wedge\beta\to\alpha)\to(\nec(\alpha\wedge\beta)\to\nec\alpha)&\hyperref[MB1]{$\mathbf{B_1}$}\\
                \fa\entails\nec(\alpha\wedge\beta)\to\nec\alpha&$\hyperref[detachment]{\mathbf{R_1}}\;\set{2, 3}$\\
                \fa\entails\alpha\wedge\beta\to\beta&\hyperref[MA5]{$\mathbf{A_5}$}\\
                \fa\entails\nec(\alpha\wedge\beta\to\beta)&$\hyperref[necessitation]{\mathbf{R_2}}\;\set{5}$\\
                \fa\entails\nec(\alpha\wedge\beta\to\beta)\to(\nec(\alpha\wedge\beta)\to\nec\beta)&\hyperref[MB1]{$\mathbf{B_1}$}\\
                \fa\entails\nec(\alpha\wedge\beta)\to\nec\beta&$\hyperref[detachment]{\mathbf{R_1}}\;\set{6, 7}$\\
                \fa\entails(\nec(\alpha\wedge\beta)\to\nec\alpha)\to(\nec(\alpha\wedge\beta)\to\nec\beta)\to\nec(\alpha\wedge\beta)\to\nec\alpha\wedge\nec\beta&\refer{and-intro}{L}\\
                \fa\entails(\nec(\alpha\wedge\beta)\to\nec\beta)\to\nec(\alpha\wedge\beta)\to\nec\alpha\wedge\nec\beta&$\hyperref[detachment]{\mathbf{R_1}}\;\set{4, 9}$\\
                \fa\entails\nec(\alpha\wedge\beta)\to\nec\alpha\wedge\nec\beta&$\hyperref[detachment]{\mathbf{R_1}}\;\set{6,9}$\\
            \end{fitch}
            \normalsize
            Estando assim demonstrada a proposição.
        \end{proof}
    \end{lemma}

    \begin{lemma}\label{nec-undistr}
        $\entails\nec\alpha\wedge\nec\beta\to\nec(\alpha\wedge\beta)$.
        \begin{proof}
            Pode ser provado pela seguinte sucessão de dedução:
            \footnotesize
            \begin{fitch}
                \fb\set{\nec\alpha,\nec\beta}\entails\nec\alpha&$\mathbf{P_2}$\\
                \fa\set{\nec\alpha,\nec\beta}\entails\nec\alpha\to\alpha&\hyperref[MB2]{$\mathbf{B_2}$}\\
                \fa\set{\nec\alpha,\nec\beta}\entails\alpha&$\hyperref[detachment]{\mathbf{R_1}}\;\set{1,2}$\\
                \fa\set{\nec\alpha,\nec\beta}\entails\nec\beta&$\mathbf{P_1}$\\
                \fa\set{\nec\alpha,\nec\beta}\entails\nec\beta\to\beta&\hyperref[MB2]{$\mathbf{B_2}$}\\
                \fa\set{\nec\alpha,\nec\beta}\entails\beta&$\hyperref[detachment]{\mathbf{R_1}}\;\set{4,5}$\\
                \fa\set{\nec\alpha,\nec\beta}\entails\alpha\to\beta\to\alpha\wedge\beta&\hyperref[MA3]{$\mathbf{A_3}$}\\
                \fa\set{\nec\alpha,\nec\beta}\entails\beta\to\alpha\wedge\beta&$\hyperref[detachment]{\mathbf{R_1}}\;\set{3,7}$\\
                \fa\set{\nec\alpha,\nec\beta}\entails\alpha\wedge\beta&$\hyperref[detachment]{\mathbf{R_1}}\;\set{6,8}$\\
                \fa\set{\nec\alpha,\nec\beta}\entails\nec(\alpha\wedge\beta)&$\hyperref[gen-nec]{\mathbf{T_{\getrefnumber{gen-nec}}}}\;\set{9}$\\
                \fa\set{\nec\alpha\wedge\nec\beta}\entails\nec(\alpha\wedge\beta)&$\hyperref[conjunctiondeduction]{\mathbf{T_{\getrefnumber{conjunctiondeduction}}}}\;\set{10}$\\
                \fa\entails\nec\alpha\wedge\nec\beta\to\nec(\alpha\wedge\beta)&$\hyperref[deduction]{\mathbf{T_{\getrefnumber{deduction}}}}\;\set{11}$\\
            \end{fitch}
            \normalsize
            Estando assim demonstrada a proposição.
        \end{proof}
    \end{lemma}

    \begin{lemma}
        $\vdash\nec(\alpha\to\beta)\to\nec\alpha\to\beta$.
        \begin{proof}
            Pode ser provado pela seguinte sucessão de dedução:
            \footnotesize
            \begin{fitch}
                \fb\set{\nec(\alpha\to\beta),\nec\alpha}\entails\nec\alpha&$\mathbf{P_2}$\\
                \fa\set{\nec(\alpha\to\beta),\nec\alpha}\entails\nec\alpha\to\alpha&$\hyperref[MB2]{\mathbf{B_2}}$\\
                \fa\set{\nec(\alpha\to\beta),\nec\alpha}\entails\alpha&$\hyperref[detachment]{\mathbf{R_1}}\;\set{1,2}$\\
                \fa\set{\nec(\alpha\to\beta),\nec\alpha}\entails\nec(\alpha\to\beta)&$\mathbf{P_1}$\\
                \fa\set{\nec(\alpha\to\beta),\nec\alpha}\entails\nec(\alpha\to\beta)\to\alpha\to\beta&$\hyperref[MB2]{\mathbf{B_2}}$\\
                \fa\set{\nec(\alpha\to\beta),\nec\alpha}\entails\alpha\to\beta&$\hyperref[detachment]{\mathbf{R_1}}\;\set{4,5}$\\
                \fa\set{\nec(\alpha\to\beta),\nec\alpha}\entails\beta&$\hyperref[detachment]{\mathbf{R_1}}\;\set{3,6}$\\
                \fa\set{\nec(\alpha\to\beta)}\entails\nec\alpha\to\beta&$\hyperref[deduction]{\mathbf{T_1}}\;\set{7}$\\
                \fa\entails\nec(\alpha\to\beta)\to\nec\alpha\to\beta&$\hyperref[deduction]{\mathbf{T_1}}\;\set{8}$.
            \end{fitch}
            \vspace*{-18pt-0.7em}
            \qedhere
        \end{proof}
    \end{lemma}

    \begin{lemma}\label{comp}
        $\entails(\alpha\to\beta)\to(\beta\to\gamma)\to\alpha\to\gamma$
        \begin{proof}
            Pode ser provado pela seguinte sucessão de dedução:
            \footnotesize
            \begin{fitch}
                \fb\set{\alpha\to\beta,\beta\to\gamma,\alpha}\entails\alpha&$\mathbf{P_1}$\\
                \fa\set{\alpha\to\beta,\beta\to\gamma,\alpha}\entails\alpha\to\beta&$\mathbf{P_3}$\\
                \fa\set{\alpha\to\beta,\beta\to\gamma,\alpha}\entails\beta&$\hyperref[detachment]{\mathbf{R_1}}\;\set{1,2}$\\
                \fa\set{\alpha\to\beta,\beta\to\gamma,\alpha}\entails\beta\to\gamma&$\mathbf{P_2}$\\
                \fa\set{\alpha\to\beta,\beta\to\gamma,\alpha}\entails\gamma&$\hyperref[detachment]{\mathbf{R_1}}\;\set{3,4}$\\
                \fa\set{\alpha\to\beta,\beta\to\gamma}\entails\alpha\to\gamma&$\hyperref[deduction]{\mathbf{T_{\getrefnumber{deduction}}}}\;\set{5}$\\
                \fa\set{\alpha\to\beta}\entails(\beta\to\gamma)\to\alpha\to\gamma&$\hyperref[deduction]{\mathbf{T_{\getrefnumber{deduction}}}}\;\set{6}$\\
                \fa\entails(\alpha\to\beta)\to(\beta\to\gamma)\to\alpha\to\gamma&$\hyperref[deduction]{\mathbf{T_{\getrefnumber{deduction}}}}\;\set{7}$\\
            \end{fitch}
            \normalsize
            Estando assim demonstrada a proposição.
        \end{proof}
    \end{lemma}

    \begin{lemma}\label{neg-intro}
        $\entails\alpha\to\neg\neg\alpha$
        \begin{proof}
            Pode ser provado pela seguinte sucessão de dedução:
            \footnotesize
            \begin{fitch}
                \fb\set{\alpha,\neg\alpha}\entails\alpha&$\mathbf{P_2}$\\
                \fa\set{\alpha,\neg\alpha}\entails\alpha\to\bot&$\mathbf{P_1}$\\
                \fa\set{\alpha,\neg\alpha}\entails\bot&$\hyperref[detachment]{\mathbf{R_1}}\;\set{1,2}$\\
                \fa\set{\alpha}\entails\neg\neg\alpha&$\hyperref[deduction]{\mathbf{T_{\getrefnumber{deduction}}}}\;\set{3}$\\
                \fa\entails\alpha\to\neg\neg\alpha&$\hyperref[deduction]{\mathbf{T_{\getrefnumber{deduction}}}}\;\set{4}$\\
            \end{fitch}
            \normalsize
            Estando assim demonstrada a proposição.
        \end{proof}
    \end{lemma}

    \begin{lemma}\label{or-left}
        $\entails(\alpha\to\beta)\to\alpha\to\beta\vee\gamma$.
        \begin{proof}
            Pode ser provado pela seguinte sucessão de dedução:
            \footnotesize 
            \begin{fitch}
                \fb\set{\alpha\to\beta,\alpha}\entails\alpha&$\mathbf{P_1}$\\
                \fa\set{\alpha\to\beta,\alpha}\entails\alpha\to\beta&$\mathbf{P_2}$\\
                \fa\set{\alpha\to\beta,\alpha}\entails\beta&$\hyperref[detachment]{\mathbf{R_1}}\;\set{1,2}$\\
                \fa\set{\alpha\to\beta,\alpha}\entails\beta\to\beta\vee\gamma&$\hyperref[MA6]{\mathbf{A_6}}$\\
                \fa\set{\alpha\to\beta,\alpha}\entails\beta\vee\gamma&$\hyperref[detachment]{\mathbf{R_1}}\;\set{3,4}$\\
                \fa\set{\alpha\to\beta}\entails\alpha\to\beta\vee\gamma&$\hyperref[deduction]{\mathbf{T_{\getrefnumber{deduction}}}}\;\set{5}$\\
                \fa\entails(\alpha\to\beta)\to\alpha\to\beta\vee\gamma&$\hyperref[deduction]{\mathbf{T_{\getrefnumber{deduction}}}}\;\set{6}$\\
            \end{fitch}
            \normalsize
            Estando assim demonstrada a proposição.
        \end{proof}
    \end{lemma}

    \begin{lemma}\label{or-right}
        $\entails(\alpha\to\beta)\to\alpha\to\gamma\vee\beta$.
        \begin{proof}
            Pode ser provado pela seguinte sucessão de dedução:
            \footnotesize 
            \begin{fitch}
                \fb\set{\alpha\to\beta,\alpha}\entails\alpha&$\mathbf{P_1}$\\
                \fa\set{\alpha\to\beta,\alpha}\entails\alpha\to\beta&$\mathbf{P_2}$\\
                \fa\set{\alpha\to\beta,\alpha}\entails\beta&$\hyperref[detachment]{\mathbf{R_1}}\;\set{1,2}$\\
                \fa\set{\alpha\to\beta,\alpha}\entails\beta\to\gamma\vee\beta&$\hyperref[MA7]{\mathbf{A_7}}$\\
                \fa\set{\alpha\to\beta,\alpha}\entails\gamma\vee\beta&$\hyperref[detachment]{\mathbf{R_1}}\;\set{3,4}$\\
                \fa\set{\alpha\to\beta}\entails\alpha\to\gamma\vee\beta&$\hyperref[deduction]{\mathbf{T_{\getrefnumber{deduction}}}}\;\set{5}$\\
                \fa\entails(\alpha\to\beta)\to\alpha\to\gamma\vee\beta&$\hyperref[deduction]{\mathbf{T_{\getrefnumber{deduction}}}}\;\set{6}$\\
            \end{fitch}
            \normalsize
            Estando assim demonstrada a proposição.
        \end{proof}
    \end{lemma}

    \begin{lemma}\label{or-subst}
        $\entails(\alpha\to\gamma)\to(\beta\to\delta)\to\alpha\vee\beta\to\gamma\vee\delta$.
        \begin{proof}
            Pode ser provado pela seguinte sucessão de dedução:
            \footnotesize 
            \begin{fitch}
                \fb\set{\alpha\to\gamma,\beta\to\delta,\alpha\vee\beta}\entails\alpha\to\gamma&$\mathbf{P_3}$\\
                \fa\set{\alpha\to\gamma,\beta\to\delta,\alpha\vee\beta}\entails(\alpha\to\gamma)\to\alpha\to\gamma\vee\delta&$\hyperref[or-left]{\mathbf{L_{\getrefnumber{or-left}}}}$\\
                \fa\set{\alpha\to\gamma,\beta\to\delta,\alpha\vee\beta}\entails\alpha\to\gamma\vee\delta&$\hyperref[detachment]{\mathbf{R_1}}\;\set{1,2}$\\
                \fa\set{\alpha\to\gamma,\beta\to\delta,\alpha\vee\beta}\entails\beta\to\delta&$\mathbf{P_2}$\\
                \fa\set{\alpha\to\gamma,\beta\to\delta,\alpha\vee\beta}\entails(\beta\to\delta)\to\beta\to\gamma\vee\delta&$\hyperref[or-right]{\mathbf{L_{\getrefnumber{or-right}}}}$\\
                \fa\set{\alpha\to\gamma,\beta\to\delta,\alpha\vee\beta}\entails\beta\to\gamma\vee\delta&$\hyperref[detachment]{\mathbf{R_1}}\;\set{4,5}$\\
                \fa\set{\alpha\to\gamma,\beta\to\delta,\alpha\vee\beta}\entails\alpha\vee\beta&$\mathbf{P_1}$\\
                \fa\set{\alpha\to\gamma,\beta\to\delta,\alpha\vee\beta}\entails(\alpha\to\gamma\vee\delta)\to(\beta\to\gamma\vee\delta)\to\alpha\vee\beta\to\gamma\vee\delta&$\hyperref[MA8]{\mathbf{A_8}}$\\
                \fa\set{\alpha\to\gamma,\beta\to\delta,\alpha\vee\beta}\entails(\beta\to\gamma\vee\delta)\to\alpha\vee\beta\to\gamma\vee\delta&$\hyperref[detachment]{\mathbf{R_1}}\;\set{3,8}$\\
                \fa\set{\alpha\to\gamma,\beta\to\delta,\alpha\vee\beta}\entails\alpha\vee\beta\to\gamma\vee\delta&$\hyperref[detachment]{\mathbf{R_1}}\;\set{6,9}$\\
                \fa\set{\alpha\to\gamma,\beta\to\delta,\alpha\vee\beta}\entails\gamma\vee\delta&$\hyperref[detachment]{\mathbf{R_1}}\;\set{7,10}$\\
                \fa\set{\alpha\to\gamma,\beta\to\delta}\entails\alpha\vee\beta\to\gamma\vee\delta&$\hyperref[deduction]{\mathbf{T_{\getrefnumber{deduction}}}}\;\set{11}$\\
                \fa\set{\alpha\to\gamma}\entails(\beta\to\delta)\to\alpha\vee\beta\to\gamma\vee\delta&$\hyperref[deduction]{\mathbf{T_{\getrefnumber{deduction}}}}\;\set{12}$\\
                \fa\entails(\alpha\to\gamma)\to(\beta\to\delta)\to\alpha\vee\beta\to\gamma\vee\delta&$\hyperref[deduction]{\mathbf{T_{\getrefnumber{deduction}}}}\;\set{13}$\\
            \end{fitch}
            \normalsize
            Estando assim demonstrada a proposição.
        \end{proof}
    \end{lemma}

    \begin{lemma}\label{or-undistr}
        $\entails\nec\alpha\vee\nec\beta\to\nec(\alpha\vee\beta)$.
        \begin{proof}
            Pode ser provado pela seguinte sucessão de dedução:
            \footnotesize 
            \begin{fitch}
                \fb\set{\nec\alpha}\entails\nec\alpha&$\mathbf{P_1}$\\
                \fa\set{\nec\alpha}\entails\nec\alpha\to\alpha&$\hyperref[MB2]{\mathbf{B_2}}$\\
                \fa\set{\nec\alpha}\entails\alpha&$\hyperref[detachment]{\mathbf{R_1}}\;\set{1,2}$\\
                \fa\set{\nec\alpha}\entails\alpha\to\alpha\vee\beta&$\hyperref[MA4]{\mathbf{A_4}}$\\
                \fa\set{\nec\alpha}\entails\alpha\vee\beta&$\hyperref[detachment]{\mathbf{R_1}}\;\set{3,4}$\\
                \fa\set{\nec\alpha}\entails\nec(\alpha\vee\beta)&$\hyperref[necessitation]{\mathbf{R_2}}\;\set{5}$\\
                \fa\set{\nec\beta}\entails\nec\beta&$\mathbf{P_1}$\\
                \fa\set{\nec\beta}\entails\nec\beta\to\beta&$\hyperref[MB2]{\mathbf{B_2}}$\\
                \fa\set{\nec\beta}\entails\beta&$\hyperref[detachment]{\mathbf{R_1}}\;\set{7,8}$\\
                \fa\set{\nec\beta}\entails\beta\to\alpha\vee\beta&$\hyperref[MA5]{\mathbf{A_5}}$\\
                \fa\set{\nec\beta}\entails\alpha\vee\beta&$\hyperref[detachment]{\mathbf{R_1}}\;\set{9,10}$\\
                \fa\set{\nec\beta}\entails\nec(\alpha\vee\beta)&$\hyperref[necessitation]{\mathbf{R_2}}\;\set{11}$\\
                \fa\set{\nec\alpha\vee\nec\beta}\entails\nec(\alpha\vee\beta)&$\hyperref[disjunctiondeduction]{\mathbf{T_{\getrefnumber{deduction}}}}\;\set{6,12}$\\
                \fa\entails\nec\alpha\vee\nec\beta\to\nec(\alpha\vee\beta)&$\hyperref[deduction]{\mathbf{T_{\getrefnumber{deduction}}}}\;\set{13}$\\
            \end{fitch}
            \normalsize
            Estando assim demonstrada a proposição.
        \end{proof}
    \end{lemma}
