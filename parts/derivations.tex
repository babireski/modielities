\section{Derivações}
    Nesta seção apresentaremos alguns lemas e teoremas para os sistemas modais que permitirão simplificar muito as provas apresentadas no decorrer deste trabalho.
    Primeiramente, provaremos que, dada uma sentença qualquer, esta sempre implica a si mesma. A este lema daremos o nome de identidade\footnote{Em analogia ao combinador $\mathbf{I}$.} e, em seguida, usaremo-no para a prova da regra da dedução.

    \begin{lemma}\label{identity}
        $\entails\alpha\to\alpha$.
        \begin{proof}
            Pode ser provado pela seguinte sucessão de dedução:
            \footnotesize
            \begin{fitch}
                \fb\vdash\alpha\to\alpha\to\alpha&\hyperref[MA1]{$\mathbf{A_1}$}\\
                \fa\vdash\alpha\to(\alpha\to\alpha)\to\alpha&\hyperref[MA1]{$\mathbf{A_1}$}\\
                \fa\vdash(\alpha\to(\alpha\to\alpha)\to\alpha)\to(\alpha\to\alpha\to\alpha)\to\alpha\to\alpha&\hyperref[MA2]{$\mathbf{A_2}$}\\
                \fa\vdash(\alpha\to\alpha\to\alpha)\to\alpha\to\alpha&$\hyperref[detachment]{\mathbf{R_2}}\;\set{2,3}$\\
                \fa\vdash\alpha\to\alpha&$\hyperref[detachment]{\mathbf{R_2}}\;\set{1,4}$.
            \end{fitch}
            \normalsize
            Estando assim demonstrada a proposição.
        \end{proof}
    \end{lemma}

    Tendo-se provado o lema da identidade, agora provaremos a regra da dedução para os sistemas modais com base na prova apresentada por~\cite{Hakli}.
    Pequenas alterações foram feitas de modo a garantir a adequação da prova com a axiomatização provida na Definição~\ref{m-axioms}.

    \begin{theorem}[Dedução]\label{deduction}
        $\text{Se }\Gamma\cup\set{\alpha}\vdash\beta\text{, então }\Gamma\vdash\alpha\to\beta$.

        \begin{proof}
            Prova por indução forte sobre o tamanho da sucessão de dedução.\footnote{Nota-se que, para a indução forte, não se faz preciso provar nenhuma base \citep{Velleman}.}
            Assim, suponhamos que o teorema da dedução valha para qualquer sucessão de dedução de tamanho $n<k$.
            Demonstraremos analisando-se os casos e valendo-se da suposição acima --- doravante chamada $\mathbf{H}$ --- o passo de indução, ou seja, que o teorema da dedução vale para sucessões de dedução de tamanho $n=k$.

            \begin{case}
                \textsc{Caso 1.}
                Se a linha derradeira da sucessão de dedução que prova $\Gamma\cup\set{\alpha}\vdash\beta$ tenha sido a evocação de alguma premissa, sabe-se que $\beta\in\Gamma\cup\set{\alpha}$.
                Deste modo, existem dois casos a serem analisados.
            \end{case}

            \begin{subcase}
                \textsc{Caso 1.1.}
                Se a linha derradeira da sucessão de dedução que prova $\Gamma\cup\set{\alpha}\vdash\beta$ tenha sido a evocação de alguma premissa do conjunto $\Gamma$, sabe-se que $\beta\in\Gamma$. Deste modo, podemos demonstrar que $\Gamma\vdash\alpha\to\beta$ pela seguinte sucessão de dedução:

                \footnotesize
                \begin{fitch}
                    \fb\Gamma\vdash\beta&$\hyperref[premisse]{\mathbf{R_1}}$\\
                    \fa\Gamma\vdash\beta\to\alpha\to\beta&$\hyperref[MA1]{\mathbf{A_1}}$\\
                    \fa\Gamma\vdash\alpha\to\beta&$\hyperref[detachment]{\mathbf{R_2}}\;\set{1,2}$.
                \end{fitch}
                \normalsize
            \end{subcase}

            \begin{subcase}
                \textsc{Caso 1.2.}
                Se a linha derradeira da sucessão de dedução que prova $\Gamma\cup\set{\alpha}\vdash\beta$ tenha sido a evocação da premissa $\alpha$, sabe-se que $\beta=\alpha$.
                Deste modo, basta demonstrar que $\Gamma\vdash\alpha\to\alpha$, que consiste num enfraquecimento do lema \refer{identity}{L}.
            \end{subcase}

            \begin{case}
                \textsc{Caso 2.}
                Se a linha derradeira da sucessão de dedução que prova $\Gamma\cup\set{\alpha}\vdash\beta$ tenha sido a evocação de algum axioma, sabe-se que existe algum esquema $\mathbf{A_\beta}\in\mathcal{A}$ que instancia $\beta$.
                Deste modo, podemos demonstrar que $\Gamma\vdash\alpha\to\beta$ pela seguinte sucessão de dedução:

                \footnotesize
                \begin{fitch}
                    \fb\Gamma\vdash\beta&$\mathbf{A_\beta}$\\
                    \fa\Gamma\vdash\beta\to\alpha\to\beta&$\hyperref[MA1]{\mathbf{A_1}}$\\
                    \fa\Gamma\vdash\alpha\to\beta&$\hyperref[detachment]{\mathbf{R_2}}\;\set{1,2}$.
                \end{fitch}
                \normalsize
            \end{case}

            \begin{case}
                \textsc{Caso 3.}
                Se a linha derradeira da sucessão de dedução que prova $\Gamma\cup\set{\alpha}\vdash\beta$ tenha sido gerada pela aplicação da regra da necessitação a uma linha anterior $\mathbf{H_1}$, sabe-se que $\beta=\nec\varphi$ e que $\mathbf{H_1}={\entails\varphi}$.
                Deste modo, podemos demonstrar que $\Gamma\vdash\alpha\to\nec\varphi$ pela seguinte sucessão de dedução:

                \footnotesize
                \begin{fitch}
                    \fb\vdash\varphi&$\mathbf{H_1}$\\
                    \fa\Gamma\vdash\nec\varphi&$\hyperref[necessitation]{\mathbf{R_3}}\;\set{1}$\\
                    \fa\Gamma\vdash\nec\varphi\to\alpha\to\nec\varphi&$\hyperref[MA1]{\mathbf{A_1}}$\\
                    \fa\Gamma\vdash\alpha\to\nec\varphi&$\hyperref[detachment]{\mathbf{R_2}}\;\set{2,3}$.
                \end{fitch}
                \normalsize
            \end{case}

            \begin{case}
                \textsc{Caso 4.} Seja a sentença $\varphi_n=\beta$ gerada pela aplicação da regra do \emph{modus ponens} a duas sentenças $\varphi_i$ e $\varphi_j$ com $i<j<n$. Assumiremos, sem perda de generalidade, que $\varphi_j=\varphi_i\to\varphi_n$.
                Assim, a partir de $\mathbf{H}$ temos que $\mathbf{H_1}=\Gamma\entails\alpha\to\varphi_i$ e que $\mathbf{H_2}=\Gamma\entails\alpha\to\varphi_i\to\varphi_n$.
                Deste modo, podemos demonstrar que $\Gamma\vdash\alpha\to\nec\varphi$ pela seguinte sucessão de dedução:

                \footnotesize
                \begin{fitch}
                    \fb\Gamma\entails\alpha\to\varphi_j&$\mathbf{H_1}$\\
                    \fa\Gamma\entails\alpha\to\varphi_j\to\beta&$\mathbf{H_2}$\\
                    \fa\Gamma\entails(\alpha\to\varphi_j\to\beta)\to(\alpha\to\varphi_j)\to(\alpha\to\beta)&$\hyperref[MA2]{\mathbf{A_2}}$\\
                    \fa\Gamma\entails(\alpha\to\varphi_j)\to(\alpha\to\beta)&$\hyperref[detachment]{\mathbf{R_2}}\;\set{2,3}$\\
                    \fa\Gamma\entails\alpha\to\beta&$\hyperref[detachment]{\mathbf{R_2}}\;\set{1,4}$.
                \end{fitch}
                \normalsize
            \end{case}
            Uma vez provada a propriedade para todos os casos do passo de indução, provamos que o teorema da dedução vale para o sistema $\mathbf{S4}$.
        \end{proof}
    \end{theorem}

    \begin{theorem}[Enfraquecimento]\label{weakening}
        Se $\Gamma\subseteq\Delta$ e $\Gamma\vdash\alpha$, então $\Delta\vdash\alpha$.
        \begin{proof}
            Prova por indução forte sobre o tamanho da sucessão de dedução.
            Assim, suponhamos que o teorema do enfraquecimento valha para qualquer sucessão de dedução de tamanho $n<k$.
            Demonstraremos analisando-se os casos e valendo-se da suposição acima --- doravante chamada $\mathbf{H}$ --- o passo de indução, ou seja, que o teorema do enfraquecimento vale para sucessões de dedução de tamanho $n=k$.

            \begin{case}
                \textsc{Caso 1.}
                Se a linha derradeira da sucessão de dedução que prova $\Gamma\vdash\alpha$ tenha sido a invocação de alguma premissa $\alpha$, sabe-se que $\alpha\in\Gamma$.
                Como $\Gamma\subseteq\Delta$, sabe-se que $\alpha\in\Delta$.
                Deste modo, pode-se provar $\Delta\vdash\alpha$ pela invocação da mesma premissa $\alpha$.
            \end{case}

            \begin{case}
                \textsc{Caso 2.}
                Se a linha derradeira da sucessão de dedução que prova $\Gamma\vdash\alpha$ tenha sido a invocação de algum axioma, sabe-se que existe algum esquema $\mathbf{A_\alpha}\in\mathcal{A}$ que instancia $\alpha$.
                Deste modo, podemos demonstrar que $\Delta\vdash\alpha$ pela invocação do mesmo axioma $\mathbf{A_\alpha}$.
            \end{case}

            \begin{case}
                \textsc{Caso 3.}
                Se a linha derradeira da sucessão de dedução que prova $\Gamma\vdash\alpha$ tenha sido gerada pela aplicação da regra da necessitação a uma linha anterior, sabe-se que $\alpha=\nec\varphi$ e que ${\entails\varphi}$. Como pode-se provar $\varphi$ sem o uso de premissas, podemos aplicar a regra da necessitação a ${\entails\varphi}$ de modo a provar ${\Delta\entails\nec\varphi}$.
            \end{case}

            \begin{case}
                \textsc{Caso 4.} Seja a sentença $\varphi_n=\alpha$ gerada pela aplicação da regra do \emph{modus ponens} a duas sentenças $\varphi_i$ e $\varphi_j$ com $i<j<n$. Assumiremos, sem perda de generalidade, que $\varphi_j=\varphi_i\to\varphi_n$.
                Assim, a partir de $\mathbf{H}$ temos que $\mathbf{H_1}=\Delta\entails\varphi_i$ e que $\mathbf{H_2}=\Delta\entails\varphi_i\to\varphi_n$.
                Deste modo, podemos demonstrar que $\Delta\vdash\alpha$ pela aplicação da regra do \emph{modus ponens} a $\mathbf{H_1}$ e $\mathbf{H_2}$.
            \end{case}
            Uma vez provada a propriedade para todos os casos do passo de indução, provamos que o teorema do enfraquecimento vale para o sistema $\mathbf{S4}$.
        \end{proof}
    \end{theorem}

    Tendo-se provado o teorema da dedução, provaremos o teorema da generalização da regra da necessitação, conforme sugerido por~\cite{Troelstra}.
    Como apresentado abaixo, este teorema afirma que, caso possamos deduzir alguma sentença $\alpha$ a partir de um conjunto necessariamente verdadeiro de premissas, podemos concluir a necessidade desta sentença $\alpha$.

    \begin{theorem}[Generalização da necessitação]\label{gen-nec}
        Se $\nec\Gamma\entails\alpha$, então $\nec\Gamma\entails\nec\alpha$.
        \begin{proof}
            Prova por indução fraca sobre o tamanho $n$ do conjunto $\Gamma$ \citep{Troelstra}.
            Deste modo, a prova consiste em dois casos: a base de indução e o passo de indução.
            \begin{case}
                \textsc{Caso 1.}
                Seja $|\Gamma|=0$.
                Sabe-se que $\Gamma=\varnothing$ e, portanto, que $\mathbf{H_1}=\proves\alpha$.
                Deste modo, pode-se demonstrar que $\entails\nec\alpha$ trivialmente pela aplicação da regra da necessitação \hyperref[necessitation]{$\mathbf{R_3}$} sobre $\mathbf{H_1}$.
            \end{case}
            \begin{case}
                \textsc{Caso 2.} 
                Para o passo, suponhamos que a generalização da regra da necessitação valha para qualquer conjunto $\Delta$ de tamanho $n=k$.
                Demonstraremos, valendo-se da suposição acima --- doravante chamada $\mathbf{H_2}$ --- e pela sucessão de dedução apresentada abaixo, que a generalização da regra da necessitação vale para conjuntos $\Delta$ de tamanho $n=k+1$.

                \footnotesize
                \begin{fitch}
                    \fb\nec\Delta\entails\beta&$\mathbf{H_1}$\\
                    \fa\nec\Delta\entails\beta\to\nec\alpha\to\beta&$\hyperref[MA1]{\mathbf{A_1}}$\\
                    \fa\nec\Delta\entails\nec\alpha\to\beta&$\hyperref[detachment]{\mathbf{R_2}}\;\set{1,2}$\\
                    \fa\nec\Delta\entails\nec(\nec\alpha\to\beta)&$\mathbf{H_2}\;\set{3}$\\
                    \fa\nec\Delta\entails\nec(\nec\alpha\to\beta)\to\nec\nec\alpha\to\nec\beta&$\hyperref[MB1]{\mathbf{B_1}}$\\
                    \fa\nec\Delta\entails\nec\nec\alpha\to\nec\beta&$\hyperref[detachment]{\mathbf{R_2}}\;\set{4,5}$\\
                    \fa\nec\Delta\entails\nec\alpha\to\nec\nec\alpha&$\hyperref[MB3]{\mathbf{B_3}}$\\
                    \fa\nec\Delta\entails(\nec\alpha\to\nec\nec\alpha)\to(\nec\nec\alpha\to\nec\beta)\to\nec\alpha\to\nec\beta&$\hyperref[MA2]{\mathbf{A_2}}$\\
                    \fa\nec\Delta\entails(\nec\nec\alpha\to\nec\beta)\to\nec\alpha\to\nec\beta&$\hyperref[detachment]{\mathbf{R_2}}\;\set{7,8}$\\
                    \fa\nec\Delta\entails\nec\alpha\to\nec\beta&$\hyperref[detachment]{\mathbf{R_1}}\;\set{6,9}$\\
                    \fa\nec\Delta\cup\set{\nec\alpha}\entails\nec\alpha&$\hyperref[premisse]{\mathbf{R_1}}$\\
                    \fa\nec\Delta\cup\set{\nec\alpha}\entails\nec\alpha\to\nec\beta&$\hyperref[weakening]{\mathbf{T_2}}\;\set{10}$\\
                    \fa\nec\Delta\cup\set{\nec\alpha}\entails\nec\beta&$\hyperref[detachment]{\mathbf{R_2}}\;\set{11,12}$.
                \end{fitch}
            \end{case}
            \vspace{.5\baselineskip}
            Tendo-se provado a base e o passo de indução, podemos concluir que generalização da regra da necessitação vale, ou seja, que se $\nec\Gamma\entails\alpha$, então $\nec\Gamma\entails\nec\alpha$.
        \end{proof}
    \end{theorem}

    Uma vez provada a generalização da regra da necessitação, a prova da regra da dedução estrita --- conforme descrito por~\cite{Barcan, Marcus} --- torna-se trivial, como pode ser visto abaixo. Esta regra afirma que, dada uma dedução de $\beta$ partindo de um conjunto de premissas necessariamente verdadeiras e uma premissa $\alpha$, podemos deduzir $\nec(\alpha\to\beta)$ a partir desse conjunto de premissas necessariamente verdadeiras. Isso nos permite simplificar as provas de correção das traduções, uma vez que uma das traduções apresentadas mapeia implicações materiais do sistema intuicionista em implicações estritas.

    \begin{theorem}[Dedução estrita]\label{strictdeduction}
        $\text{Se }\nec\Gamma\cup\set{\alpha}\entails\beta\text{, então }\nec\Gamma\entails\nec(\alpha\to\beta)$.
        \begin{proof}
            Pode ser provado pela seguinte sucessão de dedução:
            \footnotesize
            \begin{fitch}
                \fb\nec\Gamma\cup\set{\alpha}\entails\beta&$\mathbf{H_1}$\\
                \fa\nec\Gamma\entails\alpha\to\beta&\refer{deduction}{T}$\;\set{1}$\\
                \fa\nec\Gamma\entails\nec(\alpha\to\beta)&\refer{gen-nec}{T}$\;\set{2}$
            \end{fitch}
            \normalsize
            Estando assim demonstrada a proposição.
        \end{proof}
    \end{theorem}

    Agora, provaremos a aplicação da regra do \emph{modus ponens} a uma implicação estrita. Essa regra afirma que, dada uma prova de $\alpha$ e uma prova de $\nec(\alpha\to\beta)$ a partir de um conjunto de premissas, sabe-se que deve haver alguma prova de $\beta$ a partir desse mesmo conjunto de premissas.

    \begin{theorem}[Separação estrita]\label{strictsep}
        Se $\Gamma\entails\alpha$ e $\Gamma\entails\nec(\alpha\to\beta)$, então $\Gamma\entails\beta$.
        \begin{proof}
            Pode ser provado pela seguinte sucessão de dedução:
            \footnotesize
            \begin{fitch}
                \fb\Gamma\entails\alpha&$\mathbf{H_1}$\\
                \fa\Gamma\entails\nec(\alpha\to\beta)&$\mathbf{H_2}$\\
                \fa\Gamma\entails\nec(\alpha\to\beta)\to\alpha\to\beta&$\hyperref[MB2]{\mathbf{B_2}}$\\
                \fa\Gamma\entails\alpha\to\beta&$\hyperref[detachment]{\mathbf{R_2}}\;\set{2,3}$\\
                \fa\Gamma\entails\beta&$\hyperref[detachment]{\mathbf{R_2}}\;\set{1,4}$
            \end{fitch}
            \normalsize
            Estando assim demonstrada a proposição.
        \end{proof}
    \end{theorem}

    \begin{theorem}[Importação]\label{importation}
        Se $\Gamma\entails\alpha\to\beta\to\gamma$ então $\Gamma\entails\alpha\wedge\beta\to\gamma$.
        \begin{proof}
            Pode ser provado pela seguinte sucessão de dedução:
            \footnotesize
            \begin{fitch}
                \fb\Gamma\entails\alpha\to\beta\to\gamma&$\mathbf{H_1}$\\
                \fa\Gamma\cup\set{\alpha\wedge\beta}\entails\alpha\wedge\beta&$\hyperref[premisse]{\mathbf{R_1}}$\\
                \fa\Gamma\cup\set{\alpha\wedge\beta}\entails\alpha\wedge\beta\to\alpha&\hyperref[MA4]{$\mathbf{A_4}$}\\
                \fa\Gamma\cup\set{\alpha\wedge\beta}\entails\alpha&$\hyperref[detachment]{\mathbf{R_2}}\;\set{2,3}$\\
                \fa\Gamma\cup\set{\alpha\wedge\beta}\entails\alpha\wedge\beta\to\beta&\hyperref[MA5]{$\mathbf{A_5}$}\\
                \fa\Gamma\cup\set{\alpha\wedge\beta}\entails\beta&$\hyperref[detachment]{\mathbf{R_2}}\;\set{2,5}$\\
                \fa\Gamma\cup\set{\alpha\wedge\beta}\entails\alpha\to\beta\to\gamma&\refer{weakening}{T}$\;\set{1}$\\
                \fa\Gamma\cup\set{\alpha\wedge\beta}\entails\beta\to\gamma&$\hyperref[detachment]{\mathbf{R_2}}\;\set{4,7}$\\
                \fa\Gamma\cup\set{\alpha\wedge\beta}\entails\gamma&$\hyperref[detachment]{\mathbf{R_2}}\;\set{6,8}$\\
                \fa\Gamma\entails\alpha\wedge\beta\to\gamma&\refer{deduction}{T}$\;\set{9}$\\
            \end{fitch}
            \normalsize
            Estando assim demonstrada a proposição.
        \end{proof}
    \end{theorem}

    Os lemas 2 a 13 abaixo serão demonstrados a fim de diminuir o tamanho das provas futuras acerca do isomorfismo entre as traduções e a correção da tradução \emph{call-by-value}.

    \begin{lemma}\label{explosion}
        $\entails\bot\to\alpha$.
        \begin{proof}
            Pode ser provado pela seguinte sucessão de dedução:
            \footnotesize
            \begin{fitch}
                \fb\set{\bot}\entails\bot&$\hyperref[premisse]{\mathbf{R_1}}$\\
                \fa\set{\bot}\entails\bot\to(\alpha\to\bot)\to\bot&$\hyperref[MA1]{\mathbf{A_1}}$\\
                \fa\set{\bot}\entails\neg\neg\alpha&$\hyperref[detachment]{\mathbf{R_2}}\;\set{1,2}$\\
                \fa\set{\bot}\entails\neg\neg\alpha\to\alpha&$\hyperref[MANEG]{\mathbf{A_\neg}}$\\
                \fa\set{\bot}\entails\alpha&$\hyperref[detachment]{\mathbf{R_2}}\;\set{3,4}$\\
                \fa\entails\bot\to\alpha&$\hyperref[deduction]{\mathbf{T_1}}\;\set{5}$.
            \end{fitch}
            \normalsize
            Estando assim demonstrada a proposição.
        \end{proof}
    \end{lemma}

    \begin{lemma}\label{contrapositive}
        $\entails(\alpha\to\beta)\to(\neg\beta\to\neg\alpha)$.
        \begin{proof}
            Pode ser provado pela seguinte sucessão de dedução:
            \footnotesize
            \begin{fitch}
                \fb\set{\alpha\to\beta,\neg\beta}\entails\beta\to\bot&$\hyperref[premisse]{\mathbf{R_1}}$\\
                \fa\set{\alpha\to\beta,\neg\beta}\entails(\beta\to\bot)\to\alpha\to(\beta\to\bot)&\hyperref[MA1]{$\mathbf{A_1}$}\\
                \fa\set{\alpha\to\beta,\neg\beta}\entails\alpha\to\beta\to\bot&$\hyperref[detachment]{\mathbf{R_2}}\;\set{1,2}$\\
                \fa\set{\alpha\to\beta,\neg\beta}\entails(\alpha\to\beta\to\bot)\to(\alpha\to\beta)\to(\alpha\to\bot)&\hyperref[MA2]{$\mathbf{A_2}$}\\
                \fa\set{\alpha\to\beta,\neg\beta}\entails\alpha\to\beta&$\hyperref[premisse]{\mathbf{R_1}}$\\
                \fa\set{\alpha\to\beta,\neg\beta}\entails(\alpha\to\beta)\to(\alpha\to\bot)&$\hyperref[detachment]{\mathbf{R_2}}\;\set{3,4}$\\
                \fa\set{\alpha\to\beta,\neg\beta}\entails\neg\alpha&$\hyperref[detachment]{\mathbf{R_2}}\;\set{5,6}$\\
                \fa\set{\alpha\to\beta}\entails\neg\beta\to\neg\alpha&\refer{deduction}{T}$\;\set{7}$\\
                \fa\entails(\alpha\to\beta)\to(\neg\beta\to\neg\alpha)&\refer{deduction}{T}$\;\set{8}$.
            \end{fitch}
            \normalsize
            Estando assim demonstrada a proposição.
        \end{proof}
    \end{lemma}

    \begin{lemma}\label{nec-distr}
        $\Gamma\entails\nec(\alpha\wedge\beta)\to\nec\alpha\wedge\nec\beta$.
        \begin{proof}
            Pode ser provado pela seguinte sucessão de dedução:
            
            \footnotesize
            \begin{fitch}
                \fb\set{\nec(\alpha\wedge\beta)}\entails\nec(\alpha\wedge\beta)&$\hyperref[premisse]{\mathbf{R_1}}$\\
                \fa\set{\nec(\alpha\wedge\beta)}\entails\nec(\alpha\wedge\beta)\to\alpha\wedge\beta&\hyperref[MB2]{$\mathbf{B_2}$}\\
                \fa\set{\nec(\alpha\wedge\beta)}\entails\alpha\wedge\beta&$\hyperref[detachment]{\mathbf{R_2}}\;\set{1,2}$\\
                \fa\set{\nec(\alpha\wedge\beta)}\entails\alpha\wedge\beta\to\alpha&\hyperref[MA4]{$\mathbf{A_4}$}\\
                \fa\set{\nec(\alpha\wedge\beta)}\entails\alpha&$\hyperref[detachment]{\mathbf{R_2}}\;\set{3,4}$\\
                \fa\set{\nec(\alpha\wedge\beta)}\entails\nec\alpha&\refer{gen-nec}{T}$\;\set{5}$\\
                \fa\set{\nec(\alpha\wedge\beta)}\entails\alpha\wedge\beta\to\beta&\hyperref[MA5]{$\mathbf{A_5}$}\\
                \fa\set{\nec(\alpha\wedge\beta)}\entails\beta&$\hyperref[detachment]{\mathbf{R_2}}\;\set{3,7}$\\
                \fa\set{\nec(\alpha\wedge\beta)}\entails\nec\beta&\refer{gen-nec}{T}$\;\set{8}$\\
                \fa\set{\nec(\alpha\wedge\beta)}\entails\nec\alpha\to\nec\beta\to\nec\alpha\wedge\nec\beta&\hyperref[MA3]{$\mathbf{A_3}$}\\
                \fa\set{\nec(\alpha\wedge\beta)}\entails\nec\beta\to\nec\alpha\wedge\nec\beta&$\hyperref[detachment]{\mathbf{R_2}}\;\set{6,10}$\\
                \fa\set{\nec(\alpha\wedge\beta)}\entails\nec\alpha\wedge\nec\beta&$\hyperref[detachment]{\mathbf{R_2}}\;\set{9,11}$\\
                \fa\entails\nec(\alpha\wedge\beta)\to\nec\alpha\wedge\nec\beta&\refer{deduction}{T}$\;\set{12}$\\
                \fa\Gamma\entails\nec(\alpha\wedge\beta)\to\nec\alpha\wedge\nec\beta&\refer{weakening}{T}$\;\set{13}$\\
            \end{fitch}
            \normalsize

            Estando assim demonstrada a proposição.
        \end{proof}
    \end{lemma}

    \begin{lemma}\label{nec-undistr}
        $\Gamma\entails\nec\alpha\wedge\nec\beta\to\nec(\alpha\wedge\beta)$.
        \begin{proof}
            Pode ser provado pela seguinte sucessão de dedução:
            
            \footnotesize
            \begin{fitch}
                \fb\set{\nec\alpha,\nec\beta}\entails\nec\alpha&$\hyperref[premisse]{\mathbf{R_1}}$\\
                \fa\set{\nec\alpha,\nec\beta}\entails\nec\alpha\to\alpha&\hyperref[MB2]{$\mathbf{B_2}$}\\
                \fa\set{\nec\alpha,\nec\beta}\entails\alpha&$\hyperref[detachment]{\mathbf{R_2}}\;\set{1,2}$\\
                \fa\set{\nec\alpha,\nec\beta}\entails\nec\beta&$\hyperref[premisse]{\mathbf{R_1}}$\\
                \fa\set{\nec\alpha,\nec\beta}\entails\nec\beta\to\beta&\hyperref[MB2]{$\mathbf{B_2}$}\\
                \fa\set{\nec\alpha,\nec\beta}\entails\beta&$\hyperref[detachment]{\mathbf{R_2}}\;\set{4,5}$\\
                \fa\set{\nec\alpha,\nec\beta}\entails\alpha\to\beta\to\alpha\wedge\beta&\hyperref[MA3]{$\mathbf{A_3}$}\\
                \fa\set{\nec\alpha,\nec\beta}\entails\beta\to\alpha\wedge\beta&$\hyperref[detachment]{\mathbf{R_2}}\;\set{3,7}$\\
                \fa\set{\nec\alpha,\nec\beta}\entails\alpha\wedge\beta&$\hyperref[detachment]{\mathbf{R_2}}\;\set{6,8}$\\
                \fa\set{\nec\alpha,\nec\beta}\entails\nec(\alpha\wedge\beta)&\refer{gen-nec}{T}$\;\set{9}$\\
                \fa\set{\nec\alpha}\entails\nec\beta\to\nec(\alpha\wedge\beta)&\refer{deduction}{T}$\;\set{10}$\\
                \fa\entails\nec\alpha\to\nec\beta\to\nec(\alpha\wedge\beta)&\refer{deduction}{T}$\;\set{11}$\\
                \fa\entails\nec\alpha\wedge\nec\beta\to\nec(\alpha\wedge\beta)&\refer{importation}{T}$\;\set{12}$\\
                \fa\Gamma\entails\nec\alpha\wedge\nec\beta\to\nec(\alpha\wedge\beta)&\refer{weakening}{T}$\;\set{13}$\\
            \end{fitch}
            \normalsize

            Estando assim demonstrada a proposição.
        \end{proof}
    \end{lemma}

    \begin{lemma}\label{comp}
        Se $\Gamma\entails\alpha\to\beta$ e $\Gamma\entails\beta\to\gamma$, então $\Gamma\entails\alpha\to\gamma$.
        \begin{proof}
            Pode ser provado pela seguinte sucessão de dedução:
            
            \footnotesize
            \begin{fitch}
                \fb\Gamma\entails\alpha\to\beta&$\mathbf{H_1}$\\
                \fa\Gamma\entails\beta\to\gamma&$\mathbf{H_2}$\\
                \fa\Gamma\cup\set{\alpha}\entails\alpha&$\hyperref[premisse]{\mathbf{R_1}}$\\
                \fa\Gamma\cup\set{\alpha}\entails\alpha\to\beta&\refer{weakening}{T}$\;\set{1}$\\
                \fa\Gamma\cup\set{\alpha}\entails\beta&$\hyperref[detachment]{\mathbf{R_2}}\;\set{3,4}$\\
                \fa\Gamma\cup\set{\alpha}\entails\beta\to\gamma&\refer{weakening}{T}$\;\set{2}$\\
                \fa\Gamma\cup\set{\alpha}\entails\gamma&$\hyperref[detachment]{\mathbf{R_2}}\;\set{5,6}$\\
                \fa\Gamma\entails\alpha\to\gamma&\refer{deduction}{T}$\;\set{8}$\\
            \end{fitch}
            \normalsize
            
            Estando assim demonstrada a proposição.
        \end{proof}
    \end{lemma}

    \begin{lemma}\label{neg-intro}
        $\entails\alpha\to\neg\neg\alpha$
        \begin{proof}
            Pode ser provado pela seguinte sucessão de dedução:
            \footnotesize
            \begin{fitch}
                \fb\set{\alpha,\neg\alpha}\entails\alpha&$\hyperref[premisse]{\mathbf{R_1}}$\\
                \fa\set{\alpha,\neg\alpha}\entails\alpha\to\bot&$\hyperref[premisse]{\mathbf{R_1}}$\\
                \fa\set{\alpha,\neg\alpha}\entails\bot&$\hyperref[detachment]{\mathbf{R_2}}\;\set{1,2}$\\
                \fa\set{\alpha}\entails\neg\neg\alpha&$\hyperref[deduction]{\mathbf{T_{\getrefnumber{deduction}}}}\;\set{3}$\\
                \fa\entails\alpha\to\neg\neg\alpha&$\hyperref[deduction]{\mathbf{T_{\getrefnumber{deduction}}}}\;\set{4}$\\
            \end{fitch}
            \normalsize
            Estando assim demonstrada a proposição.
        \end{proof}
    \end{lemma}

    \begin{lemma}[Dilema construtivo]\label{or-subst}
        Se $\Gamma\entails\alpha\to\gamma$ e $\Gamma\entails\beta\to\delta$, então $\Gamma\entails\alpha\vee\beta\to\gamma\vee\delta$.
        \begin{proof}
            Pode ser provado pela seguinte sucessão de dedução:
            \footnotesize 
            \begin{fitch}
                \fb\Gamma\cup\set{\alpha\vee\beta}\entails\alpha\to\gamma&$\mathbf{H_1}$\\
                \fa\Gamma\cup\set{\alpha\vee\beta}\entails(\alpha\to\gamma)\to\alpha\to\gamma\vee\delta&$\hyperref[or-left]{\mathbf{L_{\getrefnumber{or-left}}}}$\\
                \fa\Gamma\cup\set{\alpha\vee\beta}\entails\alpha\to\gamma\vee\delta&$\hyperref[detachment]{\mathbf{R_2}}\;\set{1,2}$\\
                \fa\Gamma\cup\set{\alpha\vee\beta}\entails\beta\to\delta&$\mathbf{H_1}$\\
                \fa\Gamma\cup\set{\alpha\vee\beta}\entails(\beta\to\delta)\to\beta\to\gamma\vee\delta&$\hyperref[or-right]{\mathbf{L_{\getrefnumber{or-right}}}}$\\
                \fa\Gamma\cup\set{\alpha\vee\beta}\entails\beta\to\gamma\vee\delta&$\hyperref[detachment]{\mathbf{R_2}}\;\set{4,5}$\\
                \fa\Gamma\cup\set{\alpha\vee\beta}\entails\alpha\vee\beta&$\hyperref[premisse]{\mathbf{R_1}}$\\
                \fa\Gamma\cup\set{\alpha\vee\beta}\entails(\alpha\to\gamma\vee\delta)\to(\beta\to\gamma\vee\delta)\to\alpha\vee\beta\to\gamma\vee\delta&$\hyperref[MA8]{\mathbf{A_8}}$\\
                \fa\Gamma\cup\set{\alpha\vee\beta}\entails(\beta\to\gamma\vee\delta)\to\alpha\vee\beta\to\gamma\vee\delta&$\hyperref[detachment]{\mathbf{R_2}}\;\set{3,8}$\\
                \fa\Gamma\cup\set{\alpha\vee\beta}\entails\alpha\vee\beta\to\gamma\vee\delta&$\hyperref[detachment]{\mathbf{R_2}}\;\set{6,9}$\\
                \fa\Gamma\cup\set{\alpha\vee\beta}\entails\gamma\vee\delta&$\hyperref[detachment]{\mathbf{R_2}}\;\set{7,10}$\\
                \fa\Gamma\entails\alpha\vee\beta\to\gamma\vee\delta&$\hyperref[deduction]{\mathbf{T_{\getrefnumber{deduction}}}}\;\set{11}$\\
            \end{fitch}
            \normalsize
            Estando assim demonstrada a proposição.
        \end{proof}
    \end{lemma}

    \begin{lemma}\label{or-distr}
        Se $\Gamma\entails\nec\alpha\vee\nec\beta$, então $\Gamma\entails\nec(\alpha\vee\beta)$.
        \begin{proof}
            Pode ser provado pela seguinte sucessão de dedução:
            \footnotesize
            \begin{fitch}
                \fb\set{\nec\alpha}\entails\nec\alpha&$\hyperref[premisse]{\mathbf{R_1}}$\\ 
                \fa\set{\nec\alpha}\entails\nec\alpha\to\alpha&$\hyperref[MB2]{\mathbf{B_2}}$\\
                \fa\set{\nec\alpha}\entails\alpha&$\hyperref[detachment]{\mathbf{R_2}}\;\set{1,2}$\\
                \fa\set{\nec\alpha}\entails\alpha\to\alpha\vee\beta&$\hyperref[MA6]{\mathbf{A_6}}$\\
                \fa\set{\nec\alpha}\entails\alpha\vee\beta&$\hyperref[detachment]{\mathbf{R_2}}\;\set{3,4}$\\
                \fa\set{\nec\alpha}\entails\nec(\alpha\vee\beta)&\refer{gen-nec}{T}$\;\set{5}$\\
                \fa\set{\nec\beta}\entails\nec\beta&$\hyperref[premisse]{\mathbf{R_1}}$\\
                \fa\set{\nec\beta}\entails\nec\beta\to\beta&$\hyperref[MB2]{\mathbf{B_2}}$\\
                \fa\set{\nec\beta}\entails\beta&$\hyperref[detachment]{\mathbf{R_2}}\;\set{7,8}$\\
                \fa\set{\nec\beta}\entails\beta\to\alpha\vee\beta&$\hyperref[MA7]{\mathbf{A_7}}$\\
                \fa\set{\nec\beta}\entails\alpha\vee\beta&$\hyperref[detachment]{\mathbf{R_2}}\;\set{9,10}$\\
                \fa\set{\nec\beta}\entails\nec(\alpha\vee\beta)&\refer{gen-nec}{T}$\;\set{11}$\\
                \fa\entails\nec\alpha\to\nec(\alpha\vee\beta)&\refer{deduction}{T}$\;\set{6}$\\
                \fa\entails\nec\beta\to\nec(\alpha\vee\beta)&\refer{deduction}{T}$\;\set{12}$\\
                \fa\entails(\nec\alpha\to\nec(\alpha\vee\beta))\to(\nec\beta\to\nec(\alpha\vee\beta))\to\nec\alpha\vee\nec\beta\to\nec(\alpha\vee\beta)&$\hyperref[MA8]{\mathbf{A_8}}$\\
                \fa\entails(\nec\beta\to\nec(\alpha\vee\beta))\to\nec\alpha\vee\nec\beta\to\nec(\alpha\vee\beta)&$\hyperref[detachment]{\mathbf{R_2}}\;\set{13,15}$\\
                \fa\entails\nec\alpha\vee\nec\beta\to\nec(\alpha\vee\beta)&$\hyperref[detachment]{\mathbf{R_2}}\;\set{14,16}$\\
                \fa\Gamma\entails\nec\alpha\vee\nec\beta\to\nec(\alpha\vee\beta)&\refer{weakening}{T}$\;\set{17}$\\ 
            \end{fitch}
            \normalsize
            Estando assim demonstrada a proposição.
        \end{proof}
    \end{lemma}
