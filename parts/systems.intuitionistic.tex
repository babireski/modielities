\section{Intuicionista}
    \begin{definition}[$\mathcal{L}_\mathbf{I}$]
        A linguagem do sistema intuicionista, denotada $\mathcal{L}_\mathbf{I}$, consiste no menor conjunto induzido a partir das seguintes regras:
        \begin{align*}
            &\bot\in\mathcal{L}_\mathbf{I} \\
            &\mathcal{P}\subseteq\mathcal{L}_\mathbf{I} \\
            &\alpha,\beta\in\mathcal{L}_\mathbf{I}\Rightarrow\alpha\circ\beta\in\mathcal{L}_\mathbf{I}\text{, para }\circ\in\set{\wedge,\vee,\to}\text{.}
            \tag*{\qed}
        \end{align*}
    \end{definition}

    \begin{notation}
        Serão usadas as seguintes abreviações:
        \begin{align*}
            \top&\coloneqq\bot\to\bot\\
            \neg\alpha&\coloneqq\alpha\to\bot\\
            \alpha\leftrightarrow\beta&\coloneqq(\alpha\to\beta)\wedge(\beta\to\alpha)
        \end{align*}
    \end{notation}

    \begin{definition}
        A axiomatização do sistema intuicionista consiste no conjunto de esquemas de axiomas $\mathcal{A}=\set{\mathbf{A}_i\mid i\in[1,8]\vee i=\bot}$ e no conjunto de regras $\mathcal{R}=\set{\mathbf{R_1}}$, definidos abaixo:
        \begin{alignat*}{3}
            & \mathbf{A_1}\quad && \alpha\to\beta\to\alpha \\
            & \mathbf{A_2}\quad && (\alpha\to\beta\to\gamma)\to(\alpha\to\beta)\to(\alpha\to\gamma) \\
            & \mathbf{A_3}\quad && \alpha\to\beta\to\alpha\wedge\beta \\
            & \mathbf{A_4}\quad && \alpha\wedge\beta\to\alpha \\
            & \mathbf{A_5}\quad && \alpha\wedge\beta\to\beta \\
            & \mathbf{A_6}\quad && \alpha\to\alpha\vee\beta \\
            & \mathbf{A_7}\quad && \beta\to\alpha\vee\beta \\
            & \mathbf{A_8}\quad && (\alpha\to\gamma)\to(\beta\to\gamma)\to(\alpha\vee\beta\to\gamma) \\
            & \mathbf{A_\bot}\quad && \bot\to\alpha \\
            & \mathbf{R_1}\quad && \text{Se }\Gamma\vdash\alpha\text{ e }\Delta\vdash\alpha\to\beta\text{, então }\Gamma\cup\Delta\vdash\beta\text{.} & \tag*{\qed}
        \end{alignat*}   
    \end{definition}

    Chamaremos $\mathbf{R_1}$ de \emph{regra da separação}.
