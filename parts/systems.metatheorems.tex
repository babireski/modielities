\section{Metateoremas}
    Nesta seção apresentaremos alguns teoremas para os sistemas modais que permitirão simplificar muito as provas apresentadas no decorrer deste trabalho.
    Primeiramente, provaremos a regra da dedução \hyperref[deduction]{$\mathbf{T_1}$} com base na prova apresentada por~\cite{Hakli}.
    Pequenas alterações na prova foram feitas para garantir a adequação com a axiomatização provida na definição \hyperref[m-axioms]{$\mathbf{D_{13}}$}.

    \begin{theorem}
        \label{deduction}
        $\text{Se }\Gamma\cup\set{\alpha}\vdash\beta\text{, então }\Gamma\vdash\alpha\to\beta$.

        \begin{proof}
            Prova por indução forte sobre o tamanho da sucessão de dedução\footnote{Note que, para a indução forte, não se faz preciso provar nenhuma base \citep{Velleman}.}.
            Assim, suponhamos que o teorema da dedução valha para qualquer sucessão dedução de tamanho $n<k$.
            Demonstraremos, analisando-se os casos, que o teorema da dedução vale para sucessões de dedução de tamanho $n=k+1$.

            \begin{case}
                \textsc{Caso 1.}
                Se a linha derradeira da sucessão de dedução que prova $\Gamma\cup\set{\alpha}\vdash\beta$ tenha sido a evocação de alguma premissa, sabe-se que $\beta\in\Gamma\cup\set{\alpha}$.
                Deste modo, existem outros dois casos a serem analisados.
            \end{case}

            \begin{subcase}
                \textsc{Caso 1.1.}
                Se a linha derradeira da sucessão de dedução que prova $\Gamma\cup\set{\alpha}\vdash\beta$ tenha sido a evocação de alguma premissa do conjunto $\Gamma$, podemos demonstrar que $\Gamma\vdash\alpha\to\beta$ pela seguinte sucessão de dedução:

                \begin{fitch}
                    \fa\Gamma\vdash\beta&$\mathbf{P}$\\
                    \fa\Gamma\vdash\beta\to\alpha\to\beta&$\mathbf{A_1}$\\
                    \fa\Gamma\vdash\alpha\to\beta&$\mathbf{R_1}\;\set{1,2}$.
                \end{fitch}
            \end{subcase}

            \begin{subcase}
                \textsc{Caso 1.1.}
                Se a linha derradeira da sucessão de dedução que prova $\Gamma\cup\set{\alpha}\vdash\beta$ tenha sido a evocação da premissa $\alpha$, sabe-se que $\beta=\alpha$.
                Deste modo, podemos demonstrar que $\Gamma\vdash\alpha\to\alpha$ pela seguinte sucessão de dedução:

                \begin{fitch}
                    \fa\Gamma\vdash\alpha\to\alpha\to\alpha&$\mathbf{A_1}$\\
                    \fa\Gamma\vdash\alpha\to(\alpha\to\alpha)\to\alpha&$\mathbf{A_1}$\\
                    \fa\Gamma\vdash(\alpha\to(\alpha\to\alpha)\to\alpha)\to(\alpha\to\alpha\to\alpha)\to\alpha\to\alpha&$\mathbf{A_2}$\\
                    \fa\Gamma\vdash(\alpha\to\alpha\to\alpha)\to\alpha\to\alpha&$\mathbf{R_1}\;\set{2,3}$\\
                    \fa\Gamma\vdash\alpha\to\alpha&$\mathbf{R_1}\;\set{1,4}$.
                \end{fitch}
            \end{subcase}

            \begin{case}
                \textsc{Caso 2.}
                Se a linha derradeira da sucessão de dedução que prova $\Gamma\cup\set{\alpha}\vdash\beta$ tenha sido a evocação de algum axioma, sabe-se que existe algum esquema $A_\beta\in\mathcal{A}$ que gera $\beta$.
                Deste modo, podemos demonstrar que $\Gamma\vdash\alpha\to\beta$ pela seguinte sucessão de dedução:

                \begin{fitch}
                    \fa\Gamma\vdash\beta&$\mathbf{A_\beta}$\\
                    \fa\Gamma\vdash\beta\to\alpha\to\beta&$\mathbf{A_1}$\\
                    \fa\Gamma\vdash\alpha\to\beta&$\mathbf{R_1}\;\set{1,2}$.
                \end{fitch}
            \end{case}

            \begin{case}
                \textsc{Caso 3.}
                Se a linha derradeira da sucessão de dedução que prova $\Gamma\cup\set{\alpha}\vdash\beta$ tenha sido gerada pela aplicação da regra da necessitação a uma linha anterior, sabe-se que $\beta=\nec\varphi$ e $\vdash\varphi$.
                Deste modo, podemos demonstrar que $\Gamma\vdash\alpha\to\nec\varphi$ pela seguinte sucessão de dedução:

                \begin{fitch}
                    \fa\vdash\varphi&$\mathbf{H}$\\
                    \fa\Gamma\vdash\nec\varphi&$\mathbf{R_2}\;\set{1}$\\
                    \fa\Gamma\vdash\nec\varphi\to\alpha\to\nec\varphi&$\mathbf{A_1}$\\
                    \fa\Gamma\vdash\alpha\to\nec\varphi&$\mathbf{R_1}\;\set{2,3}$.
                \end{fitch}
            \end{case}

            \begin{case}
                \textsc{Caso 4.} Caso $\beta$ tenha sido gerado pela regra da separação.
            \end{case}
        \end{proof}
    \end{theorem}

    \begin{theorem}
        $\text{Se }\nec\Gamma\cup\set{\nec\alpha}\vdash\beta\text{, então }\nec\Gamma\vdash\nec(\nec\alpha\to\beta)$.

        \begin{proof}
            Prova em~\cite{Marcus}.
        \end{proof}
    \end{theorem}
