\section{Completude}
    Esta seção dedica-se a provar que as traduções são completas, ou seja, que derivações de sentenças traduzidas no sistema de destino implicam em derivações no sistema de origem. Para tanto, deixaremos de basear-nos nas provas de~\cite{Troelstra} e passaremos a basear-nos na prova de~\cite{Flagg}. Este abandono deve-se ao uso de propriedades de sequentes na demontração que seriam complicadas de acomodar ao sistema de dedução usado neste trabalho, enquanto escolha da prova usada deu-se por esta ser feita construtivamente. A construtividade da prova releva por esta conter uma computação --- ou seja, um procedimento que descreve \emph{como} transformar uma prova de uma sentença traduzida no sistema de destino em uma prova traduzida no sistema de origem.

    A prova de~\cite{Flagg} baseia-se na definição de uma contratradução da tradução por necessitação foco deste trabalho. Com isso, eles reduzem o problema de provar a completude da tradução ao problema de provar a correção da contratradução, coisa que pode ser feita por indução sobre o tamanho da prova em conjunto com uma coleção de lemas. Esta nova tradução faz grande uso de implicações duplas --- ou seja, sentenças da forma $(\alpha\to\varepsilon)\to\varepsilon$ ---, motivo que justifica a introdução abaixo de uma nova notação para a implicação.

    \begin{notation}
        Sejam $\alpha,\varepsilon\in\mathcal{L}$ sentenças proposicionais bem-formadas. Denota-se por $\neg_\varepsilon\alpha$ a sentença $\alpha\to\varepsilon$, ou seja, a implicação de $\alpha$ em $\varepsilon$.
    \end{notation}

    \begin{definition}[$\bullet^\varepsilon$] Seja $\Gamma\in\wp(\mathcal{L})$ um conjunto de sentenças proposicionais bem-formadas e seja $\varepsilon\in\Gamma$ uma dessas sentenças. Define-se a tradução $\bullet^\varepsilon:\mathcal{L}_{\nec}\times\wp(\mathcal{L})\times\mathcal{L}\to\mathcal{L}$ do sistema intuicionista ao sistema modal $\mathbf{S4}$ indutivamente da seguinte maneira:
        \begin{align*}
            a^\varepsilon&\mapsto\neg_\varepsilon\neg_\varepsilon a\\
            \bot^\varepsilon&\mapsto\bot\\
            (\nec\varphi)^\varepsilon&\mapsto\neg_\varepsilon\neg_\varepsilon\textstyle\bigwedge_{\gamma}\varphi^\gamma\displaybreak[0]\\
            {(\varphi \wedge \psi)}^\varepsilon&\mapsto \varphi^\varepsilon\wedge\psi^\varepsilon\displaybreak[0]\\
            {(\varphi \vee \psi)}^\varepsilon&\mapsto \neg_\varepsilon\neg_\varepsilon(\varphi^\varepsilon\vee\psi^\varepsilon)\displaybreak[0]\\
            {(\varphi \to \psi)}^\varepsilon&\mapsto \varphi^\varepsilon\to\psi^\varepsilon\tag*{\qed}
        \end{align*}
    \end{definition}

    \begin{theorem}
        Se $\Gamma\entails\alpha$, então $\neg_\varepsilon\neg_\varepsilon\Gamma\entails\neg_\varepsilon\neg_\varepsilon\alpha$.
    \end{theorem}

    \begin{theorem}
        Se $\Gamma\entails\alpha$, então $\Gamma^\varepsilon\entails\alpha^\varepsilon$.

        \begin{proof}
            Prova por indução forte sobre o tamanho da sucessão de dedução.
            Assim, suponhamos que a contratradução seja correta para qualquer sucessão de dedução de tamanho $n < k$.
            Demonstraremos, analisando-se os casos, que a correção da contratradução vale para sucessões de dedução de tamanho $n = k$.

            \begin{case}
                \textsc{Caso 1.}
                Seja a linha derradeira da sucessão de dedução que prova $\Gamma\entails\alpha$ gerada pela invocação de alguma premissa.
                Sabe-se que $\alpha\in\Gamma$ e, portanto, que $\alpha^\varepsilon\in\Gamma^\varepsilon$.
                Deste modo, pode-se demonstrar que $\Gamma^\varepsilon\entails\alpha^\varepsilon$ trivialmente pela invocação da premissa $\alpha^\varepsilon$.
            \end{case}

            \begin{case}
                \textsc{Caso 2.}
                Seja a linha derradeira da sucessão de dedução que prova $\Gamma\entails\alpha$ gerada pela invocação de algum axioma.
                Sabe-se que existe algum esquema $\mathbf{A}\in\mathcal{A}$ que gera $\alpha$. 
                Deste modo, analisaremos os casos e demonstraremos que se pode derivar $\Gamma^\varepsilon\entails\alpha^\varepsilon$ para cada esquema $\mathbf{A}\in\mathcal{A}$.
            \end{case}

                \begin{subcase}
                    \textsc{Caso 2.1.}
                    Seja a linha derradeira da sucessão de dedução que prova $\Gamma\entails\alpha$ gerada pelo esquema $\hyperref[MA1]{\mathbf{A_1}}$.
                    Sabe-se que $\alpha=\varphi\to\psi\to\varphi$ e que $\alpha^\varepsilon=\varphi^\varepsilon\to\psi^\varepsilon\to\varphi^\varepsilon$.
                    Deste modo, pode-se provar $\Gamma^\varepsilon\entails\alpha^\varepsilon$ trivialmente pela invocação de $\hyperref[IA1]{\mathbf{A_1}}$.
                \end{subcase}

                \begin{subcase}
                    \textsc{Caso 2.2.}
                    Seja a linha derradeira da sucessão de dedução que prova $\Gamma\entails\alpha$ gerada pelo esquema $\hyperref[MA2]{\mathbf{A_2}}$.
                    Sabe-se que $\alpha=(\varphi\to\psi\to\chi)\to(\varphi\to\psi)\to\varphi\to\chi$ e que $\alpha^\varepsilon=(\varphi^\varepsilon\to\psi^\varepsilon\to\chi^\varepsilon)\to(\varphi^\varepsilon\to\psi^\varepsilon)\to\varphi^\varepsilon\to\chi^\varepsilon$.
                    Deste modo, pode-se provar $\Gamma^\varepsilon\entails\alpha^\varepsilon$ trivialmente pela invocação de $\hyperref[IA2]{\mathbf{A_2}}$.
                \end{subcase}

                \begin{subcase}
                    \textsc{Caso 2.3.}
                    Seja a linha derradeira da sucessão de dedução que prova $\Gamma\entails\alpha$ gerada pelo esquema $\hyperref[MA3]{\mathbf{A_3}}$.
                    Sabe-se que $\alpha=\varphi\to\psi\to\varphi\wedge\psi$ e que $\alpha^\varepsilon=\varphi^\varepsilon\to\psi^\varepsilon\to\varphi^\varepsilon\wedge\psi^\varepsilon$.
                    Deste modo, pode-se provar $\Gamma^\varepsilon\entails\alpha^\varepsilon$ trivialmente pela invocação de $\hyperref[IA3]{\mathbf{A_3}}$.
                \end{subcase}

                \begin{subcase}
                    \textsc{Caso 2.4.}
                    Seja a linha derradeira da sucessão de dedução que prova $\Gamma\entails\alpha$ gerada pelo esquema $\hyperref[MA4]{\mathbf{A_4}}$.
                    Sabe-se que $\alpha=\varphi\wedge\psi\to\varphi$ e que $\alpha^\varepsilon=\varphi^\varepsilon\wedge\psi^\varepsilon\to\varphi^\varepsilon$.
                    Deste modo, pode-se provar $\Gamma^\varepsilon\entails\alpha^\varepsilon$ trivialmente pela invocação de $\hyperref[IA4]{\mathbf{A_4}}$.
                \end{subcase}

                \begin{subcase}
                    \textsc{Caso 2.5.}
                    Seja a linha derradeira da sucessão de dedução que prova $\Gamma\entails\alpha$ gerada pelo esquema $\hyperref[MA5]{\mathbf{A_5}}$.
                    Sabe-se que $\alpha=\varphi\wedge\psi\to\psi$ e que $\alpha^\varepsilon=\varphi^\varepsilon\wedge\psi^\varepsilon\to\varphi^\psi$.
                    Deste modo, pode-se provar $\Gamma^\varepsilon\entails\alpha^\varepsilon$ trivialmente pela invocação de $\hyperref[IA5]{\mathbf{A_5}}$.
                \end{subcase}

                \begin{subcase}
                    \textsc{Caso 2.6.}
                    Seja a linha derradeira da sucessão de dedução que prova $\Gamma\entails\alpha$ gerada pelo esquema $\hyperref[MA6]{\mathbf{A_6}}$.
                    Sabe-se que $\alpha=\varphi\to\varphi\vee\psi$ e que $\alpha^\varepsilon=\varphi^\varepsilon\to\neg_\varepsilon\neg_\varepsilon(\varphi^\varepsilon\vee\psi^\varepsilon)$.
                \end{subcase}

                \begin{subcase}
                    \textsc{Caso 2.7.}
                    Seja a linha derradeira da sucessão de dedução que prova $\Gamma\entails\alpha$ gerada pelo esquema $\hyperref[MA7]{\mathbf{A_7}}$.
                    Sabe-se que $\alpha=\psi\to\varphi\vee\psi$ e que $\alpha^\varepsilon=\psi^\varepsilon\to\neg_\varepsilon\neg_\varepsilon(\varphi^\varepsilon\vee\psi^\varepsilon)$.
                \end{subcase}

                \begin{subcase}
                    \textsc{Caso 2.8.}
                    Seja a linha derradeira da sucessão de dedução que prova $\Gamma\entails\alpha$ gerada pelo esquema $\hyperref[MA8]{\mathbf{A_8}}$.
                    Sabe-se que $\alpha=(\varphi\to\chi)\to(\psi\to\chi)\to\varphi\vee\psi\to\chi$ e que $\alpha^\varepsilon=(\varphi^\varepsilon\to\chi^\varepsilon)\to(\psi^\varepsilon\to\chi^\varepsilon)\to\neg_\varepsilon\neg_\varepsilon(\varphi^\varepsilon\vee\psi^\varepsilon)\to\chi^\varepsilon$.
                \end{subcase}

                \begin{subcase}
                    \textsc{Caso 2.9.}
                    Seja a linha derradeira da sucessão de dedução que prova $\Gamma\entails\alpha$ gerada pelo esquema $\hyperref[MANEG]{\mathbf{A_\neg}}$.
                    Sabe-se que $\alpha=\neg\neg\varphi\to\varphi$ e que $\alpha^\varepsilon=\neg\neg\varphi^\varepsilon\to\varphi^\varepsilon$.
                \end{subcase}
        \end{proof}
    \end{theorem}

    \begin{theorem}
        Se $\Gamma^\medsquare\entails\alpha^\medsquare$ então $\Gamma\entails\alpha$.
    \end{theorem}

    \begin{theorem}
        Se $\nec\Gamma^\circ\entails\alpha^\circ$ então $\Gamma\entails\alpha$.
    \end{theorem}
