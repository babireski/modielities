\section{Traduções}
    A primeira tradução do sistema intuicionista ao sistema modal foi proposta por~\cite{Goedel} motivado pela possibilidade de leitura da necessidade como uma modalidade de construtividade. Ou seja, por meio dessa tradução, a sentença $\nec \varphi$ poderia ser lida como \textit{$\varphi$ pode ser provada construtivamente} \citep{Troelstra}. Gödel conjeiturou a corretude fraca dessa tradução, que foi posteriormente provada por~\cite{McKinsey} em conjunto com sua completude fraca.

    \begin{definition}[$\bullet^\circ$] Define-se a tradução $\bullet^\circ$ indutivamente da seguinte maneira:
        \begin{align*}
            p^\circ                     & \coloneqq p                                       \\
            \bot^\circ                  & \coloneqq \bot                                    \\
            {(\varphi \wedge \psi)}^\circ & \coloneqq \varphi^\circ \wedge \psi^\circ         \\
            {(\varphi \vee \psi)}^\circ   & \coloneqq \nec \varphi^\circ \vee \nec \psi^\circ \\
            {(\varphi \to \psi)}^\circ    & \coloneqq \nec \varphi^\circ \to \psi^\circ
            \tag*{\qed} 
        \end{align*}
    \end{definition}
    
    \begin{definition}[$\bullet^\medsquare$] Define-se a tradução $\bullet^\medsquare$ indutivamente da seguinte maneira:
        \begin{align*}
            p^\medsquare                     & \coloneqq \nec p                                        \\
            \bot^\medsquare                  & \coloneqq \bot                                          \\
            {(\varphi \wedge \psi)}^\medsquare & \coloneqq \varphi^\medsquare \wedge \psi^\medsquare     \\
            {(\varphi \vee \psi)}^\medsquare   & \coloneqq \varphi^\medsquare \vee \psi^\medsquare       \\
            {(\varphi \to \psi)}^\medsquare    & \coloneqq \nec (\varphi^\medsquare \to \psi^\medsquare)
            \tag*{\qed} 
        \end{align*}
    \end{definition}
    
    Faz-se interessante pontuar que as traduções $\bullet^\circ$ e $\bullet^\medsquare$ correspondem, respectivamente, às traduções $\bullet^\circ$ e $\bullet^*$ do sistema intuicionista ao sistema linear providas por~\cite{Girard}, sendo as primeiras correspondentes a uma ordem de avaliação por nome (\textit{call-by-name}) e as segundas a uma ordem de avaliação por valor (\textit{call-by-value}). 
    Ademais, as duas traduções providas são equivalentes, conforme demonstrado pelo teorema $\mathbf{T_2}$.

