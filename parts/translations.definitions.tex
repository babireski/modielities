\section{Definições}
    A primeira tradução do sistema intuicionista ao sistema modal foi proposta por~\cite{Goedel} motivado pela possibilidade de leitura da necessidade como uma modalidade de construtividade. Ou seja, por meio dessa tradução, a sentença $\nec \varphi$ poderia ser lida como \textit{$\varphi$ pode ser provada construtivamente} \citep{Troelstra}. Gödel conjeiturou a corretude fraca dessa tradução, que foi posteriormente provada por~\cite{McKinsey} em conjunto com sua completude fraca.

    \begin{definition}[$\bullet^\circ$] Define-se a tradução $\bullet^\circ$ indutivamente da seguinte maneira:
        \begin{align*}
            p^\circ                     & \coloneqq p                                       \\
            \bot^\circ                  & \coloneqq \bot                                    \\
            {(\varphi \wedge \psi)}^\circ & \coloneqq \varphi^\circ \wedge \psi^\circ         \\
            {(\varphi \vee \psi)}^\circ   & \coloneqq \nec \varphi^\circ \vee \nec \psi^\circ \\
            {(\varphi \to \psi)}^\circ    & \coloneqq \nec \varphi^\circ \to \psi^\circ
            \tag*{\qed} 
        \end{align*}
    \end{definition}
    
    \begin{definition}[$\bullet^\nec$] Define-se a tradução $\bullet^\nec$ indutivamente da seguinte maneira:
        \begin{align*}
            p^\nec                     & \coloneqq \nec p                                        \\
            \bot^\nec                  & \coloneqq \bot                                          \\
            {(\varphi \wedge \psi)}^\nec & \coloneqq \varphi^\nec \wedge \psi^\nec     \\
            {(\varphi \vee \psi)}^\nec   & \coloneqq \varphi^\nec \vee \psi^\nec       \\
            {(\varphi \to \psi)}^\nec    & \coloneqq \nec (\varphi^\nec \to \psi^\nec)
            \tag*{\qed} 
        \end{align*}
    \end{definition}
    
    Faz-se interessante pontuar que as traduções $\bullet^\circ$ e $\bullet^\nec$ correspondem, respectivamente, às traduções $\bullet^\circ$ e $\bullet^*$ do sistema intuicionista ao sistema linear providas por~\cite{Girard}, sendo as primeiras correspondentes a uma ordem de avaliação por nome (\textit{call-by-name}) e as segundas a uma ordem de avaliação por valor (\textit{call-by-value}). 
    Ademais, as duas traduções providas são equivalentes, conforme demonstrado pelo teorema $\mathbf{T_2}$.

    \begin{theorem}
        $\forall\alpha\in\mathcal{L}_\mathbf{I}\point\proves\nec\alpha^\circ\leftrightarrow\alpha^\nec$.

        \begin{proof}
            Prova por indução forte sobre a profundidade de $\alpha$.
            Assim, suponhamos que as traduções equivalham para qualquer $\alpha$ de profundidade $n<k$.
            Demonstraremos, analisando-se os casos, que as traduções equivalem para qualquer $\alpha$ de profundidade $n=k$.
    
            \begin{case}
                \textsc{Caso 1.}
                Se a sentença $\alpha$ for uma proposição $a$, sabe-se que $\nec a^\circ=\nec a$ e que $a^\nec =\nec a$ pelas definições das traduções.
                Deste modo, tanto a ida quanto a volta possuem a forma $\nec a\to\nec a$ e podem ser provadas pelo lema \hyperref[identity]{$\mathbf{L_\getrefnumber{identity}}$}.
            \end{case}

            \begin{case}
                \textsc{Caso 2.}
                Se a sentença $\alpha$ for a constante $\bot$, sabe-se que $\nec\bot^\circ=\nec\bot$ e que $\bot^\nec=\bot$ pelas definições das traduções.
                Deste modo, a ida $\nec\bot\to\bot$ constitui um axioma gerado pelo esquema $\mathbf{B_2}$ --- sendo assim provada trivialmente --- e a volta $\bot\to\nec\bot$ pode ser provada pelo lema \hyperref[explosion]{$\mathbf{L_2}$}.
            \end{case}
    
            \begin{case}
                \textsc{Caso 3.}
                Se a sentença $\alpha$ for o resultado da conjunção de duas outras sentenças $\varphi$ e $\psi$, sabe-se que $\nec{(\varphi\wedge\psi)}^\circ=\nec(\varphi^\circ\wedge\psi^\circ)$ e que ${(\varphi\wedge\psi)}^\nec=\varphi^\nec\wedge\psi^\nec$ pelas definições das traduções.
            \end{case}

                \begin{subcase}
                    \textsc{Caso 3.1.}
                    Assim, pode-se provar a ida $\nec(\varphi^\circ\wedge\psi^\circ)\to\varphi^\nec\wedge\psi^\nec$ pela seguinte sucessão de dedução:

                    \begin{fitch}
                        \fa\set{\nec(\varphi^\circ\wedge\psi^\circ)}\proves\nec(\varphi^\circ\wedge\psi^\circ)&$\mathbf{P}$\\
                        \fa\set{\nec(\varphi^\circ\wedge\psi^\circ)}\proves\nec(\varphi^\circ\wedge\psi^\circ)\to\nec\varphi^\circ\wedge\nec\psi^\circ&$\mathbf{L_5}$\\
                        \fa\set{\nec(\varphi^\circ\wedge\psi^\circ)}\proves\nec\varphi^\circ\wedge\nec\psi^\circ&$\mathbf{R_1}\;\sequence{1,2}$\\
                        \fa\set{\nec(\varphi^\circ\wedge\psi^\circ)}\proves\nec\varphi^\circ\wedge\nec\psi^\circ\to\nec\varphi^\circ&$\mathbf{A_4}$\\
                        \fa\set{\nec(\varphi^\circ\wedge\psi^\circ)}\proves\nec\varphi^\circ&$\mathbf{R_1}\;\sequence{3,4}$\\
                        \fa\set{\nec(\varphi^\circ\wedge\psi^\circ)}\proves\nec\varphi^\circ\to\varphi^\nec&$\mathbf{H}$\\
                        \fa\set{\nec(\varphi^\circ\wedge\psi^\circ)}\proves\varphi^\nec&$\mathbf{R_1}\;\sequence{5,6}$\\
                        \fa\set{\nec(\varphi^\circ\wedge\psi^\circ)}\proves\nec\varphi^\circ\wedge\nec\psi^\circ\to\nec\psi^\circ&$\mathbf{A_4}$\\
                        \fa\set{\nec(\varphi^\circ\wedge\psi^\circ)}\proves\nec\psi^\circ&$\mathbf{R_1}\;\sequence{3,8}$\\
                        \fa\set{\nec(\varphi^\circ\wedge\psi^\circ)}\proves\nec\psi^\circ\to\psi^\nec&$\mathbf{H}$\\
                        \fa\set{\nec(\varphi^\circ\wedge\psi^\circ)}\proves\psi^\nec&$\mathbf{R_1}\;\sequence{9,10}$\\
                        \fa\set{\nec(\varphi^\circ\wedge\psi^\circ)}\proves\varphi^\nec\to\psi^\nec\to\varphi^\nec\wedge\psi^\nec&$\mathbf{A_3}$\\
                        \fa\set{\nec(\varphi^\circ\wedge\psi^\circ)}\proves\psi^\nec\to\varphi^\nec\wedge\psi^\nec&$\mathbf{R_1}\;\sequence{7,12}$\\
                        \fa\set{\nec(\varphi^\circ\wedge\psi^\circ)}\proves\varphi^\nec\wedge\psi^\nec&$\mathbf{R_1}\;\sequence{9,13}$\\
                        \fa\proves\nec(\varphi^\circ\wedge\psi^\circ)\to\varphi^\nec\wedge\psi^\nec&$\mathbf{T_1}\;\sequence{14}$\\
                    \end{fitch}
                \end{subcase}

                \begin{subcase}
                    \textsc{Caso 3.2.}
                    Outrossim, pode-se provar a volta $\varphi^\nec\wedge\psi^\nec\to\nec(\varphi^\circ\wedge\psi^\circ)$ pela seguinte sucessão de dedução:

                    \begin{fitch}
                        \fa\set{\varphi^\nec\wedge\psi^\nec}\proves\varphi^\nec\wedge\psi^\nec&$\mathbf{P}$\\
                        \fa\set{\varphi^\nec\wedge\psi^\nec}\proves\varphi^\nec\wedge\psi^\nec\to\varphi^\nec&$\mathbf{P}$\\
                        \fa\set{\varphi^\nec\wedge\psi^\nec}\proves\varphi^\nec&$\mathbf{P}$\\
                        \fa\set{\varphi^\nec\wedge\psi^\nec}\proves\varphi^\nec\to\nec\varphi^\circ&$\mathbf{P}$\\
                        \fa\set{\varphi^\nec\wedge\psi^\nec}\proves\nec\varphi^\circ&$\mathbf{P}$\\
                        \fa\set{\varphi^\nec\wedge\psi^\nec}\proves\varphi^\nec\wedge\psi^\nec\to\psi^\nec&$\mathbf{P}$\\
                        \fa\set{\varphi^\nec\wedge\psi^\nec}\proves\psi^\nec&$\mathbf{P}$\\
                        \fa\set{\varphi^\nec\wedge\psi^\nec}\proves\psi^\nec\to\nec\psi^\circ&$\mathbf{P}$\\
                        \fa\set{\varphi^\nec\wedge\psi^\nec}\proves\nec\psi^\circ&$\mathbf{P}$\\
                        \fa\set{\varphi^\nec\wedge\psi^\nec}\proves\nec\varphi^\circ\to\nec\psi^\circ\to\nec\varphi^\circ\wedge\nec\psi^\circ&$\mathbf{P}$\\
                        \fa\set{\varphi^\nec\wedge\psi^\nec}\proves\nec\psi^\circ\to\nec\varphi^\circ\wedge\nec\psi^\circ&$\mathbf{P}$\\
                        \fa\set{\varphi^\nec\wedge\psi^\nec}\proves\nec\varphi^\circ\wedge\nec\psi^\circ&$\mathbf{P}$\\
                        \fa\set{\varphi^\nec\wedge\psi^\nec}\proves\nec\varphi^\circ\wedge\nec\psi^\circ\to\nec(\varphi^\circ\wedge\psi^\circ)&$\mathbf{P}$\\
                        \fa\set{\varphi^\nec\wedge\psi^\nec}\proves\nec(\varphi^\circ\wedge\psi^\circ)&$\mathbf{P}$\\
                        \fa\proves\varphi^\nec\wedge\psi^\nec\to\nec(\varphi^\circ\wedge\psi^\circ)&$\mathbf{P}$\\
                    \end{fitch}
                \end{subcase}

            \begin{case}
                \textsc{Caso 4.}
                Se a sentença $\alpha$ for o resultado da disjunção de duas outras sentenças $\varphi$ e $\psi$, sabe-se que $\nec{(\varphi\vee\psi)}^\circ=\nec(\nec\varphi^\circ\vee\nec\psi^\circ)$ e que ${(\varphi\vee\psi)}^\nec=\varphi^\nec\vee\psi^\nec$ pelas definições das traduções.
                Para a ida, basta aplicar o axioma $\mathbf{B_2}$.
                Para a volta, basta aplicar a regra da necessitação.

                \begin{fitch}
                    \fa\\
                \end{fitch}
            \end{case}

            \begin{case}
                \textsc{Caso 5.}
                Se a sentença $\alpha$ for o resultado da implicação de uma sentença $\varphi$ a uma sentença $\psi$, sabe-se que $\nec{(\varphi\to\psi)}^\circ=\nec(\nec\varphi^\circ\to\psi^\circ)$ e que ${(\varphi\to\psi)}^\nec=\nec(\varphi^\nec\to\psi^\nec)$ pelas definições das traduções.
                Para a ida, basta aplicar o axioma $\mathbf{B_1}$.
                Para a volta, basta aplicar o lema $\mathbf{L_4}$ seguido pela necessitação.
                \qedhere
            \end{case}
        \end{proof}
    \end{theorem}
