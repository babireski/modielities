\section{Traduções}
    A primeira tradução do sistema intuicionista ao sistema modal foi proposta por~\cite{Goedel} motivado pela possibilidade de leitura da necessidade como uma modalidade de construtividade. Ou seja, por meio dessa tradução, a sentença $\nec \varphi$ poderia ser lida como \textit{$\varphi$ pode ser provada construtivamente} \citep{Troelstra}. Gödel conjeiturou a corretude fraca dessa tradução, que foi posteriormente provada por~\cite{McKinsey} em conjunto com sua completude fraca.

    \begin{definition}[$\bullet^\circ$] Define-se a tradução $\bullet^\circ$ indutivamente da seguinte maneira:
        \begin{align*}
            p^\circ                     & \coloneqq p                                       \\
            \bot^\circ                  & \coloneqq \bot                                    \\
            {(\varphi \wedge \psi)}^\circ & \coloneqq \varphi^\circ \wedge \psi^\circ         \\
            {(\varphi \vee \psi)}^\circ   & \coloneqq \nec \varphi^\circ \vee \nec \psi^\circ \\
            {(\varphi \to \psi)}^\circ    & \coloneqq \nec \varphi^\circ \to \psi^\circ
            \tag*{\qed} 
        \end{align*}
    \end{definition}
    
    \begin{definition}[$\bullet^\nec$] Define-se a tradução $\bullet^\nec$ indutivamente da seguinte maneira:
        \begin{align*}
            p^\nec                     & \coloneqq \nec p                                        \\
            \bot^\nec                  & \coloneqq \bot                                          \\
            {(\varphi \wedge \psi)}^\nec & \coloneqq \varphi^\nec \wedge \psi^\nec     \\
            {(\varphi \vee \psi)}^\nec   & \coloneqq \varphi^\nec \vee \psi^\nec       \\
            {(\varphi \to \psi)}^\nec    & \coloneqq \nec (\varphi^\nec \to \psi^\nec)
            \tag*{\qed} 
        \end{align*}
    \end{definition}
    
    Faz-se interessante pontuar que as traduções $\bullet^\circ$ e $\bullet^\nec$ correspondem, respectivamente, às traduções $\bullet^\circ$ e $\bullet^*$ do sistema intuicionista ao sistema linear providas por~\cite{Girard}, sendo as primeiras correspondentes a uma ordem de avaliação por nome (\textit{call-by-name}) e as segundas a uma ordem de avaliação por valor (\textit{call-by-value}). 
    Ademais, as duas traduções providas são equivalentes, conforme demonstrado pelo teorema $\mathbf{T_2}$.

    \begin{theorem}
        $\forall\alpha\in\mathcal{L}_\mathbf{I}\point\nec\alpha^\circ\leftrightarrow\alpha^\nec$.

        \begin{proof}
            Prova por indução forte sobre a profundidade de $\alpha$.
            Assim, suponhamos que as traduções equivalham para qualquer $\alpha$ de profundidade $n\leq k$.
            Demonstraremos, analisando-se os casos, que as traduções equivalem para qualquer $\alpha$ de profundidade $n=k+1$.
    
            \begin{case}
                \textbf{Caso 1} (Base)\textbf{.}
                    Para $|\alpha| = 0$, existem dois casos a serem considerados.
    
                    \begin{casee}
                        \textbf{Caso 1.1} ($\alpha = a$)\textbf{.}
                        $a^\circ = a$ e $a^\nec = \nec a$, assim $\nec a^\circ = a^\nec$ e, portanto, $\nec a^\circ \leftrightarrow a^\nec$.
                    \end{casee}

                    \begin{casee}
                        \textbf{Caso 2.1} ($\alpha = \bot$)\textbf{.}
                        $\bot^\circ = \bot$ e $\bot^\nec = \bot$. A ida $\nec\bot\to\bot$ consiste em um axioma, sendo, portando provada trivialmente pela sucessão de dedução $\sequence{\nec\bot\to\bot}$.
                        A volta $\bot\to\nec\bot$ equivale a provar, por meio do teorema da dedução, que $\set{\bot}\vdash_\mathbf{M}\nec\bot$, o que pode ser provado trivialmente pela sucessão de dedução $\sequence{\bot, \nec\bot}$, que consiste na invocação da premissa e aplicação da regra da necessitação, nessa ordem.
                    \end{casee}
            \end{case}
    
            \begin{case}
                \textbf{Caso 2} (Passo)\textbf{.} No passo, deve-se demonstrar que, caso $\nec\alpha^\circ\leftrightarrow\alpha^\nec$ para $|\alpha| = n$, 
                então $\nec\alpha^\circ\leftrightarrow\alpha^\nec$ para $|\alpha| = n + 1$, onde $n \in \mathbb{N}$. Assim, seja $\nec\alpha^\circ\leftrightarrow\alpha^\nec$ uma proposição verdadeira para $|\alpha| = k$, onde $k \in \mathbb{N}$. Existem os seguintes casos a serem considerados para $|\alpha| = k + 1$.
            \end{case}
    
                \begin{casee}
                    \textbf{Caso 2.1.}
                    Neste caso, consideraremos que $\alpha = \varphi\wedge\psi$.
                    Assim sendo, sabe-se que $\nec{(\varphi\wedge\psi)}^\circ=\nec(\varphi^\circ\wedge\psi^\circ)$ e ${(\varphi\wedge\psi)}^\nec=\varphi^\nec\wedge\psi^\nec$.
                    Pelo lemas $\mathbf{L_2}$ e $\mathbf{L_3}$ temos que $\nec(\varphi^\circ\wedge\psi^\circ)\leftrightarrow\nec\varphi^\circ\wedge\nec\psi^\circ$ e pela premissa da indução temos que $\nec\varphi^\circ\leftrightarrow\varphi^\nec$ e $\nec\psi^\circ\leftrightarrow\psi^\nec$.
                    Deste modo, $\nec{(\varphi\wedge\psi)}^\circ\leftrightarrow{(\varphi\wedge\psi)}^\nec$.
                \end{casee}
    
                \begin{casee}
                    \textbf{Caso 2.2.}
                    Neste caso, consideraremos que $\alpha = \varphi\vee\psi$.
                    Assim sendo, sabe-se que $\nec{(\varphi\vee\psi)}^\circ=\nec(\nec\varphi^\circ\vee\nec\psi^\circ)$ e ${(\varphi\vee\psi)}^\nec=\varphi^\nec\vee\psi^\nec$.
                    Para a ida, basta aplicar o axioma $\mathbf{B_2}$.
                    Para a volta, basta aplicar a regra da necessitação.
                \end{casee}
    
                \begin{casee}
                    \textbf{Caso 2.3.}
                    Neste caso, consideraremos que $\alpha=\varphi\to\psi$.
                    Assim sendo, sabe-se que $\nec{(\varphi\to\psi)}^\circ=\nec(\nec\varphi^\circ\to\psi^\circ)$ e ${(\varphi\to\psi)}^\nec=\nec(\varphi^\nec\to\psi^\nec)$.
                    Para a ida, basta aplicar o axioma $\mathbf{B_1}$.
                    Para a volta, basta aplicar o lema $\mathbf{L_4}$ seguido pela necessitação.
                    \qedhere
                \end{casee}
        \end{proof}
    \end{theorem}
