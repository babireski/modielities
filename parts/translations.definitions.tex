\section{Definições}
    A primeira tradução do sistema intuicionista ao sistema modal foi proposta por~\cite{Goedel} motivado pela possibilidade de leitura da necessidade como uma modalidade de construtividade. Ou seja, por meio dessa tradução, a sentença $\nec \varphi$ poderia ser lida como \textit{$\varphi$ pode ser provada construtivamente} \citep{Troelstra}. Gödel conjeiturou a corretude fraca dessa tradução, que foi posteriormente provada por~\cite{McKinsey} em conjunto com sua completude fraca.

    \begin{definition}[$\bullet^\circ$] Define-se a tradução $\bullet^\circ$ indutivamente da seguinte maneira:
        \begin{align*}
            p^\circ                     & \coloneqq p                                       \\
            \bot^\circ                  & \coloneqq \bot                                    \\
            {(\varphi \wedge \psi)}^\circ & \coloneqq \varphi^\circ \wedge \psi^\circ         \\
            {(\varphi \vee \psi)}^\circ   & \coloneqq \nec \varphi^\circ \vee \nec \psi^\circ \\
            {(\varphi \to \psi)}^\circ    & \coloneqq \nec \varphi^\circ \to \psi^\circ
            \tag*{\qed} 
        \end{align*}
    \end{definition}
    
    \begin{definition}[$\bullet^\smallsquare$] Define-se a tradução $\bullet^\smallsquare$ indutivamente da seguinte maneira:
        \begin{align*}
            p^\smallsquare                     & \coloneqq \nec p                                        \\
            \bot^\smallsquare                  & \coloneqq \bot                                          \\
            {(\varphi \wedge \psi)}^\smallsquare & \coloneqq \varphi^\smallsquare \wedge \psi^\smallsquare     \\
            {(\varphi \vee \psi)}^\smallsquare   & \coloneqq \varphi^\smallsquare \vee \psi^\smallsquare       \\
            {(\varphi \to \psi)}^\smallsquare    & \coloneqq \nec (\varphi^\smallsquare \to \psi^\smallsquare)
            \tag*{\qed} 
        \end{align*}
    \end{definition}
    
    Faz-se interessante pontuar que as traduções $\bullet^\circ$ e $\bullet^\smallsquare$ correspondem, respectivamente, às traduções $\bullet^\circ$ e $\bullet^*$ do sistema intuicionista ao sistema linear providas por~\cite{Girard}, sendo as primeiras correspondentes a uma ordem de avaliação por nome (\textit{call-by-name}) e as segundas a uma ordem de avaliação por valor (\textit{call-by-value}). 
    Ademais, as duas traduções providas são equivalentes, conforme demonstrado pelo teorema $\mathbf{T_2}$.

    \begin{theorem}
        $\entails\nec\alpha^\circ\leftrightarrow\alpha^\smallsquare$.

        \begin{proof}
            Prova por indução forte sobre a profundidade de $\alpha$.
            Assim, suponhamos que as traduções equivalham para qualquer $\alpha$ de profundidade $n<k$.
            Demonstraremos analisando-se os casos e valendo-se da suposição acima --- doravante chamada $\mathbf{H}$ --- o passo de indução, ou seja, que as traduções equivalem para qualquer $\alpha$ de profundidade $n=k$.
    
            \begin{case}
                \textsc{Caso 1.}
                Se a sentença $\alpha$ for uma proposição $a$, sabe-se que $\nec a^\circ=\nec a$ e que $a^\smallsquare=\nec a$ pelas definições das traduções.
                Deste modo, tanto a ida quanto a volta possuem a forma $\nec a\to\nec a$ e podem ser provadas pelo lema \hyperref[identity]{$\mathbf{L_\getrefnumber{identity}}$}.
                Ambas as implicações posteriormente podem ser unidas em uma bi-implicação por meio do esquema \hyperref[MA3]{$\mathbf{A_3}$}.
            \end{case}

            \begin{case}
                \textsc{Caso 2.}
                Se a sentença $\alpha$ for a constante $\bot$, sabe-se que $\nec\bot^\circ=\nec\bot$ e que $\bot^\smallsquare=\bot$ pelas definições das traduções.
                Deste modo, a ida $\nec\bot\to\bot$ constitui um axioma gerado pelo esquema \hyperref[MB2]{$\mathbf{B_2}$} --- sendo assim provada trivialmente --- e a volta $\bot\to\nec\bot$ pode ser provada pelo lema \hyperref[explosion]{$\mathbf{L_2}$}.
                Ambas as implicações posteriormente podem ser unidas em uma bi-implicação por meio do esquema \hyperref[MA3]{$\mathbf{A_3}$}.
            \end{case}
    
            \begin{case}
                \textsc{Caso 3.}
                Se a sentença $\alpha$ for o resultado da conjunção de duas outras sentenças $\varphi$ e $\psi$, sabe-se que $\nec{(\varphi\wedge\psi)}^\circ=\nec(\varphi^\circ\wedge\psi^\circ)$ e que ${(\varphi\wedge\psi)}^\smallsquare=\varphi^\smallsquare\wedge\psi^\smallsquare$ pelas definições das traduções.
                Separaremos a prova em dois casos: um para a ida $\nec(\varphi^\circ\wedge\psi^\circ)\to\varphi^\smallsquare\wedge\psi^\smallsquare$ e outro para a volta $\varphi^\smallsquare\wedge\psi^\smallsquare\to\nec(\varphi^\circ\wedge\psi^\circ)$. Ambas as implicações posteriormente podem ser unidas em uma bi-implicação por meio do esquema da introdução da conjunção \hyperref[MA3]{$\mathbf{A_3}$}.
            \end{case}

                \begin{subcase}
                    \textsc{Caso 3.1.}
                    A partir de $\mathbf{H}$, temos que $\mathbf{H_1}={\entails\nec\varphi^\circ\to\varphi^\smallsquare}$ e que $\mathbf{H_2}={\entails\nec\psi^\circ\to\psi^\smallsquare}$ por meio dos esquemas da eliminação da conjunção e da aplicação da regra da separação.
                    Valendo-se do listado acima em conjunto com alguns lemas, pode-se provar que $\entails\nec(\varphi^\circ\wedge\psi^\circ)\to\varphi^\smallsquare\wedge\psi^\smallsquare$ pela seguinte sucessão de dedução:

                    \footnotesize
                    \begin{fitch}
                        \fb\set{\nec(\varphi^\circ\wedge\psi^\circ)}\proves\nec(\varphi^\circ\wedge\psi^\circ)&$\mathbf{P_1}$\\
                        \fa\set{\nec(\varphi^\circ\wedge\psi^\circ)}\proves\nec(\varphi^\circ\wedge\psi^\circ)\to\nec\varphi^\circ\wedge\nec\psi^\circ&$\mathbf{L_5}$\\
                        \fa\set{\nec(\varphi^\circ\wedge\psi^\circ)}\proves\nec\varphi^\circ\wedge\nec\psi^\circ&$\mathbf{R_1}\;\set{1,2}$\\
                        \fa\set{\nec(\varphi^\circ\wedge\psi^\circ)}\proves\nec\varphi^\circ\wedge\nec\psi^\circ\to\nec\varphi^\circ&$\mathbf{A_4}$\\
                        \fa\set{\nec(\varphi^\circ\wedge\psi^\circ)}\proves\nec\varphi^\circ&$\mathbf{R_1}\;\set{3,4}$\\
                        \fa\set{\nec(\varphi^\circ\wedge\psi^\circ)}\proves\nec\varphi^\circ\to\varphi^\smallsquare&$\mathbf{H_1}$\\
                        \fa\set{\nec(\varphi^\circ\wedge\psi^\circ)}\proves\varphi^\smallsquare&$\mathbf{R_1}\;\set{5,6}$\\
                        \fa\set{\nec(\varphi^\circ\wedge\psi^\circ)}\proves\nec\varphi^\circ\wedge\nec\psi^\circ\to\nec\psi^\circ&$\mathbf{A_4}$\\
                        \fa\set{\nec(\varphi^\circ\wedge\psi^\circ)}\proves\nec\psi^\circ&$\mathbf{R_1}\;\set{3,8}$\\
                        \fa\set{\nec(\varphi^\circ\wedge\psi^\circ)}\proves\nec\psi^\circ\to\psi^\smallsquare&$\mathbf{H_2}$\\
                        \fa\set{\nec(\varphi^\circ\wedge\psi^\circ)}\proves\psi^\smallsquare&$\mathbf{R_1}\;\set{9,10}$\\
                        \fa\set{\nec(\varphi^\circ\wedge\psi^\circ)}\proves\varphi^\smallsquare\to\psi^\smallsquare\to\varphi^\smallsquare\wedge\psi^\smallsquare&$\mathbf{A_3}$\\
                        \fa\set{\nec(\varphi^\circ\wedge\psi^\circ)}\proves\psi^\smallsquare\to\varphi^\smallsquare\wedge\psi^\smallsquare&$\mathbf{R_1}\;\set{7,12}$\\
                        \fa\set{\nec(\varphi^\circ\wedge\psi^\circ)}\proves\varphi^\smallsquare\wedge\psi^\smallsquare&$\mathbf{R_1}\;\set{9,13}$\\
                        \fa\proves\nec(\varphi^\circ\wedge\psi^\circ)\to\varphi^\smallsquare\wedge\psi^\smallsquare&$\mathbf{T_1}\;\set{14}$\\
                    \end{fitch}
                \end{subcase}

                \begin{subcase}
                    \textsc{Caso 3.2.}
                    A partir de $\mathbf{H}$, temos que $\mathbf{H_1}={\entails\varphi^\smallsquare\to\nec\varphi^\circ}$ e que $\mathbf{H_2}={\entails\psi^\smallsquare\to\nec\psi^\circ}$ por meio dos esquemas da eliminação da conjunção e da aplicação regra da separação.
                    Valendo-se do listado acima em conjunto com alguns lemas, pode-se provar que $\entails\varphi^\smallsquare\wedge\psi^\smallsquare\to\nec(\varphi^\circ\wedge\psi^\circ)$ pela seguinte sucessão de dedução:

                    \footnotesize
                    \begin{fitch}
                        \fb\set{\varphi^\smallsquare\wedge\psi^\smallsquare}\proves\varphi^\smallsquare\wedge\psi^\smallsquare&$\mathbf{P}$\\
                        \fa\set{\varphi^\smallsquare\wedge\psi^\smallsquare}\proves\varphi^\smallsquare\wedge\psi^\smallsquare\to\varphi^\smallsquare&$\mathbf{P}$\\
                        \fa\set{\varphi^\smallsquare\wedge\psi^\smallsquare}\proves\varphi^\smallsquare&$\mathbf{P}$\\
                        \fa\set{\varphi^\smallsquare\wedge\psi^\smallsquare}\proves\varphi^\smallsquare\to\nec\varphi^\circ&$\mathbf{P}$\\
                        \fa\set{\varphi^\smallsquare\wedge\psi^\smallsquare}\proves\nec\varphi^\circ&$\mathbf{P}$\\
                        \fa\set{\varphi^\smallsquare\wedge\psi^\smallsquare}\proves\varphi^\smallsquare\wedge\psi^\smallsquare\to\psi^\smallsquare&$\mathbf{P}$\\
                        \fa\set{\varphi^\smallsquare\wedge\psi^\smallsquare}\proves\psi^\smallsquare&$\mathbf{P}$\\
                        \fa\set{\varphi^\smallsquare\wedge\psi^\smallsquare}\proves\psi^\smallsquare\to\nec\psi^\circ&$\mathbf{P}$\\
                        \fa\set{\varphi^\smallsquare\wedge\psi^\smallsquare}\proves\nec\psi^\circ&$\mathbf{P}$\\
                        \fa\set{\varphi^\smallsquare\wedge\psi^\smallsquare}\proves\nec\varphi^\circ\to\nec\psi^\circ\to\nec\varphi^\circ\wedge\nec\psi^\circ&$\mathbf{P}$\\
                        \fa\set{\varphi^\smallsquare\wedge\psi^\smallsquare}\proves\nec\psi^\circ\to\nec\varphi^\circ\wedge\nec\psi^\circ&$\mathbf{P}$\\
                        \fa\set{\varphi^\smallsquare\wedge\psi^\smallsquare}\proves\nec\varphi^\circ\wedge\nec\psi^\circ&$\mathbf{P}$\\
                        \fa\set{\varphi^\smallsquare\wedge\psi^\smallsquare}\proves\nec\varphi^\circ\wedge\nec\psi^\circ\to\nec(\varphi^\circ\wedge\psi^\circ)&$\mathbf{P}$\\
                        \fa\set{\varphi^\smallsquare\wedge\psi^\smallsquare}\proves\nec(\varphi^\circ\wedge\psi^\circ)&$\mathbf{P}$\\
                        \fa\proves\varphi^\smallsquare\wedge\psi^\smallsquare\to\nec(\varphi^\circ\wedge\psi^\circ)&$\mathbf{P}$\\
                    \end{fitch}
                \end{subcase}

            \begin{case}
                \textsc{Caso 4.}
                Se a sentença $\alpha$ for o resultado da disjunção de duas outras sentenças $\varphi$ e $\psi$, sabe-se que $\nec{(\varphi\vee\psi)}^\circ=\nec(\nec\varphi^\circ\vee\nec\psi^\circ)$ e que ${(\varphi\vee\psi)}^\smallsquare=\varphi^\smallsquare\vee\psi^\smallsquare$ pelas definições das traduções.
                Separaremos a prova em dois casos: um para a ida $\nec(\nec\varphi^\circ\vee\nec\psi^\circ)\to\varphi^\smallsquare\vee\psi^\smallsquare$ e outro para a volta $\varphi^\smallsquare\vee\psi^\smallsquare\to\nec(\nec\varphi^\circ\vee\nec\psi^\circ)$.
                Ambas as implicações, então, podem ser unidas em uma bi-implicação por meio do esquema da introdução da conjunção \hyperref[MA3]{$\mathbf{A_3}$}.
            \end{case}

            \begin{subcase}
                \textsc{Caso 4.1.}
                A partir de $\mathbf{H}$, temos que $\mathbf{H_1}={\entails\nec\varphi^\circ\to\varphi^\smallsquare}$ e que $\mathbf{H_2}={\entails\nec\psi^\circ\to\psi^\smallsquare}$ por meio dos esquemas da eliminação da conjunção e da aplicação da regra da separação.
                Valendo-se do listado acima em conjunto com alguns lemas, pode-se provar que $\entails\nec(\nec\varphi^\circ\vee\nec\psi^\circ)\to\varphi^\smallsquare\vee\psi^\smallsquare$ pela seguinte sucessão de dedução:
                \footnotesize
                \begin{fitch}
                    \fb\set{\nec(\nec\varphi^\circ\vee\nec\psi^\circ)}\entails\nec\varphi^\circ\to\varphi^\smallsquare\\
                    \fa\set{\nec(\nec\varphi^\circ\vee\nec\psi^\circ)}\entails(\nec\varphi^\circ\to\varphi^\smallsquare)\to\nec\varphi^\circ\to\varphi^\smallsquare\vee\psi^\smallsquare\\
                    \fa\set{\nec(\nec\varphi^\circ\vee\nec\psi^\circ)}\entails\nec\varphi^\circ\to\varphi^\smallsquare\vee\psi^\smallsquare\\

                    \fa\set{\nec(\nec\varphi^\circ\vee\nec\psi^\circ)}\entails\nec\psi^\circ\to\psi^\smallsquare\\
                    \fa\set{\nec(\nec\varphi^\circ\vee\nec\psi^\circ)}\entails(\nec\psi^\circ\to\psi^\smallsquare)\to\nec\psi^\circ\to\varphi^\smallsquare\vee\psi^\smallsquare\\
                    \fa\set{\nec(\nec\varphi^\circ\vee\nec\psi^\circ)}\entails\nec\psi^\circ\to\varphi^\smallsquare\vee\psi^\smallsquare\\

                    \fa\set{\nec(\nec\varphi^\circ\vee\nec\psi^\circ)}\entails\nec(\nec\varphi^\circ\vee\nec\psi^\circ)\\
                    \fa\set{\nec(\nec\varphi^\circ\vee\nec\psi^\circ)}\entails\nec(\nec\varphi^\circ\vee\nec\psi^\circ)\to\nec\varphi^\circ\vee\nec\psi^\circ\\
                    \fa\set{\nec(\nec\varphi^\circ\vee\nec\psi^\circ)}\entails\nec\varphi^\circ\vee\nec\psi^\circ\\
                    \fa\set{\nec(\nec\varphi^\circ\vee\nec\psi^\circ)}\entails(\nec\varphi^\circ\to\varphi^\smallsquare\vee\psi^\smallsquare)\to(\nec\psi^\circ\to\varphi^\smallsquare\vee\psi^\smallsquare)\to\nec\varphi^\circ\vee\nec\psi^\circ\to\varphi^\smallsquare\vee\psi^\smallsquare\\
                    \fa\set{\nec(\nec\varphi^\circ\vee\nec\psi^\circ)}\entails(\nec\psi^\circ\to\varphi^\smallsquare\vee\psi^\smallsquare)\to\nec\varphi^\circ\vee\nec\psi^\circ\to\varphi^\smallsquare\vee\psi^\smallsquare\\
                    \fa\set{\nec(\nec\varphi^\circ\vee\nec\psi^\circ)}\entails\nec\varphi^\circ\vee\nec\psi^\circ\to\varphi^\smallsquare\vee\psi^\smallsquare\\
                    \fa\set{\nec(\nec\varphi^\circ\vee\nec\psi^\circ)}\entails\varphi^\smallsquare\vee\psi^\smallsquare\\
                    \fa\entails\nec(\nec\varphi^\circ\vee\nec\psi^\circ)\to\varphi^\smallsquare\vee\psi^\smallsquare\\
                \end{fitch}
            \end{subcase}

            \begin{subcase}
                \textsc{Caso 4.2.}
                A partir de $\mathbf{H}$, temos que $\mathbf{H_1}={\entails\varphi^\smallsquare\to\nec\varphi^\circ}$ e que $\mathbf{H_2}={\entails\psi^\smallsquare\to\nec\psi^\circ}$ por meio dos esquemas da eliminação da conjunção e da aplicação regra da separação.
                Valendo-se do listado acima em conjunto com alguns lemas, pode-se provar que $\entails\varphi^\smallsquare\vee\psi^\smallsquare\to\nec(\nec\varphi^\circ\vee\nec\psi^\circ)$ pela seguinte sucessão de dedução:
            \end{subcase}

            \begin{case}
                \textsc{Caso 5.}
                Se a sentença $\alpha$ for o resultado da implicação de uma sentença $\varphi$ a uma sentença $\psi$, sabe-se que $\nec{(\varphi\to\psi)}^\circ=\nec(\nec\varphi^\circ\to\psi^\circ)$ e que ${(\varphi\to\psi)}^\smallsquare=\nec(\varphi^\smallsquare\to\psi^\smallsquare)$ pelas definições das traduções.
                Separaremos a prova em dois casos: um para a ida $\nec(\nec\varphi^\circ\to\psi^\circ)\to\nec(\varphi^\smallsquare\to\psi^\smallsquare)$ e outro para a volta $\nec(\varphi^\smallsquare\to\psi^\smallsquare)\to\nec(\nec\varphi^\circ\to\psi^\circ)$.
                Ambas as implicações, então, podem ser unidas em uma bi-implicação por meio do esquema \hyperref[MA3]{$\mathbf{A_3}$}.
            \end{case}

                \begin{subcase}
                    \textsc{Caso 5.1.}
                    A partir de $\mathbf{H}$, temos que $\mathbf{H_1}={\entails\varphi^\smallsquare\to\nec\varphi^\circ}$ e que $\mathbf{H_2}={\entails\psi^\smallsquare\to\nec\psi^\circ}$ por meio dos esquemas da eliminação da conjunção e da aplicação regra da separação.
                    Valendo-se do listado acima em conjunto com alguns lemas, pode-se provar que $\entails\nec(\nec\varphi^\circ\to\psi^\circ)\to\nec(\psi^\smallsquare\to\psi^\smallsquare)$ pela seguinte sucessão de dedução:

                    \footnotesize
                    \begin{fitch}
                        \fb\set{\nec(\nec\varphi^\circ\to\psi^\circ)}\entails\nec(\nec\varphi^\circ\to\psi^\circ)\\
                        \fa\set{\nec(\nec\varphi^\circ\to\psi^\circ)}\entails\nec(\nec\varphi^\circ\to\psi^\circ)\to\nec\nec\varphi^\circ\to\nec\psi^\circ\\
                        \fa\set{\nec(\nec\varphi^\circ\to\psi^\circ)}\entails\nec\varphi^\circ\to\nec\nec\varphi^\circ\\
                        \fa\set{\nec(\nec\varphi^\circ\to\psi^\circ)}\entails\nec\nec\varphi^\circ\to\nec\psi^\circ\\
                        \fa\set{\nec(\nec\varphi^\circ\to\psi^\circ)}\entails(\nec\varphi^\circ\to\nec\nec\varphi^\circ)\to(\nec\nec\varphi^\circ\to\nec\psi^\circ)\to\nec\varphi^\circ\to\nec\psi^\circ\\
                        \fa\set{\nec(\nec\varphi^\circ\to\psi^\circ)}\entails(\nec\nec\varphi^\circ\to\nec\psi^\circ)\to\nec\varphi^\circ\to\nec\psi^\circ\\
                        \fa\set{\nec(\nec\varphi^\circ\to\psi^\circ)}\entails\varphi^\smallsquare\to\nec\varphi^\circ\\
                        \fa\set{\nec(\nec\varphi^\circ\to\psi^\circ)}\entails\nec\varphi^\circ\to\nec\psi^\circ\\
                        \fa\set{\nec(\nec\varphi^\circ\to\psi^\circ)}\entails(\varphi^\smallsquare\to\nec\varphi^\circ)\to(\nec\varphi^\circ\to\nec\psi^\circ)\to\varphi^\smallsquare\to\nec\psi^\circ\\
                        \fa\set{\nec(\nec\varphi^\circ\to\psi^\circ)}\entails(\nec\varphi^\circ\to\nec\psi^\circ)\to\varphi^\smallsquare\to\nec\psi^\circ\\
                        \fa\set{\nec(\nec\varphi^\circ\to\psi^\circ)}\entails\varphi^\smallsquare\to\nec\psi^\circ\\
                        \fa\set{\nec(\nec\varphi^\circ\to\psi^\circ)}\entails\nec\psi^\circ\to\psi^\smallsquare\\
                        \fa\set{\nec(\nec\varphi^\circ\to\psi^\circ)}\entails(\varphi^\smallsquare\to\nec\psi^\circ)\to(\nec\psi^\circ\to\psi^\smallsquare)\to\psi^\smallsquare\to\psi^\smallsquare\\
                        \fa\set{\nec(\nec\varphi^\circ\to\psi^\circ)}\entails(\nec\psi^\circ\to\psi^\smallsquare)\to\psi^\smallsquare\to\psi^\smallsquare\\
                        \fa\set{\nec(\nec\varphi^\circ\to\psi^\circ)}\entails\psi^\smallsquare\to\psi^\smallsquare\\
                        \fa\set{\nec(\nec\varphi^\circ\to\psi^\circ)}\entails\nec(\psi^\smallsquare\to\psi^\smallsquare)\\
                        \fa\entails\nec(\nec\varphi^\circ\to\psi^\circ)\to\nec(\psi^\smallsquare\to\psi^\smallsquare)\\
                    \end{fitch}
                \end{subcase}

                \begin{subcase}
                    \textsc{Caso 5.2.}
                    A partir de $\mathbf{H}$, temos que $\mathbf{H_1}=\varphi^\smallsquare\to\nec\varphi^\circ$ e que $\mathbf{H_2}=\psi^\smallsquare\to\nec\psi^\circ$ por meio dos esquemas da eliminação da conjunção e da aplicação regra da separação.
                    Valendo-se do listado acima em conjunto com alguns lemas, pode-se provar que $\entails\nec(\psi^\smallsquare\to\psi^\smallsquare)\to\nec(\nec\varphi^\circ\to\psi^\circ)$ pela seguinte sucessão de dedução:

                    \footnotesize
                    \begin{fitch}
                        \fb\set{\nec(\varphi^\smallsquare\to\psi^\smallsquare)}\entails\nec(\varphi^\smallsquare\to\psi^\smallsquare)\\
                        \fa\set{\nec(\varphi^\smallsquare\to\psi^\smallsquare)}\entails\nec(\varphi^\smallsquare\to\psi^\smallsquare)\to\varphi^\smallsquare\to\psi^\smallsquare\\
                        \fa\set{\nec(\varphi^\smallsquare\to\psi^\smallsquare)}\entails\varphi^\smallsquare\to\psi^\smallsquare\\
                        \fa\set{\nec(\varphi^\smallsquare\to\psi^\smallsquare)}\entails\\
                        \fa\set{\nec(\varphi^\smallsquare\to\psi^\smallsquare)}\entails\\
                        \fa\set{\nec(\varphi^\smallsquare\to\psi^\smallsquare)}\entails\\
                    \end{fitch}
                \end{subcase}
            \vspace{\baselineskip}
            Tendo-se provado todos os casos do passo de indução, podemos concluir que ambas as traduções apresentadas equivalem, ou seja, que $\entails\nec\alpha^\circ\leftrightarrow\alpha^\smallsquare$.
        \end{proof}
    \end{theorem}
