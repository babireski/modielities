\chapter{Introdução}

    Os sistemas modais consistem em um conjunto de aumentos aos sistemas proposicionais um ou mais operadores, chamados \emph{modalidades}, que qualificam sentenças.
    No caso do sistema $\mathbf{S4}$, dito $\mathfrak{M}$, são adicionadas as modalidades duais de \emph{necessidade} e \emph{possibilidade} --- denotadas $\nec$ e $\pos$ ---, bem como regras que governam o comportamento destas modalidades.
    Neste sistema, valem sentenças da forma $\nec(\alpha\to\alpha)\to\pos\alpha\to\pos\beta$, da forma $\alpha\to\pos\alpha$ e da forma $\pos\alpha\to\pos\pos\alpha$ \citep{Zach+others.2024}, que se assemelham com as transformações naturais de uma \emph{mônada}.

    \vspace{.5\baselineskip}
    As \emph{mônadas} ganharam destaque em pesquisa de liguagens de programação deste de~\cite{Moggi.1991} formalizou a metalinguagem $\lambda_c$ como uma maneira de modelar \emph{efeitos computacionais}, como parcialidade, não-determinismo e continuações.
    \cite{Pfenning+Davies.2001} notam que as modalidades de necessidade e possibilidade podem ser vistas como valores que sobrevivem a efeitos e como computações com efeitos, nesta ordem.
    Ainda, eles apresentam uma função que traduzem sentenças do sistema laxo, que corresponde com $\lambda_c$, a sentenças do sistema $\mathfrak{M}$.

    \vspace{.5\baselineskip}
    Computações com efeitos modeladas por \emph{mônadas} imergem e podem ser imersas em \emph{continuações}~\citep{Filinski.1994}.\footnote{\emph{Embedded.}}
    Como continuações são usadas como representação dentro de compiladores, uma linguagem baseada no sistema $\mathfrak{M}$ que consiga modelar efeitos mostra-se relevante do ponto de vista da compilação.
    Este trabalho busca fundamentar futuras investigações nesse sentido.

    \vspace{.5\baselineskip}
    Para tanto, apresentaremos duas traduções do sistema intuicionista $\mathfrak{I}$ ao sistema $\mathfrak{M}$ conforme definidas por~\cite{Troelstra+Schwichtenberg.2000}, bem como suas provas de interderivação e correção.
    O sistema de origem, o intuicionista, destaca-se por abarcar a noção de \emph{construção}, que em muito lembra a noção de \emph{computação}.
    As traduções apresentadas e as asserções demonstradas serão verificadas no provador de teoremas \emph{Rocq}.\footnote{Antigamente conhecido como\emph{Coq}.}

    \vspace{.5\baselineskip}
    As implementações das demonstrações deste trabalho usarão como base a biblioteca de sistemas modais normais desenvolvida em \emph{Rocq} por~\cite{Silveira+others.2022} e posteriormente aumentada por~\cite{Nunes+Roggia+Torrens.2025} de modo a permitir a fusão os sistemas modais.
    Ainda, as implementações das traduções serão afins às implementações das traduções feitas por~\cite{Sehnem.2023}, estas tendo como sistema de destino o sistema linear de~\cite{Girard}.
    Tais traduções, tal-qualmente definidas por~\cite{Girard}, assemelham-se muito com as traduções abordadas neste trabalho.

    \vspace{.5\baselineskip}
    Provadores de teoremas tem-se mostrado ferramentas poderosas para a verificação de asserções em sistemas formais.
    Tais programas analisam cada passo da demonstração e podem apontar erros que em muitos casos passariam despercebidos.
    Os artefatos da formalização ficarão dispostos para consulta de todos no endereço \texttt{\footnotesize\textbf{\url{https://github.com/babireski/embedding}}}.

    \section{Objetivos}
    Este trabalho consiste numa continuação do desenvolvimento da biblioteca de formalização de sistemas modais normais iniciado por~\cite{Silveira} e posteriormente expandida de forma a permitir a fusão de sistemas modais por~\cite{Nunes}. Nele, formalizaremos as traduções do sistema intuicionista ao sistema modal $\mathbf{S4}$ no asssitente de provas \textsc{coq} e provaremos suas propriedades. Uma formalização de traduções entre sistemas de dedução similar a nossa foi feita por~\cite{Sehnem}, neste caso tendo como alvo o sistema linear de~\cite{Girard}. Todas as formalizações citadas acima deram-se no assistente de provas \textsc{coq}, o mesmo assistente usado neste trabalho. Como objetivos específicos, listamos:

    \begin{itemize}
        \item Fornecer uma introdução ao conceito de sistemas de dedução;
        \item Fornecer uma introdução ao conceito de traduções entre sistemas;
        \item Fornecer uma introdução ao sistema intuicionista;
        \item Fornecer uma introdução aos sistemas modais, em especial o $\mathbf{S4}$;
        \item Apresentar as traduções do sistema intuicionista ao sistema $\mathbf{S4}$;
        \item Provar manualmente a correção e completude das traduções providas bem como outras propriedades pertinentes;
        \item Formalizar as provas no provador de teoremas interativo \textsc{coq}.
    \end{itemize}

    \section{Estruturação}
    Estruturaremos este trabalho em cinco partes, iniciando-se por esta introdução. O Capítulo 2 consiste numa fundamentação de conceitos basilares ao desenvolvimento deste trabalho, notadamente os conceitos de \emph{sistemas de dedução}, \emph{traduções} e \emph{provadores de teoremas}. O Capítulo 3 apresenta as definições dos sistemas e traduções relevantes a este trabalho. No Capítulo 4 são provadas todas as propriedades abarcadas no escopo deste trabalho. Por fim, o Capítulo 5 compreende considerações parciais acerca do desenvolvido até o momento.