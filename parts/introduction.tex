\chapter{Introdução}

    Os sistemas modais consistem em um conjunto de aumentos aos sistemas proposicionais um ou mais operadores, chamados \emph{modalidades}, que qualificam sentenças.
    No caso do sistema $\mathbf{S4}$, dito $\mathfrak{M}$, são adicionadas as modalidades duais de \emph{necessidade} e \emph{possibilidade} --- denotadas $\nec$ e $\pos$ ---, bem como regras que governam o comportamento destas modalidades.
    Neste sistema, valem sentenças da forma $\nec(\alpha\to\alpha)\to\pos\alpha\to\pos\beta$, da forma $\alpha\to\pos\alpha$ e da forma $\pos\alpha\to\pos\pos\alpha$ \citep{Zach+others.2019}, que se assemelham com as transformações naturais de uma \emph{mônada}.

    \vspace{.5\baselineskip}
    As \emph{mônadas} ganharam destaque em pesquisa de liguagens de programação deste de~\cite{Moggi.1991} formalizou a metalinguagem $\lambda_c$ como uma maneira de modelar \emph{efeitos computacionais}, como parcialidade, não-determinismo e continuações.
    \cite{Pfenning+Davies.2001} notam que as modalidades de necessidade e possibilidade podem ser vistas como valores que sobrevivem a efeitos e como computações com efeitos, nesta ordem.
    Ainda, eles apresentam uma função que traduzem sentenças do sistema laxo, que corresponde com $\lambda_c$, a sentenças do sistema $\mathfrak{M}$.

    \vspace{.5\baselineskip}
    Computações com efeitos modeladas por \emph{mônadas} imergem e podem ser imersas em \emph{continuações}~\citep{Filinski.1994}.\footnote{\emph{Embedded.}}
    Como continuações são usadas como representação dentro de compiladores, uma linguagem baseada no sistema $\mathfrak{M}$ que consiga modelar efeitos mostra-se relevante do ponto de vista da compilação.
    Este trabalho busca fundamentar futuras investigações nesse sentido.

    \vspace{.5\baselineskip}
    Para tanto, apresentaremos duas traduções do sistema intuicionista $\mathfrak{I}$ ao sistema $\mathfrak{M}$ conforme definidas por~\cite{Troelstra+Schwichtenberg.2000}, bem como suas provas de interderivação e correção.
    O sistema de origem, o intuicionista, destaca-se por abarcar a noção de \emph{construção}, que em muito lembra a noção de \emph{computação}.
    As traduções apresentadas e as asserções demonstradas serão verificadas no provador de teoremas \emph{Rocq}.\footnote{Antigamente conhecido como \emph{Coq}.}
    Os artefatos desta formalização ficarão dispostos para consulta no endereço \texttt{<}\url{https://github.com/babireski/embedding}\texttt{>}.

    \vspace{.5\baselineskip}
    As implementações das demonstrações deste trabalho usarão como base a biblioteca de sistemas modais normais desenvolvida em \emph{Rocq} por~\cite{Silveira+others.2022} e posteriormente aumentada por~\cite{Nunes+others.2025} de modo a permitir a fusão os sistemas modais.
    Ainda, as implementações das traduções serão afins às implementações das traduções feitas por~\cite{Sehnem.2023}, estas tendo como sistema de destino o sistema linear de~\cite{Girard.1987}.
    Tais traduções, tal-qualmente definidas por~\cite{Girard.1987}, assemelham-se muito com as traduções abordadas neste trabalho.

    \vspace{.5\baselineskip}
    Historicamente, demonstrações feitas por um autor são validadas à mão por seus pares, em um processo que pode ser laborioso e sujeito a erros.
    Neste sentido, provadores de teoremas tem-se mostrado ferramentas poderosas para a verificação de asserções em sistemas formais~\citep{Chlipala.2013}.
    Tais \emph{softwares} --- que usufruem de uma relação profunda entre demonstrações e provas~\citep{Wadler.2015} --- analisam cada passo da demonstração e podem apontar erros que em muitos casos passariam despercebidos.
    Assim, este trabalho busca fornecer uma base robusta para futuros estudos sobre o tema abordado.

    \section{Metas}
    Este trabalho busca apresentas duas traduções do sistema $\mathfrak{I}$ ao sistema $\mathfrak{M}$ conforme definidas por~\cite{Troelstra+Schwichtenberg.2000} e as suas demonstrações de interderivação e correção, bem como formalizar tais traduções e propriedades no assistente de provas interativo \emph{Rocq}.
    Para tanto, usaremos a biblioteca uma biblioteca de sistemas modais normais desenvolvida por~\cite{Silveira+others.2022} e posteriormente aumentada de forma a permitir a fusão de sistemas modais por~\cite{Nunes+others.2025}.
    Listamos as metas que seguem para este trabalho.

    \begin{enumerate}[label=\textbf{{(\alph*)}}, left=\parindent]
        \item Apresentar a definição de \emph{sistemas de dedução} e de \emph{traduções} entre tais sistemas.
        \item Apresentar as definições do sistema $\mathfrak{I}$ e do sistema $\mathfrak{M}$.
        \item Apresentar as duas traduções do sistema $\mathfrak{I}$ ao sistema $\mathfrak{M}$ citadas.
        \item Apresentar demonstrações de interderivação e correção para as traduções.
        \item Formalizar as definições e demonstrações no assistente de provas \emph{Rocq}.
    \end{enumerate}

    \section{Estruturação}
    Estruturaremos este trabalho em seis partes, a começar por esta introdução.
    Em seguida, na \textsc{Parte \ref{foundation}}, apresentaremos os conceitos basilares ao desenvolvimento deste trabalho, nomeadamente notadamente os conceitos de \emph{sistemas de dedução} (\textsc{Seção \ref{foundation.systems}}), \emph{traduções} (\textsc{Seção \ref{foundation.translations}}) e \emph{provadores de teoremas} (\textsc{Seção \ref{foundation.provers}}).
    Durante a parte \textsc{Parte \ref{systems}}, serão apresentadas as definições dos  definições dos sistemas de dedução --- o sistema $\mathfrak{I}$ (\textsc{Seção \ref{systems.intuitionistic}}) e o sistema $\mathfrak{M}$ (\textsc{Seção \ref{systems.modal}}) ---, bem como as traduções abordadas neste trabalho (\textsc{Seção \ref{systems.translations}}).
    Ainda, relacionaremos o sistema $\mathfrak{M}$ com \emph{efeitos computacionais} (\textsc{Seção \ref{systems.effects}}).
    Serão apresentados na \textsc{Parte \ref{properties}} as derivações das propriedades, divididas em \emph{metateoremas}  (\textsc{Seção \ref{properties.metaproperties}}), \emph{interderivabilidade}  (\textsc{Seção \ref{properties.interderivation}}) e \emph{correção}  (\textsc{Seção \ref{properties.soundness}}).
    Por fim, discutiremos a implementação na \textsc{Parte \ref{implementation}} e faremos considerações finais na \textsc{Parte \ref{conclusion}}.