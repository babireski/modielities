\chapter{Implementação} 

A biblioteca desenvolvida por \cite{Silveira} apresenta algumas diferenças menores de definições quando comparada ao que foi formalizado anteriormente neste trabalho.
Tais diferenças serão abordadas em detalhes nesta seção.
Primeiramente, trataremos das diferenças na linguagem dos sistemas modais.
A linguagem da biblioteca de \cite{Silveira} permite sistemas multimodais, ou seja, sistemas com uma quantidade qualquer de modalidades.
Distinguiremos estas modalidades por um valor $i\in\mathbb{N}$.
Ademais, a biblioteca se \cite{Silveira} trata a negação como uma operação primitiva -- confrontando a contradição, usada neste trabalho --, bem como trata a possibilidade do mesmo modo.
Pequenas modificações nas definições deste trabalho precisaram ser feitas para dar conta de tais diferenças.
Tais modificações serão assomadas quando relevantes.

\begin{definition}[$\mathcal{L}_{\nec}$]
    A linguagem dos sistemas multimodais com $n\in\mathbb{N}^+$ modalidades distinguidas por $i\in\mathbb{N}^+$ pode ser induzida a partir da assinatura $\Sigma_{\nec}=\sequence{\mathcal{P},\mathcal{C}_{\nec}}$, onde $\mathcal{C}_{\nec}=\set{\neg^1,\nec^1_i,\pos^1_i,\wedge^2,\vee^2,\to^2\mid i\leq n}$.
\end{definition}

Outra diferença entre as definições deste trabalho e as definições de \cite{Silveira} consistem no conjunto regras de deduções, ligeiramente maior na biblioteca, conforme mostrado abaixo.
Isso acontece pela adição da regra $\mathbf{A_9}$ -- que pode ser derivada usando as regras apresentadas neste trabalho -- e pela regra $\mathbf{B_4}$ -- neste trabalho tratada como uma definição.
Ainda, as regras $\mathbf{B_1}$, $\mathbf{B_2}$ e $\mathbf{B_3}$ precisaram ser quantificadas universalmente sobre $i\in\mathbb{N}^+$.
Todas as definições e teoremas deste trabalho serão quantificados analogamente, de modo a encaixar-se nas definições da biblioteca.

\begin{definition}[$\vdash_{\mathfrak{L}}$]
    Abaixo estão definidas as regras do sistema multimodal $\mathfrak{L}$.
\hfill\break
\begin{center}
    \footnotesize
    \AxiomC{}
    \RightLabel{\footnotesize$\mathbf{A_1}$}
    \UnaryInfC{$\Gamma\vdash\alpha\to\beta\to\alpha$}
    \DisplayProof
    \quad
    \AxiomC{}
    \RightLabel{\footnotesize$\mathbf{A_2}$}
    \UnaryInfC{$\Gamma\vdash(\alpha\to\beta\to\gamma)\to(\alpha\to\beta)\to\alpha\to\gamma$}
    \DisplayProof
\end{center}

\begin{center}
    \footnotesize
    \AxiomC{}
    \RightLabel{\footnotesize$\mathbf{A_3}$}
    \UnaryInfC{$\Gamma\vdash\alpha\to\beta\to\alpha\wedge\beta$}
    \DisplayProof
    \quad
    \AxiomC{}
    \RightLabel{\footnotesize$\mathbf{A_4}$}
    \UnaryInfC{$\Gamma\vdash\alpha\wedge\beta\to\alpha$}
    \DisplayProof
    \quad
    \AxiomC{}
    \RightLabel{\footnotesize$\mathbf{A_5}$}
    \UnaryInfC{$\Gamma\vdash\alpha\wedge\beta\to\beta$}
    \DisplayProof
\end{center}

\begin{center}
    \footnotesize
    \AxiomC{}
    \RightLabel{\footnotesize$\mathbf{A_6}$}
    \UnaryInfC{$\Gamma\vdash\alpha\to\alpha\vee\beta$}
    \DisplayProof
    \quad
    \AxiomC{}
    \RightLabel{\footnotesize$\mathbf{A_7}$}
    \UnaryInfC{$\Gamma\vdash\beta\to\alpha\vee\beta$}
    \DisplayProof
    \quad
    \AxiomC{}
    \RightLabel{\footnotesize$\mathbf{A_8}$}
    \UnaryInfC{$\Gamma\vdash(\alpha\to\gamma)\to(\beta\to\gamma)\to\alpha\vee\beta\to\gamma$}
    \DisplayProof
\end{center}

\begin{center}
    \footnotesize
    \AxiomC{}
    \RightLabel{\footnotesize$\mathbf{A_9}$}
    \UnaryInfC{$\Gamma\vdash(\neg\beta\to\neg\alpha)\to\alpha\to\beta$}
    \DisplayProof
    \quad
    \AxiomC{}
    \RightLabel{\footnotesize$\mathbf{A_\neg}$}
    \UnaryInfC{$\Gamma\vdash\neg\neg\alpha\to\alpha$}
    \DisplayProof
\end{center}

\begin{center}
    \footnotesize
    \AxiomC{}
    \RightLabel{\footnotesize$\mathbf{B_1}$}
    \UnaryInfC{$\Gamma\vdash\Box_i(\alpha\to\beta)\to\Box_i\alpha\to\Box_i\beta$}
    \DisplayProof
    \quad
    \AxiomC{}
    \RightLabel{\footnotesize$\mathbf{B_2}$}
    \UnaryInfC{$\Gamma\vdash\Box_i\alpha\to\alpha$}
    \DisplayProof
    \quad
    \AxiomC{}
    \RightLabel{\footnotesize$\mathbf{B_3}$}
    \UnaryInfC{$\Gamma\vdash\Box_i\alpha\to\Box_i\Box_i\alpha$}
    \DisplayProof
\end{center}

\begin{center}
    \footnotesize
    \AxiomC{}
    \RightLabel{\footnotesize$\mathbf{B_4}$}
    \UnaryInfC{$\Gamma\vdash\pos_i\alpha\leftrightarrow\neg\Box_i\neg\alpha$}
    \DisplayProof
\end{center}

\begin{center}
    \footnotesize
    \AxiomC{$\alpha\in\Gamma$}
    \RightLabel{\footnotesize$\mathbf{R_1}$}
    \UnaryInfC{$\Gamma\vdash\alpha$}
    \DisplayProof
    \quad
    \AxiomC{$\Gamma\vdash\alpha$}
    \AxiomC{$\Gamma\vdash\alpha\to\beta$}
    \RightLabel{\footnotesize$\mathbf{R_2}$}
    \BinaryInfC{$\Gamma\vdash\beta$}
    \DisplayProof
    \quad
    \AxiomC{$\vdash\alpha$}
    \RightLabel{\footnotesize$\mathbf{R_3}$}
    \UnaryInfC{$\Gamma\vdash\Box_i\alpha$}
    \DisplayProof
\end{center}
\end{definition}
\hfill\break
Uma vez ditas as diferenças, partiremos para a definição da linguagem intuicionista.
Para tanto, definiu-se um tipo indutivo com cinco construtores, conforme apresentado abaixo.
O construtor \texttt{\footnotesize\textbf{Contradiction}} representa a contradição $\bot$.
O construtor \texttt{\footnotesize\textbf{Proposition}} eleva um valor $n\in\mathbb{N}$ a uma sentença intuicionista que representa uma letra proposicional.
Por fim, os construtores \texttt{\footnotesize\textbf{Conjunction}}, \texttt{\footnotesize\textbf{Disjunction}} e \texttt{\footnotesize\textbf{Implication}} criam conjunções, disjunções e implicações --- nesta ordem --- a partir de sentenças menores.
Notações foram definidas de modo a replicar a notação usada neste trabalho, com letras propositionais sendo introduzidas por um operador \texttt{\footnotesize\textbf{\#}}.
Deste modo, podemos representar a sentença intuicionista $a\wedge b\to a\vee b$ como \texttt{\footnotesize\textbf{\#1 ∧ \#2 → \#1 ∨ \#2}}, para $a,b\in\mathcal{P}$.
\\
\begin{mdframed}
\noindent
\texttt
{\noindent\footnotesize\textbf{Inductive} Propositional : \textbf{Set} :=\\
| Contradiction : Propositional\\
| Proposition\ \ \ : nat -> Propositional\\
| Conjunction\ \ \ : Propositional -> Propositional -> Propositional\\
| Disjunction\ \ \ : Propositional -> Propositional -> Propositional\\
| Implication\ \ \ : Propositional -> Propositional -> Propositional.\\
\\
Notation "\ ⊥\ "\ := Contradiction.
}
\end{mdframed}
\hfill\break
Similarmente, uma dedução intuicionista foi definida como um tipo indutivo, desta vez com onze construtores.
Cada um desses construtores representa uma das regras de dedução apresentadas anteriormente.
Assim como feito com a linguagem, foi definida uma notação de modo a facilitar a escrita de asserções.
Cabe destacar que o conjunto de assunções $\Gamma$ foi representada pela sua função indicadora $\mathbf{I}_\Gamma:\mathcal{L}\to\{0,1\}$, em que $\mathbf{I}_\Gamma(\alpha)=1$ se e somente se $\alpha\in\Gamma$.
Na implementação oferecida abaixo, isso foi feito por meio de uma função do tipo \texttt{\footnotesize\textbf{Sentence}} ao tipo \texttt{\footnotesize\textbf{Prop}}, o tipo das proposições.
\hfill\break
\begin{mdframed}
\noindent
\texttt
{\noindent\footnotesize\textbf{Definition} Theory := Propositional -> Prop.\\
\\
\textbf{Reserved} \textbf{Notation} "Γ ⊢ α"\ (at level 110).\\
\\
\textbf{Inductive} Deduction : Theory -> Propositional -> \textbf{Prop} :=\\
| A₁ : \textbf{forall} Γ α β,\ \ \ Γ ⊢ α → β → α\\
| A₂ : \textbf{forall} Γ α β γ, Γ ⊢ (α → β → γ) → (α → β) → α → γ\\
| A₃ : \textbf{forall} Γ α β,\ \ \ Γ ⊢ α → β → α ∧ β\\
| A₄ : \textbf{forall} Γ α β,\ \ \ Γ ⊢ α ∧ β → α\\
| A₅ : \textbf{forall} Γ α β,\ \ \ Γ ⊢ α ∧ β → β\\
| A₆ : \textbf{forall} Γ α β,\ \ \ Γ ⊢ α → α ∨ β\\
| A₇ : \textbf{forall} Γ α β,\ \ \ Γ ⊢ β → α ∨ β\\
| A₈ : \textbf{forall} Γ α β γ, Γ ⊢ (α → γ) → (β → γ) → α ∨ β → γ\\
| A₉ : \textbf{forall} Γ α,\ \ \ \ \ Γ ⊢ ⊥ → α\\
| R₁ : \textbf{forall} Γ α,\ \ \ \ \ Γ ∈ α -> Γ ⊢ α\\
| R₂ : \textbf{forall} Γ α β,\ \ \ Γ ⊢ α → β -> Γ ⊢ α -> Γ ⊢ β\\
\textbf{where} "Γ ⊢ α" := (Deduction Γ α).\\
\\
\textbf{Notation} "Γ ⊢ α" := (Deduction Γ α) (at level 110).
}
\end{mdframed}
\hfill\break
Para ilustração do uso do tipo indutivo \texttt{\footnotesize\textbf{Deduction}}, abaixo fornecemos a prova da identidade $\Gamma\vdash\alpha\to\alpha$.
A prova tem uma relação de um para um com a prova apresentada anteriormente, tirando-se o fato desta ser feita de baixo para cima, da conclusão em direção ao assumido.
Isso acontece porque, via de regra, transformamos a meta de prova em vez das assunções das provas na aplicação.
Nada impediria, entretanto, que o oposto fosse feito.
Abaixo, usamos o comando \texttt{\footnotesize\textbf{apply}} para a aplicar regras de dedução à meta de prova.
Ao fazermos isso, estamos dizendo que esta conclusão foi gerada a partir da aplicação da dita regra.
\hfill\break
\begin{mdframed}
\noindent
\texttt
{\noindent\footnotesize\textbf{Lemma} identity : \textbf{forall} Γ α, Γ ⊢ α → α.\\
\textbf{Proof}.\\
\phantom{\ \ \ \ }\textbf{intros} Γ α.\\
\phantom{\ \ \ \ }\textbf{apply} R₂ \textbf{with} (α → α → α).\\
\phantom{\ \ \ \ }+ \textbf{apply} R₂ \textbf{with} (α → (α → α) → α).\\
\phantom{\ \ \ \ }\phantom{\ \ \ \ }* \textbf{apply} A₂.\\
\phantom{\ \ \ \ }\phantom{\ \ \ \ }* \textbf{apply} A₁.\\
\phantom{\ \ \ \ }+ \textbf{apply} A₁.\\
\textbf{Qed}.
}
\end{mdframed}
\hfill\break
Uma vez definido o sistema intuicionista, podemos definir as funções de tradução do sistema intuicionista ao sistema modal $\mathbf{S4}$, definido por \cite{Silveira}.
A tradução $\bullet^\nec$ e a tradução $\bullet^\circ$, agora chamadas \texttt{\footnotesize\textbf{square}} e \texttt{\footnotesize\textbf{circle}}, foram definidas com apenas duas alterações.
Primeiramente, ambas as funções agora recebem um argumento a mais, que identifica a modalidade usada, uma vez que estamos usando uma biblioteca para sistema multimodais.
Em segundo lugar, a contradição $\bot$ passou a ser mapeada para $a\wedge\neg a$ para algum $a\in\mathcal{P}$, uma vez que esta biblioteca não traz a contradição como um operador primitivo.
\hfill\break
\begin{mdframed}
\noindent
\texttt
{\noindent\footnotesize\textbf{Fixpoint} square (α : Propositional) (i : Index) : Multimodal :=\\
\textbf{match} α \textbf{with}\\
| ⊥\ \ \ \ \ => \#1 ∧ ¬\#1\\
| \#a\ \ \ \ => [i] \#a\\
| α ∧ β => (square α i) ∧ (square β i)\\
| α ∨ β => (square α i) ∨ (square β i)\\
| α → β => [i]((square α i) → (square β i))\\
\textbf{end}.\\
\\
\textbf{Fixpoint} circle (α : Propositional) (i : Index) : Multimodal :=\\
\textbf{match} α \textbf{with}\\
| ⊥\ \ \ \ \ => \#1 ∧ ¬\#1\\
| \#a\ \ \ \ => \#a\\
| α ∧ β => (circle α i) ∧ (circle β i)\\
| α ∨ β => [i](circle α i) ∨ [i](circle β i)\\
| α → β => [i](circle α i) → (circle β i)\\
\textbf{end}.
}
\end{mdframed}
\hfill\break
Faz-se preciso definir novos construtores de tipos para representar conjuntos cujas sentenças são resultados da aplicação de alguma função ou operação, que foram chamados de \texttt{\footnotesize\textbf{Imboxed}}, \texttt{\footnotesize\textbf{Squared}} e \texttt{\footnotesize\textbf{Circled}}.
Cada um desses possui apenas um construtor, nomeados \texttt{\footnotesize\textbf{Imboxing}}, \texttt{\footnotesize\textbf{Squaring}} e \texttt{\footnotesize\textbf{Circling}}, nesta ordem.
\texttt{\footnotesize\textbf{Imboxing}} recebe como argumento um conjunto de sentenças multimodais de tipo \texttt{\footnotesize\textbf{Multimodal -> Prop}} e prefixa a necessidade a cada uma destas sentenças, ou seja, representa o conjunto $\nec\Gamma$. \texttt{\footnotesize\textbf{Squaring}} e \texttt{\footnotesize\textbf{Circling}} recebem como argumento conjuntos de sentenças intuicionistas de tipo \texttt{\footnotesize\textbf{Propositional -> Prop}} e aplicam os seus elementos às funções de tradução \texttt{\footnotesize\textbf{square}} e \texttt{\footnotesize\textbf{circled}}, ou seja, representam os conjuntos $\Gamma^\nec$ e $\Gamma^\circ$.
\hfill\break
\begin{mdframed}
\noindent
\texttt
{\noindent\footnotesize\textbf{Inductive} Imboxed Γ i : Multimodal -> \textbf{Prop} :=\\
| Imboxing : \textbf{forall} α , Γ α  -> Imboxed Γ i ([i]α).
\\
\\
\textbf{Inductive} Squared Γ i : Multimodal -> \textbf{Prop} :=\\
| Squaring : \textbf{forall} α , Γ α  -> Squared Γ i (square α i).
\\
\\
\textbf{Inductive} Circled Γ i : Multimodal -> \textbf{Prop} :=\\
| Circling : \textbf{forall} α , Γ α  -> Circled Γ i (circle α i).
}
\end{mdframed}
\hfill\break
Agora que as definições precisas para o desenvolvimento desta formalização foram feitos, partiremos pra uma enumeração das principais asserções provadas, com considerações pontuais.
Não serão apresentadas as provas das asserções por questões de brevidade.
As provas, entretanto, são paralelas às provas apresentadas anteriormente, a não ser quando destacado o oposto.
\hfill\break
\begin{mdframed}
\texttt{\noindent\footnotesize\textbf{Theorem} generalization : \textbf{forall} Γ α i,\\Imboxed Γ i ⊢ α -> Imboxed Γ i ⊢ [i]α.}
\end{mdframed}

\begin{mdframed}
\texttt{\noindent\footnotesize\textbf{Theorem} stability : \textbf{forall} Γ α i,\\Γ ⊢ square α i → [i](square α i).}
\end{mdframed}

\begin{mdframed}
\texttt{\noindent\footnotesize\textbf{Theorem} biimplication : \textbf{forall} Γ α i,\\Γ ⊢ [i](circle α i) ↔ square α i.}
\end{mdframed}

\begin{mdframed}
\texttt{\noindent\footnotesize\textbf{Theorem} equivalence : \textbf{forall} Γ α i,\\Imboxed (Circled Γ i) i ⊢ circle α i <-> Squared Γ i ⊢ square α i.}
\end{mdframed}

\begin{mdframed}
\texttt{\noindent\footnotesize\textbf{Theorem} soundness : \textbf{forall} Γ α i,\\Γ ⊢ α -> Squared Γ i ⊢ square α i.}
\end{mdframed}

\begin{mdframed}
\texttt{\noindent\footnotesize\textbf{Theorem} soundness : \textbf{forall} Γ α i,\\Γ ⊢ α -> Imboxed (Squared Γ i) i ⊢ circle α i.}
\end{mdframed}
