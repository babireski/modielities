\documentclass{report}

\usepackage{amsthm}
\usepackage{amsmath}
\usepackage{amssymb}
\usepackage{mathtools}
\usepackage{modalities}
\usepackage{turnstile}
% \usepackage{natbib}

\newtheorem{definition}{Definição}

\begin{document}
    \title{Uma formalização da interpretação modal do sistema intuicionista}
    \author{Elian Babireski}
    \date{2024}

    \maketitle

    \chapter{Introdução}
    \chapter{Fundamentação}

    \section{Sistema}

    \begin{definition}[Sistema]
        Um sistema consiste numa tripla $\mathbf{L} = \langle \mathcal{L}, \vdash, \vDash\rangle$, onde $\mathcal{L}$ consiste em um conjunto de sentenças bem-formadas, $\vdash \: : \wp(\mathcal{L}) \times \mathcal{L} $ em uma relação de dedução e $\, \vDash \: : \wp(\mathcal{L}) \times \mathcal{L}$.
        \qed
    \end{definition}

    \begin{definition}[$\mathcal{L}_\mathbf{I}$]
        Define-se a linguagem do sistema intuicionista, denotada $\mathcal{L}_\mathbf{I}$, como o menor conjunto induzido a partir das seguintes regras:
        
        $\top, \bot \in \mathcal{L}_\mathbf{I}$
        
        $\mathcal{P} \subseteq \mathcal{L}_\mathbf{I}$
        
        $\varphi \in \mathcal{L}_\mathbf{I} \Rightarrow \neg \varphi \in \mathcal{L}_\mathbf{I}$

        $\varphi, \psi \in \mathcal{L}_\mathbf{I} \Rightarrow \varphi \circ \psi \in \mathcal{L}_\mathbf{I}$, para $\circ \in \{\wedge, \vee, \to\}$
    \end{definition}

    \begin{definition}[$\mathcal{L}_\mathbf{M}$]
        Define-se a linguagem dos sistemas modais, denotada $\mathcal{L}_\mathbf{M}$, como o menor conjunto induzido a partir das seguintes regras:
        
        $\top, \bot \in \mathcal{L}_\mathbf{I}$
        
        $\mathcal{P} \subseteq \mathcal{L}_\mathbf{I}$
        
        $\varphi \in \mathcal{L}_\mathbf{I} \Rightarrow \circ \varphi \in \mathcal{L}_\mathbf{I}$, para $\circ \in \{\nec, \pos, \neg\}$

        $\varphi, \psi \in \mathcal{L}_\mathbf{I} \Rightarrow \varphi \circ \psi \in \mathcal{L}_\mathbf{I}$, para $\circ \in \{\wedge, \vee, \to\}$
    \end{definition}

    \section{Tradução}

    Traduções entre sistemas consistem em funções que mapeiam sentenças de um sistema a sentenças de outro sistema e garantem certas propiedades. As propriedades a serem garantidas variam e ainda são discutidas na literatura, fazendo deixando a definição exata de tradução -- assim como houve com a definição de sistema -- varie de acordo com a predileção de cada autor. Nesta seção, serão abordadas historicamente noções de tradução entre sistemas, bem como serão definidos e nomeados os conceitos de tradução que serão usados no restante desta tese.

    \begin{definition}[Tradução] 
        Uma sentença $\varphi$ de um sistema $\mathbf{A} = \langle \mathcal{L}_\mathbf{A}, \vdash_\mathbf{A}\rangle$ pode ser traduzida a uma sentença $\varphi^*$ em um sistema $\mathbf{B} = \langle \mathcal{L}_\mathbf{B}, \vdash_\mathbf{B} \rangle$ caso exista uma função $\bullet^* : \mathcal{L}_\mathbf{A} \to \mathcal{L}_\mathbf{B}$ que garanta que $\Gamma \vdash_\mathbf{A} \varphi \Leftrightarrow \Gamma^* \vdash_\mathbf{B} \varphi^*$.
        \qed
    \end{definition}

    \begin{definition}[$\bullet^\neg$] Define-se a tradução $\bullet^\neg$ indutivamente da seguinte maneira:
        \begin{align*}
            p^\neg                     & \coloneqq \neg\neg p                               \\
            \bot^\neg                  & \coloneqq \bot                                     \\
            (\varphi \wedge \psi)^\neg & \coloneqq \neg\neg (\varphi^\neg \wedge \psi^\neg) \\
            (\varphi \vee \psi)^\neg   & \coloneqq \neg\neg (\varphi^\neg \vee \psi^\neg)   \\
            (\varphi \to \psi)^\neg    & \coloneqq \neg\neg (\varphi^\neg \to \psi^\neg)
            \tag*{\qed} 
        \end{align*}
    \end{definition}

    A primeira tradução do sistema intuicionista ao sistema modal foi proposta por Gödel \cite{Gödel} motivado pela possibilidade de leitura da necessidade como uma modalidade de construtividade. Ou seja, por meio dessa tradução, a sentença $\nec \varphi$ poderia ser lida como \textit{$\varphi$ pode ser provada construtivamente} \cite{Troelstra}. Gödel conjeiturou a corretude fraca dessa tradução, que foi posteriormente provada por McKinsey e Tarski \cite{McKinsey} em conjunto com sua completude fraca.

    \begin{definition}[$\bullet^\circ$] Define-se a tradução $\bullet^\circ$ indutivamente da seguinte maneira:
        \begin{align*}
            p^\circ                     & \coloneqq p                                       \\
            \bot^\circ                  & \coloneqq \bot                                    \\
            (\varphi \wedge \psi)^\circ & \coloneqq \varphi^\circ \wedge \psi^\circ         \\
            (\varphi \vee \psi)^\circ   & \coloneqq \nec \varphi^\circ \vee \nec \psi^\circ \\
            (\varphi \to \psi)^\circ    & \coloneqq \nec \varphi^\circ \to \psi^\circ       \\
            (\exists x. \varphi)^\circ  & \coloneqq \exists x. \nec \varphi^\circ           \\
            (\forall x. \varphi)^\circ  & \coloneqq \forall x. \varphi^\circ
            \tag*{\qed} 
        \end{align*}
    \end{definition}

    \begin{definition}[$\bullet^\medsquare$] Define-se a tradução $\bullet^\medsquare$ indutivamente da seguinte maneira:
        \begin{align*}
            p^\medsquare                     & \coloneqq \nec p                                        \\
            \bot^\medsquare                  & \coloneqq \bot                                          \\
            (\varphi \wedge \psi)^\medsquare & \coloneqq \varphi^\medsquare \wedge \psi^\medsquare     \\
            (\varphi \vee \psi)^\medsquare   & \coloneqq \varphi^\medsquare \vee \psi^\medsquare       \\
            (\varphi \to \psi)^\medsquare    & \coloneqq \nec (\varphi^\medsquare \to \psi^\medsquare) \\
            (\exists x. \varphi)^\medsquare  & \coloneqq \exists x. \varphi^\medsquare                 \\
            (\forall x. \varphi)^\medsquare  & \coloneqq \nec \forall x. \varphi^\medsquare
            \tag*{\qed} 
        \end{align*}
    \end{definition}

    Faz-se interessante pontuar que as traduções $\bullet^\circ$ e $\bullet^\medsquare$ correspondem, respectivamente, às traduções $\bullet^\circ$ e $\bullet^*$ do sistema intuicionista ao sistema linear providas por Girard \cite{Girard}, sendo as primeiras correspondentes a uma ordem de avaliação por nome (\textit{call-by-name}) e as segundas a uma ordem de avaliação por valor (\textit{call-by-value}). 

    \bibliographystyle{plain}
    \bibliography{bibliography}
\end{document}