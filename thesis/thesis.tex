\documentclass{report}

\usepackage{amsmath}
\usepackage{amssymb}
\usepackage{amsthm}
\usepackage{mathtools}
\usepackage{modalities}
\usepackage{turnstile}
% \usepackage{natbib}

\newtheorem{definition}{Definição}

\begin{document}
    \title{Uma formalização da interpretação modal do sistema intuicionista}
    \author{Elian Babireski}
    \date{2024}

    \maketitle


    \begin{table}[h]
        \centering
        \begin{tabular}{|c|c|c|}
            \hline
            \textbf{Axioma} & \textbf{Sentença}                                       & \textbf{Condição} \\ \hline
            \textbf{K}      & $\nec(\varphi \to \psi) \to \nec \varphi \to \nec \psi$ & Distributividade  \\ \hline
            \textbf{T}      & $\nec \varphi \to \varphi$                              & Reflexividade     \\ \hline
            \textbf{B}      & $\varphi \to \nec \pos \varphi$                         & Simetria          \\ \hline
            \textbf{D}      & $\nec \varphi \to \pos \varphi$                         & Serialidade       \\ \hline
            \textbf{4}      & $\nec \varphi \to \nec \nec \varphi$                    & Transitividade    \\ \hline
            \textbf{5}      & $\pos \varphi \to \nec \pos \varphi$                    & Euclidianidade    \\ \hline
        \end{tabular}
        \caption{Sample Table}
        \label{tab:sample_table}
    \end{table}

    \begin{definition}[Tradução] 
        Uma sentença $\varphi$ de um sistema $\mathbf{A} = \langle \mathcal{L}_\mathbf{A}, \vdash_\mathbf{A}\rangle$ pode ser traduzida a uma sentença $\varphi^*$ em um sistema $\mathbf{B} = \langle \mathcal{L}_\mathbf{B}, \vdash_\mathbf{B} \rangle$ caso exista uma função $\bullet^* : \mathcal{L}_\mathbf{A} \to \mathcal{L}_\mathbf{B}$ que garanta que $\Gamma \vdash_\mathbf{A} \varphi \Leftrightarrow \Gamma^* \vdash_\mathbf{B} \varphi^*$.
        \qed
    \end{definition}

    A primeira tradução do sistema intuicionista ao sistema modal foi definida por Gödel \cite{Gödel} motivado pela possibilidade de leitura da necessidade como uma modalidade de construtividade. Ou seja, por meio dessa tradução, a sentença $\nec \varphi$ poderia ser lida como \textit{$\varphi$ pode ser provada construtivamente} \cite{Troelstra}. Gödel demonstrou grande crença acerca da corretude fraca -- $ \varnothing \vdash_\mathbf{I} \varphi \Rightarrow \varnothing \vdash_\mathbf{M} \varphi^\medsquare $ -- dessa tradução.

    \begin{definition}[$\bullet^\medsquare$] Define-se a tradução $\bullet^\medsquare$ indutivamente da seguinte maneira:
        \begin{align*}
            p^\medsquare                     & \coloneqq \nec p                                           \\
            \bot^\medsquare                  & \coloneqq \nec \bot                                        \\
            (\varphi \wedge \psi)^\medsquare & \coloneqq \nec (\varphi^\medsquare \wedge \psi^\medsquare) \\
            (\varphi \vee \psi)^\medsquare   & \coloneqq \nec (\varphi^\medsquare \vee \psi^\medsquare)   \\
            (\varphi \to \psi)^\medsquare    & \coloneqq \nec (\varphi^\medsquare \to \psi^\medsquare)    \\
            (\exists x. \varphi)^\medsquare  & \coloneqq \nec (\exists x. \varphi^\medsquare)             \\
            (\forall x. \varphi)^\medsquare  & \coloneqq \nec (\forall x. \varphi^\medsquare)  
        \end{align*}
        \qed
    \end{definition}

    \begin{definition}[$\bullet^\circ$] Define-se a tradução $\bullet^\circ$ indutivamente da seguinte maneira:
        \begin{align*}
            p^\circ                     & \coloneqq p                                       \\
            \bot^\circ                  & \coloneqq \bot                                    \\
            (\varphi \wedge \psi)^\circ & \coloneqq \varphi^\circ \wedge \psi^\circ         \\
            (\varphi \vee \psi)^\circ   & \coloneqq \nec \varphi^\circ \vee \nec \psi^\circ \\
            (\varphi \to \psi)^\circ    & \coloneqq \nec \varphi^\circ \to \psi^\circ       \\
            (\exists x. \varphi)^\circ  & \coloneqq \exists x. \nec \varphi^\circ           \\
            (\forall x. \varphi)^\circ  & \coloneqq \forall x. \varphi^\circ
        \end{align*}
        \qed
    \end{definition}

    \begin{definition}[$\bullet^\medsquare$] Define-se a tradução $\bullet^\medsquare$ indutivamente da seguinte maneira:
        \begin{align*}
            p^\medsquare                     & \coloneqq \nec p                                        \\
            \bot^\medsquare                  & \coloneqq \bot                                          \\
            (\varphi \wedge \psi)^\medsquare & \coloneqq \varphi^\medsquare \wedge \psi^\medsquare     \\
            (\varphi \vee \psi)^\medsquare   & \coloneqq \varphi^\medsquare \vee \psi^\medsquare       \\
            (\varphi \to \psi)^\medsquare    & \coloneqq \nec (\varphi^\medsquare \to \psi^\medsquare) \\
            (\exists x. \varphi)^\medsquare  & \coloneqq \exists x. \varphi^\medsquare                 \\
            (\forall x. \varphi)^\medsquare  & \coloneqq \nec \forall x. \varphi^\medsquare
        \end{align*}
        \qed
    \end{definition}

    Faz-se interessante pontuar que as traduções $\bullet^\circ$ e $\bullet^\medsquare$ correspondem, respectivamente, às traduções $\bullet^0$ e $\bullet^*$ do sistema intuicionista ao sistema linear providas por \cite{Girard}, bastando trocar o modalidade $\nec$ pelo expoente $!$ e os conectivos intuicionistas pelas suas contrapartes lineares.

    \bibliographystyle{plain}
    \bibliography{bibliography}
\end{document}