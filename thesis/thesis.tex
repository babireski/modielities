\documentclass{report}

\usepackage{amsthm}
\usepackage{amsmath}
\usepackage{amssymb}
\usepackage{epigraph}
\usepackage{mathtools}
\usepackage{modalities}
\usepackage{ragged2e}
\usepackage{hyperref}
\usepackage{proof}
\usepackage[brazil]{babel}
\usepackage{fitch}
\usepackage{cases}
\usepackage{functions}
% \usepackage{natbib}


\newcommand\mdoubleplus{\mathbin{+\mkern-10mu+}}

\newtheorem{notation}{Notação}
\newtheorem{definition}{Definição}
\newtheorem{lemma}{Lema}
\newtheorem{theorem}{Teorema}

\begin{document}
    \title{Uma formalização da interpretação modal do sistema intuicionista}
    \author{Elian Babireski}
    \date{2024}

    \maketitle

    \begin{abstract}
        Resumo aqui.
    \end{abstract}

    \tableofcontents

    \setlength\epigraphwidth{.5\textwidth}
\setlength\epigraphrule{0pt}

\vspace*{\fill}
\epigraph{\justifying \itshape ``Oh, you can't help that,'' said the Cat: ``we're all mad here. I'm mad. You're mad.'' ``How do you know I'm mad?'' said Alice. ``You must be,'' said the Cat, ``or you wouldn't have come here.''}{---Lewis Carroll, \textit{Alice in Wonderland}}
    \chapter{Introdução}

As lógicas modais consistem em um conjunto de extensões da lógica clássica que contam com a adição de um ou mais operadores, chamados modalidades, que qualificam sentenças. No caso do sistema \textbf{S4}, são adicionadas as modalidades de necessidade ($\nec$) e possibilidade ($\pos$) em conjunto à regra da necessitação\footnote{Se $\vdash A$ então $\vdash \nec A$} e os axiomas $\mathbf{K}\text{: } \nec(A \to B) \to \nec A \to \nec B$, $\mathbf{T}\text{: } \nec A \to A$ e $\text{\textbf{4}: } \nec A \to \nec \nec A$ \citep{Troelstra}. Ademais, pode-se derivar nesse sistema, por meio da dualidade entre as modalidades\footnote{$\pos A \equiv \neg \nec \neg A $}, sentenças duais aos axiomas \textbf{T} e \textbf{4}, sendo elas $\mathbf{T}_\meddiamond \text{: } A \to \pos A$ e $\mathbf{4}_\meddiamond \text{: } \pos \pos A \to \pos A$, respectivamente~\cite{Zach}.

As mônadas ganharam destaque na área de linguagens de programação desde que~\cite{Moggi} formalizou uma metalinguagem que faz uso dessas estruturas para modelar noções de computação --- como parcialidade, não-determinismo, exceções e continuações --- de uma maneira puramente funcional. Pode-se notar uma grande semelhança entre as sentenças $\mathbf{T}_\meddiamond$ e $\mathbf{4}_\meddiamond$ e as transformações naturais monádicas $\mathbf{\eta:} 1_C \to T$ e $\mathbf{\mu:} T^2 \to T$, respectivamente. Nesse sentido,~\cite{Pfenning} demonstraram que se pode traduzir essa metalinguagem para o sistema \textbf{S4} da lógica modal, pelo qual se torna interessante analisar esse sistema como uma linguagem de programação sob a ótica do isomorfismo de Curry-Howard.

~\cite{Troelstra} apresentam duas traduções equivalentes da lógica intuicionista para o sistema \textbf{S4} da lógica modal, sendo um deles correspondente a uma abordagem \textit{call-by-name} e outra a um abordagem \textit{call-by-value}. Tais traduções possuem grande similaridade com as traduções da lógica intuicionista para a lógica linear definidas por~\cite{Girard}. Essas traduções equivalem à tradução por negação dupla que, por sua vez, equivalem a traduções \textit{continuation-passing style} (CPS) em compiladores por meio do isomorfismo de Curry-Howard~\citep{Reynolds}, o que torna esse tema interessante no ponto de vista de compilação.

Durante grande parte da história, provas lógicas e matemáticas eram validadas manualmente pela comunidade acadêmica, o que muitas vezes --- a depender do tamanho e complexidade da prova --- se mostrava ser um trabalho complexo e sujeito a erros. Hoje em dia, exitem \textit{softwares} chamados assistentes de provas que permitem verificar --- graças ao isomorfismo de Curry-Howard --- a corretude de provas~\citep{Chlipala}. O assistente de provas que será usado neste trabalho é o \textsc{coq}, que utiliza o cálculo de construções indutivas e um conjunto axiomático pequeno para permitir a escrita de provas simples e intuitivas~\citep{Coq}.

    \section{Objetivos}
    Este trabalho consiste numa continuação do desenvolvimento da biblioteca de formalização de sistemas modais normais iniciado por~\cite{Silveira} e posteriormente expandida de forma a permitir a fusão de sistemas modais por~\cite{Nunes}. Nele, formalizaremos as traduções do sistema intuicionista ao sistema modal $\mathbf{S4}$ no asssitente de provas \textsc{coq} e provaremos suas propriedades. Uma formalização de traduções entre sistemas de dedução similar a nossa foi feita por~\cite{Sehnem}, neste caso tendo como alvo o sistema linear de~\cite{Girard}. Todas as formalizações citadas acima deram-se no assistente de provas \textsc{coq}, o mesmo assistente usado neste trabalho. Como objetivos específicos, listamos:

    \begin{itemize}
        \item Fornecer uma introdução ao conceito de sistemas de dedução;
        \item Fornecer uma introdução ao conceito de traduções entre sistemas;
        \item Fornecer uma introdução ao sistema intuicionista;
        \item Fornecer uma introdução aos sistemas modais, em especial o $\mathbf{S4}$;
        \item Apresentar as traduções do sistema intuicionista ao sistema $\mathbf{S4}$;
        \item Provar manualmente a correção das traduções providas bem como outras propriedades pertinentes;
        \item Formalizar as provas no provador de teoremas interativo \textsc{coq}.
    \end{itemize}

    \section{Estruturação}
    Estruturaremos este trabalho em cinco partes. A parte \textbf{(1)} trata-se desta introdução. A parte \textbf{(2)} consiste numa fundamentação de conceitos basilares ao desenvolvimento deste trabalho, notadamente os conceitos de \emph{sistemas de dedução}, \emph{traduções} e \emph{provadores de teoremas}. A parte \textbf{(3)} apresenta as definições dos sistemas e traduções relevantes a este trabalho. Na parte \textbf{(4)} são provadas todas as propriedades abarcadas no escopo deste trabalho. Por fim, a parte \textbf{(5)} compreende considerações parciais acerca do desenvolvido até o momento.

    \chapter{Fundamentação}

    \section{Sistema}

    \begin{notation}
        A marcação $\bullet$ consiste em um marcador de posição, ou seja, pode ser trocado por qualquer valor. 
    \end{notation}

    Para este trabalho, a definição de sistema adotada será uma especialização daquela provida por Béziau \cite{Beziau}.

    \begin{definition}[Sistema]
        Um sistema consiste num par $\mathbf{L} = \sequence{\mathcal{L}, \vdash}$, onde $\mathcal{L}$ consiste em um conjunto de sentenças bem-formadas e $\, \vdash \: : \wp(\mathcal{L}) \times \mathcal{L} $ em uma relação de dedução, sem demais condições.
        \qed
    \end{definition}

    Definiremos a noção de profundidade de uma sentença para que possamos realizar indução na profundidade da sentença, conforme Troelstra \cite{Troelstra}.

    \begin{definition}[Profundidade]
        A profundidade $|\alpha|$ de uma sentença $\alpha$ consiste no comprimento do maior ramo de sua árvore de construção. Seja $\circ$ um operador qualquer, define-se a profundidade recursivamente como:
        \begin{align*}
            |p|                  & \coloneqq 0                           \\
            |\bot|               & \coloneqq 0                           \\
            |\circ \alpha|       & \coloneqq |\alpha| + 1                \\
            |\alpha \circ \beta| & \coloneqq \max(|\alpha|, |\beta|) + 1
            \tag*{\qed} 
        \end{align*}
    \end{definition}

    \begin{definition}[Esquema]
        Definição.
    \end{definition}

    % \begin{definition}[Substituição]
    %     Uma substituição consiste em uma função $\sigma : \mathcal{P} \to \mathcal{L}$ que mapeia proposições em sentenças. A aplicação de $\sigma$ em uma sentença $\varphi \in \mathcal{L}$, denotada $\varphi[\sigma]$, define-se recursivamente como a aplicação de $\sigma$ a cada proposição $p \in \mathcal{P}$ em $\varphi$.
    %     \begin{align*}
    %         p[p \mapsto \beta]                 & \coloneqq \beta                                                  \\
    %         q[p \mapsto \beta]                 & \coloneqq q                                                      \\
    %         \bot[p \mapsto \beta]              & \coloneqq \bot                                                   \\
    %         \circ\alpha[p \mapsto \beta]       & \coloneqq \circ(\alpha[p \mapsto \beta])                         \\
    %         \alpha\circ\beta[p \mapsto \beta]  & \coloneqq (\alpha[p \mapsto \beta])\circ(\beta[p \mapsto \beta])
    %         \tag*{\qed} 
    %     \end{align*}
    % \end{definition}    

    \begin{definition}[$\mathcal{L}_\mathbf{I}$]
        Define-se a linguagem do sistema intuicionista, denotada $\mathcal{L}_\mathbf{I}$, como o menor conjunto induzido a partir das seguintes regras:
    \end{definition}

    \begin{definition}[$\vdash_\mathbf{I}$]
        Define-se a relação de dedução do sistema intuicionista, denotado $\vdash_\mathbf{I}$.
    \end{definition}


    \begin{definition}[$\mathcal{L}_\mathbf{M}$]
        Define-se a linguagem dos sistemas modais, denotada $\mathcal{L}_\mathbf{M}$, como o menor conjunto induzido a partir das seguintes regras:
        
        $\top, \bot \in \mathcal{L}_\mathbf{I}$
        
        $\mathcal{P} \subseteq \mathcal{L}_\mathbf{I}$
        
        $\varphi \in \mathcal{L}_\mathbf{I} \Rightarrow \circ \varphi \in \mathcal{L}_\mathbf{I}$, para $\circ \in \{\nec, \pos, \neg\}$

        $\varphi, \psi \in \mathcal{L}_\mathbf{I} \Rightarrow \varphi \circ \psi \in \mathcal{L}_\mathbf{I}$, para $\circ \in \{\wedge, \vee, \to\}$
    \end{definition}

    \begin{definition}[Dedução]
        Uma dedução para uma linguagem $\mathcal{L}$ consiste em um par composto por um conjunto finito $\{\varphi_1, \cdots, \varphi_n\} \subseteq \mathcal{L}$, chamado de \emph{premissas}, e uma sentença $\varphi \in \mathcal{L}$, chamada de \emph{conclusão}, que pode ser notada da seguinte forma: $$\infer[.]{\varphi}{\varphi_1 \cdots \varphi_n}$$
    \end{definition}

    \begin{definition}[Sistema de Hilbert]
        Um sistema de Hilbert para um sistema $\mathbf{L} = \sequence{\mathcal{L}, \vdash}$ consiste em um par $\mathcal{H} = \sequence{\mathcal{A}, \mathcal{R}}$, sendo $\mathcal{A}$ um conjunto de axiomas e $\mathcal{R}$ um conjunto de regras de dedução. Uma sucessão $\sequence{\varphi_n | n < k}$, onde cada sentença $\varphi_n$ trata-se de um axioma $\alpha \in \mathcal{A}$, uma assunção $\gamma \in \Gamma$ ou sentenças geradas pela aplicação de regras de dedução $\rho \in \mathcal{R}$ a sentenças anteriores, consiste em uma prova de $\Gamma \vdash \varphi_n$.
    \end{definition}

    \section{Tradução}

    Traduções entre sistemas consistem em funções que mapeiam sentenças de um sistema a sentenças de outro sistema e garantem certas propriedades. As propriedades a serem garantidas variam e ainda são discutidas na literatura, deixando a definição exata de tradução -- assim como houve com a definição de sistema -- varie de acordo com a predileção de cada autor. Nesta seção, serão abordadas historicamente noções de tradução entre sistemas, bem como serão definidos e nomeados os conceitos de tradução que serão usados no restante deste trabalho.

    \begin{definition}[Tradução] 
        Uma sentença $\varphi$ de um sistema $\mathbf{A} = \langle \mathcal{L}_\mathbf{A}, \vdash_\mathbf{A}\rangle$ pode ser traduzida a uma sentença $\varphi^*$ em um sistema $\mathbf{B} = \langle \mathcal{L}_\mathbf{B}, \vdash_\mathbf{B} \rangle$ caso exista uma função $\bullet^* : \mathcal{L}_\mathbf{A} \to \mathcal{L}_\mathbf{B}$ que garanta que $\Gamma \vdash_\mathbf{A} \varphi \Leftrightarrow \Gamma^* \vdash_\mathbf{B} \varphi^*$.
        \qed
    \end{definition}

    \begin{definition}[$\bullet^\neg$] Define-se a tradução $\bullet^\neg$ indutivamente da seguinte maneira:
        \begin{align*}
            p^\neg                     & \coloneqq \neg\neg p                               \\
            \bot^\neg                  & \coloneqq \bot                                     \\
            (\varphi \wedge \psi)^\neg & \coloneqq \neg\neg (\varphi^\neg \wedge \psi^\neg) \\
            (\varphi \vee \psi)^\neg   & \coloneqq \neg\neg (\varphi^\neg \vee \psi^\neg)   \\
            (\varphi \to \psi)^\neg    & \coloneqq \neg\neg (\varphi^\neg \to \psi^\neg)
            \tag*{\qed} 
        \end{align*}
    \end{definition}

    A primeira tradução do sistema intuicionista ao sistema modal foi proposta por Gödel \cite{Goedel} motivado pela possibilidade de leitura da necessidade como uma modalidade de construtividade. Ou seja, por meio dessa tradução, a sentença $\nec \varphi$ poderia ser lida como \textit{$\varphi$ pode ser provada construtivamente} \cite{Troelstra}. Gödel conjeiturou a corretude fraca dessa tradução, que foi posteriormente provada por McKinsey e Tarski \cite{McKinsey} em conjunto com sua completude fraca.

    \begin{definition}[$\bullet^\circ$] Define-se a tradução $\bullet^\circ$ indutivamente da seguinte maneira:
        \begin{align*}
            p^\circ                     & \coloneqq p                                       \\
            \bot^\circ                  & \coloneqq \bot                                    \\
            (\varphi \wedge \psi)^\circ & \coloneqq \varphi^\circ \wedge \psi^\circ         \\
            (\varphi \vee \psi)^\circ   & \coloneqq \nec \varphi^\circ \vee \nec \psi^\circ \\
            (\varphi \to \psi)^\circ    & \coloneqq \nec \varphi^\circ \to \psi^\circ
            \tag*{\qed} 
        \end{align*}
    \end{definition}

    \begin{definition}[$\bullet^\medsquare$] Define-se a tradução $\bullet^\medsquare$ indutivamente da seguinte maneira:
        \begin{align*}
            p^\medsquare                     & \coloneqq \nec p                                        \\
            \bot^\medsquare                  & \coloneqq \bot                                          \\
            (\varphi \wedge \psi)^\medsquare & \coloneqq \varphi^\medsquare \wedge \psi^\medsquare     \\
            (\varphi \vee \psi)^\medsquare   & \coloneqq \varphi^\medsquare \vee \psi^\medsquare       \\
            (\varphi \to \psi)^\medsquare    & \coloneqq \nec (\varphi^\medsquare \to \psi^\medsquare)
            \tag*{\qed} 
        \end{align*}
    \end{definition}

    Faz-se interessante pontuar que as traduções $\bullet^\circ$ e $\bullet^\medsquare$ correspondem, respectivamente, às traduções $\bullet^\circ$ e $\bullet^*$ do sistema intuicionista ao sistema linear providas por Girard \cite{Girard}, sendo as primeiras correspondentes a uma ordem de avaliação por nome (\textit{call-by-name}) e as segundas a uma ordem de avaliação por valor (\textit{call-by-value}). 

    \begin{lemma}
        $\forall \alpha \in \mathcal{L}_\mathbf{I} . \nec\alpha^\circ \leftrightarrow \alpha^\medsquare$.
    \end{lemma}

    \begin{proof}
        Prova por indução na profundidade de $\alpha$.
    \end{proof}

    \begin{lemma}
        $\forall \alpha \in \mathcal{L}_\mathbf{I} . \nec\alpha^\medsquare \leftrightarrow \alpha^\medsquare$.
    \end{lemma}

    \begin{proof}
        A volta $\nec\alpha^\medsquare\leftarrow\alpha^\medsquare$ pode ser provada trivialmente por meio da regra da necessitação. A ida $\nec\alpha^\medsquare\rightarrow\alpha^\medsquare$ deve ser provada por indução na profundidade de $\alpha$.
    \end{proof}

    \begin{theorem}
        $\Gamma\vdash_\mathbf{M}\alpha\to\beta\Leftrightarrow\Gamma\cup\{\alpha\}\vdash_\mathbf{M}\beta$.
    \end{theorem}

    \begin{proof}
        Prova por indução no tamanho da prova.
    \end{proof}

    \begin{theorem}
        $\forall \alpha \in \mathcal{L}_\mathbf{I} . \Gamma \vdash_\mathbf{I} \alpha \Rightarrow \Gamma^\medsquare \vdash_\mathbf{M} \alpha^\medsquare$
    \end{theorem}

    \begin{proof}
        Como $\Gamma \vdash_\mathbf{I} \alpha$, sabe-se que existe uma prova $\sequence{\varphi_i | 1 \leq i \leq n}$ tal que $\varphi_n = \alpha$. A demonstração deste teorema será feita por indução no tamanho $n$ da prova.

        \begin{case}
                \textbf{Passo} ($n = 1$)\textbf{.} A prova, caso possua tamanho $n = 1$, tem obrigatoriamente a forma $\sequence{\alpha}$. Deste modo, existem duas casos a serem considerados: $\alpha$ ser um axioma ou $\alpha$ ser uma premissa.
                
                \begin{case}
                    \textbf{Caso 1} ($\alpha\in\Gamma$)\textbf{.} Como $\alpha\in\Gamma$, sabe-se que $\alpha^\medsquare\in\Gamma^\medsquare$, uma vez que $\Gamma^\medsquare = \set{\varphi^\medsquare | \varphi\in\Gamma}$. Desta forma, $\sequence{\alpha^\medsquare}$ constitui uma prova para $\Gamma^\medsquare\vdash\alpha^\medsquare$.
                \end{case}

                \begin{case}
                    \textbf{Caso 2} ($\alpha\in\mathcal{A}$)\textbf{.}

                    \begin{case}
                        \textbf{Caso 2.1} ($\mathbf{A_1}$)\textbf{.}

                        \begin{fitch}
                            \fa \alpha^\medsquare \to \beta^\medsquare \to \alpha^\medsquare & $\mathbf{A_1}$ \\
                            \fa \nec(\alpha^\medsquare \to \beta^\medsquare \to \alpha^\medsquare) & $\mathbf{R_2} \; 1$ \\
                            \fa \nec(\alpha^\medsquare \to \beta^\medsquare \to \alpha^\medsquare) \to \nec \alpha^\medsquare \to \nec (\beta^\medsquare \to \alpha^\medsquare) & $\mathbf{A_{11}}$ \\
                            \fa \nec \alpha^\medsquare \to \nec (\beta^\medsquare \to \alpha^\medsquare) & $\mathbf{R_1} \; 2 \; 3$ \\
                            \fa \nec(\nec \alpha^\medsquare \to \nec (\beta^\medsquare \to \alpha^\medsquare)) & $\mathbf{R_2} \; 4$
                        \end{fitch}
                    \end{case}

                    \begin{case}
                        \textbf{Caso 2.2} ($\mathbf{A_2}$)\textbf{.}
                    \end{case}

                    \begin{case}
                        \textbf{Caso 2.3} ($\mathbf{A_3}$)\textbf{.}

                        \begin{fitch}
                            \fa \alpha^\medsquare \to \beta^\medsquare \to \alpha^\medsquare \wedge \beta^\medsquare & $\mathbf{A_4}$ \\
                            \fa \nec(\alpha^\medsquare \to \beta^\medsquare \to \alpha^\medsquare \wedge \beta^\medsquare) & $\mathbf{R_2} \; 1$ \\
                            \fa \nec(\alpha^\medsquare \to \beta^\medsquare \to \alpha^\medsquare \wedge \beta^\medsquare) \to \nec (\alpha^\medsquare \to \beta^\medsquare) \to \nec (\alpha^\medsquare \wedge \beta^\medsquare) & $\mathbf{A_{11}}$ \\
                            \fa \nec (\alpha^\medsquare \to \beta^\medsquare) \to \nec (\alpha^\medsquare \wedge \beta^\medsquare) & $\mathbf{R_1} \; 2 \; 3$ \\
                            \fa \nec (\nec (\alpha^\medsquare \to \beta^\medsquare) \to \nec (\alpha^\medsquare \wedge \beta^\medsquare)) & $\mathbf{R_2} \; 4$
                        \end{fitch} 
                    \end{case}

                    \begin{case}
                        \textbf{Caso 2.4} ($\mathbf{A_4}$)\textbf{.}

                        \begin{fitch}
                            \fa \alpha^\medsquare\wedge\beta^\medsquare\to\alpha^\medsquare & $\mathbf{A_4}$ \\
                            \fa \nec(\alpha^\medsquare \wedge \beta^\medsquare \to \alpha^\medsquare) & $\mathbf{R_2} \; 1$
                        \end{fitch}
                    \end{case}

                    \begin{case}
                        \textbf{Caso 2.5} ($\mathbf{A_5}$)\textbf{.}

                        \begin{fitch}
                            \fa \alpha^\medsquare\wedge\beta^\medsquare\to\beta^\medsquare & $\mathbf{A_5}$ \\
                            \fa \nec(\alpha^\medsquare \wedge \beta^\medsquare \to \beta^\medsquare) & $\mathbf{R_2} \; 1$
                        \end{fitch}
                    \end{case}

                    \begin{case}
                        \textbf{Caso 2.6} ($\mathbf{A_6}$)\textbf{.}

                        \begin{fitch}
                            \fa \alpha^\medsquare\to\alpha^\medsquare\vee\beta^\medsquare & $\mathbf{A_7}$ \\
                            \fa \nec(\alpha^\medsquare\to\alpha^\medsquare\vee\beta^\medsquare) & $\mathbf{R_2} \; 1$
                        \end{fitch}
                    \end{case}

                    \begin{case}
                        \textbf{Caso 2.7} ($\mathbf{A_7}$)\textbf{.}

                        \begin{fitch}
                            \fa \beta^\medsquare\to\alpha^\medsquare\vee\beta^\medsquare & $\mathbf{A_8}$ \\
                            \fa \nec(\beta^\medsquare\to\alpha^\medsquare\vee\beta^\medsquare) & $\mathbf{R_2}  \; 1$
                        \end{fitch}
                    \end{case}

                    \begin{case}
                        \textbf{Caso 2.8} ($\mathbf{A_8}$)\textbf{.}
                    \end{case}

                    \begin{case}
                        \textbf{Caso 2.9} ($\mathbf{A_9}$)\textbf{.}
                    \end{case}
                \end{case}
        \end{case}
    \end{proof}

    % \begin{fitch}
    %     \fa (\alpha\to\beta \to \gamma) \to (\alpha \to \beta) \to \alpha \to \gamma \\
    %     \fa (\alpha^\medsquare \to \beta^\medsquare \to \gamma^\medsquare) \to (\alpha^\medsquare \to \beta^\medsquare) \to \alpha^\medsquare \to \gamma^\medsquare \\
    %     \fa \\
    %     \fa \nec(\nec(\alpha^\medsquare\to\nec(\beta^\medsquare\to\gamma^\medsquare))\to\nec(\nec(\alpha^\medsquare \to \beta^\medsquare)\to\nec(\alpha^\medsquare \to\gamma^\medsquare))) \\
    % \end{fitch}

        % $\mathbf{A^\vee_3}$

        % $            
        %     \begin{nd}
        %         \have{1}{(\alpha\to\gamma)\to(\beta\to\gamma)\to\alpha\vee\beta\to\gamma}
        %         \have{2}{(\alpha^\medsquare\to\gamma^\medsquare)\to(\beta^\medsquare\to\gamma^\medsquare)\to\alpha^\medsquare\vee\beta^\medsquare\to\gamma^\medsquare}
        %         \have{3}{}
        %         \have{n}{\nec(\nec(\alpha^\medsquare\to\gamma^\medsquare)\to\nec(\nec(\beta^\medsquare\to\gamma^\medsquare)\to\nec(\alpha^\medsquare\vee\beta^\medsquare\to\gamma^\medsquare)))}
        %     \end{nd}
        % $

        % $\mathbf{A^\bot_1}$
        
        % $
        %     \begin{nd}
        %         \have{1}{}
        %         \have{n}{\nec(\bot\to\alpha^\medsquare)}
        %     \end{nd}
        % $

    \begin{definition}[$\mathcal{H}_\mathbf{I}$]
        A axiomatização do sistema intuicionista consiste num par $\mathcal{H}_\mathbf{I} = \langle\mathcal{A}_\mathbf{I}, \mathcal{R}_\mathbf{I}\rangle$, onde $\mathcal{A}_\mathbf{I} = \{\mathbf{A_1}, \mathbf{A_2}, \mathbf{A_3}, \mathbf{A_4}, \mathbf{A_5}, \mathbf{A_6}, \mathbf{A_7}, \mathbf{A_8}, \mathbf{A_9}\}$ e $\mathcal{R}_\mathbf{I} = \{\mathbf{R_1}\}$, conforme as definições abaixo:

        \begin{align*}
            & \mathbf{A_1}\quad & \alpha\to\beta\to\alpha \\
            & \mathbf{A_2}\quad & (\alpha\to\beta\to\gamma)\to(\alpha\to\beta)\to(\alpha\to\gamma) \\
            & \mathbf{A_3}\quad & \alpha\to\beta\to\alpha\wedge\beta \\
            & \mathbf{A_4}\quad & \alpha\wedge\beta\to\alpha \\
            & \mathbf{A_5}\quad & \alpha\wedge\beta\to\beta \\
            & \mathbf{A_6}\quad & \alpha\to\alpha\vee\beta \\
            & \mathbf{A_7}\quad & \beta\to\alpha\vee\beta \\
            & \mathbf{A_8}\quad & (\alpha\to\gamma)\to(\beta\to\gamma)\to(\alpha\vee\beta\to\gamma) \\
            & \mathbf{A_9}\quad & \bot\to\alpha \\
            & \mathbf{R_1}\quad & \text{Se }\vdash\alpha\text{ e }\vdash\alpha\to\beta\text{, então }\vdash\beta
            \tag*{\qed}
        \end{align*}   
    \end{definition}

    \begin{definition}[$\mathcal{H}_\mathbf{M}$]
        A axiomatização do sistema modal consiste num par $\mathcal{H}_\mathbf{M} = \langle\mathcal{A}_\mathbf{M}, \mathcal{R}_\mathbf{M}\rangle$, onde $\mathcal{A}_\mathbf{M} = \set{\mathbf{A_1}, \mathbf{A_2}, \mathbf{A_3}, \mathbf{A_4}, \mathbf{A_5}, \mathbf{A_6}, \mathbf{A_7}, \mathbf{A_8}, \mathbf{A_9}, \mathbf{A_{10}}, \mathbf{A_{11}}, \mathbf{A_{12}}, \mathbf{A_{13}}, \mathbf{A_{14}}}$ e $\mathcal{R}_\mathbf{I} = \set{\mathbf{R_1}, \mathbf{R_2}}$, conforme as definições abaixo:
        \begin{align*}
            & \mathbf{A_1}\quad & \alpha\to\beta\to\alpha \\
            & \mathbf{A_2}\quad & (\alpha\to\beta\to\gamma)\to(\alpha\to\beta)\to(\alpha\to\gamma) \\
            & \mathbf{A_3}\quad & (\neg\alpha\to\neg\beta)\to\alpha\to\beta \\
            & \mathbf{A_4}\quad & \alpha\to\beta\to\alpha\wedge\beta \\
            & \mathbf{A_5}\quad & \alpha\wedge\beta\to\alpha \\
            & \mathbf{A_6}\quad & \alpha\wedge\beta\to\beta \\
            & \mathbf{A_7}\quad & \alpha\to\alpha\vee\beta \\
            & \mathbf{A_8}\quad & \beta\to\alpha\vee\beta \\
            & \mathbf{A_9}\quad & (\alpha\to\gamma)\to(\beta\to\gamma)\to(\alpha\vee\beta\to\gamma) \\
            & \mathbf{A_{10}}\quad & \neg\neg\alpha\to\alpha \\
            & \mathbf{A_{11}}\quad & \nec(\alpha\to\beta)\to\nec\alpha\to\nec\beta \\
            & \mathbf{A_{12}}\quad & \nec\alpha\to\alpha \\
            & \mathbf{A_{13}}\quad & \nec\alpha\to\nec\nec\alpha \\
            & \mathbf{A_{14}}\quad & \pos(\alpha\vee\beta)\to\pos\alpha\vee\pos\beta \\
            & \mathbf{R_1}\quad & \text{Se }\vdash\alpha\text{ e }\vdash\alpha\to\beta\text{, então }\vdash\beta \\
            & \mathbf{R_2}\quad & \text{Se }\vdash\alpha\text{, então }\vdash\nec\alpha
            \tag*{\qed} 
        \end{align*}   
    \end{definition}

    \begin{definition}[Sucessão]
        Uma sucessão, denotada por $\sequence{\bullet}$, consiste em uma coleção ordenada de elementos que permite repetição. A notação $\sequence{a_n | 1 \leq n \leq k}$ denota uma sucessão iniciada em $a_1$ e terminada em $a_k$. 
    \end{definition}

    \begin{definition}[Concatenação]
        A concatenação de duas sucessões $\mathbf{A}$ e $\mathbf{B}$, denotada $\mathbf{A} \mdoubleplus \mathbf{B}$ consiste em uma nova sucessão resultante da justaposição dos elementos de $\mathbf{A}$ sucedidos pelos elementos de $\mathbf{B}$, mantendo a ordem original de ambos.
    \end{definition}

    \chapter{Formalização}
    \chapter{Implementação}
    \chapter{Conclusão}

    \bibliographystyle{plain}
    \bibliography{bibliography}
\end{document}
