\title{
Plano de Trabalho de Conclusão de Curso\\
Uma formalização da tradução da lógica intuicionista para a lógica modal S4
}

\author{
Elian Gustavo Chorny Babireski -- \texttt{elian.babireski@gmail.com}\\
Karina Girardi Roggia -- \texttt{karina.roggia@udesc.br} {(\textit{orientadora})}\\
Paulo Henrique Torrens -- \texttt{paulotorrens@gnu.org} {(\textit{coorientador})}\\
~\\
Turma 2024/2 -- Joinville/SC
}

\date{23 de agosto de 2024}

\maketitle

%\singlespacing  %espaçamento simples
\onehalfspacing  %espaçamento de 1,5
%\doublespacing  %espaçamento duplo

\begin{abstract}
As lógicas modais consistem em um conjunto de extensões da lógica clássica que contam com a adição de um ou mais operadores, chamados modalidades, que qualificam sentenças. Uma lógica modal com particular interesse à computação é o sistema \textbf{S4}, uma vez que a metalinguagem de Moggi que modela noções de computação em linguagens de programação por meio de mônadas pode ser traduzida a esse sistema. Ademais, existem correspondências entre a tradução da lógica intuicionista ao sistema modal \textbf{S4} com traduções \textit{continuation-passing style} (CPS) usadas em compiladores. Este trabalho busca formalizar a derivação das sentenças $\boldsymbol{\text{\textbf{T}}_\Diamond}$ e $\boldsymbol{\text{\textbf{4}}_\Diamond}$ no sistema \textbf{S4} -- uma vez que estas correspondem às transformações naturais monádicas --, bem como formalizar duas traduções da lógica intuicionista para o sistema \textbf{S4} da lógica modal e demonstrar a equivalência entre elas. Todas as fomalizações serão feitas no assistente de provas Coq. \\\\
\textbf{Palavras-chave}: Coq, lógica intuicionista, lógica modal, S4, tradução de lógicas.
\end{abstract}