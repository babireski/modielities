\documentclass[12pt]{report}

\usepackage{amsthm}
\usepackage{amsmath}
\usepackage{amssymb}
\usepackage{epigraph}
\usepackage{mathtools}
\usepackage{styles/modalities}
\usepackage{ragged2e}
\usepackage{hyperref}
\usepackage{proof}
\usepackage[round]{natbib}
\usepackage[french, brazil]{babel}
\usepackage{styles/fitch}
\usepackage{styles/cases}
\usepackage{styles/functions}
\usepackage{lipsum}
\usepackage{enumitem} % Para mudar a itemização
\usepackage{pxfonts}
\usepackage{MnSymbol}

\usepackage[T1]{fontenc}

\usepackage[a4paper, margin=1.2in]{geometry}

\hypersetup{
    pdftitle={Uma formalização da interpretação modal do sistema intuicionista},
    pdfauthor={Elian Babireski},
    colorlinks=true,
    linkcolor=blue,
    citecolor=blue,
    filecolor=blue,
    urlcolor=blue
}

\usepackage[
    left = \flqq{},% 
    right = \frqq{},% 
    leftsub = \flq{},% 
    rightsub = \frq{} %
]{dirtytalk}

\newcommand{\entails}{\mathrel{\vdash}}
\newcommand{\point}{\mathpunct{.}}

\newtheorem*{notation}{Notação}
\newtheorem{definition}{Definição}
\newtheorem{lemma}{Lema}
\newtheorem{theorem}{Teorema}

\usepackage{titlesec}

\titleformat{\chapter}[block]
  {\normalfont\huge\bfseries}{\thechapter.}{1em}{\Huge}
\titlespacing*{\chapter}{0pt}{-19pt}{20pt}

\usepackage{setspace}
\onehalfspacing{}

\begin{document}
    \title{Uma formalização da interpretação modal do sistema intuicionista}
    \author{Elian Babireski}
    \date{2024}

    \maketitle

    \begin{abstract}
        Resumo aqui.
    \end{abstract}

    \tableofcontents

    \setlength\epigraphwidth{.5\textwidth}
\setlength\epigraphrule{0pt}

\vspace*{\fill}
\epigraph{\justifying\itshape``Oh, you can't help that,'' said the Cat: ``we're all mad here. I'm mad. You're mad.'' ``How do you know I'm mad?'' said Alice. ``You must be,'' said the Cat, ``or you wouldn't have come here.''}{---Lewis Carroll, \textit{Alice in Wonderland}}
    \chapter{Introdução}

As lógicas modais consistem em um conjunto de extensões da lógica clássica que contam com a adição de um ou mais operadores, chamados modalidades, que qualificam sentenças. No caso do sistema \textbf{S4}, são adicionadas as modalidades de necessidade ($\nec$) e possibilidade ($\pos$) em conjunto à regra da necessitação\footnote{Se $\vdash A$ então $\vdash \nec A$} e os axiomas $\mathbf{T}\text{: } \nec(A \to B) \to \nec A \to \nec B$, $\mathbf{T}\text{: } \nec A \to A$ e $\text{\textbf{4}: } \nec A \to \nec \nec A$ \citep{Troelstra}. Ademais, pode-se derivar nesse sistema, por meio da dualidade entre as modalidades\footnote{$\pos A \equiv \neg \nec \neg A $}, sentenças duais aos axiomas \textbf{T} e \textbf{4}, sendo elas $\mathbf{T}_\meddiamond \text{: } A \to \pos A$ e $\mathbf{4}_\meddiamond \text{: } \pos \pos A \to \pos A$, respectivamente~\cite{Zach}.

As mônadas ganharam destaque na área de linguagens de programação desde que~\cite{Moggi} formalizou uma metalinguagem que faz uso dessas estruturas para modelar noções de computação --- como parcialidade, não-determinismo, exceções e continuações --- de uma maneira puramente funcional. Pode-se notar uma grande semelhança entre as sentenças $\mathbf{T}_\meddiamond$ e $\mathbf{4}_\meddiamond$ e as transformações naturais monádicas $\mathbf{\eta:} 1_C \to T$ e $\mathbf{\mu:} T^2 \to T$, respectivamente. Nesse sentido,~\cite{Pfenning} demonstraram que se pode traduzir essa metalinguagem para o sistema \textbf{S4} da lógica modal, pelo qual se torna interessante analisar esse sistema como uma linguagem de programação sob a ótica do isomorfismo de Curry-Howard.

~\cite{Troelstra} apresentam duas traduções equivalentes da lógica intuicionista para o sistema \textbf{S4} da lógica modal, sendo um deles correspondente a uma abordagem \textit{call-by-name} e outra a um abordagem \textit{call-by-value}. Tais traduções possuem grande similaridade com as traduções da lógica intuicionista para a lógica linear definidas por~\cite{Girard}. Essas traduções equivalem à tradução por negação dupla que, por sua vez, equivalem a traduções \textit{continuation-passing style} (CPS) em compiladores por meio do isomorfismo de Curry-Howard~\citep{Reynolds}, o que torna esse tema interessante no ponto de vista de compilação.

Durante grande parte da história, provas lógicas e matemáticas eram validadas manualmente pela comunidade acadêmica, o que muitas vezes --- a depender do tamanho e complexidade da prova --- se mostrava ser um trabalho complexo e sujeito a erros. Hoje em dia, exitem \textit{softwares} chamados assistentes de provas que permitem verificar --- graças ao isomorfismo de Curry-Howard --- a corretude de provas~\citep{Chlipala}. O assistente de provas que será usado neste trabalho é o \textsc{coq}, que utiliza o cálculo de construções indutivas e um conjunto axiomático pequeno para permitir a escrita de provas simples e intuitivas~\citep{Coq}.

Este trabalho consiste numa continuação do desenvolvimento da biblioteca de formalização de sistemas modais normais iniciado por~\cite{Silveira} e posteriormente expandida de forma a permitir a fusão de sistemas modais por~\cite{Nunes}. Uma formalização de traduções entre sistemas de dedução similar a nossa foi feita por~\cite{Sehnem}, neste caso tendo como alvo o sistema linear de~\cite{Girard}. Todas as formalizações citadas acima deram-se no assistente de provas \textsc{coq}, o mesmo assistente usado neste trabalho.

    \section{Justificativa}
    \section{Metas}

    \section{Estruturação}
    Estruturaremos este trabalho em cinco partes. A parte \textbf{(1)} trata-se desta introdução. A parte \textbf{(2)} consiste numa fundamentação de conceitos basilares ao desenvolvimento deste trabalho, notadamente os conceitos de \emph{sistemas de dedução}, \emph{traduções} e \emph{provadores de teoremas}. A parte \textbf{(3)} apresenta as definições dos sistemas e traduções relevantes a este trabalho. Na parte \textbf{(4)} são provadas todas as propriedades abarcadas no escopo deste trabalho. Por fim, a parte \textbf{(5)} compreende considerações parciais acerca do desenvolvido até o momento.
    \chapter{Fundamentação}

Nesta parte do trabalho, serão apresentadas definições gerais que fundamentarão as definições mais estritas que serão apresentadas futuramente. Notadamente, fundamentaremos as noções de sistemas e traduções. Ademais, discorreremos acerca da noção de provadores, que serão usados para certificar as provas apresentadas posteriormente. Antes disso, entretanto, introduziremos duas notações que serão usadas copiosamente, uma para o conjunto das partes e outra para sucessões.

\begin{notation}
    Seja $A$ um conjunto, $\mathfrak{P}(A)$ denota o conjunto $\set{X\mid X\subseteq A}$.
\end{notation}

\begin{notation}
    Seja $i\in\mathbb{N}^+$ e $n\in\mathbb{N}$, $\sequence{a_i\mid i\leq n}$ denota uma sucessão de $n$ elementos de modo que o elemento $a_i$ encontra-se na posição $i$.
\end{notation}

\section{Sistemas}

Sistemas de dedução buscam formalizar e sistematizar o processo de razoamento. Estudos acerca disso datam da antiguidade, dentre os quais destaca-se~\cite{Aristotle}. Considera-se que os estudos modernos neste campo foram, dentre outras pessoas, fundados por~\cite{Frege} e continuados por~\cite{Russel-A,Russel-B,Russel-C}. Estas investigações --- bem como outras --- levaram ao desenvolvimento do sistema hoje tido como padrão. Posteriormente a isso, viu-se o surgimento de diversos sistemas não-padrões, fato que --- conforme~\cite{Beziau-B} --- justifica uma conceituação de sistema de dedução, que apresentaremos nesta seção.

Ainda segundo~\cite{Beziau-B}, os primeiros desenvolvimentos neste sentido foram feitos por~\cite{Tarski}, que define o conceito de dedução com base num operador de fecho $C\mathrel{:}\mathfrak{P}(\mathcal{L})\to\mathfrak{P}(\mathcal{L})$, sendo $\mathcal{L}$ um conjunto qualquer. Neste trabalho entretanto usaremos a definição proposta por~\cite{Beziau} baseada numa relação de dedução ${\vdash}\subseteq\mathfrak{P}(\mathcal{L})\times\mathcal{L}$, uma vez que, por sua simplicidade, não traz elementos irrelevantes aos intuitos deste. Cabe destacar, conforme apontam~\cite{Font}, que ambas as definições são equivalentes\footnote{Destaca-se, entretanto, que a definição de~\cite{Tarski} requer a satisfação de postulados não requeridos por~\cite{Beziau}, sendo portanto menos geralista.}, uma vez que $\Gamma\entails\alpha$ se e somente se $\alpha\in C(\Gamma)$.

\begin{definition}[Sistema]
    Um sistema de dedução consiste num par $\mathfrak{S} = \sequence{\mathcal{L}, \vdash}$, onde $\mathcal{L}$ consiste em um conjunto e ${\vdash}\subseteq\mathfrak{P}(\mathcal{L})\times\mathcal{L}$ em uma relação sobre o produto cartesiano do conjunto das partes de $\mathcal{L}$ e o conjunto $\mathcal{L}$, sem demais condições.
    \qed{}
\end{definition}

Conforme~\cite{Beziau} aponta, a qualidade e quantidade dos elementos de um sistema $\mathfrak{S}=\sequence{\mathcal{L}, \vdash}$ não são especificados, portanto sendo esta uma definição de grande generalidade. Neste sentido, com base no escopo deste trabalho, restringiremos a definição do conjunto $\mathcal{L}$ --- dito \emph{linguagem} --- a linguagens proposicionais. Os elementos destas, aos quais daremos o nome de \emph{sentenças}, notabilizam-se por serem formadas por \emph{letras} --- que consistem em proposições indivisas --- e \emph{operadores} --- que podem gerar proposições maiores a partir de proposições menores. Ao par formado por letras e operadores daremos o nome \emph{assinatura}, conforme abaixo.

\begin{definition}[Assinatura]
    Uma assinatura proposicional consiste num par $\Sigma=\sequence{\mathcal{P},\mathcal{C}}$, onde $\mathcal{P}$ consiste num conjunto letras e $\mathcal{C}=\bigcup\set{\mathcal{C}_i\mid i\in\mathbb{N}}$ num conjunto de operadores de modo que $\mathalpha{\circ}\in\mathcal{C}_n$ se e somente se $\mathalpha{\circ}$ possuir aridade $n$.
    \qed{}
\end{definition}

\begin{notation}
    Seja $\mathcal{C}$ um conjunto de operadores, $\mathalpha{\circ}^n$ denota um operador $\mathalpha{\circ}\in\mathcal{C}_n$.
\end{notation}

Podemos interpretar os conjuntos $\mathcal{P}$ e $\mathcal{C}$ de uma assinatura $\Sigma=\sequence{\mathcal{P},\mathcal{C}}$ como construtores de sentenças.
Neste sentido, o conjunto $\mathcal{C}_0$ assemelha-se mais ao conjunto $\mathcal{P}$, uma vez que seus elementos --- ditos \emph{constantes} --- não geram sentenças maiores partindo de sentenças menores.
Nota-se que uma assinatura constitui um elemento suficiente para definirmos indutivamente a linguagem de um sistema, conforme definido abaixo de maneira similar a~\cite{Franks}.
Por fim, destacamos que, para todos os sistemas apresentados neste trabalho, usaremos o conjunto de letras $\mathcal{P}=\set{p_i\mid i\in\mathbb{N}}$ e letras romanas em caixa-baixa para representar seus elementos.

\begin{definition}[Linguagem]
    Seja $\Sigma=\sequence{\mathcal{P},\mathcal{C}}$ uma assinatura proposicional. Uma linguagem proposicional $\mathcal{L}$ induzida a partir de $\Sigma$ consiste no menor conjunto de sentenças bem-formadas induzido a partir das seguintes regras:
    \begin{enumerate}[label=\textbf{\emph{(\alph*)}}, left=\parindent]
        \item$\mathcal{P}\subseteq\mathcal{L}$
        \item\text{Se }$\mathalpha{\circ}\in\mathcal{C}_n\text{ e }\set{\varphi_i\mid i\leq n}\subseteq\mathcal{L}\text{, então }\circ\sequence{\varphi_i\mid i\leq n}\in\mathcal{L}$.\qed{}
    \end{enumerate}
\end{definition}

Neste trabalho, representaremos sentenças por letras gregas em caixa-baixa e conjuntos de sentenças por letras gregas em caixa-alta.\footnote{Desconsiderando-se o $\Sigma$, usado para representar assinaturas.}
Ademais, impõe-se definir a noção de profundidade de uma sentença. Esta noção, em termos simples, consiste no comprimento do maior ramo da construção da dada sentença. A definição provida abaixo consiste numa generalização para quaisquer aridades da definição dada por~\cite{Troelstra}. Usaremos essa definição futuramente para fazer demonstrações por meio provas indutivas sobre esta propriedade.

\begin{definition}[Profundidade]
    Seja $\mathfrak{S} = \sequence{\mathcal{L}, \vdash}$ um sistema com linguagem induzida a partir de uma assinatura $\Sigma=\sequence{\mathcal{P},\mathcal{C}}$. Considerando-se uma proposição $a\in\mathcal{P}$, um operador ${\circ}\in\mathcal{C}$ e uma aridade $n>0$, definimos a profundidade $|\alpha|$ de uma sentença $\alpha\in\mathcal{L}$ indutivamente da seguinte maneira:
    \begin{align*}
        |a|&\coloneqq 0\\
        |{\circ^0}|&\coloneqq 0\\
        |{\circ^n\sequence{\varphi_i\mid i\leq n}}|&\coloneqq\max\set{|\varphi_i|\mid i\leq n}+1.
        \tag*{\qed} 
    \end{align*}
\end{definition}

Com isso, encerram-se as definições relacionadas a linguagens de sistemas de dedução. Agora, apresentaremos definições relacionadas a relações de dedução, que gozam da mesma generalidade dada a liguagens. Deste modo, a relação $\mathalpha{\vdash}$ pode ser tanto uma relação de \emph{derivação} --- definida sintaticamente --- quanto uma relação de \emph{satisfação}\footnote{Sendo esta denotada por $\mathalpha{\vDash}$.} --- definida semanticamente. Neste trabalho, serão abordados apenas sistemas definidos sobre relações de derivação. Cabe destacar, entretanto, que nada na definição de tradução impede que esta seja feita sobre relações de satisfação, conforme veremos com mais detalhes futuramente.

Neste trabalho, definiremos a relações de dedução baseada em axiomatizações, ou seja, em conjuntos de \emph{axiomas} --- sentenças postuladas como verdadeiras --- e conjuntos de \emph{regras de dedução} --- que permitem derivar mais sentenças verdadeiras caso certas condições sejam satisfeitas. Axiomatizações consistem numa abordagem hilbertiana de dedução que, segundo~\cite{Troelstra}, distinguem-se por conter um conjunto reduzido de regras de dedução que nunca descartam premissas. Ainda baseando-se em~\cite{Troelstra} e em contraste a~\cite{Frege} e~\cite{Hilbert-A, Hilbert-B}, preferiremos esquemas de axiomas a axiomas individuais de modo a eliminarmos a necessidade de instanciações.

\begin{definition}[Axiomatização]
    Seja $\mathfrak{S}=\sequence{\mathcal{L},\vdash}$ um sistema. Uma axiomatização para o sistema $\mathfrak{S}$ consiste num par $\mathcal{H}=\sequence{\mathcal{A},\mathcal{R}}$, sendo $\mathcal{A}$ um conjunto de esquemas de axiomas e $\mathcal{R}$ um conjunto de regras de dedução.
\end{definition}

\begin{definition}[Dedução]
    Seja um sistema $\mathfrak{S} = \sequence{\mathcal{L},{\vdash}}$ com uma relação de dedução definida sobre uma axiomatização $\mathcal{H} = \sequence{\mathcal{A},\mathcal{R}}$ e  seja um conjunto de sentenças $\Gamma\cup\set{\alpha}\subseteq\mathcal{L}$.
    A dedução $\Gamma\vdash\alpha$ vale se e somente se houver sucessão de sentenças $\sequence{\varphi_i\in\mathcal{L}\mid i\leq n}$ de modo que $\varphi_n=\alpha$ e que cada sentença $\varphi_i$ tenha sido gerada ou por algum esquema $\mathbf{A}\in\mathcal{A}$ ou pela aplicação de alguma regra $\mathbf{R}\in\mathcal{R}$ a sentenças anteriores.
    \qed{}
\end{definition}

\section{Traduções}

Traduções entre sistemas consistem em funções que mapeiam sentenças de um sistema a sentenças de outro, garantindo certas propriedades. As propriedades a serem garantidas variam e ainda são discutidas na literatura, deixando que a definição exata de tradução --- assim como houve com a definição de sistema --- varie de acordo com a predileção e as necessidades de cada autor. Nesta seção, serão abordadas historicamente noções de tradução entre sistemas, bem como serão definidos e nomeados os conceitos de tradução que serão usados no restante deste trabalho.

\begin{definition}[Condições]
    Chamaremos a condição $\varnothing\entails_\mathbf{A}\alpha$ implica em $\varnothing\entails_\mathbf{B}\alpha^*$ de correção fraca e a condição $\varnothing\entails_\mathbf{B}\alpha^*$ implica em $\varnothing\entails_\mathbf{A}\alpha$ de completude fraca. Analogamente, considerando-se dedução com premissas, chamaremos a condição $\Gamma\entails_\mathbf{A}\alpha$ implica em $\Gamma^*\entails_\mathbf{B}\alpha^*$ de correção forte e a condição $\Gamma^*\entails_\mathbf{B}\alpha^*$ implica em $\Gamma\entails_\mathbf{A}\alpha$ de completude forte.
\end{definition}

Historicamente, autores usaram diferentes combinações das condições apresentadas acima e, em certos casos, outras. Neste trabalho, adotaremos uma noção forte de tradução que requer tanto a correção forte quanto a completude forte, conforme~\cite{Coniglio}. Definiremos, ainda, uma notação que nos permite aplicar sucintamente a tradução a todos os elementos de um conjunto.

\begin{definition}[Tradução] 
    Uma sentença $\varphi$ de um sistema $\mathfrak{A} = \langle\mathcal{A}, \vdash_\mathfrak{A}\rangle$ pode ser traduzida a uma sentença $\varphi^*$ em um sistema $\mathfrak{B} = \langle\mathcal{B}, \vdash_\mathfrak{B} \rangle$ caso exista uma função $\bullet^* : \mathcal{A} \to \mathcal{B}$ que garanta que $\Gamma\vdash_\mathfrak{A}\varphi$ se e somente se $\Gamma^*\vdash_\mathfrak{B}\varphi^*$.
    \qed{}
\end{definition}

\begin{notation}
    Seja $\Gamma\in\mathfrak{P}(\mathcal{A})$ um conjunto de sentenças bem-formadas e $\bullet^*\mathrel{:}\mathcal{A}\to\mathcal{B}$ uma tradução. $\Gamma^*$ denota o conjunto $\set{\alpha^*\mid\alpha\in\Gamma}\in\mathfrak{P}(\mathcal{B})$, ou seja, a aplicação da tradução a todos os elementos do conjunto $\Gamma$.
    \qed{}
\end{notation}

A primeira tradução entre dois sistemas conhecida na literatura foi definida por~\cite{Kolmogorov} como uma maneira de demonstrar que o uso da \emph{lei do terceiro excluso}\footnote{Definido como $\entails\alpha\vee\neg\alpha$.} não leva a contradições. Essa definição consiste basicamente em prefixar uma dupla negação a cada elemento da construção de uma dada sentença \citep{Coniglio}, motivo pelo qual chamaremos essa tradução de \emph{tradução de negação dupla}. Essa mesma tradução foi também descoberta independentemente por Gödel e por Getzen. Curiosamente, essa tradução mostra-se relevante para o escopo deste trabalho, uma vez que consiste na contraparte da passagem por continuações segundo a interpretação prova-programa.

\begin{example} Define-se a tradução $\bullet^\neg:\mathcal{L}_\mathbf{C}\to\mathcal{L}_\mathbf{I}$ do sistema clássico ao sistema intuicionista indutivamente da seguinte maneira:
    \begin{align*}
        p^\neg&\coloneqq\neg\neg p\\
        \bot^\neg&\coloneqq\bot\\
        {(\varphi\wedge\psi)}^\neg&\coloneqq\neg\neg(\varphi^\neg \wedge \psi^\neg)\\
        {(\varphi\vee\psi)}^\neg&\coloneqq\neg\neg (\varphi^\neg \vee \psi^\neg)\\
        {(\varphi\to\psi)}^\neg&\coloneqq\neg\neg (\varphi^\neg \to \psi^\neg)
        \tag*{\qed} 
    \end{align*}
\end{example}

\section{Provadores}

A primeira prova de destaque a ser realizada com grande uso de computadores foi a do teorema das quatro cores\footnote{Que afirma que \emph{qualquer mapa planar tem uma quatro-coloração}.}, feita por~\cite{Appel}, motivado pela grande quantidade de casos a serem analisados. Conforme~\cite{Wilson} afirma, esta prova foi, por uns, recebida com entusiasmo e por outros, devido ao uso de computadores, com cetistismo e desapontamento. Dentre aqueles que compartilharam destas visões opositoras, destaca-se~\cite{Tymoczko}. Ainda segundo~\cite{Wilson}, o teorema tornou-se mais aceito com o passar do tempo e foi, posteriormente, formalizado em um provador de teoremas por~\cite{Gonthier}.

Provadores de teoremas consistem em programas de computador que verificam a validade de teoremas. Dentre estes, podemos destacar as classes dos provadores \emph{automáticos} e dos provadores \emph{interativos}. Os primeiros buscam provar teoremas de maneira que requeira a menor quantidade de intervenção humana, enquanto os segundos --- que ganharam destaque depois das limitações dos primeiros ficarem evidentes --- delegam-se a verificar rigorosamente provas desenvolvidas por humanos em sua linguagem. Formalizaremos as provas apresentadas neste trabalho no provador de teoremas interativo \textsc{coq}, o mesmo \emph{software} usado por~\cite{Gonthier}.

O \textsc{coq} trata-se de um provador de teoremas interativo baseado no \emph{cálculo de construções}. Este sistema formal fornece uma estrutura unificada para definir funções, tipos e proposições, permitindo a construção e verificação de provas dentro do mesmo formalismo. No \textsc{coq}, entretanto, este formalismo foi estendido de modo a permitir tipos indutivos, criando o dito \emph{cálculo de construções indutivas}. Neste, pode-se definir tipos de dados estruturados e funções e provas recursivas. Essa fundação alinha-se com isomorfismo de Curry-Howard, onde programas correspondem a provas e tipos correspondem a proposições, tornando o \textsc{coq} uma ferramenta poderosa de formalização e verificação. Para um maior aprofundamento acerca do provador de teoremas \textsc{coq}, recomenda-se a leitura de~\cite{Chlipala},~\cite{Pierce} e~\cite{Coq}.


    \chapter{Definições}
        \section{Intuicionista}
    \begin{definition}[$\mathcal{L}_\mathbf{I}$]
        A linguagem do sistema intuicionista, denotada $\mathcal{L}_\mathbf{I}$, consiste no menor conjunto induzido a partir das seguintes regras:
        \begin{align*}
            &\bot\in\mathcal{L}_\mathbf{I} \\
            &\mathcal{P}\subseteq\mathcal{L}_\mathbf{I} \\
            &\alpha,\beta\in\mathcal{L}_\mathbf{I}\Rightarrow\alpha\circ\beta\in\mathcal{L}_\mathbf{I}\text{, para }\circ\in\set{\wedge,\vee,\to}\text{.}
            \tag*{\qed}
        \end{align*}
    \end{definition}

    \begin{notation}
        Serão usadas as seguintes abreviações:
        \begin{align*}
            \top&\coloneqq\bot\to\bot\\
            \neg\alpha&\coloneqq\alpha\to\bot\\
            \alpha\leftrightarrow\beta&\coloneqq(\alpha\to\beta)\wedge(\beta\to\alpha)
        \end{align*}
    \end{notation}

    \begin{definition}
        A axiomatização do sistema intuicionista consiste nos seguintes esquemas e regras:
        \begin{alignat*}{3}
            & \mathbf{A_1}\quad && \alpha\to\beta\to\alpha \\
            & \mathbf{A_2}\quad && (\alpha\to\beta\to\gamma)\to(\alpha\to\beta)\to(\alpha\to\gamma) \\
            & \mathbf{A_3}\quad && \alpha\to\beta\to\alpha\wedge\beta \\
            & \mathbf{A_4}\quad && \alpha\wedge\beta\to\alpha \\
            & \mathbf{A_5}\quad && \alpha\wedge\beta\to\beta \\
            & \mathbf{A_6}\quad && \alpha\to\alpha\vee\beta \\
            & \mathbf{A_7}\quad && \beta\to\alpha\vee\beta \\
            & \mathbf{A_8}\quad && (\alpha\to\gamma)\to(\beta\to\gamma)\to(\alpha\vee\beta\to\gamma) \\
            & \mathbf{A_\bot}\quad && \bot\to\alpha \\
            & \mathbf{R_1}\quad && \text{Se }\Gamma\vdash\alpha\text{ e }\Delta\vdash\alpha\to\beta\text{, então }\Gamma\cup\Delta\vdash\beta\text{.} & \tag*{\qed}
        \end{alignat*}   
    \end{definition}

    Chamaremos $\mathbf{R_1}$ de \emph{regra da separação}.

        \section{Modalismo}
    \babireski{\cite{Blackburn} traz uma visão da evolução dos sistemas modais.}

    Os sistemas modais consistem em extensões do sistema proposicional com a adição de modalidades que representam \emph{necessidade} --- denotada como $\nec$ --- e \emph{possibilidade} --- denotada como $\pos$ --- bem como esquemas e regras que dizem respeito a elas. Deste modo, estão contidas na linguagem do sistema sentenças da forma $\nec\alpha$ e $\pos\alpha$ --- lidas \emph{necessariamente} $\alpha$ e \emph{possivelmente} $\alpha$, respeitivamente. Intuitivamente, uma necessidade deve ser verdade em todos os casos, enquanto uma possibilidade deve ser verdade em algum caso.

    Os primeiros desenvolvimentos acerca das modalidades acima foram feitos pelos gregos antigos, que anteciparam muitos dos preceitos aceitos modernamente. O fundador do modalismo moderno foi~\cite{Lewis}, motivado pela sua insatisfação com a implicação material, uma vez que sua definição\footnote{Definida como $\alpha\to\beta\equiv\neg\alpha\vee\beta$.} permite que sentenças intuitivamente falsas em linguagem natural seja valoradas como verdade. Este sistema foi posteriormente melhor desenvolvido por~\cite{Langford}, onde foram apresentados os sistemas $\mathbf{S_1}$ a $\mathbf{S_5}$ --- sendo $\mathbf{S_4}$ o abordado neste trabalho.

    Primeiramente, definiremos a assinatura proposicional dos sistemas modais, a partir da qual sua linguagem deriva trivialmente.

    \begin{definition}[$\mathcal{L}_\mathbf{M}$]
        A linguagem dos sistemas modais, denotada $\mathcal{L}_\mathbf{M}$, pode ser induzida a partir da assinatura $\Sigma_\mathbf{M}=\sequence{\mathcal{P},\mathcal{C}_\mathbf{M}}$, onde $\mathcal{C}_\mathbf{M}=\set{\bot^0,\nec^1,\wedge^2,\vee^2,\to^2}$.
    \end{definition}

    \begin{notation}
        Serão usadas as seguintes abreviações:
        \begin{align*}
            \top&\coloneqq\bot\to\bot\\
            \neg\alpha&\coloneqq\alpha\to\bot\\
            \pos\alpha&\coloneqq\neg\nec\neg\alpha\\
            \alpha\strictif\beta&\coloneqq\nec(\alpha\to\beta)\\
            \alpha\leftrightarrow\beta&\coloneqq(\alpha\to\beta)\wedge(\beta\to\alpha)
        \end{align*}
    \end{notation}

    \begin{notation}
        Seja $\Gamma\in\wp(\mathcal{L}_\mathbf{M})$ um conjunto de sentenças bem-formadas.
        $\nec\Gamma$ denota o conjunto $\set{\nec\alpha\mid\alpha\in\Gamma}\in\wp(\mathcal{L}_\mathbf{M})$, ou seja, a prefixação da necessitação a todos os elementos do conjunto $\Gamma$.
    \end{notation}

    \begin{definition}\label{m-axioms}
        A axiomatização do sistema modal consiste no conjunto de esquemas de axiomas $\mathcal{A}=\set{\mathbf{A}_i\mid i\in[1,8]\vee i=\neg}\cup\set{\mathbf{B_1},\mathbf{B_2},\mathbf{B_3}}$ e no conjunto de regras $\mathcal{R}=\set{\mathbf{R_1},\mathbf{R_2}}$, definidos abaixo:
        \begin{alignat}{3}
            &\mathbf{A_1}\quad&&\alpha\to\beta\to\alpha\label{MA1}\tag*{}\\
            &\mathbf{A_2}\quad&&(\alpha\to\beta\to\gamma)\to(\alpha\to\beta)\to(\alpha\to\gamma)\label{MA2}\tag*{}\\
            &\mathbf{A_3}\quad&&\alpha\to\beta\to\alpha\wedge\beta\label{MA3}\tag*{}\\
            &\mathbf{A_4}\quad&&\alpha\wedge\beta\to\alpha\label{MA4}\tag*{}\\
            &\mathbf{A_5}\quad&&\alpha\wedge\beta\to\beta\label{MA5}\tag*{}\\
            &\mathbf{A_6}\quad&&\alpha\to\alpha\vee\beta\label{MA6}\tag*{}\\
            &\mathbf{A_7}\quad&&\beta\to\alpha\vee\beta\label{MA7}\tag*{}\\
            &\mathbf{A_8}\quad&&(\alpha\to\gamma)\to(\beta\to\gamma)\to(\alpha\vee\beta\to\gamma)\label{MA8}\tag*{}\\
            &\mathbf{A_\neg}\quad&&\neg\neg\alpha\to\alpha\label{MANEG}\tag*{}\\
            &\mathbf{B_1}\quad&&\nec(\alpha\to\beta)\to\nec\alpha\to\nec\beta\label{MB1}\tag*{}\\
            &\mathbf{B_2}\quad&&\nec\alpha\to\alpha\label{MB2}\tag*{}\\
            &\mathbf{B_3}\quad&&\nec\alpha\to\nec\nec\alpha\label{MB3}\tag*{}\\
            &\mathbf{R_1}\quad&&\text{Se }\Gamma\entails\alpha\text{ e }\Gamma\entails\alpha\to\beta\text{, então }\Gamma\entails\beta\label{detachment}\tag*{}\\
            &\mathbf{R_2}\quad&&\text{Se }\entails\alpha\text{, então }\Gamma\entails\nec\alpha\text{.}\tag*{\qed}\label{necessitation} 
        \end{alignat}   
    \end{definition}

    \babireski{Falar aqui sobre como a axiomatização consiste nos esquemas clássicos mais os esquemas modais}. 

    Assim como feito para o sistema intuicionista, nomearemos os esquemas e regras acima de modo a facilitar a comunicação.
    Aos axiomas e regras que correspondem aos axiomas e regras intuicionistas receberão os mesmos nomes. Ademais, chamaremos $\mathbf{B_1}$ de axiomas da normalidade, $\mathbf{B_2}$ de axiomas da reflexividade e $\mathbf{B_3}$ de axiomas da transitividade.\footnote{Em analogia às condições relacionais impostas nos enquadramentos.} Nomearemos $\mathbf{A_\neg}$ como chamaremos de axiomas da eliminação da negação e a $\mathbf{R_2}$ como regra da necessitação.

    A definição das regras de dedução em relação a conjuntos de sentenças baseia-se tanto em~\cite{Troelstra} como em~\cite{Hakli}. Ao decorrer do texto, ocasionalmente chamaremos $\mathbf{R_1}$ de \emph{regra da separação} e $\mathbf{R_2}$ de \emph{regra da necessitação}. A definição da regra da necessitação deve ser cuidadosa de modo a permitir a prova do metateorema da dedução, feita futuramente neste trabalho. Neste sentido, restringimos a aplicação desta regra apenas a teoremas.\footnote{Para uma discussão mais aprofudada, ver~\cite{Hakli}.}

        \section{Traduções}
    A primeira tradução do sistema intuicionista ao sistema modal foi proposta por~\cite{Goedel} motivado pela possibilidade de leitura da necessidade como uma modalidade de construtividade. Ou seja, por meio dessa tradução, a sentença $\nec \varphi$ poderia ser lida como \textit{$\varphi$ pode ser provada construtivamente} \citep{Troelstra}. Gödel conjeiturou a corretude fraca dessa tradução, que foi posteriormente provada por~\cite{McKinsey} em conjunto com sua completude fraca.

    \begin{definition}[$\bullet^\circ$] Define-se a tradução $\bullet^\circ$ indutivamente da seguinte maneira:
        \begin{align*}
            p^\circ                     & \coloneqq p                                       \\
            \bot^\circ                  & \coloneqq \bot                                    \\
            {(\varphi \wedge \psi)}^\circ & \coloneqq \varphi^\circ \wedge \psi^\circ         \\
            {(\varphi \vee \psi)}^\circ   & \coloneqq \nec \varphi^\circ \vee \nec \psi^\circ \\
            {(\varphi \to \psi)}^\circ    & \coloneqq \nec \varphi^\circ \to \psi^\circ
            \tag*{\qed} 
        \end{align*}
    \end{definition}
    
    \begin{definition}[$\bullet^\nec$] Define-se a tradução $\bullet^\nec$ indutivamente da seguinte maneira:
        \begin{align*}
            p^\nec                     & \coloneqq \nec p                                        \\
            \bot^\nec                  & \coloneqq \bot                                          \\
            {(\varphi \wedge \psi)}^\nec & \coloneqq \varphi^\nec \wedge \psi^\nec     \\
            {(\varphi \vee \psi)}^\nec   & \coloneqq \varphi^\nec \vee \psi^\nec       \\
            {(\varphi \to \psi)}^\nec    & \coloneqq \nec (\varphi^\nec \to \psi^\nec)
            \tag*{\qed} 
        \end{align*}
    \end{definition}
    
    Faz-se interessante pontuar que as traduções $\bullet^\circ$ e $\bullet^\nec$ correspondem, respectivamente, às traduções $\bullet^\circ$ e $\bullet^*$ do sistema intuicionista ao sistema linear providas por~\cite{Girard}, sendo as primeiras correspondentes a uma ordem de avaliação por nome (\textit{call-by-name}) e as segundas a uma ordem de avaliação por valor (\textit{call-by-value}). 
    Ademais, as duas traduções providas são equivalentes, conforme demonstrado pelo teorema $\mathbf{T_2}$.

    \begin{theorem}
        $\forall\alpha\in\mathcal{L}_\mathbf{I}\point\nec\alpha^\circ\leftrightarrow\alpha^\nec$.

        \begin{proof}
            Prova por indução forte sobre a profundidade de $\alpha$.
            Assim, suponhamos que as traduções equivalham para qualquer $\alpha$ de profundidade $n\leq k$.
            Demonstraremos, analisando-se os casos, que as traduções equivalem para qualquer $\alpha$ de profundidade $n=k+1$.
    
            \begin{case}
                \textbf{Caso 1} (Base)\textbf{.}
                    Para $|\alpha| = 0$, existem dois casos a serem considerados.
    
                    \begin{subcase}
                        \textbf{Caso 1.1} ($\alpha = a$)\textbf{.}
                        $a^\circ = a$ e $a^\nec = \nec a$, assim $\nec a^\circ = a^\nec$ e, portanto, $\nec a^\circ \leftrightarrow a^\nec$.
                    \end{subcase}

                    \begin{subcase}
                        \textbf{Caso 2.1} ($\alpha = \bot$)\textbf{.}
                        $\bot^\circ = \bot$ e $\bot^\nec = \bot$. A ida $\nec\bot\to\bot$ consiste em um axioma, sendo, portando provada trivialmente pela sucessão de dedução $\sequence{\nec\bot\to\bot}$.
                        A volta $\bot\to\nec\bot$ equivale a provar, por meio do teorema da dedução, que $\set{\bot}\vdash_\mathbf{M}\nec\bot$, o que pode ser provado trivialmente pela sucessão de dedução $\sequence{\bot, \nec\bot}$, que consiste na invocação da premissa e aplicação da regra da necessitação, nessa ordem.
                    \end{subcase}
            \end{case}
    
            \begin{case}
                \textbf{Caso 2} (Passo)\textbf{.} No passo, deve-se demonstrar que, caso $\nec\alpha^\circ\leftrightarrow\alpha^\nec$ para $|\alpha| = n$, 
                então $\nec\alpha^\circ\leftrightarrow\alpha^\nec$ para $|\alpha| = n + 1$, onde $n \in \mathbb{N}$. Assim, seja $\nec\alpha^\circ\leftrightarrow\alpha^\nec$ uma proposição verdadeira para $|\alpha| = k$, onde $k \in \mathbb{N}$. Existem os seguintes casos a serem considerados para $|\alpha| = k + 1$.
            \end{case}
    
                \begin{subcase}
                    \textbf{Caso 2.1.}
                    Neste caso, consideraremos que $\alpha = \varphi\wedge\psi$.
                    Assim sendo, sabe-se que $\nec{(\varphi\wedge\psi)}^\circ=\nec(\varphi^\circ\wedge\psi^\circ)$ e ${(\varphi\wedge\psi)}^\nec=\varphi^\nec\wedge\psi^\nec$.
                    Pelo lemas $\mathbf{L_2}$ e $\mathbf{L_3}$ temos que $\nec(\varphi^\circ\wedge\psi^\circ)\leftrightarrow\nec\varphi^\circ\wedge\nec\psi^\circ$ e pela premissa da indução temos que $\nec\varphi^\circ\leftrightarrow\varphi^\nec$ e $\nec\psi^\circ\leftrightarrow\psi^\nec$.
                    Deste modo, $\nec{(\varphi\wedge\psi)}^\circ\leftrightarrow{(\varphi\wedge\psi)}^\nec$.
                \end{subcase}
    
                \begin{subcase}
                    \textbf{Caso 2.2.}
                    Neste caso, consideraremos que $\alpha = \varphi\vee\psi$.
                    Assim sendo, sabe-se que $\nec{(\varphi\vee\psi)}^\circ=\nec(\nec\varphi^\circ\vee\nec\psi^\circ)$ e ${(\varphi\vee\psi)}^\nec=\varphi^\nec\vee\psi^\nec$.
                    Para a ida, basta aplicar o axioma $\mathbf{B_2}$.
                    Para a volta, basta aplicar a regra da necessitação.
                \end{subcase}
    
                \begin{subcase}
                    \textbf{Caso 2.3.}
                    Neste caso, consideraremos que $\alpha=\varphi\to\psi$.
                    Assim sendo, sabe-se que $\nec{(\varphi\to\psi)}^\circ=\nec(\nec\varphi^\circ\to\psi^\circ)$ e ${(\varphi\to\psi)}^\nec=\nec(\varphi^\nec\to\psi^\nec)$.
                    Para a ida, basta aplicar o axioma $\mathbf{B_1}$.
                    Para a volta, basta aplicar o lema $\mathbf{L_4}$ seguido pela necessitação.
                    \qedhere
                \end{subcase}
        \end{proof}
    \end{theorem}


    \chapter{Formalização}
        Todas as derivações a seguir serão no sistema $\mathbf{S_4}$, motivo pelo qual denotaremos a relação de derivação $\entails_{\mathbf{4}}$ apenas como $\entails$.

        \section{Derivações}
    Nesta seção apresentaremos alguns lemas e teoremas para os sistemas modais que permitirão simplificar muito as provas apresentadas no decorrer deste trabalho.
    Primeiramente, provaremos que, dada uma sentença qualquer, esta sempre implica a si mesma. A este lema daremos o nome de identidade\footnote{Em analogia ao combinador $\mathbf{I}$.} e, em seguida, usaremo-no para a prova da regra da dedução.

    \begin{lemma}\label{identity}
        $\entails\alpha\to\alpha$.
        \begin{proof}
            Pode ser provado pela seguinte sucessão de dedução:
            \footnotesize
            \begin{fitch}   
                \fb\vdash\alpha\to\alpha\to\alpha&\hyperref[MA1]{$\mathbf{A_1}$}\\
                \fa\vdash\alpha\to(\alpha\to\alpha)\to\alpha&\hyperref[MA1]{$\mathbf{A_1}$}\\
                \fa\vdash(\alpha\to(\alpha\to\alpha)\to\alpha)\to(\alpha\to\alpha\to\alpha)\to\alpha\to\alpha&\hyperref[MA2]{$\mathbf{A_2}$}\\
                \fa\vdash(\alpha\to\alpha\to\alpha)\to\alpha\to\alpha&$\hyperref[detachment]{\mathbf{R_1}}\;\set{2,3}$\\
                \fa\vdash\alpha\to\alpha&$\hyperref[detachment]{\mathbf{R_1}}\;\set{1,4}$
            \end{fitch}
            \normalsize
            Estando assim demonstrada a proposição.
        \end{proof}
    \end{lemma}

    Tendo-se provado o lema da identidade, agora provaremos a regra da dedução para os sistemas modais com base na prova apresentada por~\cite{Hakli}.
    Pequenas alterações foram feitas de modo a garantir a adequação da prova com a axiomatização provida na Definição~\ref{m-axioms}.

    \begin{theorem}[Metateorema da dedução]\label{deduction}
        $\text{Se }\Gamma\cup\set{\alpha}\vdash\beta\text{, então }\Gamma\vdash\alpha\to\beta$.

        \begin{proof}
            Prova por indução forte sobre o tamanho da sucessão de dedução.\footnote{Nota-se que, para a indução forte, não se faz preciso provar nenhuma base \citep{Velleman}.}
            Assim, suponhamos que o teorema da dedução valha para qualquer sucessão de dedução de tamanho $n<k$.
            Demonstraremos analisando-se os casos e valendo-se da suposição acima --- doravante chamada $\mathbf{H}$ --- o passo de indução, ou seja, que o teorema da dedução vale para sucessões de dedução de tamanho $n=k$.

            \begin{case}
                \textsc{Caso 1.}
                Se a linha derradeira da sucessão de dedução que prova $\Gamma\cup\set{\alpha}\vdash\beta$ tenha sido a evocação de alguma premissa, sabe-se que $\beta\in\Gamma\cup\set{\alpha}$.
                Deste modo, existem dois casos a serem analisados.
            \end{case}

            \begin{subcase}
                \textsc{Caso 1.1.}
                Se a linha derradeira da sucessão de dedução que prova $\Gamma\cup\set{\alpha}\vdash\beta$ tenha sido a evocação de alguma premissa do conjunto $\Gamma$, sabe-se que $\beta\in\Gamma$. Deste modo, podemos demonstrar que $\Gamma\vdash\alpha\to\beta$ pela seguinte sucessão de dedução:

                \footnotesize
                \begin{fitch}
                    \fb\Gamma\vdash\beta&$\mathbf{P_\beta}$\\
                    \fa\Gamma\vdash\beta\to\alpha\to\beta&$\hyperref[MA1]{\mathbf{A_1}}$\\
                    \fa\Gamma\vdash\alpha\to\beta&$\hyperref[detachment]{\mathbf{R_1}}\;\set{1,2}$.
                \end{fitch}
            \end{subcase}

            \begin{subcase}
                \textsc{Caso 1.2.}
                Se a linha derradeira da sucessão de dedução que prova $\Gamma\cup\set{\alpha}\vdash\beta$ tenha sido a evocação da premissa $\alpha$, sabe-se que $\beta=\alpha$.
                Deste modo, basta demonstrar que $\Gamma\vdash\alpha\to\alpha$, que consiste num enfraquecimento do lema \refer{identity}{L}.
            \end{subcase}

            \begin{case}
                \textsc{Caso 2.}
                Se a linha derradeira da sucessão de dedução que prova $\Gamma\cup\set{\alpha}\vdash\beta$ tenha sido a evocação de algum axioma, sabe-se que existe algum esquema $\mathbf{A_\beta}\in\mathcal{A}$ que instancia $\beta$.
                Deste modo, podemos demonstrar que $\Gamma\vdash\alpha\to\beta$ pela seguinte sucessão de dedução:

                \footnotesize
                \begin{fitch}
                    \fb\Gamma\vdash\beta&$\mathbf{A_\beta}$\\
                    \fa\Gamma\vdash\beta\to\alpha\to\beta&$\hyperref[MA1]{\mathbf{A_1}}$\\
                    \fa\Gamma\vdash\alpha\to\beta&$\hyperref[detachment]{\mathbf{R_1}}\;\set{1,2}$.
                \end{fitch}
            \end{case}

            \begin{case}
                \textsc{Caso 3.}
                Se a linha derradeira da sucessão de dedução que prova $\Gamma\cup\set{\alpha}\vdash\beta$ tenha sido gerada pela aplicação da regra da necessitação a uma linha anterior $\mathbf{H_1}$, sabe-se que $\beta=\nec\varphi$ e que $\mathbf{H_1}={\entails\varphi}$.
                Deste modo, podemos demonstrar que $\Gamma\vdash\alpha\to\nec\varphi$ pela seguinte sucessão de dedução:

                \footnotesize
                \begin{fitch}
                    \fb\vdash\varphi&$\mathbf{H_1}$\\
                    \fa\Gamma\vdash\nec\varphi&$\hyperref[necessitation]{\mathbf{R_2}}\;\set{1}$\\
                    \fa\Gamma\vdash\nec\varphi\to\alpha\to\nec\varphi&$\hyperref[MA1]{\mathbf{A_1}}$\\
                    \fa\Gamma\vdash\alpha\to\nec\varphi&$\hyperref[detachment]{\mathbf{R_1}}\;\set{2,3}$.
                \end{fitch}
            \end{case}

            \begin{case}
                \textsc{Caso 4.} Seja a sentença $\varphi_n=\beta$ gerada pela aplicação da regra do \emph{modus ponens} a duas sentenças $\varphi_i$ e $\varphi_j$ com $i<j<n$. Assumiremos, sem perda de generalidade, que $\varphi_j=\varphi_i\to\varphi_n$.
                Assim, a partir de $\mathbf{H}$ temos que $\mathbf{H_1}=\Gamma\entails\alpha\to\varphi_i$ e que $\mathbf{H_2}=\Gamma\entails\alpha\to\varphi_i\to\varphi_n$.
                Deste modo, podemos demonstrar que $\Gamma\vdash\alpha\to\nec\varphi$ pela seguinte sucessão de dedução:

                \footnotesize
                \begin{fitch}
                    \fb\Gamma\entails\alpha\to\varphi_j&$\mathbf{H_1}$\\
                    \fa\Gamma\entails\alpha\to\varphi_j\to\beta&$\mathbf{H_2}$\\
                    \fa\Gamma\entails(\alpha\to\varphi_j\to\beta)\to(\alpha\to\varphi_j)\to(\alpha\to\beta)&$\hyperref[MA2]{\mathbf{A_2}}$\\
                    \fa\Gamma\entails(\alpha\to\varphi_j)\to(\alpha\to\beta)&$\hyperref[detachment]{\mathbf{R_1}}\;\set{2,3}$\\
                    \fa\Gamma\entails\alpha\to\beta&$\hyperref[detachment]{\mathbf{R_1}}\;\set{1,4}$.
                \end{fitch}
            \end{case}
            Uma vez provada a propriedade para todos os casos do passo de indução, provamos que o teorema da dedução vale para o sistema $\mathbf{S4}$.
        \end{proof}
    \end{theorem}

    Tendo-se provado o teorema da dedução, provaremos o teorema da generalização da regra da necessitação, conforme sugerido por~\cite{Troelstra}.
    Como apresentado abaixo, este teorema afirma que, caso possamos deduzir alguma sentença $\alpha$ a partir de um conjunto necessariamente verdadeiro de premissas, podemos concluir a necessidade desta sentença $\alpha$.

    \begin{theorem}[Generalização da necessitação]\label{gen-nec}
        Se $\nec\Gamma\entails\alpha$, então $\nec\Gamma\entails\nec\alpha$.
        \begin{proof}
            Prova por indução fraca sobre o tamanho $n$ do conjunto $\Gamma$ \citep{Troelstra}. A prova consiste em dois casos: um para a base da indução e outro para o passo da indução. No que segue, denotaremos $\mathbf{H_1}=\nec\Gamma\entails\alpha$.
            \begin{case}
                \textsc{Caso 1.} Para a base, consideraremos que $\Gamma=\varnothing$.
                Assim, sabemos que o conjunto possui tamanho nulo e que $\entails\alpha$. Portanto, sabe-se que existe uma sucessão de dedução $\sequence{\varphi_i\mid 0\leq i\leq n}$ com $\varphi_n=\alpha$
                Deste modo, pode-se demonstrar que $\entails\nec\alpha$ trivialmente pela aplicação da regra da necessitação \hyperref[necessitation]{$\mathbf{R_2}$} sobre a sentença $\varphi_n$.
            \end{case}
            \begin{case}
                \textsc{Caso 2.} 
                Para o passo, suponhamos que a generalização da regra da necessitação valha para qualquer conjunto $\Gamma$ de tamanho $n=k$.
                Demonstraremos, valendo-se da suposição acima --- doravante chamada $\mathbf{H_2}$ --- e pela sucessão de dedução apresentada abaixo, que a generalização da regra da necessitação vale para conjuntos $\Gamma$ de tamanho $n=k+1$.
                \footnotesize
                \begin{fitch}
                    \fb\nec\Gamma\cup\set{\nec\alpha}\entails\beta&$\mathbf{H_1}$\\
                    \fa\nec\Gamma\entails\nec\alpha\to\beta&$\hyperref[deduction]{\mathbf{T_1}}\;\set{1}$\\
                    \fa\nec\Gamma\entails\nec(\nec\alpha\to\beta)&$\mathbf{H_2}\;\set{2}$\\
                    \fa\nec\Gamma\entails\nec(\nec\alpha\to\beta)\to\nec\nec\alpha\to\nec\beta&$\hyperref[MB1]{\mathbf{K}}$\\
                    \fa\nec\Gamma\entails\nec\nec\alpha\to\nec\beta&$\hyperref[detachment]{\mathbf{R_1}}\;\set{3,4}$\\
                    \fa\nec\Gamma\entails\nec\alpha\to\nec\nec\alpha&$\hyperref[MB3]{\mathbf{4}}$\\
                    \fa\nec\Gamma\entails(\nec\alpha\to\nec\nec\alpha)\to(\nec\nec\alpha\to\nec\beta)\to\nec\alpha\to\nec\beta&$\hyperref[MA2]{\mathbf{A_2}}$\\
                    \fa\nec\Gamma\entails(\nec\nec\alpha\to\nec\beta)\to\nec\alpha\to\nec\beta&$\hyperref[detachment]{\mathbf{R_1}}\;\set{6,7}$\\
                    \fa\nec\Gamma\entails\nec\alpha\to\nec\beta&$\hyperref[detachment]{\mathbf{R_1}}\;\set{5,8}$\\
                    \fa\nec\Gamma\cup\set{\nec\alpha}\entails\nec\alpha&$\mathbf{P_1}$\\
                    \fa\nec\Gamma\cup\set{\nec\alpha}\entails\nec\alpha\to\nec\beta&$\mathbf{E_1}\;\set{9}$\\
                    \fa\nec\Gamma\cup\set{\nec\alpha}\entails\nec\beta&$\hyperref[detachment]{\mathbf{R_1}}\;\set{10,11}$.
                \end{fitch}
            \end{case}
            \vspace{.5\baselineskip}
            Tendo-se provado a base e o passo de indução, podemos concluir que generalização da regra da necessitação vale, ou seja, que se $\nec\Gamma\entails\alpha$, então $\nec\Gamma\entails\nec\alpha$.
        \end{proof}
    \end{theorem}

    Uma vez provada a generalização da regra da necessitação, a prova da regra da dedução estrita --- conforme descrito por~\cite{Barcan, Marcus} --- torna-se trivial, como pode ser visto abaixo. Esta regra afirma que, dada uma dedução de $\beta$ partindo de um conjunto de premissas necessariamente verdadeiras e uma premissa $\alpha$, podemos deduzir $\nec(\alpha\to\beta)$ a partir desse conjunto de premissas necessariamente verdadeiras. Isso nos permite simplificar as provas de correção das traduções, uma vez que uma das traduções apresentadas mapeia implicações materiais do sistema intuicionista em implicações estritas.

    \begin{theorem}\label{strictdeduction}
        $\text{Se }\nec\Gamma\cup\set{\alpha}\entails\beta\text{, então }\nec\Gamma\entails\nec(\alpha\to\beta)$.
        \begin{proof}
            Pode ser provado pela seguinte sucessão de dedução:
            \footnotesize
            \begin{fitch}
                \fb\nec\Gamma\cup\set{\alpha}\entails\beta&$\mathbf{H_1}$\\
                \fa\nec\Gamma\entails\alpha\to\beta&\refer{deduction}{T}$\;\set{1}$\\
                \fa\nec\Gamma\entails\nec(\alpha\to\beta)&\refer{gen-nec}{T}$\;\set{2}$
            \end{fitch}
            \normalsize
            Estando assim demonstrada a proposição.
        \end{proof}
    \end{theorem}

    Agora, provaremos a aplicação da regra do \emph{modus ponens} a uma implicação estrita. Essa regra afirma que, dada uma prova de $\alpha$ e uma prova de $\nec(\alpha\to\beta)$ a partir de um conjunto de premissas, sabe-se que deve haver alguma prova de $\beta$ a partir desse mesmo conjunto de premissas.

    \begin{theorem}\label{strictsep}
        Se $\Gamma\entails\alpha$ e $\Gamma\entails\nec(\alpha\to\beta)$, então $\Gamma\entails\beta$.
        \begin{proof}
            Pode ser provado pela seguinte sucessão de dedução:
            \footnotesize
            \begin{fitch}
                \fb\Gamma\entails\alpha&$\mathbf{H_1}$\\
                \fa\Gamma\entails\nec(\alpha\to\beta)&$\mathbf{H_2}$\\
                \fa\Gamma\entails\nec(\alpha\to\beta)\to\alpha\to\beta&$\hyperref[MB2]{\mathbf{T}}$\\
                \fa\Gamma\entails\alpha\to\beta&$\hyperref[detachment]{\mathbf{R_1}}\;\set{2,3}$\\
                \fa\Gamma\entails\beta&$\hyperref[detachment]{\mathbf{R_1}}\;\set{1,4}$
            \end{fitch}
            \normalsize
            Estando assim demonstrada a proposição.
        \end{proof}
    \end{theorem}

    Abaixo, demonstraremos que, caso haja uma prova de alguma sentença $\gamma$ a partir de um par de premissas, sabe-se que deve haver alguma prova desta mesma sentença $\gamma$ a partir da conjunção deste par de premissas.

    \begin{theorem}\label{conjunctiondeduction}
        Se $\set{\alpha,\beta}\entails\gamma$, então $\set{\alpha\wedge\beta}\entails\gamma$.
        \begin{proof}
            Seja $\mathbf{H_1}=\set{\alpha,\beta}\entails\gamma$. A proposição pode ser provada pela seguinte sucessão de dedução:
            \footnotesize
            \begin{fitch}
                \fb\set{\alpha,\beta}\entails\gamma&$\mathbf{H_1}$\\
                \fa\set{\alpha}\entails\beta\to\gamma&$\hyperref[deduction]{\mathbf{T_1}}\;\set{1}$\\
                \fa\entails\alpha\to\beta\to\gamma&$\hyperref[deduction]{\mathbf{T_1}}\;\set{2}$\\
                \fa\set{\alpha\wedge\beta}\entails\alpha\wedge\beta&$\mathbf{P_1}$\\
                \fa\set{\alpha\wedge\beta}\entails\alpha\wedge\beta\to\alpha&$\hyperref[MA4]{\mathbf{A_4}}$\\
                \fa\set{\alpha\wedge\beta}\entails\alpha&$\hyperref[detachment]{\mathbf{R_1}}\;\set{4,5}$\\
                \fa\set{\alpha\wedge\beta}\entails\alpha\wedge\beta\to\beta&$\hyperref[MA5]{\mathbf{A_5}}$\\
                \fa\set{\alpha\wedge\beta}\entails\beta&$\hyperref[detachment]{\mathbf{R_1}}\;\set{4,7}$\\
                \fa\set{\alpha\wedge\beta}\entails\alpha\to\beta\to\gamma&${\mathbf{E_1}}\;\set{3}$\\
                \fa\set{\alpha\wedge\beta}\entails\beta\to\gamma&$\hyperref[detachment]{\mathbf{R_1}}\;\set{6,9}$\\
                \fa\set{\alpha\wedge\beta}\entails\gamma&$\hyperref[detachment]{\mathbf{R_1}}\;\set{8,10}$
            \end{fitch}
            \normalsize
            Estando assim demonstrada a proposição.
        \end{proof}
    \end{theorem}

    Analogamente ao teorema anterior, demonstraremos que, caso haja uma prova de $\gamma$ partindo-se da premissa $\alpha$ e uma prova de $\gamma$ partindo-se da premissa $\beta$, então sabe-se que deve haver uma prova de $\gamma$ partindo-se da premissa $\alpha\vee\beta$.

    \begin{theorem}\label{disjunctiondeduction}
        Se $\set{\alpha}\entails \gamma$ e $\set{\beta}\entails \gamma$, então $\set{\alpha \vee \beta}\entails \gamma$.
        \begin{proof}
            Seja $\mathbf{H_1}=\set{\alpha}\entails\gamma$ e $\mathbf{H_2}=\set{\beta}\entails\gamma$. A proposição pode ser provada pela seguinte sucessão de dedução:
            \footnotesize
            \begin{fitch}
                \fb\set{\alpha}\entails\gamma&$\mathbf{H_1}$\\
                \fa\set{\beta}\entails\gamma&$\mathbf{H_2}$\\
                \fa\entails\alpha\to\gamma&$\hyperref[deduction]{\mathbf{T_1}}\;\set{1}$\\
                \fa\entails\beta\to\gamma&$\hyperref[deduction]{\mathbf{T_1}}\;\set{2}$\\
                \fa\set{\alpha\vee\beta}\entails\alpha\to\gamma&${\mathbf{E_1}}\;\set{3}$\\
                \fa\set{\alpha\vee\beta}\entails\beta\to\gamma&${\mathbf{E_1}}\;\set{4}$\\
                \fa\set{\alpha\vee\beta}\entails\alpha\vee\beta&$\mathbf{P_1}$\\
                \fa\set{\alpha\vee\beta}\entails(\alpha\to\gamma)\to(\beta\to\gamma)\to\alpha\vee\beta\to\gamma&$\hyperref[MA8]{\mathbf{A_8}}$\\
                \fa\set{\alpha\vee\beta}\entails(\beta\to\gamma)\to\alpha\vee\beta\to\gamma&$\hyperref[detachment]{\mathbf{R_1}}\;\set{5,8}$\\
                \fa\set{\alpha\vee\beta}\entails\alpha\vee\beta\to\gamma&$\hyperref[detachment]{\mathbf{R_1}}\;\set{4,9}$\\
                \fa\set{\alpha\vee\beta}\entails\gamma&$\hyperref[detachment]{\mathbf{R_1}}\;\set{7,10}$
            \end{fitch}
            \normalsize
            Estando assim demonstrada a proposição.
        \end{proof}
    \end{theorem}

    Os lemas 2 a 13 abaixo serão demonstrados a fim de diminuir o tamanho das provas futuras acerca do isomorfismo entre as traduções e a correção da tradução \emph{call-by-value}.

    \begin{lemma}\label{explosion}
        $\entails\bot\to\alpha$.
        \begin{proof}
            Pode ser provado pela seguinte sucessão de dedução:
            \footnotesize
            \begin{fitch}
                \fb\set{\bot}\entails\bot&$\mathbf{P_1}$\\
                \fa\set{\bot}\entails\bot\to(\alpha\to\bot)\to\bot&$\hyperref[MA1]{\mathbf{A_1}}$\\
                \fa\set{\bot}\entails\neg\neg\alpha&$\hyperref[detachment]{\mathbf{R_1}}\;\set{1,2}$\\
                \fa\set{\bot}\entails\neg\neg\alpha\to\alpha&$\hyperref[MANEG]{\mathbf{A_\neg}}$\\
                \fa\set{\bot}\entails\alpha&$\hyperref[detachment]{\mathbf{R_1}}\;\set{3,4}$\\
                \fa\entails\bot\to\alpha&$\hyperref[deduction]{\mathbf{T_1}}\;\set{5}$.
            \end{fitch}
            \normalsize
            Estando assim demonstrada a proposição.
        \end{proof}
    \end{lemma}

    \begin{lemma}\label{contrapositive}
        $\entails(\alpha\to\beta)\to(\neg\beta\to\neg\alpha)$.
        \begin{proof}
            Pode ser provado pela seguinte sucessão de dedução:
            \footnotesize
            \begin{fitch}
                \fb\set{\alpha\to\beta,\neg\beta}\entails\beta\to\bot&$\mathbf{P_1}$\\
                \fa\set{\alpha\to\beta,\neg\beta}\entails(\beta\to\bot)\to\alpha\to(\beta\to\bot)&\hyperref[MA1]{$\mathbf{A_1}$}\\
                \fa\set{\alpha\to\beta,\neg\beta}\entails\alpha\to\beta\to\bot&$\hyperref[detachment]{\mathbf{R_1}}\;\set{1,2}$\\
                \fa\set{\alpha\to\beta,\neg\beta}\entails(\alpha\to\beta\to\bot)\to(\alpha\to\beta)\to(\alpha\to\bot)&\hyperref[MA2]{$\mathbf{A_2}$}\\
                \fa\set{\alpha\to\beta,\neg\beta}\entails\alpha\to\beta&$\mathbf{P_2}$\\
                \fa\set{\alpha\to\beta,\neg\beta}\entails(\alpha\to\beta)\to(\alpha\to\bot)&$\hyperref[detachment]{\mathbf{R_1}}\;\set{3,4}$\\
                \fa\set{\alpha\to\beta,\neg\beta}\entails\neg\alpha&$\hyperref[detachment]{\mathbf{R_1}}\;\set{5,6}$\\
                \fa\set{\alpha\to\beta}\entails\neg\beta\to\neg\alpha&\refer{deduction}{T}$\;\set{7}$\\
                \fa\entails(\alpha\to\beta)\to(\neg\beta\to\neg\alpha)&\refer{deduction}{T}$\;\set{8}$.
            \end{fitch}
            \normalsize
            Estando assim demonstrada a proposição.
        \end{proof}
    \end{lemma}

    \begin{lemma}\label{and-intro}
        $\vdash(\alpha\to\beta)\to(\alpha\to\gamma)\to\alpha\to\beta\wedge\gamma$.
        \begin{proof}
            Pode ser provado pela seguinte sucessão de dedução:
            \footnotesize
            \begin{fitch}
                \fb\set{\alpha\to\beta,\alpha\to\gamma,\alpha}\vdash\alpha&$\mathbf{P_1}$\\
                \fa\set{\alpha\to\beta,\alpha\to\gamma,\alpha}\vdash\alpha\to\beta&$\mathbf{P_3}$\\
                \fa\set{\alpha\to\beta,\alpha\to\gamma,\alpha}\vdash\beta&$\hyperref[detachment]{\mathbf{R_1}}\;\set{1, 2}$\\
                \fa\set{\alpha\to\beta,\alpha\to\gamma,\alpha}\vdash\alpha\to\gamma&$\mathbf{P_2}$\\
                \fa\set{\alpha\to\beta,\alpha\to\gamma,\alpha}\vdash\gamma&$\hyperref[detachment]{\mathbf{R_1}}\;\set{1, 4}$\\
                \fa\set{\alpha\to\beta,\alpha\to\gamma,\alpha}\vdash\beta\to\gamma\to\beta\wedge\gamma&\hyperref[MA3]{$\mathbf{A_3}$}\\
                \fa\set{\alpha\to\beta,\alpha\to\gamma,\alpha}\vdash\gamma\to\beta\wedge\gamma&$\hyperref[detachment]{\mathbf{R_1}}\;\set{3, 6}$\\
                \fa\set{\alpha\to\beta,\alpha\to\gamma,\alpha}\vdash\beta\wedge\gamma&$\hyperref[detachment]{\mathbf{R_1}}\;\set{5, 7}$\\
                \fa\set{\alpha\to\beta,\alpha\to\gamma}\vdash\alpha\to\beta\wedge\gamma&\refer{deduction}{T}$\;\set{8}$\\
                \fa\set{\alpha\to\beta}\vdash(\alpha\to\gamma)\to\alpha\to\beta\wedge\gamma&\refer{deduction}{T}$\;\set{9}$\\
                \fa\vdash(\alpha\to\beta)\to(\alpha\to\gamma)\to\alpha\to\beta\wedge\gamma&\refer{deduction}{T}$\;\set{10}$.
            \end{fitch}
            \normalsize
            Estando assim demonstrada a proposição.
        \end{proof}
    \end{lemma}

    \begin{lemma}\label{nec-distr}
        $\vdash\nec(\alpha\wedge\beta)\to\nec\alpha\wedge\nec\beta$.
        \begin{proof}
            Pode ser provado pela seguinte sucessão de dedução:
            \footnotesize
            \begin{fitch}
                \fb\entails\alpha\wedge\beta\to\alpha&\hyperref[MA4]{$\mathbf{A_4}$}\\
                \fa\entails\nec(\alpha\wedge\beta\to\alpha)&$\hyperref[necessitation]{\mathbf{R_2}}\;\set{1}$\\
                \fa\entails\nec(\alpha\wedge\beta\to\alpha)\to(\nec(\alpha\wedge\beta)\to\nec\alpha)&\hyperref[MB1]{$\mathbf{K}$}\\
                \fa\entails\nec(\alpha\wedge\beta)\to\nec\alpha&$\hyperref[detachment]{\mathbf{R_1}}\;\set{2, 3}$\\
                \fa\entails\alpha\wedge\beta\to\beta&\hyperref[MA5]{$\mathbf{A_5}$}\\
                \fa\entails\nec(\alpha\wedge\beta\to\beta)&$\hyperref[necessitation]{\mathbf{R_2}}\;\set{5}$\\
                \fa\entails\nec(\alpha\wedge\beta\to\beta)\to(\nec(\alpha\wedge\beta)\to\nec\beta)&\hyperref[MB1]{$\mathbf{K}$}\\
                \fa\entails\nec(\alpha\wedge\beta)\to\nec\beta&$\hyperref[detachment]{\mathbf{R_1}}\;\set{6, 7}$\\
                \fa\entails(\nec(\alpha\wedge\beta)\to\nec\alpha)\to(\nec(\alpha\wedge\beta)\to\nec\beta)\to\nec(\alpha\wedge\beta)\to\nec\alpha\wedge\nec\beta&\refer{and-intro}{L}\\
                \fa\entails(\nec(\alpha\wedge\beta)\to\nec\beta)\to\nec(\alpha\wedge\beta)\to\nec\alpha\wedge\nec\beta&$\hyperref[detachment]{\mathbf{R_1}}\;\set{4, 9}$\\
                \fa\entails\nec(\alpha\wedge\beta)\to\nec\alpha\wedge\nec\beta&$\hyperref[detachment]{\mathbf{R_1}}\;\set{6,9}$\\
            \end{fitch}
            \normalsize
            Estando assim demonstrada a proposição.
        \end{proof}
    \end{lemma}

    \begin{lemma}\label{nec-undistr}
        $\entails\nec\alpha\wedge\nec\beta\to\nec(\alpha\wedge\beta)$.
        \begin{proof}
            Pode ser provado pela seguinte sucessão de dedução:
            \footnotesize
            \begin{fitch}
                \fb\set{\nec\alpha,\nec\beta}\entails\nec\alpha&$\mathbf{P_2}$\\
                \fa\set{\nec\alpha,\nec\beta}\entails\nec\alpha\to\alpha&\hyperref[MB2]{$\mathbf{T}$}\\
                \fa\set{\nec\alpha,\nec\beta}\entails\alpha&$\hyperref[detachment]{\mathbf{R_1}}\;\set{1,2}$\\
                \fa\set{\nec\alpha,\nec\beta}\entails\nec\beta&$\mathbf{P_1}$\\
                \fa\set{\nec\alpha,\nec\beta}\entails\nec\beta\to\beta&\hyperref[MB2]{$\mathbf{T}$}\\
                \fa\set{\nec\alpha,\nec\beta}\entails\beta&$\hyperref[detachment]{\mathbf{R_1}}\;\set{4,5}$\\
                \fa\set{\nec\alpha,\nec\beta}\entails\alpha\to\beta\to\alpha\wedge\beta&\hyperref[MA3]{$\mathbf{A_3}$}\\
                \fa\set{\nec\alpha,\nec\beta}\entails\beta\to\alpha\wedge\beta&$\hyperref[detachment]{\mathbf{R_1}}\;\set{3,7}$\\
                \fa\set{\nec\alpha,\nec\beta}\entails\alpha\wedge\beta&$\hyperref[detachment]{\mathbf{R_1}}\;\set{6,8}$\\
                \fa\set{\nec\alpha,\nec\beta}\entails\nec(\alpha\wedge\beta)&$\hyperref[gen-nec]{\mathbf{T_{\getrefnumber{gen-nec}}}}\;\set{9}$\\
                \fa\set{\nec\alpha\wedge\nec\beta}\entails\nec(\alpha\wedge\beta)&$\hyperref[conjunctiondeduction]{\mathbf{T_{\getrefnumber{conjunctiondeduction}}}}\;\set{10}$\\
                \fa\entails\nec\alpha\wedge\nec\beta\to\nec(\alpha\wedge\beta)&$\hyperref[deduction]{\mathbf{T_{\getrefnumber{deduction}}}}\;\set{11}$\\
            \end{fitch}
            \normalsize
            Estando assim demonstrada a proposição.
        \end{proof}
    \end{lemma}

    \begin{lemma}
        $\vdash\nec(\alpha\to\beta)\to\nec\alpha\to\beta$.
        \begin{proof}
            Pode ser provado pela seguinte sucessão de dedução:
            \footnotesize
            \begin{fitch}
                \fb\set{\nec(\alpha\to\beta),\nec\alpha}\entails\nec\alpha&$\mathbf{P_2}$\\
                \fa\set{\nec(\alpha\to\beta),\nec\alpha}\entails\nec\alpha\to\alpha&$\hyperref[MB2]{\mathbf{T}}$\\
                \fa\set{\nec(\alpha\to\beta),\nec\alpha}\entails\alpha&$\hyperref[detachment]{\mathbf{R_1}}\;\set{1,2}$\\
                \fa\set{\nec(\alpha\to\beta),\nec\alpha}\entails\nec(\alpha\to\beta)&$\mathbf{P_1}$\\
                \fa\set{\nec(\alpha\to\beta),\nec\alpha}\entails\nec(\alpha\to\beta)\to\alpha\to\beta&$\hyperref[MB2]{\mathbf{T}}$\\
                \fa\set{\nec(\alpha\to\beta),\nec\alpha}\entails\alpha\to\beta&$\hyperref[detachment]{\mathbf{R_1}}\;\set{4,5}$\\
                \fa\set{\nec(\alpha\to\beta),\nec\alpha}\entails\beta&$\hyperref[detachment]{\mathbf{R_1}}\;\set{3,6}$\\
                \fa\set{\nec(\alpha\to\beta)}\entails\nec\alpha\to\beta&$\hyperref[deduction]{\mathbf{T_1}}\;\set{7}$\\
                \fa\entails\nec(\alpha\to\beta)\to\nec\alpha\to\beta&$\hyperref[deduction]{\mathbf{T_1}}\;\set{8}$.
            \end{fitch}
            \vspace*{-18pt-0.7em}
            \qedhere
        \end{proof}
    \end{lemma}

    \begin{lemma}\label{comp}
        $\entails(\alpha\to\beta)\to(\beta\to\gamma)\to\alpha\to\gamma$
        \begin{proof}
            Pode ser provado pela seguinte sucessão de dedução:
            \footnotesize
            \begin{fitch}
                \fb\set{\alpha\to\beta,\beta\to\gamma,\alpha}\entails\alpha&$\mathbf{P_1}$\\
                \fa\set{\alpha\to\beta,\beta\to\gamma,\alpha}\entails\alpha\to\beta&$\mathbf{P_3}$\\
                \fa\set{\alpha\to\beta,\beta\to\gamma,\alpha}\entails\beta&$\hyperref[detachment]{\mathbf{R_1}}\;\set{1,2}$\\
                \fa\set{\alpha\to\beta,\beta\to\gamma,\alpha}\entails\beta\to\gamma&$\mathbf{P_2}$\\
                \fa\set{\alpha\to\beta,\beta\to\gamma,\alpha}\entails\gamma&$\hyperref[detachment]{\mathbf{R_1}}\;\set{3,4}$\\
                \fa\set{\alpha\to\beta,\beta\to\gamma}\entails\alpha\to\gamma&$\hyperref[deduction]{\mathbf{T_{\getrefnumber{deduction}}}}\;\set{5}$\\
                \fa\set{\alpha\to\beta}\entails(\beta\to\gamma)\to\alpha\to\gamma&$\hyperref[deduction]{\mathbf{T_{\getrefnumber{deduction}}}}\;\set{6}$\\
                \fa\entails(\alpha\to\beta)\to(\beta\to\gamma)\to\alpha\to\gamma&$\hyperref[deduction]{\mathbf{T_{\getrefnumber{deduction}}}}\;\set{7}$\\
            \end{fitch}
            \normalsize
            Estando assim demonstrada a proposição.
        \end{proof}
    \end{lemma}

    \begin{lemma}\label{neg-intro}
        $\entails\alpha\to\neg\neg\alpha$
        \begin{proof}
            Pode ser provado pela seguinte sucessão de dedução:
            \footnotesize
            \begin{fitch}
                \fb\set{\alpha,\neg\alpha}\entails\alpha&$\mathbf{P_2}$\\
                \fa\set{\alpha,\neg\alpha}\entails\alpha\to\bot&$\mathbf{P_1}$\\
                \fa\set{\alpha,\neg\alpha}\entails\bot&$\hyperref[detachment]{\mathbf{R_1}}\;\set{1,2}$\\
                \fa\set{\alpha}\entails\neg\neg\alpha&$\hyperref[deduction]{\mathbf{T_{\getrefnumber{deduction}}}}\;\set{3}$\\
                \fa\entails\alpha\to\neg\neg\alpha&$\hyperref[deduction]{\mathbf{T_{\getrefnumber{deduction}}}}\;\set{4}$\\
            \end{fitch}
            \normalsize
            Estando assim demonstrada a proposição.
        \end{proof}
    \end{lemma}

    \begin{lemma}\label{or-left}
        $\entails(\alpha\to\beta)\to\alpha\to\beta\vee\gamma$.
        \begin{proof}
            Pode ser provado pela seguinte sucessão de dedução:
            \footnotesize 
            \begin{fitch}
                \fb\set{\alpha\to\beta,\alpha}\entails\alpha&$\mathbf{P_1}$\\
                \fa\set{\alpha\to\beta,\alpha}\entails\alpha\to\beta&$\mathbf{P_2}$\\
                \fa\set{\alpha\to\beta,\alpha}\entails\beta&$\hyperref[detachment]{\mathbf{R_1}}\;\set{1,2}$\\
                \fa\set{\alpha\to\beta,\alpha}\entails\beta\to\beta\vee\gamma&$\hyperref[MA6]{\mathbf{A_6}}$\\
                \fa\set{\alpha\to\beta,\alpha}\entails\beta\vee\gamma&$\hyperref[detachment]{\mathbf{R_1}}\;\set{3,4}$\\
                \fa\set{\alpha\to\beta}\entails\alpha\to\beta\vee\gamma&$\hyperref[deduction]{\mathbf{T_{\getrefnumber{deduction}}}}\;\set{5}$\\
                \fa\entails(\alpha\to\beta)\to\alpha\to\beta\vee\gamma&$\hyperref[deduction]{\mathbf{T_{\getrefnumber{deduction}}}}\;\set{6}$\\
            \end{fitch}
            \normalsize
            Estando assim demonstrada a proposição.
        \end{proof}
    \end{lemma}

    \begin{lemma}\label{or-right}
        $\entails(\alpha\to\beta)\to\alpha\to\gamma\vee\beta$.
        \begin{proof}
            Pode ser provado pela seguinte sucessão de dedução:
            \footnotesize 
            \begin{fitch}
                \fb\set{\alpha\to\beta,\alpha}\entails\alpha&$\mathbf{P_1}$\\
                \fa\set{\alpha\to\beta,\alpha}\entails\alpha\to\beta&$\mathbf{P_2}$\\
                \fa\set{\alpha\to\beta,\alpha}\entails\beta&$\hyperref[detachment]{\mathbf{R_1}}\;\set{1,2}$\\
                \fa\set{\alpha\to\beta,\alpha}\entails\beta\to\gamma\vee\beta&$\hyperref[MA7]{\mathbf{A_7}}$\\
                \fa\set{\alpha\to\beta,\alpha}\entails\gamma\vee\beta&$\hyperref[detachment]{\mathbf{R_1}}\;\set{3,4}$\\
                \fa\set{\alpha\to\beta}\entails\alpha\to\gamma\vee\beta&$\hyperref[deduction]{\mathbf{T_{\getrefnumber{deduction}}}}\;\set{5}$\\
                \fa\entails(\alpha\to\beta)\to\alpha\to\gamma\vee\beta&$\hyperref[deduction]{\mathbf{T_{\getrefnumber{deduction}}}}\;\set{6}$\\
            \end{fitch}
            \normalsize
            Estando assim demonstrada a proposição.
        \end{proof}
    \end{lemma}

    \begin{lemma}\label{or-subst}
        $\entails(\alpha\to\gamma)\to(\beta\to\delta)\to\alpha\vee\beta\to\gamma\vee\delta$.
        \begin{proof}
            Pode ser provado pela seguinte sucessão de dedução:
            \footnotesize 
            \begin{fitch}
                \fb\set{\alpha\to\gamma,\beta\to\delta,\alpha\vee\beta}\entails\alpha\to\gamma&$\mathbf{P_3}$\\
                \fa\set{\alpha\to\gamma,\beta\to\delta,\alpha\vee\beta}\entails(\alpha\to\gamma)\to\alpha\to\gamma\vee\delta&$\hyperref[or-left]{\mathbf{L_{\getrefnumber{or-left}}}}$\\
                \fa\set{\alpha\to\gamma,\beta\to\delta,\alpha\vee\beta}\entails\alpha\to\gamma\vee\delta&$\hyperref[detachment]{\mathbf{R_1}}\;\set{1,2}$\\
                \fa\set{\alpha\to\gamma,\beta\to\delta,\alpha\vee\beta}\entails\beta\to\delta&$\mathbf{P_2}$\\
                \fa\set{\alpha\to\gamma,\beta\to\delta,\alpha\vee\beta}\entails(\beta\to\delta)\to\beta\to\gamma\vee\delta&$\hyperref[or-right]{\mathbf{L_{\getrefnumber{or-right}}}}$\\
                \fa\set{\alpha\to\gamma,\beta\to\delta,\alpha\vee\beta}\entails\beta\to\gamma\vee\delta&$\hyperref[detachment]{\mathbf{R_1}}\;\set{4,5}$\\
                \fa\set{\alpha\to\gamma,\beta\to\delta,\alpha\vee\beta}\entails\alpha\vee\beta&$\mathbf{P_1}$\\
                \fa\set{\alpha\to\gamma,\beta\to\delta,\alpha\vee\beta}\entails(\alpha\to\gamma\vee\delta)\to(\beta\to\gamma\vee\delta)\to\alpha\vee\beta\to\gamma\vee\delta&$\hyperref[MA8]{\mathbf{A_8}}$\\
                \fa\set{\alpha\to\gamma,\beta\to\delta,\alpha\vee\beta}\entails(\beta\to\gamma\vee\delta)\to\alpha\vee\beta\to\gamma\vee\delta&$\hyperref[detachment]{\mathbf{R_1}}\;\set{3,8}$\\
                \fa\set{\alpha\to\gamma,\beta\to\delta,\alpha\vee\beta}\entails\alpha\vee\beta\to\gamma\vee\delta&$\hyperref[detachment]{\mathbf{R_1}}\;\set{6,9}$\\
                \fa\set{\alpha\to\gamma,\beta\to\delta,\alpha\vee\beta}\entails\gamma\vee\delta&$\hyperref[detachment]{\mathbf{R_1}}\;\set{7,10}$\\
                \fa\set{\alpha\to\gamma,\beta\to\delta}\entails\alpha\vee\beta\to\gamma\vee\delta&$\hyperref[deduction]{\mathbf{T_{\getrefnumber{deduction}}}}\;\set{11}$\\
                \fa\set{\alpha\to\gamma}\entails(\beta\to\delta)\to\alpha\vee\beta\to\gamma\vee\delta&$\hyperref[deduction]{\mathbf{T_{\getrefnumber{deduction}}}}\;\set{12}$\\
                \fa\entails(\alpha\to\gamma)\to(\beta\to\delta)\to\alpha\vee\beta\to\gamma\vee\delta&$\hyperref[deduction]{\mathbf{T_{\getrefnumber{deduction}}}}\;\set{13}$\\
            \end{fitch}
            \normalsize
            Estando assim demonstrada a proposição.
        \end{proof}
    \end{lemma}

    \begin{lemma}\label{or-undistr}
        $\entails\nec\alpha\vee\nec\beta\to\nec(\alpha\vee\beta)$.
        \begin{proof}
            Pode ser provado pela seguinte sucessão de dedução:
            \footnotesize 
            \begin{fitch}
                \fb\set{\nec\alpha}\entails\nec\alpha&$\mathbf{P_1}$\\
                \fa\set{\nec\alpha}\entails\nec\alpha\to\alpha&$\hyperref[MB2]{\mathbf{T}}$\\
                \fa\set{\nec\alpha}\entails\alpha&$\hyperref[detachment]{\mathbf{R_1}}\;\set{1,2}$\\
                \fa\set{\nec\alpha}\entails\alpha\to\alpha\vee\beta&$\hyperref[MA4]{\mathbf{A_4}}$\\
                \fa\set{\nec\alpha}\entails\alpha\vee\beta&$\hyperref[detachment]{\mathbf{R_1}}\;\set{3,4}$\\
                \fa\set{\nec\alpha}\entails\nec(\alpha\vee\beta)&$\hyperref[necessitation]{\mathbf{R_2}}\;\set{5}$\\
                \fa\set{\nec\beta}\entails\nec\beta&$\mathbf{P_1}$\\
                \fa\set{\nec\beta}\entails\nec\beta\to\beta&$\hyperref[MB2]{\mathbf{T}}$\\
                \fa\set{\nec\beta}\entails\beta&$\hyperref[detachment]{\mathbf{R_1}}\;\set{7,8}$\\
                \fa\set{\nec\beta}\entails\beta\to\alpha\vee\beta&$\hyperref[MA5]{\mathbf{A_5}}$\\
                \fa\set{\nec\beta}\entails\alpha\vee\beta&$\hyperref[detachment]{\mathbf{R_1}}\;\set{9,10}$\\
                \fa\set{\nec\beta}\entails\nec(\alpha\vee\beta)&$\hyperref[necessitation]{\mathbf{R_2}}\;\set{11}$\\
                \fa\set{\nec\alpha\vee\nec\beta}\entails\nec(\alpha\vee\beta)&$\hyperref[disjunctiondeduction]{\mathbf{T_{\getrefnumber{deduction}}}}\;\set{6,12}$\\
                \fa\entails\nec\alpha\vee\nec\beta\to\nec(\alpha\vee\beta)&$\hyperref[deduction]{\mathbf{T_{\getrefnumber{deduction}}}}\;\set{13}$\\
            \end{fitch}
            \normalsize
            Estando assim demonstrada a proposição.
        \end{proof}
    \end{lemma}

        \section{Dualidades}

    \babireski{Ver~\cite{Zach} acerca dos axiomas duais e suas derivações.}

    % \begin{theorem}
    %     $\vdash\nec(\alpha\to\beta)\to\pos\alpha\to\pos\beta$.

    %     \begin{proof}
    %         Pode ser provado pela seguinte sucessão de dedução:

    %         \begin{fitch}
    %             \fa\set{\nec(\alpha\to\beta),\pos\alpha}\vdash\pos\beta\\
    %             \fa\set{\nec(\alpha\to\beta)}\vdash\pos\alpha\to\pos\beta\\
    %             \fa\vdash\nec(\alpha\to\beta)\to\pos\alpha\to\pos\beta\\
    %         \end{fitch}
    %         \vspace*{-18pt-0.7em}
    %         \qedhere
    %     \end{proof}
    % \end{theorem}

    \begin{theorem}
        $\vdash\alpha\to\pos\alpha$.
        \begin{proof}
            Pode ser provado pela seguinte sucessão de dedução:

            \begin{fitch}
                \fa\entails\nec\neg\alpha\to\neg\alpha&$\hyperref[MB2]{\mathbf{B_2}}$\\
                \fa\entails(\nec\neg\alpha\to\neg\alpha)\to\neg\neg\alpha\to\pos\alpha&$\hyperref[contrapositive]{\mathbf{L_3}}$\\
                \fa\entails\neg\neg\alpha\to\pos\alpha&$\hyperref[detachment]{\mathbf{R_1}}\;\set{1,2}$\\
                \fa\entails(\neg\neg\alpha\to\neg\nec\neg\alpha)\to\alpha\to(\neg\neg\alpha\to\pos\alpha)&$\hyperref[MA1]{\mathbf{A_1}}$\\
                \fa\entails\alpha\to\neg\neg\alpha\to\pos\alpha&$\hyperref[detachment]{\mathbf{R_1}}\;\set{3,4}$\\
                \fa\entails(\alpha\to\neg\neg\alpha\to\pos\alpha)\to(\alpha\to\neg\neg\alpha)\to(\alpha\to\pos\alpha)&$\hyperref[MA2]{\mathbf{A_2}}$\\
                \fa\entails\alpha\to\neg\neg\alpha&\babireski{Provar.}\\
                \fa\entails(\alpha\to\neg\neg\alpha)\to(\alpha\to\pos\alpha)&$\hyperref[detachment]{\mathbf{R_1}}\;\set{5,6}$\\
                \fa\entails\alpha\to\pos\alpha&$\hyperref[detachment]{\mathbf{R_1}}\;\set{7,8}$.
            \end{fitch}
            \vspace*{-18pt-0.7em}
            \qedhere
        \end{proof}
    \end{theorem}

    \begin{theorem}
        $\vdash\pos\pos\alpha\to\pos\alpha$.
        \begin{proof}
            Pode ser provado pela seguinte sucessão de dedução:

            \begin{fitch}
                
                \fa\entails\neg\nec\alpha\to\neg\nec\pos\alpha\\
                \fa\entails\pos\pos\alpha\to\pos\alpha\\
            \end{fitch}
            \vspace*{-18pt-0.7em}
            \qedhere
        \end{proof}
    \end{theorem}

    Apesar da similaridades com as transformações naturais, deve-se destacar que as noções de computação não podem ser interpretadas simplesmente como necessidade ou possibilidade, uma vez que apresenta propriedades presente em ambas as modalidades. Neste sentido, a modalidade de \emph{laxidade} --- que combina noções de necessidade e possibilidade --- mostra-se uma melhor representação de efeitos computacionais sobre a interpretação programa-prova.
    
    Ao sistema que comporta essa modalidade --- denotada $\lax$ --- damos o nome de sistema laxo ou simplesmente $\mathbf{L}$. Este sistema foi primeiramente considerado por~\cite{Curry-A,Curry-B} e posteriormente redescoberto por~\cite{Fairtlough,Mendler} como uma tentativa de representar correção dentro de restrições na verificação formal de \emph{hardware} de computadores. Pode ser definido formalmente por meio da assinatura $\Sigma_\mathbf{L}=\sequence{\mathcal{P},\mathcal{C}_\mathbf{L}}$ e da axiomatização $\mathcal{H}=\sequence{\mathcal{A}_\mathbf{L},\mathcal{R}_\mathbf{I}}$, onde $\mathcal{C}_\mathbf{L}=\mathcal{C}_\mathbf{I}\cup\set{\circ^1}$ e $\mathcal{A}_\mathbf{L}=\mathcal{A}_\mathbf{I}\cup\set{\mathbf{C_1},\mathbf{C_2},\mathbf{C_3}}$, considerando-se os esquemas abaixo:
    \begin{alignat*}{3}
        &\mathbf{C_1}\quad&&\alpha\to\lax\alpha\\
        &\mathbf{C_2}\quad&&\lax\lax\alpha\to\lax\alpha\\
        &\mathbf{C_3}\quad&&(\alpha\to\beta)\to\lax\alpha\to\lax\beta
    \end{alignat*}

    Esse sistema, entretanto, pode ser interpretado modalmente por meio da seguinte tradução \citep{Pfenning}:

    \begin{definition}[$\bullet^+$] A tradução $\bullet^+:\mathcal{L}_\mathbf{M}\to\mathcal{L}_\mathbf{L}$ do sistema $\mathbf{S_4}$ intuicionista ao sistema $\mathbf{L}$ pode ser definida indutivamente da seguinte maneira:
        \begin{align*}
            a^+&\coloneq a\\
            \bot^+&\coloneq\bot\\
            {(\lax\alpha)}^+&\coloneq\pos\nec\alpha^+\\
            {(\alpha\to\beta)}^+&\coloneq\nec\alpha^+\to\beta^+
            \tag*{\qed} 
        \end{align*}
    \end{definition}
        \section{Isomorfismo entre as traduções}

Conforme afirmado anteriormente, ambas as traduções apresentadas neste trabalho equivalem --- ou seja, são isomorfas --- na forma $\entails\nec\alpha^\circ\leftrightarrow\alpha^\medsquare$. Nesta seção, provaremos este isomorfismo que, não somente constitui puramente um resultado de interesse, como permite tornar a prova de propriedades de uma tradução triviais caso tais propriedades valham para a outra tradução.

\begin{theorem}\label{isomorphism}
    $\entails\nec\alpha^\circ\leftrightarrow\alpha^\medsquare$.

    \begin{proof}
        Prova por indução forte sobre a profundidade de $\alpha$ \citep{Troelstra}.
        Assim, suponhamos que as traduções equivalham para qualquer $\alpha$ de profundidade $n<k$.
        Demonstraremos analisando-se os casos e valendo-se da suposição acima --- doravante chamada $\mathbf{H}$ --- o passo de indução, ou seja, que as traduções equivalem para qualquer $\alpha$ de profundidade $n=k$.

        \begin{case}
            \textsc{Caso 1.}
            Se a sentença $\alpha$ for uma proposição $a\in\mathcal{P}$, sabe-se que $\nec a^\circ=\nec a$ e que $a^\medsquare=\nec a$ pelas definições das traduções.
            Deste modo, tanto a ida quanto a volta possuem a forma $\nec a\to\nec a$ e podem ser provadas pelo lema \hyperref[identity]{$\mathbf{L_\getrefnumber{identity}}$}.
            Ambas as implicações posteriormente podem ser unidas em uma bi-implicação por meio do esquema \hyperref[MA3]{$\mathbf{A_3}$}.
        \end{case}

        \begin{case}
            \textsc{Caso 2.}
            Se a sentença $\alpha$ for a constante $\bot$, sabe-se que $\nec\bot^\circ=\nec\bot$ e que $\bot^\medsquare=\bot$ pelas definições das traduções.
            Deste modo, a ida $\nec\bot\to\bot$ constitui um axioma gerado pelo esquema \hyperref[MB2]{$\mathbf{B_2}$} --- sendo assim provada trivialmente --- e a volta $\bot\to\nec\bot$ pode ser provada pelo lema \hyperref[explosion]{$\mathbf{L_2}$}.
            Ambas as implicações posteriormente podem ser unidas em uma bi-implicação por meio do esquema \hyperref[MA3]{$\mathbf{A_3}$}.
        \end{case}

        \begin{case}
            \textsc{Caso 3.}
            Se a sentença $\alpha$ for o resultado da conjunção de duas outras sentenças $\varphi$ e $\psi$, sabe-se que $\nec{(\varphi\wedge\psi)}^\circ=\nec(\varphi^\circ\wedge\psi^\circ)$ e que ${(\varphi\wedge\psi)}^\medsquare=\varphi^\medsquare\wedge\psi^\medsquare$ pelas definições das traduções.
            Separaremos a prova em dois casos: um para a ida $\nec(\varphi^\circ\wedge\psi^\circ)\to\varphi^\medsquare\wedge\psi^\medsquare$ e outro para a volta $\varphi^\medsquare\wedge\psi^\medsquare\to\nec(\varphi^\circ\wedge\psi^\circ)$. Ambas as implicações posteriormente podem ser unidas em uma bi-implicação por meio do esquema da introdução da conjunção \hyperref[MA3]{$\mathbf{A_3}$}.
        \end{case}

            \begin{subcase}
                \textsc{Caso 3.1.}
                A partir de $\mathbf{H}$, temos que $\mathbf{H_1}={\entails\nec\varphi^\circ\to\varphi^\medsquare}$ e que $\mathbf{H_2}={\entails\nec\psi^\circ\to\psi^\medsquare}$ por meio dos esquemas da eliminação da conjunção e da aplicação da regra do \emph{modus ponens}.
                Valendo-se do listado acima em conjunto com alguns lemas, pode-se provar que $\entails\nec(\varphi^\circ\wedge\psi^\circ)\to\varphi^\medsquare\wedge\psi^\medsquare$ pela seguinte sucessão de dedução:

                \footnotesize
                \begin{fitch}
                    \fb\set{\nec(\varphi^\circ\wedge\psi^\circ)}\proves\nec(\varphi^\circ\wedge\psi^\circ)&$\mathbf{P_1}$\\
                    \fa\set{\nec(\varphi^\circ\wedge\psi^\circ)}\proves\nec(\varphi^\circ\wedge\psi^\circ)\to\nec\varphi^\circ\wedge\nec\psi^\circ&\refer{nec-distr}{L}\\
                    \fa\set{\nec(\varphi^\circ\wedge\psi^\circ)}\proves\nec\varphi^\circ\wedge\nec\psi^\circ&$\hyperref[detachment]{\mathbf{R_1}}\;\set{1,2}$\\
                    \fa\set{\nec(\varphi^\circ\wedge\psi^\circ)}\proves\nec\varphi^\circ\wedge\nec\psi^\circ\to\nec\varphi^\circ&\hyperref[MA4]{${\mathbf{A_4}}$}\\
                    \fa\set{\nec(\varphi^\circ\wedge\psi^\circ)}\proves\nec\varphi^\circ&$\hyperref[detachment]{\mathbf{R_1}}\;\set{3,4}$\\
                    \fa\set{\nec(\varphi^\circ\wedge\psi^\circ)}\proves\nec\varphi^\circ\to\varphi^\medsquare&$\mathbf{H_1}$\\
                    \fa\set{\nec(\varphi^\circ\wedge\psi^\circ)}\proves\varphi^\medsquare&$\hyperref[detachment]{\mathbf{R_1}}\;\set{5,6}$\\
                    \fa\set{\nec(\varphi^\circ\wedge\psi^\circ)}\proves\nec\varphi^\circ\wedge\nec\psi^\circ\to\nec\psi^\circ&\hyperref[MA4]{$\mathbf{A_4}$}\\
                    \fa\set{\nec(\varphi^\circ\wedge\psi^\circ)}\proves\nec\psi^\circ&$\hyperref[detachment]{\mathbf{R_1}}\;\set{3,8}$\\
                    \fa\set{\nec(\varphi^\circ\wedge\psi^\circ)}\proves\nec\psi^\circ\to\psi^\medsquare&$\mathbf{H_2}$\\
                    \fa\set{\nec(\varphi^\circ\wedge\psi^\circ)}\proves\psi^\medsquare&$\hyperref[detachment]{\mathbf{R_1}}\;\set{9,10}$\\
                    \fa\set{\nec(\varphi^\circ\wedge\psi^\circ)}\proves\varphi^\medsquare\to\psi^\medsquare\to\varphi^\medsquare\wedge\psi^\medsquare&\hyperref[MA3]{$\mathbf{A_3}$}\\
                    \fa\set{\nec(\varphi^\circ\wedge\psi^\circ)}\proves\psi^\medsquare\to\varphi^\medsquare\wedge\psi^\medsquare&$\hyperref[detachment]{\mathbf{R_1}}\;\set{7,12}$\\
                    \fa\set{\nec(\varphi^\circ\wedge\psi^\circ)}\proves\varphi^\medsquare\wedge\psi^\medsquare&$\hyperref[detachment]{\mathbf{R_1}}\;\set{9,13}$\\
                    \fa\proves\nec(\varphi^\circ\wedge\psi^\circ)\to\varphi^\medsquare\wedge\psi^\medsquare&$\hyperref[deduction]{\mathbf{T_\getrefnumber{deduction}}}\;\set{14}$\\
                \end{fitch}
            \end{subcase} 

            \begin{subcase}
                \textsc{Caso 3.2.}
                A partir de $\mathbf{H}$, temos que $\mathbf{H_1}={\entails\varphi^\medsquare\to\nec\varphi^\circ}$ e que $\mathbf{H_2}={\entails\psi^\medsquare\to\nec\psi^\circ}$ por meio dos esquemas da eliminação da conjunção e da aplicação regra do \emph{modus ponens}.
                Valendo-se do listado acima em conjunto com alguns lemas, pode-se provar que $\entails\varphi^\medsquare\wedge\psi^\medsquare\to\nec(\varphi^\circ\wedge\psi^\circ)$ pela seguinte sucessão de dedução:

                \footnotesize
                \begin{fitch}
                    \fb\set{\varphi^\medsquare\wedge\psi^\medsquare}\proves\varphi^\medsquare\wedge\psi^\medsquare&$\mathbf{P_1}$\\
                    \fa\set{\varphi^\medsquare\wedge\psi^\medsquare}\proves\varphi^\medsquare\wedge\psi^\medsquare\to\varphi^\medsquare&\hyperref[MA4]{${\mathbf{A_4}}$}\\
                    \fa\set{\varphi^\medsquare\wedge\psi^\medsquare}\proves\varphi^\medsquare&$\hyperref[detachment]{\mathbf{R_1}}\;\set{1,2}$\\
                    \fa\set{\varphi^\medsquare\wedge\psi^\medsquare}\proves\varphi^\medsquare\to\nec\varphi^\circ&$\mathbf{H_1}$\\
                    \fa\set{\varphi^\medsquare\wedge\psi^\medsquare}\proves\nec\varphi^\circ&$\hyperref[detachment]{\mathbf{R_1}}\;\set{3,4}$\\
                    \fa\set{\varphi^\medsquare\wedge\psi^\medsquare}\proves\varphi^\medsquare\wedge\psi^\medsquare\to\psi^\medsquare&\hyperref[MA5]{${\mathbf{A_5}}$}\\
                    \fa\set{\varphi^\medsquare\wedge\psi^\medsquare}\proves\psi^\medsquare&$\hyperref[detachment]{\mathbf{R_1}}\;\set{1,6}$\\
                    \fa\set{\varphi^\medsquare\wedge\psi^\medsquare}\proves\psi^\medsquare\to\nec\psi^\circ&$\mathbf{H_2}$\\
                    \fa\set{\varphi^\medsquare\wedge\psi^\medsquare}\proves\nec\psi^\circ&$\hyperref[detachment]{\mathbf{R_1}}\;\set{7,8}$\\
                    \fa\set{\varphi^\medsquare\wedge\psi^\medsquare}\proves\nec\varphi^\circ\to\nec\psi^\circ\to\nec\varphi^\circ\wedge\nec\psi^\circ&\hyperref[MA3]{${\mathbf{A_3}}$}\\
                    \fa\set{\varphi^\medsquare\wedge\psi^\medsquare}\proves\nec\psi^\circ\to\nec\varphi^\circ\wedge\nec\psi^\circ&$\hyperref[detachment]{\mathbf{R_1}}\;\set{5,10}$\\
                    \fa\set{\varphi^\medsquare\wedge\psi^\medsquare}\proves\nec\varphi^\circ\wedge\nec\psi^\circ&$\hyperref[detachment]{\mathbf{R_1}}\;\set{9,11}$\\
                    \fa\set{\varphi^\medsquare\wedge\psi^\medsquare}\proves\nec\varphi^\circ\wedge\nec\psi^\circ\to\nec(\varphi^\circ\wedge\psi^\circ)&\refer{nec-undistr}{L}\\
                    \fa\set{\varphi^\medsquare\wedge\psi^\medsquare}\proves\nec(\varphi^\circ\wedge\psi^\circ)&$\hyperref[detachment]{\mathbf{R_1}}\;\set{12,13}$\\
                    \fa\proves\varphi^\medsquare\wedge\psi^\medsquare\to\nec(\varphi^\circ\wedge\psi^\circ)&$\hyperref[deduction]{\mathbf{T_\getrefnumber{deduction}}}\;\set{14}$\\
                \end{fitch}
            \end{subcase}

        \begin{case}
            \textsc{Caso 4.}
            Se a sentença $\alpha$ for o resultado da disjunção de duas outras sentenças $\varphi$ e $\psi$, sabe-se que $\nec{(\varphi\vee\psi)}^\circ=\nec(\nec\varphi^\circ\vee\nec\psi^\circ)$ e que ${(\varphi\vee\psi)}^\medsquare=\varphi^\medsquare\vee\psi^\medsquare$ pelas definições das traduções.
            Separaremos a prova em dois casos: um para a ida $\nec(\nec\varphi^\circ\vee\nec\psi^\circ)\to\varphi^\medsquare\vee\psi^\medsquare$ e outro para a volta $\varphi^\medsquare\vee\psi^\medsquare\to\nec(\nec\varphi^\circ\vee\nec\psi^\circ)$.
            Ambas as implicações, então, podem ser unidas em uma bi-implicação por meio do esquema da introdução da conjunção \hyperref[MA3]{$\mathbf{A_3}$}.
        \end{case}

        \begin{subcase}
            \textsc{Caso 4.1.}
            A partir de $\mathbf{H}$, temos que $\mathbf{H_1}={\entails\nec\varphi^\circ\to\varphi^\medsquare}$ e que $\mathbf{H_2}={\entails\nec\psi^\circ\to\psi^\medsquare}$ por meio dos esquemas da eliminação da conjunção e da aplicação da regra do \emph{modus ponens}.
            Valendo-se do listado acima em conjunto com alguns lemas, pode-se provar que $\entails\nec(\nec\varphi^\circ\vee\nec\psi^\circ)\to\varphi^\medsquare\vee\psi^\medsquare$ pela seguinte sucessão de dedução, sendo $\chi=\nec(\nec\varphi^\circ\vee\nec\psi^\circ)$:
            \footnotesize
            \begin{fitch}
                \fb\set{\chi}\entails\nec\varphi^\circ\to\varphi^\medsquare&$\mathbf{H_1}$\\
                \fa\set{\chi}\entails\nec\psi^\circ\to\psi^\medsquare&$\mathbf{H_2}$\\
                \fa\set{\chi}\entails\nec(\nec\varphi^\circ\vee\nec\psi^\circ)&$\mathbf{P_1}$\\
                \fa\set{\chi}\entails\nec(\nec\varphi^\circ\vee\nec\psi^\circ)\to\nec\varphi^\circ\vee\nec\psi^\circ&\hyperref[MB2]{${\mathbf{B_2}}$}\\
                \fa\set{\chi}\entails\nec\varphi^\circ\vee\nec\psi^\circ&$\hyperref[detachment]{\mathbf{R_1}}\;\set{3,4}$\\
                \fa\set{\chi}\entails(\nec\varphi^\circ\to\varphi^\medsquare)\to(\nec\psi^\circ\to\psi^\medsquare)\to\nec\varphi^\circ\vee\nec\psi^\circ\to\varphi^\medsquare\vee\psi^\medsquare&\refer{or-subst}{L}\\
                \fa\set{\chi}\entails(\nec\psi^\circ\to\psi^\medsquare)\to\nec\varphi^\circ\vee\nec\psi^\circ\to\varphi^\medsquare\vee\psi^\medsquare&$\hyperref[detachment]{\mathbf{R_1}}\;\set{1,6}$\\
                \fa\set{\chi}\entails\nec\varphi^\circ\vee\nec\psi^\circ\to\varphi^\medsquare\vee\psi^\medsquare&$\hyperref[detachment]{\mathbf{R_1}}\;\set{2,7}$\\
                \fa\set{\chi}\entails\varphi^\medsquare\vee\psi^\medsquare&$\hyperref[detachment]{\mathbf{R_1}}\;\set{5,8}$\\
                \fa\entails\nec(\nec\varphi^\circ\vee\nec\psi^\circ)\to\varphi^\medsquare\vee\psi^\medsquare&$\hyperref[deduction]{\mathbf{T_\getrefnumber{deduction}}}\;\set{8}$\\
            \end{fitch}
        \end{subcase}

        \begin{subcase}
            \textsc{Caso 4.2.}
            A partir de $\mathbf{H}$, temos que $\mathbf{H_1}={\entails\varphi^\medsquare\to\nec\varphi^\circ}$ e que $\mathbf{H_2}={\entails\psi^\medsquare\to\nec\psi^\circ}$ por meio dos esquemas da eliminação da conjunção e da aplicação regra do \emph{modus ponens}.
            Valendo-se do listado acima em conjunto com alguns lemas, pode-se provar que $\entails\varphi^\medsquare\vee\psi^\medsquare\to\nec(\nec\varphi^\circ\vee\nec\psi^\circ)$ pela seguinte sucessão de dedução, sendo $\chi=\varphi^\medsquare\vee\psi^\medsquare$.
            \footnotesize
            \begin{fitch}
                \fb\set{\chi}\entails\varphi^\medsquare\to\nec\varphi^\circ&$\mathbf{H_1}$\\
                \fa\set{\chi}\entails\nec\varphi^\circ\to\nec\nec\varphi^\circ&\hyperref[MB3]{${\mathbf{B_3}}$}\\
                \fa\set{\chi}\entails(\varphi^\medsquare\to\nec\varphi^\circ)\to(\nec\varphi^\circ\to\nec\nec\varphi^\circ)\to\varphi^\medsquare\to\nec\nec\varphi^\circ&\refer{comp}{L}\\
                \fa\set{\chi}\entails(\nec\varphi^\circ\to\nec\nec\varphi^\circ)\to\varphi^\medsquare\to\nec\nec\varphi^\circ&$\hyperref[detachment]{\mathbf{R_1}}\;\set{1,3}$\\
                \fa\set{\chi}\entails\varphi^\medsquare\to\nec\nec\varphi^\circ&$\hyperref[detachment]{\mathbf{R_1}}\;\set{2,4}$\\
                \fa\set{\chi}\entails\psi^\medsquare\to\nec\psi^\circ&$\mathbf{H_2}$\\
                \fa\set{\chi}\entails\nec\psi^\circ\to\nec\nec\psi^\circ&\hyperref[MB3]{${\mathbf{B_3}}$}\\
                \fa\set{\chi}\entails(\psi^\medsquare\to\nec\psi^\circ)\to(\nec\psi^\circ\to\nec\nec\psi^\circ)\to\psi^\medsquare\to\nec\nec\psi^\circ&\refer{comp}{L}\\
                \fa\set{\chi}\entails(\nec\psi^\circ\to\nec\nec\psi^\circ)\to\psi^\medsquare\to\nec\nec\psi^\circ&$\hyperref[detachment]{\mathbf{R_1}}\;\set{6,8}$\\
                \fa\set{\chi}\entails\psi^\medsquare\to\nec\nec\psi^\circ&$\hyperref[detachment]{\mathbf{R_1}}\;\set{7,9}$\\

                \fa\set{\chi}\entails\varphi^\medsquare\vee\psi^\medsquare&$\mathbf{P_1}$\\
                \fa\set{\chi}\entails(\varphi^\medsquare\to\nec^2\varphi^\circ)\to(\psi^\medsquare\to\nec^2\psi^\circ)\to\varphi^\medsquare\vee\psi^\medsquare\to\nec^2\varphi^\circ\vee\nec^2\psi^\circ&\refer{or-subst}{L}\\
                \fa\set{\chi}\entails(\psi^\medsquare\to\nec\nec\psi^\circ)\to\varphi^\medsquare\vee\psi^\medsquare\to\nec\nec\varphi^\circ\vee\nec\nec\psi^\circ&$\hyperref[detachment]{\mathbf{R_1}}\;\set{5,12}$\\
                \fa\set{\chi}\entails\varphi^\medsquare\vee\psi^\medsquare\to\nec\nec\varphi^\circ\vee\nec\nec\psi^\circ&$\hyperref[detachment]{\mathbf{R_1}}\;\set{10,13}$\\
                \fa\set{\chi}\entails\nec\nec\varphi^\circ\vee\nec\nec\psi^\circ&$\hyperref[detachment]{\mathbf{R_1}}\;\set{11,14}$\\
                \fa\set{\chi}\entails\nec\nec\varphi^\circ\vee\nec\nec\psi^\circ\to\nec(\nec\varphi^\circ\vee\nec\psi^\circ)&\refer{or-undistr}{L}\\
                \fa\set{\chi}\entails\nec(\nec\varphi^\circ\vee\nec\psi^\circ)&$\hyperref[detachment]{\mathbf{R_1}}\;\set{15,16}$\\
                \fa\entails\varphi^\medsquare\vee\psi^\medsquare\to\nec(\nec\varphi^\circ\vee\nec\psi^\circ)&$\hyperref[deduction]{\mathbf{T_\getrefnumber{deduction}}}\;\set{17}$
            \end{fitch}
        \end{subcase}

        \begin{case}
            \textsc{Caso 5.}
            Se a sentença $\alpha$ for o resultado da implicação de uma sentença $\varphi$ a uma sentença $\psi$, sabe-se que $\nec{(\varphi\to\psi)}^\circ=\nec(\nec\varphi^\circ\to\psi^\circ)$ e que ${(\varphi\to\psi)}^\medsquare=\nec(\varphi^\medsquare\to\psi^\medsquare)$ pelas definições das traduções.
            Separaremos a prova em dois casos: um para a ida $\nec(\nec\varphi^\circ\to\psi^\circ)\to\nec(\varphi^\medsquare\to\psi^\medsquare)$ e outro para a volta $\nec(\varphi^\medsquare\to\psi^\medsquare)\to\nec(\nec\varphi^\circ\to\psi^\circ)$.
            Ambas as implicações, então, podem ser unidas em uma bi-implicação por meio do esquema \hyperref[MA3]{$\mathbf{A_3}$}.
        \end{case}

            \begin{subcase}
                \textsc{Caso 5.1.}
                A partir de $\mathbf{H}$, temos que $\mathbf{H_1}={\entails\varphi^\medsquare\to\nec\varphi^\circ}$ e que $\mathbf{H_2}={\entails\nec\psi^\circ\to\psi^\medsquare}$ por meio dos esquemas da eliminação da conjunção e da aplicação regra do \emph{modus ponens}.
                Valendo-se do listado acima em conjunto com alguns lemas, pode-se provar que $\entails\nec(\nec\varphi^\circ\to\psi^\circ)\to\nec(\psi^\medsquare\to\psi^\medsquare)$ pela seguinte sucessão de dedução:

                \footnotesize
                \begin{fitch}
                    \fb\set{\nec(\nec\varphi^\circ\to\psi^\circ)}\entails\nec(\nec\varphi^\circ\to\psi^\circ)&$\mathbf{P_1}$\\
                    \fa\set{\nec(\nec\varphi^\circ\to\psi^\circ)}\entails\nec(\nec\varphi^\circ\to\psi^\circ)\to\nec\nec\varphi^\circ\to\nec\psi^\circ&\hyperref[MB1]{${\mathbf{B_1}}$}\\
                    \fa\set{\nec(\nec\varphi^\circ\to\psi^\circ)}\entails\nec\varphi^\circ\to\nec\nec\varphi^\circ&\hyperref[MB3]{${\mathbf{B_3}}$}\\
                    \fa\set{\nec(\nec\varphi^\circ\to\psi^\circ)}\entails\nec\nec\varphi^\circ\to\nec\psi^\circ&\hyperref[MB2]{${\mathbf{B_2}}$}\\
                    \fa\set{\nec(\nec\varphi^\circ\to\psi^\circ)}\entails(\nec\varphi^\circ\to\nec^2\varphi^\circ)\to(\nec^2\varphi^\circ\to\nec\psi^\circ)\to\nec\varphi^\circ\to\nec\psi^\circ&\refer{comp}{L}\\
                    \fa\set{\nec(\nec\varphi^\circ\to\psi^\circ)}\entails(\nec\nec\varphi^\circ\to\nec\psi^\circ)\to\nec\varphi^\circ\to\nec\psi^\circ&$\hyperref[detachment]{\mathbf{R_1}}\;\set{3,5}$\\
                    \fa\set{\nec(\nec\varphi^\circ\to\psi^\circ)}\entails\varphi^\medsquare\to\nec\varphi^\circ&$\mathbf{H_1}$\\
                    \fa\set{\nec(\nec\varphi^\circ\to\psi^\circ)}\entails\nec\varphi^\circ\to\nec\psi^\circ&$\hyperref[detachment]{\mathbf{R_1}}\;\set{4,6}$\\
                    \fa\set{\nec(\nec\varphi^\circ\to\psi^\circ)}\entails(\varphi^\medsquare\to\nec\varphi^\circ)\to(\nec\varphi^\circ\to\nec\psi^\circ)\to\varphi^\medsquare\to\nec\psi^\circ&\refer{comp}{L}\\
                    \fa\set{\nec(\nec\varphi^\circ\to\psi^\circ)}\entails(\nec\varphi^\circ\to\nec\psi^\circ)\to\varphi^\medsquare\to\nec\psi^\circ&$\hyperref[detachment]{\mathbf{R_1}}\;\set{7,9}$\\
                    \fa\set{\nec(\nec\varphi^\circ\to\psi^\circ)}\entails\varphi^\medsquare\to\nec\psi^\circ&$\hyperref[detachment]{\mathbf{R_1}}\;\set{8,10}$\\
                    \fa\set{\nec(\nec\varphi^\circ\to\psi^\circ)}\entails\nec\psi^\circ\to\psi^\medsquare&$\mathbf{H_2}$\\
                    \fa\set{\nec(\nec\varphi^\circ\to\psi^\circ)}\entails(\varphi^\medsquare\to\nec\psi^\circ)\to(\nec\psi^\circ\to\psi^\medsquare)\to\varphi^\medsquare\to\psi^\medsquare&\refer{comp}{L}\\
                    \fa\set{\nec(\nec\varphi^\circ\to\psi^\circ)}\entails(\nec\psi^\circ\to\psi^\medsquare)\to\varphi^\medsquare\to\psi^\medsquare&$\hyperref[detachment]{\mathbf{R_1}}\;\set{11,13}$\\
                    \fa\set{\nec(\nec\varphi^\circ\to\psi^\circ)}\entails\varphi^\medsquare\to\psi^\medsquare&$\hyperref[detachment]{\mathbf{R_1}}\;\set{12,14}$\\
                    \fa\set{\nec(\nec\varphi^\circ\to\psi^\circ)}\entails\nec(\varphi^\medsquare\to\psi^\medsquare)&$\hyperref[gen-nec]{\mathbf{T_{\getrefnumber{gen-nec}}}}\;\set{15}$\\
                    \fa\entails\nec(\nec\varphi^\circ\to\psi^\circ)\to\nec(\varphi^\medsquare\to\psi^\medsquare)&$\hyperref[deduction]{\mathbf{T_\getrefnumber{deduction}}}\;\set{16}$
                \end{fitch}
            \end{subcase}

            \begin{subcase}
                \textsc{Caso 5.2.}
                A partir de $\mathbf{H}$, temos que $\mathbf{H_1}=\nec\varphi^\circ\to\varphi^\medsquare$ e que $\mathbf{H_2}=\psi^\medsquare\to\nec\psi^\circ$ por meio dos esquemas da eliminação da conjunção e da aplicação regra do \emph{modus ponens}.
                Valendo-se do listado acima em conjunto com alguns lemas, pode-se provar que $\entails\nec(\psi^\medsquare\to\psi^\medsquare)\to\nec(\nec\varphi^\circ\to\psi^\circ)$ pela seguinte sucessão de dedução:

                \footnotesize
                \begin{fitch}
                    \fb\set{\nec(\varphi^\medsquare\to\psi^\medsquare),\nec\varphi^\circ}\entails\nec\varphi^\circ\to\varphi^\medsquare&$\mathbf{H_1}$\\
                    \fa\set{\nec(\varphi^\medsquare\to\psi^\medsquare),\nec\varphi^\circ}\entails\nec(\varphi^\medsquare\to\psi^\medsquare)&$\mathbf{P_2}$\\
                    \fa\set{\nec(\varphi^\medsquare\to\psi^\medsquare),\nec\varphi^\circ}\entails\nec(\varphi^\medsquare\to\psi^\medsquare)\to\varphi^\medsquare\to\psi^\medsquare&\hyperref[MB2]{${\mathbf{B_2}}$}\\
                    \fa\set{\nec(\varphi^\medsquare\to\psi^\medsquare),\nec\varphi^\circ}\entails\varphi^\medsquare\to\psi^\medsquare&$\hyperref[detachment]{\mathbf{R_1}}\;\set{2,3}$\\
                    \fa\set{\nec(\varphi^\medsquare\to\psi^\medsquare),\nec\varphi^\circ}\entails(\nec\varphi^\circ\to\varphi^\medsquare)\to(\varphi^\medsquare\to\psi^\medsquare)\to\nec\varphi^\circ\to\psi^\medsquare&\refer{comp}{L}\\
                    \fa\set{\nec(\varphi^\medsquare\to\psi^\medsquare),\nec\varphi^\circ}\entails(\varphi^\medsquare\to\psi^\medsquare)\to\nec\varphi^\circ\to\psi^\medsquare&$\hyperref[detachment]{\mathbf{R_1}}\;\set{1,5}$\\
                    \fa\set{\nec(\varphi^\medsquare\to\psi^\medsquare),\nec\varphi^\circ}\entails\nec\varphi^\circ&$\mathbf{P_1}$\\
                    \fa\set{\nec(\varphi^\medsquare\to\psi^\medsquare),\nec\varphi^\circ}\entails\nec\varphi^\circ\to\psi^\medsquare&$\hyperref[detachment]{\mathbf{R_1}}\;\set{4,6}$\\
                    \fa\set{\nec(\varphi^\medsquare\to\psi^\medsquare),\nec\varphi^\circ}\entails\psi^\medsquare&$\hyperref[detachment]{\mathbf{R_1}}\;\set{7,8}$\\
                    \fa\set{\nec(\varphi^\medsquare\to\psi^\medsquare),\nec\varphi^\circ}\entails\psi^\medsquare\to\nec\psi^\circ&$\mathbf{H_2}$\\
                    \fa\set{\nec(\varphi^\medsquare\to\psi^\medsquare),\nec\varphi^\circ}\entails\nec\psi^\circ&$\hyperref[detachment]{\mathbf{R_1}}\;\set{9,10}$\\
                    \fa\set{\nec(\varphi^\medsquare\to\psi^\medsquare),\nec\varphi^\circ}\entails\nec\psi^\circ\to\psi^\circ&\hyperref[MB2]{${\mathbf{B_2}}$}\\
                    \fa\set{\nec(\varphi^\medsquare\to\psi^\medsquare),\nec\varphi^\circ}\entails\psi^\circ&$\hyperref[detachment]{\mathbf{R_1}}\;\set{11,12}$\\
                    \fa\set{\nec(\varphi^\medsquare\to\psi^\medsquare)}\entails\nec(\nec\varphi^\circ\to\psi^\circ)&$\hyperref[strictdeduction]{\mathbf{R_\getrefnumber{strictdeduction}}}\;\set{13}$\\
                    \fa\entails\nec(\varphi^\medsquare\to\psi^\medsquare)\to\nec(\nec\varphi^\circ\to\psi^\circ)&$\hyperref[deduction]{\mathbf{T_\getrefnumber{deduction}}}\;\set{14}$
                \end{fitch}
            \end{subcase}
        \vspace{.5\baselineskip}
        Tendo-se provado todos os casos do passo de indução, podemos concluir que ambas as traduções apresentadas equivalem, ou seja, que $\entails\nec\alpha^\circ\leftrightarrow\alpha^\medsquare$.
    \end{proof}
\end{theorem}
        \section{Correção}
    \begin{theorem}
        $\text{Se }\Gamma\vdash_\mathbf{I}\alpha\text{, então }\Gamma^\smallsquare\vdash_\mathbf{4}\alpha^\smallsquare$.
    \end{theorem}

    \begin{proof}
        Prova por indução forte sobre o tamanho da sucessão de dedução.
        Assim, suponhamos que a tradução seja correta para qualquer sucessão dedução de tamanho $n<k$.
        Demonstraremos, analisando-se os casos, que o a correção da tradução vale para sucessões de dedução de tamanho $n=k+1$.

        \begin{case}
            \textsc{Caso 1.}
            Se a linha derradeira da sucessão de dedução que prova $\Gamma\entails\alpha$ tenha sido a evocação de alguma premissa, sabe-se que $\alpha\in\Gamma$ e, portanto, que $\alpha^\smallsquare\in\Gamma^\smallsquare$. Desde modo, pode-se demonstrar que $\Gamma^\smallsquare\vdash\alpha^\smallsquare$ trivialmente pela evocação da premissa $\alpha^\smallsquare$.
        \end{case}

        \begin{case}
            \textsc{Caso 2.}
            Se a linha derradeira da sucessão de dedução que prova $\Gamma\entails\alpha$ tenha sido a evocação de algum axioma, sabe-se que existe algum esquema $\mathbf{A_\alpha}\in\mathcal{A}$ que gera $\alpha$. Deste modo, devemos demonstrar que para cada esquema $\mathbf{A}\in\mathcal{A}$, pode-se derivar $\Gamma^\smallsquare\entails_\mathbf{4}\mathbf{A}^\smallsquare$.
        \end{case}

            \begin{subcase}
                \textsc{Caso 2.1} ($\mathbf{A_1}$).

                \begin{fitch}
                    \fa\Gamma^\smallsquare\cup\set{\nec{}a,\nec{}b}\entails\nec{}a&$\mathbf{P_1}$\\
                    \fa\Gamma^\smallsquare\cup\set{\nec{}a}\entails\nec{}b\fishhook\nec{}a&\refer{strictdeduction}{T}$\;\set{1}$\\
                    \fa\Gamma^\smallsquare\entails\nec{}a\fishhook\nec{}b\fishhook\nec{}a&\refer{strictdeduction}{T}$\;\set{2}$.
                \end{fitch}
            \end{subcase}

            \begin{subcase}
                \textsc{Caso 2.2} ($\mathbf{A_2}$).

                \begin{fitch}
                    \fa\Gamma^\smallsquare\cup\set{\nec{a}\fishhook\nec{b}\fishhook\nec{c},\nec{a}\fishhook\nec{b},\nec{a}}\entails\nec{}a\\
                    \fa\Gamma^\smallsquare\cup\set{\nec{a}\fishhook\nec{b}\fishhook\nec{c},\nec{a}\fishhook\nec{b},\nec{a}}\entails\nec{}a\fishhook\nec{}b\\
                    \fa\Gamma^\smallsquare\cup\set{\nec{a}\fishhook\nec{b}\fishhook\nec{c},\nec{a}\fishhook\nec{b},\nec{a}}\entails\nec{}b\\
                    \fa\Gamma^\smallsquare\cup\set{\nec{a}\fishhook\nec{b}\fishhook\nec{c},\nec{a}\fishhook\nec{b},\nec{a}}\entails\nec{a}\fishhook\nec{b}\fishhook\nec{c}\\
                    \fa\Gamma^\smallsquare\cup\set{\nec{a}\fishhook\nec{b}\fishhook\nec{c},\nec{a}\fishhook\nec{b},\nec{a}}\entails\nec{b}\fishhook\nec{c}\\
                    \fa\Gamma^\smallsquare\cup\set{\nec{a}\fishhook\nec{b}\fishhook\nec{c},\nec{a}\fishhook\nec{b},\nec{a}}\entails\nec{c}\\
                    \fa\Gamma^\smallsquare\cup\set{\nec{a}\fishhook\nec{b}\fishhook\nec{c},\nec{a}\fishhook\nec{b}}\entails\nec{a}\fishhook\nec{c}\\
                    \fa\Gamma^\smallsquare\cup\set{\nec{a}\fishhook\nec{b}\fishhook\nec{c}}\entails(\nec{a}\fishhook\nec{b})\fishhook\nec{a}\fishhook\nec{c}\\
                    \fa\Gamma^\smallsquare\entails(\nec{a}\fishhook\nec{b}\fishhook\nec{c})\fishhook(\nec{a}\fishhook\nec{b})\fishhook\nec{a}\fishhook\nec{c}\\

                \end{fitch}
            \end{subcase}

            \begin{subcase}
                \textsc{Caso 2.3} ($\mathbf{A_3}$).

                \begin{fitch}
                    \fa\Gamma^\smallsquare\cup\set{\nec{a},\nec{b}}\entails\nec{a}\\
                    \fa\Gamma^\smallsquare\cup\set{\nec{a},\nec{b}}\entails\nec{b}\\
                    \fa\Gamma^\smallsquare\cup\set{\nec{a},\nec{b}}\entails\nec{a}\to\nec{b}\to\nec{a}\wedge\nec{b}\\
                    \fa\Gamma^\smallsquare\cup\set{\nec{a},\nec{b}}\entails\nec{b}\to\nec{a}\wedge\nec{b}\\
                    \fa\Gamma^\smallsquare\cup\set{\nec{a},\nec{b}}\entails\nec{a}\wedge\nec{b}\\
                    \fa\Gamma^\smallsquare\cup\set{\nec{a},\nec{b}}\entails\nec{a}\wedge\nec{b}\\
                    \fa\Gamma^\smallsquare\cup\set{\nec{a}}\entails\nec{b}\fishhook\nec{a}\wedge\nec{b}\\
                    \fa\Gamma^\smallsquare\entails\nec{a}\fishhook\nec{b}\fishhook\nec{a}\wedge\nec{b}\\
                \end{fitch} 
            \end{subcase}

            \begin{subcase}
                \textsc{Caso 2.4} ($\mathbf{A_4}$).

                \begin{fitch}
                    \fa\Gamma^\smallsquare\entails\nec{a}\wedge\nec{b}\to\nec{a}&$\hyperref[MA4]{\mathbf{A_4}}$\\
                    \fa\Gamma^\smallsquare\entails\nec{a}\wedge\nec{b}\fishhook\nec{a}&$\hyperref[necessitation]{\mathbf{R_2}}\;\set{1}$.
                \end{fitch}
            \end{subcase}

            \begin{subcase}
                \textsc{Caso 2.5} ($\mathbf{A_5}$).

                \begin{fitch}
                    \fa\Gamma^\smallsquare\entails\nec{a}\wedge\nec{b}\to\nec{b}&$\hyperref[MA5]{\mathbf{A_5}}$\\
                    \fa\Gamma^\smallsquare\entails\nec{a}\wedge\nec{b}\fishhook\nec{b}&$\hyperref[necessitation]{\mathbf{R_2}}\;\set{1}$.
                \end{fitch}
            \end{subcase}

            \begin{subcase}
                \textsc{Caso 2.6} ($\mathbf{A_6}$).

                \begin{fitch}
                    \fa\Gamma^\smallsquare\entails\nec{a}\to\nec{a}\vee\nec{b}&$\hyperref[MA6]{\mathbf{A_6}}$\\
                    \fa\Gamma^\smallsquare\entails\nec{a}\fishhook\nec{a}\vee\nec{b}&$\hyperref[necessitation]{\mathbf{R_2}}\;\set{1}$.
                \end{fitch}
            \end{subcase}

            \begin{subcase}
                \textsc{Caso 2.7} ($\mathbf{A_7}$).

                \begin{fitch}
                    \fa\Gamma^\smallsquare\entails\nec{b}\to\nec{a}\vee\nec{b}&$\hyperref[MA7]{\mathbf{A_7}}$\\
                    \fa\Gamma^\smallsquare\entails\nec{b}\fishhook\nec{a}\vee\nec{b}&$\hyperref[necessitation]{\mathbf{R_2}}\;\set{1}$.
                \end{fitch}
            \end{subcase}

            \begin{subcase}
                \textsc{Caso 2.8} ($\mathbf{A_8}$).
                
                \footnotesize
                \begin{fitch}
                    \fa\Gamma^\smallsquare\cup\set{\nec{a}\fishhook\nec{c},\nec{b}\fishhook\nec{c},\nec{a}\vee\nec{b}}\entails\nec{a}\fishhook\nec{c}\\
                    \fa\Gamma^\smallsquare\cup\set{\nec{a}\fishhook\nec{c},\nec{b}\fishhook\nec{c},\nec{a}\vee\nec{b}}\entails(\nec{a}\fishhook\nec{c})\to\nec{a}\to\nec{c}\\
                    \fa\Gamma^\smallsquare\cup\set{\nec{a}\fishhook\nec{c},\nec{b}\fishhook\nec{c},\nec{a}\vee\nec{b}}\entails\nec{a}\to\nec{c}\\
                    \fa\Gamma^\smallsquare\cup\set{\nec{a}\fishhook\nec{c},\nec{b}\fishhook\nec{c},\nec{a}\vee\nec{b}}\entails\nec{b}\fishhook\nec{c}\\
                    \fa\Gamma^\smallsquare\cup\set{\nec{a}\fishhook\nec{c},\nec{b}\fishhook\nec{c},\nec{a}\vee\nec{b}}\entails(\nec{b}\fishhook\nec{c})\to\nec{b}\to\nec{c}\\
                    \fa\Gamma^\smallsquare\cup\set{\nec{a}\fishhook\nec{c},\nec{b}\fishhook\nec{c},\nec{a}\vee\nec{b}}\entails\nec{b}\to\nec{c}\\
                    \fa\Gamma^\smallsquare\cup\set{\nec{a}\fishhook\nec{c},\nec{b}\fishhook\nec{c},\nec{a}\vee\nec{b}}\entails\nec{a}\vee\nec{b}\\
                    \fa\Gamma^\smallsquare\cup\set{\nec{a}\fishhook\nec{c},\nec{b}\fishhook\nec{c},\nec{a}\vee\nec{b}}\entails(\nec{a}\to\nec{c})\to(\nec{b}\to\nec{c})\to\nec{a}\vee\nec{b}\to\nec{c}\\
                    \fa\Gamma^\smallsquare\cup\set{\nec{a}\fishhook\nec{c},\nec{b}\fishhook\nec{c},\nec{a}\vee\nec{b}}\entails(\nec{b}\to\nec{c})\to\nec{a}\vee\nec{b}\to\nec{c}\\
                    \fa\Gamma^\smallsquare\cup\set{\nec{a}\fishhook\nec{c},\nec{b}\fishhook\nec{c},\nec{a}\vee\nec{b}}\entails\nec{a}\vee\nec{b}\to\nec{c}\\
                    \fa\Gamma^\smallsquare\cup\set{\nec{a}\fishhook\nec{c},\nec{b}\fishhook\nec{c}}\entails\nec{a}\vee\nec{b}\fishhook\nec{c}\\
                    \fa\Gamma^\smallsquare\cup\set{\nec{a}\fishhook\nec{c}}\entails(\nec{b}\fishhook\nec{c})\fishhook\nec{a}\vee\nec{b}\fishhook\nec{c}\\
                    \fa\Gamma^\smallsquare\entails(\nec{a}\fishhook\nec{c})\fishhook(\nec{b}\fishhook\nec{c})\fishhook\nec{a}\vee\nec{b}\fishhook\nec{c}\\
                \end{fitch}
            \end{subcase}

            \begin{subcase}
                \textsc{Caso 2.9} ($\mathbf{A_\bot}$).

                \begin{fitch}
                    \fa\Gamma^\smallsquare\entails\bot\to\nec{a}&\refer{explosion}{L}\\
                    \fa\Gamma^\smallsquare\entails\bot\fishhook\nec{a}&$\hyperref[necessitation]{\mathbf{R_2}}\;\set{1}$.\\
                \end{fitch}
            \end{subcase}

        \begin{case}
            \textsc{Caso 3.}
            Deve-se demonstrar que, se $\vdash\nec(\alpha^\smallsquare\to\beta^\smallsquare)$ ($\mathbf{H_1}$) e $\vdash\alpha^\smallsquare$ ($\mathbf{H_2}$), então $\beta^\smallsquare$.
            Isso pode ser feito pela seguinte sucessão de dedução:

            \begin{fitch}
                \fa\nec(\alpha^\smallsquare\to\beta^\smallsquare)\to\alpha^\smallsquare\to\beta^\smallsquare&$\mathbf{B_2}$\\
                \fa\nec(\alpha^\smallsquare\to\beta^\smallsquare)&$\mathbf{H_1}$\\
                \fa\alpha^\smallsquare\to\beta^\smallsquare&$\mathbf{R_1}\;\sequence{1, 2}$\\
                \fa\alpha^\smallsquare&$\mathbf{H_2}$\\
                \fa\beta^\smallsquare&$\mathbf{R_1}\;\sequence{3, 4}$.
            \end{fitch}
        \end{case}
    \end{proof}
        \section{Completude}
    \babireski{Não vai rolar de provar a completude como~\cite{Troelstra}. Vou precisar procurar outros artigos.}

    \babireski{Ver~\cite{Benton},~\cite{Pfenning} e~\cite{Fairtlough} acerca do sistema laxo}.

    \bibliographystyle{plainnat}
    \bibliography{bibliography}
\end{document}
