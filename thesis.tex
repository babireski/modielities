\documentclass{report}

\usepackage{amsthm}
\usepackage{amsmath}
\usepackage{amssymb}
\usepackage{epigraph}
\usepackage{mathtools}
\usepackage{styles/modalities}
\usepackage{ragged2e}
\usepackage{hyperref}
\usepackage{proof}
\usepackage[round]{natbib}
\usepackage[french, brazil]{babel}
\usepackage{styles/fitch}
\usepackage{styles/cases}
\usepackage{styles/functions}

\usepackage[T1]{fontenc}

\usepackage[
    left = \flqq{},% 
    right = \frqq{},% 
    leftsub = \flq{},% 
    rightsub = \frq{} %
]{dirtytalk}

\newcommand{\entails}{\mathrel{\vdash}}
\newcommand{\point}{\mathrel{.}}

\newtheorem*{notation}{Notação}
\newtheorem{definition}{Definição}
\newtheorem{lemma}{Lema}
\newtheorem{theorem}{Teorema}

\begin{document}
    \title{Uma formalização da interpretação modal do sistema intuicionista}
    \author{Elian Babireski}
    \date{2024}

    \maketitle

    \begin{abstract}
        Resumo aqui.
    \end{abstract}

    \tableofcontents

    \setlength\epigraphwidth{.5\textwidth}
\setlength\epigraphrule{0pt}

\vspace*{\fill}
\epigraph{\justifying\say{\say{Oh, you can't help that,} said the Cat: \say{we're all mad here. I'm mad. You're mad.} \say{How do you know I'm mad?} said Alice. \say{You must be,} said the Cat, \say{or you wouldn't have come here.}}}{--- Lewis Carroll, \textit{Alice in Wonderland}}
    \chapter{Introdução}

As lógicas modais consistem em um conjunto de extensões da lógica clássica que contam com a adição de um ou mais operadores, chamados modalidades, que qualificam sentenças. No caso do sistema \textbf{S4}, são adicionadas as modalidades de necessidade ($\nec$) e possibilidade ($\pos$) em conjunto à regra da necessitação\footnote{Se $\vdash A$ então $\vdash \nec A$} e os axiomas $\mathbf{K}\text{: } \nec(A \to B) \to \nec A \to \nec B$, $\mathbf{T}\text{: } \nec A \to A$ e $\text{\textbf{4}: } \nec A \to \nec \nec A$ \citep{Troelstra}. Ademais, pode-se derivar nesse sistema, por meio da dualidade entre as modalidades\footnote{$\pos A \equiv \neg \nec \neg A $}, sentenças duais aos axiomas \textbf{T} e \textbf{4}, sendo elas $\mathbf{T}_\meddiamond \text{: } A \to \pos A$ e $\mathbf{4}_\meddiamond \text{: } \pos \pos A \to \pos A$, respectivamente~\cite{Zach}.

As mônadas ganharam destaque na área de linguagens de programação desde que~\cite{Moggi} formalizou uma metalinguagem que faz uso dessas estruturas para modelar noções de computação --- como parcialidade, não-determinismo, exceções e continuações --- de uma maneira puramente funcional. Pode-se notar uma grande semelhança entre as sentenças $\mathbf{T}_\meddiamond$ e $\mathbf{4}_\meddiamond$ e as transformações naturais monádicas $\mathbf{\eta:} 1_C \to T$ e $\mathbf{\mu:} T^2 \to T$, respectivamente. Nesse sentido,~\cite{Pfenning} demonstraram que se pode traduzir essa metalinguagem para o sistema \textbf{S4} da lógica modal, pelo qual se torna interessante analisar esse sistema como uma linguagem de programação sob a ótica do isomorfismo de Curry-Howard.

~\cite{Troelstra} apresentam duas traduções equivalentes da lógica intuicionista para o sistema \textbf{S4} da lógica modal, sendo um deles correspondente a uma abordagem \textit{call-by-name} e outra a um abordagem \textit{call-by-value}. Tais traduções possuem grande similaridade com as traduções da lógica intuicionista para a lógica linear definidas por~\cite{Girard}. Essas traduções equivalem à tradução por negação dupla que, por sua vez, equivalem a traduções \textit{continuation-passing style} (CPS) em compiladores por meio do isomorfismo de Curry-Howard~\citep{Reynolds}, o que torna esse tema interessante no ponto de vista de compilação.

Durante grande parte da história, provas lógicas e matemáticas eram validadas manualmente pela comunidade acadêmica, o que muitas vezes --- a depender do tamanho e complexidade da prova --- se mostrava ser um trabalho complexo e sujeito a erros. Hoje em dia, exitem \textit{softwares} chamados assistentes de provas que permitem verificar --- graças ao isomorfismo de Curry-Howard --- a corretude de provas~\citep{Chlipala}. O assistente de provas que será usado neste trabalho é o \textsc{coq}, que utiliza o cálculo de construções indutivas e um conjunto axiomático pequeno para permitir a escrita de provas simples e intuitivas~\citep{Coq}.

    \section{Objetivos}
    Este trabalho consiste numa continuação do desenvolvimento da biblioteca de formalização de sistemas modais normais iniciado por~\cite{Silveira} e posteriormente expandida de forma a permitir a fusão de sistemas modais por~\cite{Nunes}. Nele, formalizaremos as traduções do sistema intuicionista ao sistema modal $\mathbf{S4}$ no asssitente de provas \textsc{coq} e provaremos suas propriedades. Uma formalização de traduções entre sistemas de dedução similar a nossa foi feita por~\cite{Sehnem}, neste caso tendo como alvo o sistema linear de~\cite{Girard}. Todas as formalizações citadas acima deram-se no assistente de provas \textsc{coq}, o mesmo assistente usado neste trabalho. Como objetivos específicos, listamos:

    \begin{itemize}
        \item Fornecer uma introdução ao conceito de sistemas de dedução;
        \item Fornecer uma introdução ao conceito de traduções entre sistemas;
        \item Fornecer uma introdução ao sistema intuicionista;
        \item Fornecer uma introdução aos sistemas modais, em especial o $\mathbf{S4}$;
        \item Apresentar as traduções do sistema intuicionista ao sistema $\mathbf{S4}$;
        \item Provar manualmente a correção das traduções providas bem como outras propriedades pertinentes;
        \item Formalizar as provas no provador de teoremas interativo \textsc{coq}.
    \end{itemize}

    \section{Estruturação}
    Estruturaremos este trabalho em cinco partes. A parte \textbf{(1)} trata-se desta introdução. A parte \textbf{(2)} consiste numa fundamentação de conceitos basilares ao desenvolvimento deste trabalho, notadamente os conceitos de \emph{sistemas de dedução}, \emph{traduções} e \emph{provadores de teoremas}. A parte \textbf{(3)} apresenta as definições dos sistemas e traduções relevantes a este trabalho. Na parte \textbf{(4)} são provadas todas as propriedades abarcadas no escopo deste trabalho. Por fim, a parte \textbf{(5)} compreende considerações parciais acerca do desenvolvido até o momento.
    \chapter{Fundamentação}

Nesta parte do trabalho, serão apresentadas definições gerais que fundamentarão as definições mais estritas que serão apresentadas futuramente. Notadamente, fundamentaremos as noções de sistemas e traduções. Ademais, discorreremos acerca da noção de provadores, que serão usados para certificar as provas apresentadas posteriormente. Antes disso, entretanto, introduziremos duas notações que serão usadas copiosamente, uma para o conjunto das partes e outra para sucessões.

\begin{notation}
    Seja $A$ um conjunto, $\wp(A)$ denota o conjunto $\set{X\mid X\subseteq A}$.
\end{notation}

\begin{notation}
    Seja $i\in\mathbb{N}^+$ e $n\in\mathbb{N}$, $\sequence{a_i\mid i\leq n}$ denota uma sucessão de $n$ elementos de modo que o elemento $a_i$ encontra-se na posição $i$.
\end{notation}

\section{Sistemas}

Sistemas de dedução buscam formalizar e sistematizar o processo de razoamento. Estudos acerca disso datam da antiguidade, dentre os quais destaca-se~\cite{Aristotle}. Considera-se que os estudos modernos neste campo foram, dentre outras pessoas, fundados por~\cite{Frege} e continuados por~\cite{Russel-A,Russel-B,Russel-C}. Estas investigações --- bem como outras --- levaram ao desenvolvimento do sistema hoje tido como padrão. Posteriormente a isso, viu-se o surgimento de diversos sistemas não-padrões, fato que --- conforme~\cite{Beziau-B} --- justifica uma conceituação de sistema de dedução, que apresentaremos nesta seção.

Ainda segundo~\cite{Beziau-B}, os primeiros desenvolvimentos neste sentido foram feitos por~\cite{Tarski}, que define o conceito de dedução com base num operador de fecho $C\mathrel{:}\wp(\mathcal{L})\to\wp(\mathcal{L})$, sendo $\mathcal{L}$ um conjunto qualquer. Neste trabalho entretanto usaremos a definição proposta por~\cite{Beziau} baseada numa relação de dedução ${\vdash}\subseteq\wp(\mathcal{L})\times\mathcal{L}$, uma vez que, por sua simplicidade, não traz elementos irrelevantes aos intuitos deste. Cabe destacar, conforme apontam~\cite{Font}, que ambas as definições são equivalentes\footnote{Destaca-se, entretanto, que a definição de~\cite{Tarski} requer a satisfação de postulados não requeridos por~\cite{Beziau}, sendo portanto menos geralista.}, uma vez que $\Gamma\entails\alpha$ se e somente se $\alpha\in C(\Gamma)$.

\begin{definition}[Sistema]
    Um sistema de dedução consiste num par $\mathbf{L} = \sequence{\mathcal{L}, \vdash}$, onde $\mathcal{L}$ consiste em um conjunto e ${\vdash}\subseteq\wp(\mathcal{L})\times\mathcal{L}$ em uma relação sobre o produto cartesiano do conjunto das partes de $\mathcal{L}$ e o conjunto $\mathcal{L}$, sem demais condições.
    \qed{}
\end{definition}

Conforme~\cite{Beziau} aponta, a qualidade e quantidade dos elementos de um sistema $\mathbf{L}=\sequence{\mathcal{L}, \vdash}$ não são especificados, portanto sendo esta uma definição de grande generalidade. Neste sentido, com base no escopo deste trabalho, restringiremos a definição do conjunto $\mathcal{L}$ --- dito \emph{linguagem} --- a linguagens proposicionais. Os elementos destas, aos quais daremos o nome de \emph{sentenças}, notabilizam-se por serem formadas por \emph{letras} --- que consistem em proposições indivisas --- e \emph{operadores} --- que podem gerar proposições maiores a partir de proposições menores. Ao par formado por letras e operadores daremos o nome \emph{assinatura}, conforme abaixo.

\begin{definition}[Assinatura]
    Uma assinatura proposicional consiste num par $\Sigma=\sequence{\mathcal{P},\mathcal{C}}$, onde $\mathcal{P}$ consiste num conjunto letras e $\mathcal{C}=\bigcup\set{\mathcal{C}_i\mid i\in\mathbb{N}}$ num conjunto de operadores de modo que $\mathalpha{\bullet}\in\mathcal{C}_n$ se e somente se $\mathalpha{\bullet}$ possuir aridade $n$.
    \qed{}
\end{definition}

\begin{notation}
    Seja $\mathcal{C}$ um conjunto de operadores, $\mathalpha{\bullet}^n$ denota um operador $\mathalpha{\bullet}\in\mathcal{C}_n$.
\end{notation}

Podemos interpretar os conjuntos $\mathcal{P}$ e $\mathcal{C}$ de uma assinatura $\Sigma=\sequence{\mathcal{P},\mathcal{C}}$ como contrutores de sentenças.
Neste sentido, o conjunto $\mathcal{C}_0$ assemelha-se mais ao conjunto $\mathcal{P}$, uma vez que seus elementos --- ditos \emph{constantes} --- não geram sentenças maiores partindo de sentenças menores.
Nota-se que uma assinatura constitui um elemento suficiente para definirmos indutivamente a linguagem de um sistema, conforme definido abaixo de maneira similar a~\cite{Franks}.
Por fim, destacamos que, para todos os sistemas apresentados neste trabalho, usaremos o conjunto de letras $\mathcal{P}=\set{p_i\mid i\in\mathbb{N}}$ e letras romanas em caixa-baixa para representar seus elementos.

\begin{definition}[Linguagem]
    Seja $\Sigma=\sequence{\mathcal{P},\mathcal{C}}$ uma assinatura proposicional. Uma linguagem proposicional $\mathcal{L}$ induzida a partir de $\Sigma$ consiste no menor conjunto de sentenças bem-formadas induzido a partir das seguintes regras:
    \begin{enumerate}[label=\textbf{\emph{(\alph*)}}, left=\parindent]
        \item$\mathcal{P}\subseteq\mathcal{L}$
        \item\text{Se }$\mathalpha{\bullet}\in\mathcal{C}_n\text{ e }\set{\varphi_i\mid i\leq n}\subseteq\mathcal{L}\text{, então }\bullet\sequence{\varphi_i\mid i\leq n}\in\mathcal{L}$.\qed{}
    \end{enumerate}
\end{definition}

Neste trabalho, representaremos sentenças por letras gregas em caixa-baixa e conjuntos de sentenças por letras gregas em caixa-alta.\footnote{Desconsiderando-se o $\Sigma$, usado para representar assinaturas.}
Ademais, impõe-se definir a noção de profundidade de uma sentença. Esta noção, em termos simples, consiste no comprimento do maior ramo da construção da dada sentença. A definição provida abaixo consiste numa generalização para quaisquer aridades da definição dada por~\cite{Troelstra}. Usaremos essa definição futuramente para fazer demonstrações por meio provas indutivas sobre esta propriedade.

\begin{definition}[Profundidade]
    Seja $\mathbf{L} = \sequence{\mathcal{L}, \vdash}$ um sistema com linguagem induzida a partir de uma assinatura $\Sigma=\sequence{\mathcal{P},\mathcal{C}}$. Considerando-se uma proposição $a\in\mathcal{P}$, um operador ${\bullet}\in\mathcal{C}$ e uma aridade $n>0$, definimos a profundidade $|\alpha|$ de uma sentença $\alpha\in\mathcal{L}$ indutivamente da seguinte maneira:
    \begin{align*}
        |a|&\coloneqq 0\\
        |{\bullet^0}|&\coloneqq 0\\
        |{\bullet^n\sequence{\varphi_i\mid i\leq n}}|&\coloneqq\max\set{|\varphi_i|\mid i\leq n}+1.
        \tag*{\qed} 
    \end{align*}
\end{definition}

Com isso, encerram-se as definições relacionadas a linguagens de sistemas de dedução. Agora, apresentaremos definições relacionadas a relações de dedução, que gozam da mesma generalidade dada a liguagens. Deste modo, a relação $\mathalpha{\vdash}$ pode ser tanto uma relação de \emph{derivação} --- definida sintaticamente --- quanto uma relação de \emph{satisfação}\footnote{Sendo esta denotada por $\mathalpha{\vDash}$.} --- definida semanticamente. Neste trabalho, serão abordados apenas sistemas definidos sobre relações de derivação. Cabe destacar, entretanto, que nada na definição de tradução impede que esta seja feita sobre relações de satisfação, conforme veremos com mais detalhes futuramente.

Neste trabalho, definiremos a relações de dedução baseada em axiomatizações, ou seja, em conjuntos de \emph{axiomas} --- sentenças postuladas como verdadeiras --- e conjuntos de \emph{regras de dedução} --- que permitem derivar mais sentenças verdadeiras caso certas condições sejam satisfeitas. Axiomatizações consistem numa abordagem hilbertiana de dedução que, segundo~\cite{Troelstra}, distinguem-se por conter um conjunto reduzido de regras de dedução que nunca descartam premissas. Ainda baseando-se em~\cite{Troelstra} e em contraste a~\cite{Frege} e~\cite{Hilbert-A, Hilbert-B}, preferiremos esquemas de axiomas a axiomas individuais de modo a eliminarmos a necessidade de instanciações.

\begin{definition}[Axiomatização]
    Seja $\mathbf{L}=\sequence{\mathcal{L},\vdash}$ um sistema. Uma axiomatização para o sistema $\mathbf{L}$ consiste num par $\mathcal{H}=\sequence{\mathcal{A},\mathcal{R}}$, sendo $\mathcal{A}$ um conjunto de esquemas de axiomas e $\mathcal{R}$ um conjunto de regras de dedução.
\end{definition}


Neste trabalho, consideraremos axiomatizações definidas em relação a conjuntos de premissas $\Gamma\subseteq\mathcal{L}$. 
Por este motivo, todas as axiomatizações apresentadas futuramente neste trabalho conterão implicitamente a \emph{regra da premissa} $\mathbf{P}$ --- sendo esta regra definida como se $\alpha\in\Gamma$, então $\Gamma\entails\alpha$.
Similarmente, faremos o mesmo para a \emph{regra do enfraquecimento} $\mathbf{E}$ --- definida aqui como se $\Gamma\entails\alpha$, então $\Gamma\cup\Delta\entails\alpha$ ---, tendo em vista que todos os sistemas apresentados neste trabaho são estruturais. Assim, tendo-se claro o conceito de axiomatização, podemos finalmente o conceito de dedução.

\begin{definition}[Dedução]
    Seja um sistema $\mathbf{L} = \sequence{\mathcal{L},{\vdash}}$ com uma relação de dedução definida sobre uma axiomatização $\mathcal{H} = \sequence{\mathcal{A},\mathcal{R}}$ e  seja um conjunto de sentenças $\Gamma\cup\set{\alpha}\subseteq\mathcal{L}$.
    A dedução $\Gamma\vdash\alpha$ vale se e somente se houver sucessão de sentenças $\sequence{\varphi_i\in\mathcal{L}\mid i\leq n}$ de modo que $\varphi_n=\alpha$ e que cada sentença $\varphi_i$ ou tenha sido gerada ou por algum esquema $\mathbf{A}\in\mathcal{A}$ ou pela aplicação de alguma regra $\mathbf{R}\in\mathcal{R}$ a sentenças anteriores.
    \qed{}
\end{definition}

\section{Traduções}

Traduções entre sistemas consistem em funções que mapeiam sentenças de um sistema a sentenças de outro, garantindo certas propriedades. As propriedades a serem garantidas variam e ainda são discutidas na literatura, deixando a definição exata de tradução --- assim como houve com a definição de sistema --- varie de acordo com a predileção e as necessidades de cada autor. Nesta seção, serão abordadas historicamente noções de tradução entre sistemas, bem como serão definidos e nomeados os conceitos de tradução que serão usados no restante deste trabalho.

\begin{definition}[Condições]
    Chamaremos a condição $\varnothing\entails_\mathbf{A}\alpha$ implica em $\varnothing\entails_\mathbf{B}\alpha^*$ de correção fraca e a condição $\varnothing\entails_\mathbf{B}\alpha^*$ implica em $\varnothing\entails_\mathbf{A}\alpha$ de completude fraca. Analogamente, considerando-se dedução com premissas, chamaremos a condição $\Gamma\entails_\mathbf{A}\alpha$ implica em $\Gamma^*\entails_\mathbf{B}\alpha^*$ de correção forte e a condição $\Gamma^*\entails_\mathbf{B}\alpha^*$ implica em $\Gamma\entails_\mathbf{A}\alpha$ de completude forte.
\end{definition}

Historicamente, autores usaram diferentes combinações das condições apresentadas acima e, em certos casos, outras. Neste trabalho, adotaremos uma noção forte de tradução que requer tanto a correção forte quanto a completude forte, conforme~\cite{Coniglio}. Definiremos, ainda, uma notação que nos permite aplicar sucintamente a tradução a todos os elementos de um conjunto.

\begin{definition}[Tradução] 
    Uma sentença $\varphi$ de um sistema $\mathbf{A} = \langle \mathcal{L}_\mathbf{A}, \vdash_\mathbf{A}\rangle$ pode ser traduzida a uma sentença $\varphi^*$ em um sistema $\mathbf{B} = \langle \mathcal{L}_\mathbf{B}, \vdash_\mathbf{B} \rangle$ caso exista uma função $\bullet^* : \mathcal{L}_\mathbf{A} \to \mathcal{L}_\mathbf{B}$ que garanta que $\Gamma \vdash_\mathbf{A} \varphi \Leftrightarrow \Gamma^* \vdash_\mathbf{B} \varphi^*$.
    \qed{}
\end{definition}

\begin{notation}
    Seja $\Gamma\in\wp(\mathcal{L}_\mathbf{A})$ um conjunto de sentenças bem-formadas e $\bullet^*\mathrel{:}\mathcal{L}_\mathbf{A}\to\mathcal{L}_\mathbf{B}$ uma tradução. $\Gamma^*$ denota o conjunto $\set{\alpha^*\mid\alpha\in\Gamma}\in\wp(\mathcal{L}_\mathbf{B})$, ou seja, a aplicação da tradução a todos os elementos do conjunto $\Gamma$.
    \qed{}
\end{notation}

A primeira tradução entre dois sistemas conhecida na literatura foi definida por~\cite{Kolmogorov} como uma maneira de demonstrar que o uso da \emph{lei do terceiro excluso}\footnote{Definido como $\entails\alpha\vee\neg\alpha$.} não leva a contradições. Essa definição consiste basicamente em prefixar uma dupla negação a cada elemento da construção de uma dada sentença, motivo pelo qual chamaremos essa tradução de \emph{tradução de negação dupla} \citep{Coniglio}. Essa mesma tradução foi também descoberta independentemente por Gödel e por Getzen. Curiosamente, essa tradução mostra-se relevante para o escopo deste trabalho, uma vez que consiste na contraparte da passagem por continuações segundo a interpretação prova-programa.

\begin{definition}[$\bullet^\neg$] Define-se a tradução $\bullet^\neg:\mathcal{L}_\mathbf{C}\to\mathcal{L}_\mathbf{I}$ do sistema clássico ao sistema intuicionista indutivamente da seguinte maneira:
    \begin{align*}
        p^\neg&\coloneqq\neg\neg p\\
        \bot^\neg&\coloneqq\bot\\
        {(\varphi\wedge\psi)}^\neg&\coloneqq\neg\neg(\varphi^\neg \wedge \psi^\neg)\\
        {(\varphi\vee\psi)}^\neg&\coloneqq\neg\neg (\varphi^\neg \vee \psi^\neg)\\
        {(\varphi\to\psi)}^\neg&\coloneqq\neg\neg (\varphi^\neg \to \psi^\neg)
        \tag*{\qed} 
    \end{align*}
\end{definition}

\section{Provadores}

A primeira prova de destaque a ser realizada com grande uso de computadores foi a do teorema das quatro cores\footnote{Que afirma que \emph{qualquer mapa planar tem uma quatro-coloração}.}, feita por~\cite{Appel}, motivado pela grande quantidade de casos a serem analisados.
Conforme~\cite{Wilson} afirma, esta prova foi por uns recebida com entusiasmo e por outros, devido ao uso de computadores, com cetistismo e desapontamento.
Dentre aqueles que compartilharam destas visões opositoras, destaca-se~\cite{Tymoczko}.
Ainda segundo~\cite{Wilson}, o teorema tornou-se mais aceito com o passar do tempo e foi, posteriormente, formalizado em um provador de teoremas por~\cite{Gonthier}.

Provadores de teoremas consistem em programas de computador que verificam a validade de teoremas. Dentre estes, podemos destacar as classes dos provadores \emph{automados} e dos provadores \emph{interativos}. Os primeiros buscam provar teoremas de maneira que requeira a menor quantidade de intervenção humana, enquanto os segundos --- que ganharam destaque depois das limitações dos primeiros ficarem evidentes --- delegam-se a verificar rigorosamente provas desenvolvidas por humanos em sua linguagem~(\babireski{Citação}). Formalizaremos das provas apresentadas neste trabalho no provador de teoremas interativo \textsc{coq}.


    \chapter{Formalização}
        \section{Sistemas}
    \begin{definition}[$\mathcal{L}_\mathbf{I}$]
        A linguagem do sistema intuicionista, denotada $\mathcal{L}_\mathbf{I}$, consiste no menor conjunto induzido a partir das seguintes regras:
        \begin{align*}
            &\bot\in\mathcal{L}_\mathbf{I} \\
            &\mathcal{P}\subseteq\mathcal{L}_\mathbf{I} \\
            &\alpha,\beta\in\mathcal{L}_\mathbf{I}\Rightarrow\alpha\circ\beta\in\mathcal{L}_\mathbf{I}\text{, para }\circ\in\set{\wedge,\vee,\to}\text{.}
            \tag*{\qed}
        \end{align*}
    \end{definition}

    \begin{notation}
        Serão usadas as seguintes abreviações:
        \begin{align*}
            \top&\coloneqq\bot\to\bot\\
            \neg\alpha&\coloneqq\alpha\to\bot\\
            \alpha\leftrightarrow\beta&\coloneqq(\alpha\to\beta)\wedge(\beta\to\alpha)
        \end{align*}
    \end{notation}

    \begin{definition}
        A axiomatização do sistema intuicionista consiste nos seguintes esquemas e regras:
        \begin{alignat*}{3}
            & \mathbf{A_1}\quad && \alpha\to\beta\to\alpha \\
            & \mathbf{A_2}\quad && (\alpha\to\beta\to\gamma)\to(\alpha\to\beta)\to(\alpha\to\gamma) \\
            & \mathbf{A_3}\quad && \alpha\to\beta\to\alpha\wedge\beta \\
            & \mathbf{A_4}\quad && \alpha\wedge\beta\to\alpha \\
            & \mathbf{A_5}\quad && \alpha\wedge\beta\to\beta \\
            & \mathbf{A_6}\quad && \alpha\to\alpha\vee\beta \\
            & \mathbf{A_7}\quad && \beta\to\alpha\vee\beta \\
            & \mathbf{A_8}\quad && (\alpha\to\gamma)\to(\beta\to\gamma)\to(\alpha\vee\beta\to\gamma) \\
            & \mathbf{A_\bot}\quad && \bot\to\alpha \\
            & \mathbf{R_1}\quad && \text{Se }\vdash\alpha\text{ e }\vdash\alpha\to\beta\text{, então }\vdash\beta\text{.} & \tag*{\qed}
        \end{alignat*}   
    \end{definition}

    \babireski{\cite{Blackburn} traz uma visão da evolução dos sistemas modais.}

    \begin{definition}[$\mathcal{L}_\mathbf{M}$]
        A linguagem dos sistemas modais, denotada $\mathcal{L}_\mathbf{M}$, consiste no menor conjunto induzido a partir das seguintes regras:
        \begin{align*}
            &\bot\in\mathcal{L}_\mathbf{M} \\
            &\mathcal{P}\subseteq\mathcal{L}_\mathbf{M} \\
            &\alpha\in\mathcal{L}_\mathbf{M}\Rightarrow\nec\alpha\in\mathcal{L}_\mathbf{M} \\
            &\alpha,\beta\in\mathcal{L}_\mathbf{M}\Rightarrow\alpha\circ\beta\in\mathcal{L}_\mathbf{M}\text{, para }\circ\in\set{\wedge,\vee,\to}\text{.}
            \tag*{\qed}
        \end{align*}
    \end{definition}

    \begin{notation}
        Serão usadas as seguintes abreviações:
        \begin{align*}
            \top&\coloneqq\bot\to\bot\\
            \neg\alpha&\coloneqq\alpha\to\bot\\
            \pos\alpha&\coloneqq\neg\nec\neg\alpha\\
            \alpha\fishhook\beta&\coloneqq\nec(\alpha\to\beta)\\
            \alpha\leftrightarrow\beta&\coloneqq(\alpha\to\beta)\wedge(\beta\to\alpha)
        \end{align*}
    \end{notation}

    \begin{definition}
        A axiomatização do sistema modal consiste nos seguintes esquemas e regras:
        \begin{alignat*}{3}
            & \mathbf{A_1}\quad && \alpha\to\beta\to\alpha \\
            & \mathbf{A_2}\quad && (\alpha\to\beta\to\gamma)\to(\alpha\to\beta)\to(\alpha\to\gamma) \\
            & \mathbf{A_3}\quad && \alpha\to\beta\to\alpha\wedge\beta \\
            & \mathbf{A_4}\quad && \alpha\wedge\beta\to\alpha \\
            & \mathbf{A_5}\quad && \alpha\wedge\beta\to\beta \\
            & \mathbf{A_6}\quad && \alpha\to\alpha\vee\beta \\
            & \mathbf{A_7}\quad && \beta\to\alpha\vee\beta \\
            & \mathbf{A_8}\quad && (\alpha\to\gamma)\to(\beta\to\gamma)\to(\alpha\vee\beta\to\gamma) \\
            & \mathbf{A_\neg}\quad && \neg\neg\alpha\to\alpha \\
            & \mathbf{B_1}\quad && \nec(\alpha\to\beta)\to\nec\alpha\to\nec\beta \\
            & \mathbf{B_2}\quad && \nec\alpha\to\alpha \\
            & \mathbf{B_3}\quad && \nec\alpha\to\nec\nec\alpha \\
            & \mathbf{R_1}\quad && \text{Se }\vdash\alpha\text{ e }\vdash\alpha\to\beta\text{, então }\vdash\beta \\
            & \mathbf{R_2}\quad && \text{Se }\vdash\alpha\text{, então }\vdash\nec\alpha\text{.} & \tag*{\qed} 
        \end{alignat*}   
    \end{definition}
        \section{Metateoremas}
    Provaremos, para o sistema modal apresentado, uma variação do teorema da dedução baseado na implicação estrita, conforme proposto por~\cite{Marcus} de modo a permitir simplificar muitas das demonstrações apresentadas futuramente neste trabalho.
    \babireski{\citet{Hakli} apresentam uma discussão sobre a validade do metateorema da dedução e suas variantes nos sistemas modais. Quem sabe valha a pena escrever um pouco sobre isso, pois se trata de uma leitura interessante.}

    \begin{theorem}
        $\forall\,\Gamma\cup\set{\alpha,\beta}\in\wp(\mathcal{L}_\mathbf{M})\point\Gamma\cup\set{\alpha}\vdash\beta\Rightarrow  \Gamma\vdash\nec(\alpha\to\beta)$.
    \end{theorem}

    \begin{proof}
        Seja $\Gamma$ um conjunto de sentenças e sejam $\alpha$ e $\beta$ sentenças de modo que $\Gamma\cup\set{\alpha}\vdash\beta$, deve-se provar que $\Gamma\vdash\nec(\alpha\to\beta)$. Como $\Gamma\cup\set{\alpha}\vdash\beta$, existe uma prova de $\beta$ a partir de $\Gamma\cup\set{\alpha}$. A prova baseia-se numa indução sobre o tamanho $n$ da prova.

        \begin{case}
            \textbf{Caso 1} (Base)\textbf{.}
            Para a base requer-se considerar os seguintes casos: 
                \textbf{(1)} $\beta$ consiste num axioma,
                \textbf{(2)} $\beta\in\Gamma$ e
                \textbf{(3)} $\beta=\alpha$.

            \begin{case}
                \textbf{Caso 1.1} ($\beta\in\mathcal{A}$)\textbf{.}
                
                \begin{fitch}
                    \fa\beta\to\alpha\to\beta&$\mathbf{A_1}$\\
                    \fa\beta&$\mathbf{A_\beta}$\\
                    \fa\alpha\to\beta&$\mathbf{R_1}\;\sequence{1, 2}$\\
                    \fa\nec(\alpha\to\beta)&$\mathbf{R_2}\;\sequence{3}$
                \end{fitch}
            \end{case}

            \begin{case}
                \textbf{Caso 1.2} ($\beta\in\Gamma$)\textbf{.}

                \begin{fitch}
                    \fa\beta\to\alpha\to\beta&$\mathbf{A_1}$\\
                    \fa\beta&$\mathbf{P}$\\
                    \fa\alpha\to\beta&$\mathbf{R_1}\;\sequence{1, 2}$\\
                    \fa\nec(\alpha\to\beta)&$\mathbf{R_2}\;\sequence{3}$
                \end{fitch}
            \end{case}

            \begin{case}
                \textbf{Caso 1.3} ($\beta=\alpha$)\textbf{.}
                
                \begin{fitch}
                    \fa\alpha\to\alpha&$\mathbf{L_1}$\\
                    \fa\nec(\alpha\to\alpha)&$\mathbf{R_2}\;\sequence{1}$
                \end{fitch}
            \end{case}
        \end{case}

        \begin{case}
            \textbf{Caso 2} (Passo)\textbf{.}
            Supõe-se que, para qualquer prova de $\Gamma\cup\set{\alpha}\vdash\beta$ com tamanho $k$, tem-se que $\Gamma\vdash\nec(\alpha\to\beta)$.
            Deve-se mostrar que a proposição segue verdadeira caso a prova tenha tamanho $k+1$. 
            Assim, sendo $\sequence{\varphi_i\mid1\leq i\leq k+1}$ uma sucessão de dedução com $\varphi_{k+1}=\beta$, requer-se considerar os seguintes casos:

            \begin{case}
                \textbf{Caso 2.1} ($\beta\in\mathcal{A}$)\textbf{.} \textit{Vide} caso $\mathbf{C_{1.1}}$.
            \end{case}

            \begin{case}
                \textbf{Caso 2.2} ($\beta\in\Gamma$)\textbf{.} \textit{Vide} caso $\mathbf{C_{1.2}}$.
            \end{case}

            \begin{case}
                \textbf{Caso 2.3} ($\beta=\alpha$)\textbf{.} \textit{Vide} caso $\mathbf{C_{1.3}}$.
            \end{case}

            \begin{case}
                \textbf{Caso 2.4} ($\mathbf{R_1}$)\textbf{.} \babireski{Aqui basta transcrever a prova da aula.}
            \end{case}

            \begin{case}
                \textbf{Caso 2.5} ($\mathbf{R_2}$)\textbf{.} \babireski{Estou sofrendo nesse caso.}
            \end{case}
        \end{case}
    \end{proof}
        \section{Derivações}
    \begin{lemma}
        $\vdash\alpha\to\alpha$
        \begin{proof}
            Pode ser provado pela seguinte sucessão de dedução:
        
            \begin{fitch}
                \fa(\alpha\to(\alpha\to\alpha)\to\alpha)\to(\alpha\to\alpha\to\alpha)\to\alpha\to\alpha&$\mathbf{A_2}$\\
                \fa\alpha\to(\alpha\to\alpha)\to\alpha&$\mathbf{A_1}$\\
                \fa(\alpha\to\alpha\to\alpha)\to\alpha\to\alpha&$\mathbf{R_1}\;\set{1, 2}$\\
                \fa\alpha\to\alpha\to\alpha&$\mathbf{A_1}$\\
                \fa\alpha\to\alpha&$\mathbf{R_1}\;\set{3, 4}$\\
            \end{fitch}
            \vspace*{-18pt}
            \qedhere
        \end{proof}
    \end{lemma}

    \begin{lemma}
        $\vdash\bot\to\alpha$
        \begin{proof}
            Pode ser provado pela seguinte sucessão de dedução:
        
            \begin{fitch}
                \fa(((\alpha\to\bot)\to\bot)\to\alpha)\to\bot\to((\alpha\to\bot)\to\bot)\to\alpha&$\mathbf{A_1}$\\
                \fa((\alpha\to\bot)\to\bot)\to\alpha&$\mathbf{A_\neg}$\\
                \fa\bot\to((\alpha\to\bot)\to\bot)\to\alpha&$\mathbf{R_1}\;\set{1, 2}$\\
                \fa(\bot\to(\alpha\to\bot)\to\bot)\to\bot\to\bot\to(\alpha\to\bot)\to\bot&$\mathbf{A_1}$\\
                \fa\bot\to(\alpha\to\bot)\to\bot&$\mathbf{A_1}$\\
                \fa\bot\to\bot\to(\alpha\to\bot)\to\bot&$\mathbf{R_1}\;\set{4, 5}$\\
                \fa\bot\to\bot&$\mathbf{L_1}$\\
                \fa\\
                \fa\\
                \fa\bot\to(\alpha\to\bot)\to\bot\\
                \fa\\
                \fa\\
                \fa\bot\to\alpha\\
            \end{fitch}
            \vspace*{-18pt}
            \qedhere
        \end{proof}
    \end{lemma}

    \begin{lemma}
        $\vdash(\alpha\to\beta)\to(\alpha\to\gamma)\to\alpha\to\beta\wedge\gamma$.
    \end{lemma}

    \begin{lemma}
        $\vdash\nec(\alpha\wedge\beta)\to\nec\alpha\wedge\nec\beta$.

        \begin{proof}
            Pode ser provado pela seguinte sucessão de dedução:

            \begin{fitch}
                \fa\alpha\wedge\beta\to\alpha\\
                \fa\nec(\alpha\wedge\beta\to\alpha)\\
                \fa\nec(\alpha\wedge\beta\to\alpha)\to(\nec(\alpha\wedge\beta)\to\nec\alpha)\\
                \fa\nec(\alpha\wedge\beta)\to\nec\alpha\\
                \fa\alpha\wedge\beta\to\beta\\
                \fa\nec(\alpha\wedge\beta\to\beta)\\
                \fa\nec(\alpha\wedge\beta\to\beta)\to(\nec(\alpha\wedge\beta)\to\nec\beta)\\
                \fa\nec(\alpha\wedge\beta)\to\nec\beta\\
                \fa(\nec(\alpha\wedge\beta)\to\nec\alpha)\to(\nec(\alpha\wedge\beta)\to\nec\beta)\to\nec(\alpha\wedge\beta)\to\nec\alpha\wedge\nec\beta\\
                \fa(\nec(\alpha\wedge\beta)\to\nec\beta)\to\nec(\alpha\wedge\beta)\to\nec\alpha\wedge\nec\beta\\
                \fa\nec(\alpha\wedge\beta)\to\nec\alpha\wedge\nec\beta{}
            \end{fitch}
            \vspace*{-18pt}
            \qedhere
        \end{proof}
    \end{lemma}

    \begin{lemma}    
        $\vdash\nec\alpha\wedge\nec\beta\to\nec(\alpha\wedge\beta)$.
    \end{lemma}

    \begin{lemma}
        $\vdash\nec(\alpha\to\beta)\to\nec\alpha\to\beta$.
    \end{lemma}
        \section{Dualidades}

    \babireski{Ver~\cite{Zach} acerca dos axiomas duais e suas derivações.}

    \begin{theorem}
        A partir dos axiomas modais $\mathbf{B_1}$, $\mathbf{B_2}$ e $\mathbf{B_3}$, podem-se derivar os seguintes axiomas duais:
        \begin{align*}
            \mathbf{B_1^\diamond}&\coloneqq\nec(\alpha\to\beta)\to\pos\alpha\to\pos\beta\\
            \mathbf{B_2^\diamond}&\coloneqq\alpha\to\pos\alpha\\
            \mathbf{B_3^\diamond}&\coloneqq\pos\pos\alpha\to\pos\alpha\text{.}
        \end{align*}
    \end{theorem}

    \begin{theorem}
        $\vdash\nec(\alpha\to\beta)\to\pos\alpha\to\pos\beta$.

        \begin{proof}
            Pode ser provado pela seguinte sucessão de dedução:
            \begin{fitch}
                \fa(\nec\neg\alpha\to\neg\alpha)\to\neg\neg\alpha\to\neg\nec\neg\alpha\\
                \fa\nec\neg\alpha\to\neg\alpha\\
                \fa\neg\neg\alpha\to\neg\nec\neg\alpha{}
            \end{fitch}
        \end{proof}
    \end{theorem}

    \begin{theorem}
        $\vdash\alpha\to\pos\alpha$.
        \begin{proof}
            Pode ser provado pela seguinte sucessão de dedução:

            \begin{fitch}
                \fa(\nec\neg\alpha\to\neg\alpha)\to\neg\neg\alpha\to\neg\nec\neg\alpha\\
                \fa\nec\neg\alpha\to\neg\alpha\\
                \fa\neg\neg\alpha\to\neg\nec\neg\alpha{}
            \end{fitch}
        \end{proof}
    \end{theorem}

    \begin{theorem}
        $\vdash\pos\pos\alpha\to\pos\alpha$.
        \begin{proof}
            Isso equivale, por reescrita, a demonstrar $\vdash\neg\nec\neg\neg\nec\neg\alpha\to\neg\nec\neg\alpha$, o que pode ser feito pela seguinte sucessão de dedução:

            \begin{fitch}
                \fa(\nec\neg\alpha\to\neg\alpha)\to\neg\neg\alpha\to\neg\nec\neg\alpha\\
                \fa\nec\neg\alpha\to\neg\alpha\\
                \fa\neg\neg\alpha\to\neg\nec\neg\alpha{}
            \end{fitch}
        \end{proof}
    \end{theorem}
        \section{Traduções}
    A primeira tradução do sistema intuicionista ao sistema modal foi proposta por~\cite{Goedel} motivado pela possibilidade de leitura da necessidade como uma modalidade de construtividade. Ou seja, por meio dessa tradução, a sentença $\nec \varphi$ poderia ser lida como \textit{$\varphi$ pode ser provada construtivamente} \citep{Troelstra}. Gödel conjeiturou a corretude fraca dessa tradução, que foi posteriormente provada por~\cite{McKinsey} em conjunto com sua completude fraca.

    \begin{definition}[$\bullet^\circ$] Define-se a tradução $\bullet^\circ$ indutivamente da seguinte maneira:
        \begin{align*}
            p^\circ                     & \coloneqq p                                       \\
            \bot^\circ                  & \coloneqq \bot                                    \\
            {(\varphi \wedge \psi)}^\circ & \coloneqq \varphi^\circ \wedge \psi^\circ         \\
            {(\varphi \vee \psi)}^\circ   & \coloneqq \nec \varphi^\circ \vee \nec \psi^\circ \\
            {(\varphi \to \psi)}^\circ    & \coloneqq \nec \varphi^\circ \to \psi^\circ
            \tag*{\qed} 
        \end{align*}
    \end{definition}
    
    \begin{definition}[$\bullet^\nec$] Define-se a tradução $\bullet^\nec$ indutivamente da seguinte maneira:
        \begin{align*}
            p^\nec                     & \coloneqq \nec p                                        \\
            \bot^\nec                  & \coloneqq \bot                                          \\
            {(\varphi \wedge \psi)}^\nec & \coloneqq \varphi^\nec \wedge \psi^\nec     \\
            {(\varphi \vee \psi)}^\nec   & \coloneqq \varphi^\nec \vee \psi^\nec       \\
            {(\varphi \to \psi)}^\nec    & \coloneqq \nec (\varphi^\nec \to \psi^\nec)
            \tag*{\qed} 
        \end{align*}
    \end{definition}
    
    Faz-se interessante pontuar que as traduções $\bullet^\circ$ e $\bullet^\nec$ correspondem, respectivamente, às traduções $\bullet^\circ$ e $\bullet^*$ do sistema intuicionista ao sistema linear providas por~\cite{Girard}, sendo as primeiras correspondentes a uma ordem de avaliação por nome (\textit{call-by-name}) e as segundas a uma ordem de avaliação por valor (\textit{call-by-value}). 
    Ademais, as duas traduções providas são equivalentes, conforme demonstrado pelo teorema $\mathbf{T_2}$.

    \begin{theorem}
        $\forall\alpha\in\mathcal{L}_\mathbf{I}\point\nec\alpha^\circ\leftrightarrow\alpha^\nec$.

        \begin{proof}
            Prova por indução na profundidade de $\alpha$.
    
            \begin{case}
                \textbf{Caso 1} (Base)\textbf{.}
                    Para $|\alpha| = 0$, existem dois casos a serem considerados.
    
                    \begin{casee}
                        \textbf{Caso 1.1} ($\alpha = a$)\textbf{.}
                        $a^\circ = a$ e $a^\nec = \nec a$, assim $\nec a^\circ = a^\nec$ e, portanto, $\nec a^\circ \leftrightarrow a^\nec$.
                    \end{casee}

                    \begin{casee}
                        \textbf{Caso 2.1} ($\alpha = \bot$)\textbf{.}
                        $\bot^\circ = \bot$ e $\bot^\nec = \bot$. A ida $\nec\bot\to\bot$ consiste em um axioma, sendo, portando provada trivialmente pela sucessão de dedução $\sequence{\nec\bot\to\bot}$.
                        A volta $\bot\to\nec\bot$ equivale a provar, por meio do teorema da dedução, que $\set{\bot}\vdash_\mathbf{M}\nec\bot$, o que pode ser provado trivialmente pela sucessão de dedução $\sequence{\bot, \nec\bot}$, que consiste na invocação da premissa e aplicação da regra da necessitação, nessa ordem.
                    \end{casee}
            \end{case}
    
            \begin{case}
                \textbf{Caso 2} (Passo)\textbf{.} No passo, deve-se demonstrar que, caso $\nec\alpha^\circ\leftrightarrow\alpha^\nec$ para $|\alpha| = n$, 
                então $\nec\alpha^\circ\leftrightarrow\alpha^\nec$ para $|\alpha| = n + 1$, onde $n \in \mathbb{N}$. Assim, seja $\nec\alpha^\circ\leftrightarrow\alpha^\nec$ uma proposição verdadeira para $|\alpha| = k$, onde $k \in \mathbb{N}$. Existem os seguintes casos a serem considerados para $|\alpha| = k + 1$.
            \end{case}
    
                \begin{casee}
                    \textbf{Caso 2.1.}
                    Neste caso, consideraremos que $\alpha = \varphi\wedge\psi$.
                    Assim sendo, sabe-se que $\nec{(\varphi\wedge\psi)}^\circ=\nec(\varphi^\circ\wedge\psi^\circ)$ e ${(\varphi\wedge\psi)}^\nec=\varphi^\nec\wedge\psi^\nec$.
                    Pelo lemas $\mathbf{L_2}$ e $\mathbf{L_3}$ temos que $\nec(\varphi^\circ\wedge\psi^\circ)\leftrightarrow\nec\varphi^\circ\wedge\nec\psi^\circ$ e pela premissa da indução temos que $\nec\varphi^\circ\leftrightarrow\varphi^\nec$ e $\nec\psi^\circ\leftrightarrow\psi^\nec$.
                    Deste modo, $\nec{(\varphi\wedge\psi)}^\circ\leftrightarrow{(\varphi\wedge\psi)}^\nec$.
                \end{casee}
    
                \begin{casee}
                    \textbf{Caso 2.2.}
                    Neste caso, consideraremos que $\alpha = \varphi\vee\psi$.
                    Assim sendo, sabe-se que $\nec{(\varphi\vee\psi)}^\circ=\nec(\nec\varphi^\circ\vee\nec\psi^\circ)$ e ${(\varphi\vee\psi)}^\nec=\varphi^\nec\vee\psi^\nec$.
                    Para a ida, basta aplicar o axioma $\mathbf{B_2}$.
                    Para a volta, basta aplicar a regra da necessitação.
                \end{casee}
    
                \begin{casee}
                    \textbf{Caso 2.3.}
                    Neste caso, consideraremos que $\alpha=\varphi\to\psi$.
                    Assim sendo, sabe-se que $\nec{(\varphi\to\psi)}^\circ=\nec(\nec\varphi^\circ\to\psi^\circ)$ e ${(\varphi\to\psi)}^\nec=\nec(\varphi^\nec\to\psi^\nec)$.
                    Para a ida, basta aplicar o axioma $\mathbf{B_1}$.
                    Para a volta, basta aplicar o lema $\mathbf{L_4}$ seguido pela necessitação.
                    \qedhere
                \end{casee}
        \end{proof}
    \end{theorem}

        \section{Corretude}
    \begin{theorem}
        $\forall \alpha \in \mathcal{L}_\mathbf{I} \point \Gamma \vdash_\mathbf{I} \alpha \Rightarrow \Gamma^\nec \vdash_\mathbf{M} \alpha^\nec$
    \end{theorem}

    \begin{proof}
        Como $\Gamma \vdash_\mathbf{I} \alpha$, sabe-se que existe uma prova $\sequence{\varphi_i\mid 1 \leq i \leq n}$ tal que $\varphi_n = \alpha$. A demonstração deste teorema será feita por indução no tamanho $n$ da prova.

        \begin{case}
            \textbf{Passo} ($n = 1$)\textbf{.} A prova, caso possua tamanho $n = 1$, tem obrigatoriamente a forma $\sequence{\alpha}$. Deste modo, existem duas casos a serem considerados: $\alpha$ ser um axioma ou $\alpha$ ser uma premissa.
        \end{case}

            \begin{casee}
                \textbf{Caso 1} ($\alpha\in\Gamma$)\textbf{.} Como $\alpha\in\Gamma$, sabe-se que $\alpha^\nec\in\Gamma^\nec$, uma vez que $\Gamma^\nec = \set{\varphi^\nec\mid\varphi\in\Gamma}$. Desta forma, $\sequence{\alpha^\nec}$ constitui uma prova para $\Gamma^\nec\vdash\alpha^\nec$.
            \end{casee}

            \begin{casee}
                \textbf{Caso 2} ($\alpha\in\mathcal{A}$)\textbf{.}
            \end{casee}

                \begin{caseee}
                    \textbf{Caso 2.1} ($\mathbf{A_1}$)\textbf{.}
                    Deve-se demonstrar que $\vdash\nec(\alpha^\nec\to\nec(\beta^\nec\to\alpha^\nec))$.
                    Pelo teorema $\mathbf{T_1}$, basta provar que $\set{\alpha^\nec,\beta^\nec}\vdash\alpha^\nec$, o que pode ser feito pela seguinte sucessão de dedução:

                    \begin{fitch}
                        \fa\alpha^\nec&$\mathbf{P}$
                    \end{fitch}
                \end{caseee}

                \begin{caseee}
                    \textbf{Caso 2.2} ($\mathbf{A_2}$)\textbf{.}
                    Deve-se demonstrar que $\vdash\nec(\nec(\alpha^\nec\to\nec(\beta^\nec\to\gamma^\nec))\to\nec(\nec(\alpha^\nec\to\beta^\nec)\to\nec(\alpha^\nec\to\gamma^\nec)))$.

                    \begin{fitch}
                        \fa\nec(\alpha^\nec\to\beta^\nec)\to\alpha^\nec\to\beta^\nec&$\mathbf{B_2}$\\
                        \fa\nec(\alpha^\nec\to\beta^\nec)&$\mathbf{P}$\\
                        \fa\alpha^\nec\to\beta^\nec&$\mathbf{R_1}\;\sequence{1,2}$\\
                        \fa\alpha^\nec&$\mathbf{P}$\\
                        \fa\beta^\nec&$\mathbf{R_1}\;\sequence{3,4}$\\
                        \fa\nec(\alpha^\nec\to\nec(\beta^\nec\to\alpha^\nec))\to\alpha^\nec\to\nec(\beta^\nec\to\alpha^\nec)&$\mathbf{B_2}$\\
                        \fa\nec(\alpha^\nec\to\nec(\beta^\nec\to\alpha^\nec))&$\mathbf{P}$\\
                        \fa\alpha^\nec\to\nec(\beta^\nec\to\alpha^\nec)&$\mathbf{R_1}\;\sequence{6,7}$\\
                        \fa\nec(\beta^\nec\to\alpha^\nec)&$\mathbf{R_1}\;\sequence{4,8}$\\
                        \fa\nec(\beta^\nec\to\alpha^\nec)\to\beta^\nec\to\gamma^\nec&$\mathbf{B_2}$\\
                        \fa\beta^\nec\to\gamma^\nec&$\mathbf{R_1}\;\sequence{10,9}$\\
                        \fa\gamma^\nec&$\mathbf{R_1}\;\sequence{11,5}${}
                    \end{fitch}
                \end{caseee}

                \begin{caseee}
                    \textbf{Caso 2.3} ($\mathbf{A_3}$)\textbf{.}

                    \begin{fitch}
                        \fa\alpha^\nec\to\beta^\nec\to\alpha^\nec\wedge\beta^\nec&$\mathbf{A_3}$\\
                        \fa\alpha^\nec&$\mathbf{P}$\\
                        \fa\beta^\nec\to\alpha^\nec\wedge\beta^\nec&$\mathbf{R_1}\;\sequence{1,2}$\\
                        \fa\beta^\nec&$\mathbf{P}$\\
                        \fa\alpha^\nec\wedge\beta^\nec&$\mathbf{R_1}\;\sequence{3,4}$
                    \end{fitch} 
                \end{caseee}

                \begin{caseee}
                    \textbf{Caso 2.4} ($\mathbf{A_4}$)\textbf{.}
                    Deve-se demonstrar que $\vdash\nec(\alpha^\nec\wedge\beta^\nec\to\alpha^\nec)$, o que pode ser feito ela seguinte sucessão de dedução:

                    \begin{fitch}
                        \fa\alpha^\nec\wedge\beta^\nec\to\alpha^\nec&$\mathbf{A_4}$\\
                        \fa\nec(\alpha^\nec\wedge\beta^\nec\to\alpha^\nec)&$\mathbf{R_2}\;\sequence{1}$
                    \end{fitch}
                \end{caseee}

                \begin{caseee}
                    \textbf{Caso 2.5} ($\mathbf{A_5}$)\textbf{.}
                    Deve-se demonstrar que $\vdash\nec(\alpha^\nec\wedge\beta^\nec\to\beta^\nec)$, o que pode ser feito ela seguinte sucessão de dedução:

                    \begin{fitch}
                        \fa\alpha^\nec\wedge\beta^\nec\to\beta^\nec&$\mathbf{A_5}$\\
                        \fa\nec(\alpha^\nec\wedge\beta^\nec\to\beta^\nec)&$\mathbf{R_2}\;\sequence{1}$
                    \end{fitch}
                \end{caseee}

                \begin{caseee}
                    \textbf{Caso 2.6} ($\mathbf{A_6}$)\textbf{.}
                    Deve-se demonstrar que $\vdash\nec(\alpha^\nec\to\alpha^\nec\vee\beta^\nec)$, o que pode ser feito ela seguinte sucessão de dedução:

                    \begin{fitch}
                        \fa\alpha^\nec\to\alpha^\nec\vee\beta^\nec&$\mathbf{A_6}$ \\
                        \fa\nec(\alpha^\nec\to\alpha^\nec\vee\beta^\nec)&$\mathbf{R_2}\;\sequence{1}$
                    \end{fitch}
                \end{caseee}

                \begin{caseee}
                    \textbf{Caso 2.7} ($\mathbf{A_7}$)\textbf{.}
                    Deve-se demonstrar que $\vdash\nec(\beta^\nec\to\alpha^\nec\vee\beta^\nec)$, o que pode ser feito ela seguinte sucessão de dedução:

                    \begin{fitch}
                        \fa\beta^\nec\to\alpha^\nec\vee\beta^\nec&$\mathbf{A_7}$\\
                        \fa\nec(\beta^\nec\to\alpha^\nec\vee\beta^\nec)&$\mathbf{R_2}\;\sequence{1}$
                    \end{fitch}
                \end{caseee}

                \begin{caseee}
                    \textbf{Caso 2.8} ($\mathbf{A_8}$)\textbf{.}
                    Deve-se demonstrar que $\vdash(\alpha^\nec\fishhook\gamma^\nec)\fishhook(\beta^\nec\fishhook\gamma^\nec)\fishhook\alpha^\nec\vee\beta^\nec\fishhook\gamma^\nec$.
                    Pelo teorema $\mathbf{T_1}$, basta provar que $\set{\alpha\fishhook\gamma,\beta\fishhook\gamma,\alpha\vee\gamma}\vdash\alpha^\nec$, o que pode ser feito pela seguinte sucessão de dedução:
                    
                    % $\vdash\nec(\nec(\alpha^\nec\to\gamma^\nec)\to\nec(\nec(\beta^\nec\to\gamma)\to\nec(\alpha^\nec\vee\beta^\nec\to\gamma^\nec)))$.
                    
                    \begin{fitch}
                        \fa(\alpha^\nec\to\gamma^\nec)\to(\beta^\nec\to\gamma^\nec)\to\alpha^\nec\vee\beta^\nec\to\gamma^\nec&$\mathbf{A_8}$\\
                        \fa\nec(\alpha^\nec\to\gamma^\nec)\to\alpha^\nec\to\gamma^\nec&$\mathbf{B_2}$\\
                        \fa\nec(\alpha^\nec\to\gamma^\nec)&$\mathbf{P}$\\
                        \fa\alpha^\nec\to\gamma^\nec&$\mathbf{R_1}\;\sequence{2,3}$\\
                        \fa(\beta^\nec\to\gamma^\nec)\to\alpha^\nec\vee\beta^\nec\to\gamma^\nec&$\mathbf{R_1}\;\sequence{1,4}$\\
                        \fa\nec(\beta^\nec\to\gamma^\nec)\to\beta^\nec\to\gamma^\nec&$\mathbf{B_2}$\\
                        \fa\nec(\beta^\nec\to\gamma^\nec)&$\mathbf{P}$\\
                        \fa\beta^\nec\to\gamma^\nec&$\mathbf{R_1}\;\sequence{6,7}$\\
                        \fa\alpha^\nec\vee\beta^\nec\to\gamma^\nec&$\mathbf{R_1}\;\sequence{5,8}$\\
                        \fa\alpha^\nec\vee\beta^\nec&$\mathbf{P}$\\
                        \fa\gamma^\nec&$\mathbf{R_1}\;\sequence{9,10}${}
                    \end{fitch}
                \end{caseee}

                \begin{caseee}
                    \textbf{Caso 2.9} ($\mathbf{A_\bot}$)\textbf{.}
                    Deve-se demonstrar que $\vdash\nec(\bot\to\alpha^\nec)$.
                    Pelo teorema $\mathbf{T_1}$, basta provar que $\set{\bot}\vdash\alpha^\nec$, o que pode ser feito pela seguinte sucessão de dedução:

                    \begin{fitch}
                        \fa((\alpha^\nec\to\bot)\to\bot)\to\alpha^\nec&$\mathbf{A_\neg}$\\
                        \fa\bot\to(\alpha^\nec\to\bot)\to\bot&$\mathbf{A_1}$\\
                        \fa\bot&$\mathbf{P}$\\
                        \fa(\alpha^\nec\to\bot)\to\bot&$\mathbf{R_1}\;\sequence{2, 3}$\\
                        \fa\alpha^\nec&$\mathbf{R_1}\;\sequence{1, 4}$.
                    \end{fitch}
                \end{caseee}

                \begin{caseee}
                    \textbf{Caso 2.9} ($\mathbf{R_1}$)\textbf{.}
                    Deve-se demonstrar que, se $\vdash\nec(\alpha^\nec\to\beta^\nec)$ ($\mathbf{H_1}$) e $\vdash\alpha^\nec$ ($\mathbf{H_2}$), então $\beta^\nec$.
                    Isso pode ser feito pela seguinte sucessão de dedução:

                    \begin{fitch}
                        \fa\nec(\alpha^\nec\to\beta^\nec)\to\alpha^\nec\to\beta^\nec&$\mathbf{B_2}$\\
                        \fa\nec(\alpha^\nec\to\beta^\nec)&$\mathbf{H_1}$\\
                        \fa\alpha^\nec\to\beta^\nec&$\mathbf{R_1}\;\sequence{1, 2}$\\
                        \fa\alpha^\nec&$\mathbf{H_2}$\\
                        \fa\beta^\nec&$\mathbf{R_1}\;\sequence{3, 4}$.
                    \end{fitch}
                \end{caseee}
    \end{proof}
        \section{Completude}
    \babireski{Não vai rolar de provar a completude como~\cite{Troelstra}. Vou precisar procurar outros artigos.}

    \bibliographystyle{plainnat}
    \bibliography{bibliography}
\end{document}
