\documentclass[12pt]{report}

\PassOptionsToPackage{table,xcdraw}{xcolor}

% Package imports
\usepackage{amsthm}
\usepackage{amsmath}
\usepackage{amssymb}
\usepackage{epigraph}
\usepackage{ragged2e}
\usepackage[round]{natbib}
\usepackage[english, french, brazil]{babel}
\usepackage{styles/cases}
\usepackage{styles/fitch}
\usepackage{styles/functions}
\usepackage{styles/modalities}
\usepackage{enumitem}
\usepackage[bottom]{footmisc}
\usepackage[T1]{fontenc}
\usepackage[a4paper, margin=1.2in]{geometry}
\usepackage{dirtytalk}
\usepackage{titlesec}
\usepackage{setspace}
\usepackage{hyperref}
\usepackage{lipsum}
\usepackage{stackengine}
\usepackage{multirow}
\usepackage[table,xcdraw]{xcolor}
\usepackage{listings}
\usepackage{lstfiracode}
\usepackage{fontspec}
\usepackage{styles/coq}

\newfontfamily\Fira{Fira Code}[Contextuals=Alternate]
\lstset{language=Coq, basicstyle=\Fira}

% Math font settings
\DeclareMathAlphabet{\mathbbm}{U}{bbm}{m}{n}

% Command definitions
\newcommand{\entails}{\mathrel{\vdash}}
\newcommand{\point}{\mathpunct{.}}

% Theorem styles
\newtheorem*{notation}{Notação}
\newtheorem{example}{Exemplo}
\newtheorem{definition}{Definição}
\newtheorem{lemma}{Lema}
\newtheorem{theorem}{Teorema}

% Chapter title formatting
\titleformat{\chapter}[block]
  {\normalfont\huge\bfseries}{\thechapter.}{1em}{\Huge}
\titlespacing*{\chapter}{0pt}{-19pt}{20pt}

% Abstract title formatting
\titleformat{\abstract}[block]
  {\normalfont\huge\bfseries}{\abstractname}{1em}{\Huge}
\titlespacing*{\abstract}{0pt}{-19pt}{20pt}

% Hyperref settings
\hypersetup{
    pdftitle={Uma formalização da interpretação modal do sistema intuicionista},
    pdfauthor={Elian Babireski},
    colorlinks=true,
    linkcolor=blue,
    citecolor=blue,
    filecolor=blue,
    urlcolor=blue
}

% Spacing settings
\onehalfspacing{}


\begin{document}
    \title{Uma formalização assistida por computador da interpretação modal do sistema intuicionista}
    \author{Elian Babireski}
    \date{2024}

    \maketitle

    UNIVERSIDADE DO ESTADO DE SANTA CATARINA

CENTRO DE CIÊNCIAS TECNOLÓGICAS

BACHARELADO EM CIÊNCIA DA COMPUTAÇÃO

ELIAN GUSTAVO CHORNY BABIRESKI

UMA FORMALIZAÇÃO ASSISTIDA POR COMPUTADOR DA INTERPRETAÇÃO MODAL DO SISTEMA INTUICIONISTA

JOINVILLE

2025

    \begin{titlepage}
\vspace*{\fill}
\noindent
\begin{center}
\textit{Aos meus pais, e a todos os amigos feitos durante esta jornada.}
\end{center}
\vspace*{\fill}
\end{titlepage}
    	\vspace*{\fill}					% Posição vertical
	\hrule							% Linha horizontal
	\begin{center}					% Minipage Centralizado
	\begin{minipage}[c]{12.5cm}		% Largura
	
	Elian Gustavo Chorny Babireski
	
	\hspace{0.5cm} Uma formalização assistida por computador da imersão modal do sistema intuicionista  / Elian Gustavo Chorny Babireski. --
	Joinville, 2025-
	
	\hspace{0.5cm} \pageref{LastPage} p. : il. (algumas color.) ; 30 cm.\\
	
	\hspace{0.5cm} Orientadora:~Karina Girardi Roggia\\
	
	\hspace{0.5cm}
	\parbox[t]{\textwidth}{Trabalho de conclusão de curso~--~Universidade do Estado de Santa Catarina,
	2025.}\\
	
	\hspace{0.5cm}
		1. Tradução.
		2. Imersão.
        3. Formalização.
        4. Intuicionismo.
        5. Modalidades.
        6. Efeitos.
        7. Compilação.
		I. Karina Girardi Roggia.
		II. Universidade do Estado de Santa Catarina.
		III. Departamento de Computação.
		IV. Uma formalização assistida por computador da imersão modal do sistema intuicionista.\\ 			
	
	\hspace{8.75cm} CDU 02:141:005.7\\
	
	\end{minipage}
	\end{center}
	\hrule

    \begin{abstract}
        Os sistemas modais possuem atributos interessantes para a representação de efeitos computacionais.
        Uma linguagem baseada em algum destes sistemas, podendo representar efeitos monadicamente, imerge e pode ser imersa em continuações.
        O estilo de passagem por continuações trata-se de uma das diversas representações usadas em compiladores, o que torna estes sistemas relevantes do ponto de vista da compilação.
        O sistema intuicionista, do mesmo modo, possui interesses computacionais, uma vez que a noção intuicionista de \emph{construção} assemelha-se com a noção de \emph{computação}.
        Este trabalho apresenta duas traduções do sistema intuicionista a um dos sistemas modais em conjunto com as demonstrações das propriedades de interderivação e correção.
        Todas as definições e asserções são implementadas e verificadas pela provador interativo de teoremas \emph{Rocq}.

        \textit{Palavras-chave} --- Tradução. Imersão. Formalização. Intuitionismo. Modalidades. Efeitos. Compilação.
    \end{abstract}

    \begin{otherlanguage}{english} 
        \begin{abstract}
            One of the first translations of a deduction system into another presented in the literature involves the translation of the intuitionistic system into the modal system $\mathbf{S4}$ to interpret the modality of \emph{necessity} as a modality of \emph{provability}. Decades later, a metalanguage was introduced to semantically represent notions of computation such as partiality, non-determinism, exceptions, and continuations. The encoding of these notions shows great similarities with the modal axioms of the $\mathbf{S4}$ system, making the mentioned translation relevant within a proof-program interpretation perspective. Thus, inspired by the various cases of computer-assisted formalizations, this work aims to formally verify this translation.

            \textit{Keywords} --- translation, embedding, formalization, intuitionism, modalities, effects, compilation.
        \end{abstract}
    \end{otherlanguage}

    \tableofcontents

    \setlength\epigraphwidth{.5\textwidth}
\setlength\epigraphrule{0pt}

\vspace*{\fill}
\epigraph{\justifying\say{\say{Oh, you can't help that,} said the Cat: \say{we're all mad here. I'm mad. You're mad.} \say{How do you know I'm mad?} said Alice. \say{You must be,} said the Cat, \say{or you wouldn't have come here.}}}{--- Lewis Carroll, \textit{Alice in Wonderland}}
    \chapter{Introdução}

As lógicas modais consistem em um conjunto de extensões da lógica clássica que contam com a adição de um ou mais operadores, chamados modalidades, que qualificam sentenças. No caso do sistema \textbf{S4}, são adicionadas as modalidades de necessidade ($\nec$) e possibilidade ($\pos$) em conjunto à regra da necessitação\footnote{Se $\vdash A$ então $\vdash \nec A$} e os axiomas $\mathbf{K}\text{: } \nec(A \to B) \to \nec A \to \nec B$, $\mathbf{T}\text{: } \nec A \to A$ e $\text{\textbf{4}: } \nec A \to \nec \nec A$ \citep{Troelstra}. Ademais, pode-se derivar nesse sistema, por meio da dualidade entre as modalidades\footnote{$\pos A \equiv \neg \nec \neg A $}, sentenças duais aos axiomas \textbf{T} e \textbf{4}, sendo elas $\mathbf{T}_\meddiamond \text{: } A \to \pos A$ e $\mathbf{4}_\meddiamond \text{: } \pos \pos A \to \pos A$, respectivamente~\cite{Zach}.

As mônadas ganharam destaque na área de linguagens de programação desde que~\cite{Moggi} formalizou uma metalinguagem que faz uso dessas estruturas para modelar noções de computação --- como parcialidade, não-determinismo, exceções e continuações --- de uma maneira puramente funcional. Pode-se notar uma grande semelhança entre as sentenças $\mathbf{T}_\meddiamond$ e $\mathbf{4}_\meddiamond$ e as transformações naturais monádicas $\mathbf{\eta:} 1_C \to T$ e $\mathbf{\mu:} T^2 \to T$, respectivamente. Nesse sentido,~\cite{Pfenning} demonstraram que se pode traduzir essa metalinguagem para o sistema \textbf{S4} da lógica modal, pelo qual se torna interessante analisar esse sistema como uma linguagem de programação sob a ótica do isomorfismo de Curry-Howard.

~\cite{Troelstra} apresentam duas traduções equivalentes da lógica intuicionista para o sistema \textbf{S4} da lógica modal, sendo um deles correspondente a uma abordagem \textit{call-by-name} e outra a um abordagem \textit{call-by-value}. Tais traduções possuem grande similaridade com as traduções da lógica intuicionista para a lógica linear definidas por~\cite{Girard}. Essas traduções equivalem à tradução por negação dupla que, por sua vez, equivalem a traduções \textit{continuation-passing style} (CPS) em compiladores por meio do isomorfismo de Curry-Howard~\citep{Reynolds}, o que torna esse tema interessante no ponto de vista de compilação.

Durante grande parte da história, provas lógicas e matemáticas eram validadas manualmente pela comunidade acadêmica, o que muitas vezes --- a depender do tamanho e complexidade da prova --- se mostrava ser um trabalho complexo e sujeito a erros. Hoje em dia, exitem \textit{softwares} chamados assistentes de provas que permitem verificar --- graças ao isomorfismo de Curry-Howard --- a corretude de provas~\citep{Chlipala}. O assistente de provas que será usado neste trabalho é o \textsc{coq}, que utiliza o cálculo de construções indutivas e um conjunto axiomático pequeno para permitir a escrita de provas simples e intuitivas~\citep{Coq}.

    \section{Objetivos}
    Este trabalho consiste numa continuação do desenvolvimento da biblioteca de formalização de sistemas modais normais iniciado por~\cite{Silveira} e posteriormente expandida de forma a permitir a fusão de sistemas modais por~\cite{Nunes}. Nele, formalizaremos as traduções do sistema intuicionista ao sistema modal $\mathbf{S4}$ no asssitente de provas \textsc{coq} e provaremos suas propriedades. Uma formalização de traduções entre sistemas de dedução similar a nossa foi feita por~\cite{Sehnem}, neste caso tendo como alvo o sistema linear de~\cite{Girard}. Todas as formalizações citadas acima deram-se no assistente de provas \textsc{coq}, o mesmo assistente usado neste trabalho. Como objetivos específicos, listamos:

    \begin{itemize}
        \item Fornecer uma introdução ao conceito de sistemas de dedução;
        \item Fornecer uma introdução ao conceito de traduções entre sistemas;
        \item Fornecer uma introdução ao sistema intuicionista;
        \item Fornecer uma introdução aos sistemas modais, em especial o $\mathbf{S4}$;
        \item Apresentar as traduções do sistema intuicionista ao sistema $\mathbf{S4}$;
        \item Provar manualmente a correção das traduções providas bem como outras propriedades pertinentes;
        \item Formalizar as provas no provador de teoremas interativo \textsc{coq}.
    \end{itemize}

    \section{Estruturação}
    Estruturaremos este trabalho em cinco partes. A parte \textbf{(1)} trata-se desta introdução. A parte \textbf{(2)} consiste numa fundamentação de conceitos basilares ao desenvolvimento deste trabalho, notadamente os conceitos de \emph{sistemas de dedução}, \emph{traduções} e \emph{provadores de teoremas}. A parte \textbf{(3)} apresenta as definições dos sistemas e traduções relevantes a este trabalho. Na parte \textbf{(4)} são provadas todas as propriedades abarcadas no escopo deste trabalho. Por fim, a parte \textbf{(5)} compreende considerações parciais acerca do desenvolvido até o momento.
    \chapter{Fundamentação}

Nesta parte do trabalho, serão apresentadas definições gerais que fundamentarão as definições mais estritas que serão apresentadas futuramente. Notadamente, fundamentaremos as noções de sistemas e traduções. Ademais, discorreremos acerca da noção de provadores, que serão usados para certificar as provas apresentadas posteriormente. Antes disso, entretanto, introduziremos duas notações que serão usadas copiosamente, uma para o conjunto das partes e outra para sucessões.

\begin{notation}
    Seja $A$ um conjunto, $\wp(A)$ denota o conjunto $\set{X\mid X\subseteq A}$.
\end{notation}

\begin{notation}
    Seja $i\in\mathbb{N}^+$ e $n\in\mathbb{N}$, $\sequence{a_i\mid i\leq n}$ denota uma sucessão de $n$ elementos de modo que o elemento $a_i$ encontra-se na posição $i$.
\end{notation}

\section{Sistemas}

Sistemas de dedução buscam formalizar e sistematizar o processo de razoamento. Estudos acerca disso datam da antiguidade, dentre os quais destaca-se~\cite{Aristotle}. Considera-se que os estudos modernos neste campo foram, dentre outras pessoas, fundados por~\cite{Frege} e continuados por~\cite{Russel-A,Russel-B,Russel-C}. Estas investigações --- bem como outras --- levaram ao desenvolvimento do sistema hoje tido como padrão. Posteriormente a isso, viu-se o surgimento de diversos sistemas não-padrões, fato que --- conforme~\cite{Beziau-B} --- justifica uma conceituação de sistema de dedução, que apresentaremos nesta seção.

Ainda segundo~\cite{Beziau-B}, os primeiros desenvolvimentos neste sentido foram feitos por~\cite{Tarski}, que define o conceito de dedução com base num operador de fecho $C\mathrel{:}\wp(\mathcal{L})\to\wp(\mathcal{L})$, sendo $\mathcal{L}$ um conjunto qualquer. Neste trabalho entretanto usaremos a definição proposta por~\cite{Beziau} baseada numa relação de dedução ${\vdash}\subseteq\wp(\mathcal{L})\times\mathcal{L}$, uma vez que, por sua simplicidade, não traz elementos irrelevantes aos intuitos deste. Cabe destacar, conforme apontam~\cite{Font}, que ambas as definições são equivalentes\footnote{Destaca-se, entretanto, que a definição de~\cite{Tarski} requer a satisfação de postulados não requeridos por~\cite{Beziau}, sendo portanto menos geralista.}, uma vez que $\Gamma\entails\alpha$ se e somente se $\alpha\in C(\Gamma)$.

\begin{definition}[Sistema]
    Um sistema de dedução consiste num par $\mathbf{L} = \sequence{\mathcal{L}, \vdash}$, onde $\mathcal{L}$ consiste em um conjunto e ${\vdash}\subseteq\wp(\mathcal{L})\times\mathcal{L}$ em uma relação sobre o produto cartesiano do conjunto das partes de $\mathcal{L}$ e o conjunto $\mathcal{L}$, sem demais condições.
    \qed{}
\end{definition}

Conforme~\cite{Beziau} aponta, a qualidade e quantidade dos elementos de um sistema $\mathbf{L}=\sequence{\mathcal{L}, \vdash}$ não são especificados, portanto sendo esta uma definição de grande generalidade. Neste sentido, com base no escopo deste trabalho, restringiremos a definição do conjunto $\mathcal{L}$ --- dito \emph{linguagem} --- a linguagens proposicionais. Os elementos destas, aos quais daremos o nome de \emph{sentenças}, notabilizam-se por serem formadas por \emph{letras} --- que consistem em proposições indivisas --- e \emph{operadores} --- que podem gerar proposições maiores a partir de proposições menores. Ao par formado por letras e operadores daremos o nome \emph{assinatura}, conforme abaixo.

\begin{definition}[Assinatura]
    Uma assinatura proposicional consiste num par $\Sigma=\sequence{\mathcal{P},\mathcal{C}}$, onde $\mathcal{P}$ consiste num conjunto letras e $\mathcal{C}=\bigcup\set{\mathcal{C}_i\mid i\in\mathbb{N}}$ num conjunto de operadores de modo que $\mathalpha{\bullet}\in\mathcal{C}_n$ se e somente se $\mathalpha{\bullet}$ possuir aridade $n$.
    \qed{}
\end{definition}

\begin{notation}
    Seja $\mathcal{C}$ um conjunto de operadores, $\mathalpha{\bullet}^n$ denota um operador $\mathalpha{\bullet}\in\mathcal{C}_n$.
\end{notation}

Podemos interpretar os conjuntos $\mathcal{P}$ e $\mathcal{C}$ de uma assinatura $\Sigma=\sequence{\mathcal{P},\mathcal{C}}$ como contrutores de sentenças.
Neste sentido, o conjunto $\mathcal{C}_0$ assemelha-se mais ao conjunto $\mathcal{P}$, uma vez que seus elementos --- ditos \emph{constantes} --- não geram sentenças maiores partindo de sentenças menores.
Nota-se que uma assinatura constitui um elemento suficiente para definirmos indutivamente a linguagem de um sistema, conforme definido abaixo de maneira similar a~\cite{Franks}.
Por fim, destacamos que, para todos os sistemas apresentados neste trabalho, usaremos o conjunto de letras $\mathcal{P}=\set{p_i\mid i\in\mathbb{N}}$ e letras romanas em caixa-baixa para representar seus elementos.

\begin{definition}[Linguagem]
    Seja $\Sigma=\sequence{\mathcal{P},\mathcal{C}}$ uma assinatura proposicional. Uma linguagem proposicional $\mathcal{L}$ induzida a partir de $\Sigma$ consiste no menor conjunto de sentenças bem-formadas induzido a partir das seguintes regras:
    \begin{enumerate}[label=\textbf{\emph{(\alph*)}}, left=\parindent]
        \item$\mathcal{P}\subseteq\mathcal{L}$
        \item\text{Se }$\mathalpha{\bullet}\in\mathcal{C}_n\text{ e }\set{\varphi_i\mid i\leq n}\subseteq\mathcal{L}\text{, então }\bullet\sequence{\varphi_i\mid i\leq n}\in\mathcal{L}$.\qed{}
    \end{enumerate}
\end{definition}

Neste trabalho, representaremos sentenças por letras gregas em caixa-baixa e conjuntos de sentenças por letras gregas em caixa-alta.\footnote{Desconsiderando-se o $\Sigma$, usado para representar assinaturas.}
Ademais, impõe-se definir a noção de profundidade de uma sentença. Esta noção, em termos simples, consiste no comprimento do maior ramo da construção da dada sentença. A definição provida abaixo consiste numa generalização para quaisquer aridades da definição dada por~\cite{Troelstra}. Usaremos essa definição futuramente para fazer demonstrações por meio provas indutivas sobre esta propriedade.

\begin{definition}[Profundidade]
    Seja $\mathbf{L} = \sequence{\mathcal{L}, \vdash}$ um sistema com linguagem induzida a partir de uma assinatura $\Sigma=\sequence{\mathcal{P},\mathcal{C}}$. Considerando-se uma proposição $a\in\mathcal{P}$, um operador ${\bullet}\in\mathcal{C}$ e uma aridade $n>0$, definimos a profundidade $|\alpha|$ de uma sentença $\alpha\in\mathcal{L}$ indutivamente da seguinte maneira:
    \begin{align*}
        |a|&\coloneqq 0\\
        |{\bullet^0}|&\coloneqq 0\\
        |{\bullet^n\sequence{\varphi_i\mid i\leq n}}|&\coloneqq\max\set{|\varphi_i|\mid i\leq n}+1.
        \tag*{\qed} 
    \end{align*}
\end{definition}

Com isso, encerram-se as definições relacionadas a linguagens de sistemas de dedução. Agora, apresentaremos definições relacionadas a relações de dedução, que gozam da mesma generalidade dada a liguagens. Deste modo, a relação $\mathalpha{\vdash}$ pode ser tanto uma relação de \emph{derivação} --- definida sintaticamente --- quanto uma relação de \emph{satisfação}\footnote{Sendo esta denotada por $\mathalpha{\vDash}$.} --- definida semanticamente. Neste trabalho, serão abordados apenas sistemas definidos sobre relações de derivação. Cabe destacar, entretanto, que nada na definição de tradução impede que esta seja feita sobre relações de satisfação, conforme veremos com mais detalhes futuramente.

Neste trabalho, definiremos a relações de dedução baseada em axiomatizações, ou seja, em conjuntos de \emph{axiomas} --- sentenças postuladas como verdadeiras --- e conjuntos de \emph{regras de dedução} --- que permitem derivar mais sentenças verdadeiras caso certas condições sejam satisfeitas. Axiomatizações consistem numa abordagem hilbertiana de dedução que, segundo~\cite{Troelstra}, distinguem-se por conter um conjunto reduzido de regras de dedução que nunca descartam premissas. Ainda baseando-se em~\cite{Troelstra} e em contraste a~\cite{Frege} e~\cite{Hilbert-A, Hilbert-B}, preferiremos esquemas de axiomas a axiomas individuais de modo a eliminarmos a necessidade de instanciações.

\begin{definition}[Axiomatização]
    Seja $\mathbf{L}=\sequence{\mathcal{L},\vdash}$ um sistema. Uma axiomatização para o sistema $\mathbf{L}$ consiste num par $\mathcal{H}=\sequence{\mathcal{A},\mathcal{R}}$, sendo $\mathcal{A}$ um conjunto de esquemas de axiomas e $\mathcal{R}$ um conjunto de regras de dedução.
\end{definition}


Neste trabalho, consideraremos axiomatizações definidas em relação a conjuntos de premissas $\Gamma\subseteq\mathcal{L}$. 
Por este motivo, todas as axiomatizações apresentadas futuramente neste trabalho conterão implicitamente a \emph{regra da premissa} $\mathbf{P}$ --- sendo esta regra definida como se $\alpha\in\Gamma$, então $\Gamma\entails\alpha$.
Similarmente, faremos o mesmo para a \emph{regra do enfraquecimento} $\mathbf{E}$ --- definida aqui como se $\Gamma\entails\alpha$, então $\Gamma\cup\Delta\entails\alpha$ ---, tendo em vista que todos os sistemas apresentados neste trabaho são estruturais. Assim, tendo-se claro o conceito de axiomatização, podemos finalmente o conceito de dedução.

\begin{definition}[Dedução]
    Seja um sistema $\mathbf{L} = \sequence{\mathcal{L},{\vdash}}$ com uma relação de dedução definida sobre uma axiomatização $\mathcal{H} = \sequence{\mathcal{A},\mathcal{R}}$ e  seja um conjunto de sentenças $\Gamma\cup\set{\alpha}\subseteq\mathcal{L}$.
    A dedução $\Gamma\vdash\alpha$ vale se e somente se houver sucessão de sentenças $\sequence{\varphi_i\in\mathcal{L}\mid i\leq n}$ de modo que $\varphi_n=\alpha$ e que cada sentença $\varphi_i$ ou tenha sido gerada ou por algum esquema $\mathbf{A}\in\mathcal{A}$ ou pela aplicação de alguma regra $\mathbf{R}\in\mathcal{R}$ a sentenças anteriores.
    \qed{}
\end{definition}

\section{Traduções}

Traduções entre sistemas consistem em funções que mapeiam sentenças de um sistema a sentenças de outro, garantindo certas propriedades. As propriedades a serem garantidas variam e ainda são discutidas na literatura, deixando a definição exata de tradução --- assim como houve com a definição de sistema --- varie de acordo com a predileção e as necessidades de cada autor. Nesta seção, serão abordadas historicamente noções de tradução entre sistemas, bem como serão definidos e nomeados os conceitos de tradução que serão usados no restante deste trabalho.

\begin{definition}[Condições]
    Chamaremos a condição $\varnothing\entails_\mathbf{A}\alpha$ implica em $\varnothing\entails_\mathbf{B}\alpha^*$ de correção fraca e a condição $\varnothing\entails_\mathbf{B}\alpha^*$ implica em $\varnothing\entails_\mathbf{A}\alpha$ de completude fraca. Analogamente, considerando-se dedução com premissas, chamaremos a condição $\Gamma\entails_\mathbf{A}\alpha$ implica em $\Gamma^*\entails_\mathbf{B}\alpha^*$ de correção forte e a condição $\Gamma^*\entails_\mathbf{B}\alpha^*$ implica em $\Gamma\entails_\mathbf{A}\alpha$ de completude forte.
\end{definition}

Historicamente, autores usaram diferentes combinações das condições apresentadas acima e, em certos casos, outras. Neste trabalho, adotaremos uma noção forte de tradução que requer tanto a correção forte quanto a completude forte, conforme~\cite{Coniglio}. Definiremos, ainda, uma notação que nos permite aplicar sucintamente a tradução a todos os elementos de um conjunto.

\begin{definition}[Tradução] 
    Uma sentença $\varphi$ de um sistema $\mathbf{A} = \langle \mathcal{L}_\mathbf{A}, \vdash_\mathbf{A}\rangle$ pode ser traduzida a uma sentença $\varphi^*$ em um sistema $\mathbf{B} = \langle \mathcal{L}_\mathbf{B}, \vdash_\mathbf{B} \rangle$ caso exista uma função $\bullet^* : \mathcal{L}_\mathbf{A} \to \mathcal{L}_\mathbf{B}$ que garanta que $\Gamma \vdash_\mathbf{A} \varphi \Leftrightarrow \Gamma^* \vdash_\mathbf{B} \varphi^*$.
    \qed{}
\end{definition}

\begin{notation}
    Seja $\Gamma\in\wp(\mathcal{L}_\mathbf{A})$ um conjunto de sentenças bem-formadas e $\bullet^*\mathrel{:}\mathcal{L}_\mathbf{A}\to\mathcal{L}_\mathbf{B}$ uma tradução. $\Gamma^*$ denota o conjunto $\set{\alpha^*\mid\alpha\in\Gamma}\in\wp(\mathcal{L}_\mathbf{B})$, ou seja, a aplicação da tradução a todos os elementos do conjunto $\Gamma$.
    \qed{}
\end{notation}

A primeira tradução entre dois sistemas conhecida na literatura foi definida por~\cite{Kolmogorov} como uma maneira de demonstrar que o uso da \emph{lei do terceiro excluso}\footnote{Definido como $\entails\alpha\vee\neg\alpha$.} não leva a contradições. Essa definição consiste basicamente em prefixar uma dupla negação a cada elemento da construção de uma dada sentença, motivo pelo qual chamaremos essa tradução de \emph{tradução de negação dupla} \citep{Coniglio}. Essa mesma tradução foi também descoberta independentemente por Gödel e por Getzen. Curiosamente, essa tradução mostra-se relevante para o escopo deste trabalho, uma vez que consiste na contraparte da passagem por continuações segundo a interpretação prova-programa.

\begin{definition}[$\bullet^\neg$] Define-se a tradução $\bullet^\neg:\mathcal{L}_\mathbf{C}\to\mathcal{L}_\mathbf{I}$ do sistema clássico ao sistema intuicionista indutivamente da seguinte maneira:
    \begin{align*}
        p^\neg&\coloneqq\neg\neg p\\
        \bot^\neg&\coloneqq\bot\\
        {(\varphi\wedge\psi)}^\neg&\coloneqq\neg\neg(\varphi^\neg \wedge \psi^\neg)\\
        {(\varphi\vee\psi)}^\neg&\coloneqq\neg\neg (\varphi^\neg \vee \psi^\neg)\\
        {(\varphi\to\psi)}^\neg&\coloneqq\neg\neg (\varphi^\neg \to \psi^\neg)
        \tag*{\qed} 
    \end{align*}
\end{definition}

\section{Provadores}

A primeira prova de destaque a ser realizada com grande uso de computadores foi a do teorema das quatro cores\footnote{Que afirma que \emph{qualquer mapa planar tem uma quatro-coloração}.}, feita por~\cite{Appel}, motivado pela grande quantidade de casos a serem analisados.
Conforme~\cite{Wilson} afirma, esta prova foi por uns recebida com entusiasmo e por outros, devido ao uso de computadores, com cetistismo e desapontamento.
Dentre aqueles que compartilharam destas visões opositoras, destaca-se~\cite{Tymoczko}.
Ainda segundo~\cite{Wilson}, o teorema tornou-se mais aceito com o passar do tempo e foi, posteriormente, formalizado em um provador de teoremas por~\cite{Gonthier}.

Provadores de teoremas consistem em programas de computador que verificam a validade de teoremas. Dentre estes, podemos destacar as classes dos provadores \emph{automados} e dos provadores \emph{interativos}. Os primeiros buscam provar teoremas de maneira que requeira a menor quantidade de intervenção humana, enquanto os segundos --- que ganharam destaque depois das limitações dos primeiros ficarem evidentes --- delegam-se a verificar rigorosamente provas desenvolvidas por humanos em sua linguagem~(\babireski{Citação}). Formalizaremos das provas apresentadas neste trabalho no provador de teoremas interativo \textsc{coq}.


    \chapter{Sistemas}

        Nesta parte do trabalho, uma vez apresentada a fundamentação, introduziremos as definições dos sistemas de origem e de destino das traduções apreciadas neste trabalho, nomeadamente o sistema intuicionista e um dos sistemas modais.
        Ainda, relacionaremos as modalidades dos sistemas modais com efeitos computacionais de modo a justificar investigações acerca destes a partir um ponto de vista da computação, especialmente aquele da compilação.
        Por fim, apresentaremos duas traduções que levam sentenças intuicionistas a sentenças modais.

        \section{Sistema intuicionista}
    Nesta seção, definiremos os sistema intuicionista $\mathfrak{B}=\langle\mathcal{L},\vdash_\mathfrak{B}\rangle$, cuja linguagem consiste no conjunto de origem das traduções foco deste trabalho.
    Este sistema surge da rejeição da lei do \textit{tertium non datur}, ou seja $\Gamma\vdash\alpha\vee\neg\alpha$ não vale para todos os casos.

    O sistema intuicionista consiste no sistema resultante da rejeição de algumas sentenças classicamente tidas como verdadeiras, como a sentença $\alpha\vee\neg\alpha$ e a sentença $\neg\neg\alpha\to\alpha$.
    Esse sistema foi inicialmente formalizado por~\cite{Kolmogorov}, Heyting e~\cite{Glivenko-A, Glivenko-B} com inspiração nos trabalhos de~\cite{Brouwer-A, Brouwer-B} acerca do intuicionismo.
    Nesta seção, definiremos este sistema conforme~\cite{Troelstra} e traremos um breve contexto de seu uso na computação.

    \begin{tcolorbox}[enhanced jigsaw, breakable, sharp corners, colframe=black, colback=white, boxrule=0.5pt, left=1.5mm, right=1.5mm, top=1.5mm, bottom=1.5mm]
    \begin{definition}[$\mathcal{L}$]\label{intuitionistic.language}
        A linguagem do sistema intuicionista, denotada $\mathcal{L}$, pode ser induzida a partir da assinatura $\Sigma=\sequence{\mathcal{P},\mathcal{C}}$, onde $\mathcal{C}=\set{\bot^0,\wedge^2,\vee^2,\to^2}$.
        \qed{}
    \end{definition}
    \end{tcolorbox}

    \begin{tcolorbox}[enhanced jigsaw, breakable, sharp corners, colframe=black, colback=white, boxrule=0.5pt, left=1.5mm, right=1.5mm, top=1.5mm, bottom=1.5mm]
    \begin{notation}
        Seja uma sentença $\alpha\in\mathcal{L}$, $\neg\alpha$ denota a sua negação $\alpha\to\bot$.
    \end{notation}
    \end{tcolorbox}

\vspace{.5\baselineskip}
\begin{tcolorbox}[enhanced jigsaw, breakable, sharp corners, colframe=black, colback=white, boxrule=0.5pt, left=1.5mm, right=1.5mm, top=1.5mm, bottom=1.5mm]
\begin{definition}[$\vdash_{\mathfrak{B}}$]\label{intuitionistic.deduction}
    Abaixo estão definidas as regras do sistema intuicionista $\mathfrak{B}$.
\vspace{.5\baselineskip}
\begin{center}
    \footnotesize
    \AxiomC{}
    \RightLabel{\footnotesize$\mathbf{A_1}$}
    \UnaryInfC{$\Gamma\vdash\alpha\to\beta\to\alpha$}
    \DisplayProof\label{intuitionistic.axiom.1}
    \quad
    \AxiomC{}
    \RightLabel{\footnotesize$\mathbf{A_2}$}
    \UnaryInfC{$\Gamma\vdash(\alpha\to\beta\to\gamma)\to(\alpha\to\beta)\to\alpha\to\gamma$}
    \DisplayProof\label{intuitionistic.axiom.2}
\end{center}

\begin{center}
    \footnotesize
    \AxiomC{}
    \RightLabel{\footnotesize$\mathbf{A_3}$}
    \UnaryInfC{$\Gamma\vdash\alpha\to\beta\to\alpha\wedge\beta$}
    \DisplayProof\label{intuitionistic.axiom.3}
    \quad
    \AxiomC{}
    \RightLabel{\footnotesize$\mathbf{A_4}$}
    \UnaryInfC{$\Gamma\vdash\alpha\wedge\beta\to\alpha$}
    \DisplayProof\label{intuitionistic.axiom.4}
    \quad
    \AxiomC{}
    \RightLabel{\footnotesize$\mathbf{A_5}$}
    \UnaryInfC{$\Gamma\vdash\alpha\wedge\beta\to\beta$}
    \DisplayProof\label{intuitionistic.axiom.5}
\end{center}

\begin{center}
    \footnotesize
    \AxiomC{}
    \RightLabel{\footnotesize$\mathbf{A_6}$}
    \UnaryInfC{$\Gamma\vdash\alpha\to\alpha\vee\beta$}
    \DisplayProof\label{intuitionistic.axiom.6}
    \quad
    \AxiomC{}
    \RightLabel{\footnotesize$\mathbf{A_7}$}
    \UnaryInfC{$\Gamma\vdash\beta\to\alpha\vee\beta$}
    \DisplayProof\label{intuitionistic.axiom.7}
    \quad
    \AxiomC{}
    \RightLabel{\footnotesize$\mathbf{A_8}$}
    \UnaryInfC{$\Gamma\vdash(\alpha\to\gamma)\to(\beta\to\gamma)\to\alpha\vee\beta\to\gamma$}
    \DisplayProof\label{intuitionistic.axiom.8}
\end{center}

\begin{center}
    \footnotesize
    \AxiomC{}
    \RightLabel{\footnotesize$\mathbf{A_\bot}$}
    \UnaryInfC{$\Gamma\vdash\bot\to\alpha$}
    \DisplayProof\label{intuitionistic.axiom.contradiction}
\end{center}

\begin{center}
    \footnotesize
    \AxiomC{$\alpha\in\Gamma$}
    \RightLabel{\footnotesize$\mathbf{R_1}$}
    \UnaryInfC{$\Gamma\vdash\alpha$}
    \DisplayProof\label{intuitionistic.rule.1}
    \quad
    \AxiomC{$\Gamma\vdash\alpha$}
    \AxiomC{$\Gamma\vdash\alpha\to\beta$}
    \RightLabel{\footnotesize$\mathbf{R_2}$}
    \BinaryInfC{$\Gamma\vdash\beta$}\label{intuitionistic.rule.1}
    \DisplayProof
\end{center}
\end{definition}
\end{tcolorbox}

    De modo a facilitar a comunicação no decorrer deste trabalho, chamaremos $\mathbf{R_1}$ de regra da separação ou \emph{modus ponens}.

    O sistema intuicionista calca-se numa visão construtivista que fundamenta aplicações importantes na computação. O isomorfismo de Curry-Howard estabelece uma associação entre provas e programas e entre proposições e tipos.
    Enquanto isso, a interpretação de Brouwer-Heyting-Kolmogorov --- definida abaixo segundo~\cite{Troelstra} --- exige que provas sejam construtivas, garantindo a realidade efetiva dos elementos provados. Tais propriedades são largamente usadas, por exemplo, na prova de teoremas computacionalmente e na construção de compiladores robustos.

    \begin{enumerate}[label=\textbf{(\alph*)}, left=\parindent]
        \item Não existe prova de $\bot$.
        \item Uma prova de $\alpha\wedge\beta$ consiste num par $\sequence{A,B}$, sendo $A$ uma prova de $\alpha$ e $B$ uma prova de $\beta$.
        \item Uma prova de $\alpha\vee\beta$ consiste ou num par $\sequence{0,A}$, sendo $A$ uma prova de $\alpha$, ou num par $\sequence{1,B}$, sendo $B$ uma prova de $\beta$.
        \item Uma prova de $\alpha\to\beta$ consiste numa construção $C$ que transforma uma prova $A$ de $\alpha$ numa prova $B$ de $\beta$.
    \end{enumerate}

        \section{Modalismo}
    \babireski{\cite{Blackburn} traz uma visão da evolução dos sistemas modais.}

    Os sistemas modais consistem em extensões do sistema proposicional com a adição de modalidades que representam \emph{necessidade} --- denotada como $\nec$ --- e \emph{possibilidade} --- denotada como $\pos$ --- bem como esquemas e regras que dizem respeito a elas. Deste modo, estão contidas na linguagem do sistema sentenças da forma $\nec\alpha$ e $\pos\alpha$ --- lidas \emph{necessariamente} $\alpha$ e \emph{possivelmente} $\alpha$, respeitivamente. Intuitivamente, uma necessidade deve ser verdade em todos os casos, enquanto uma possibilidade deve ser verdade em algum caso. Nesta seção, contextualizaremos esses sistemas e, em seguida, definiremo-lo formalmente na sua versão $\mathbf{S_4}$.

    Os primeiros desenvolvimentos acerca das modalidades acima foram feitos pelos gregos antigos, que anteciparam muitos dos preceitos aceitos modernamente e dentre os quais destacamos novamente~\cite{Aristotle}. O fundador do modalismo moderno foi~\cite{Lewis}, motivado pela sua insatisfação com o conceito vigente de implicação, uma vez que sua definição\footnote{Definida como $\alpha\to\beta\equiv\neg\alpha\vee\beta$.} permite que sentenças intuitivamente falsas em linguagem natural seja valoradas como verdade. Este sistema foi posteriormente melhor desenvolvido por~\cite{Langford}, onde foram apresentados os sistemas $\mathbf{S_1}$ a $\mathbf{S_5}$ --- sendo $\mathbf{S_4}$ o abordado neste trabalho.

    \begin{definition}[$\mathcal{L}_\mathbf{M}$]
        A linguagem dos sistemas modais, denotada $\mathcal{L}_\mathbf{M}$, pode ser induzida a partir da assinatura $\Sigma_\mathbf{M}=\sequence{\mathcal{P},\mathcal{C}_\mathbf{M}}$, onde $\mathcal{C}_\mathbf{M}=\set{\bot^0,\nec^1,\wedge^2,\vee^2,\to^2}$.
    \end{definition}

    \begin{notation}
        Serão usadas as seguintes abreviações:
        \begin{align*}
            \top&\coloneqq\bot\to\bot\\
            \neg\alpha&\coloneqq\alpha\to\bot\\
            \pos\alpha&\coloneqq\neg\nec\neg\alpha\\
            \alpha\strictif\beta&\coloneqq\nec(\alpha\to\beta)\\
            \alpha\leftrightarrow\beta&\coloneqq(\alpha\to\beta)\wedge(\beta\to\alpha)
        \end{align*}
    \end{notation}

    \begin{notation}
        Seja $\Gamma\in\wp(\mathcal{L}_\mathbf{M})$ um conjunto de sentenças bem-formadas.
        $\nec\Gamma$ denota o conjunto $\set{\nec\alpha\mid\alpha\in\Gamma}\in\wp(\mathcal{L}_\mathbf{M})$, ou seja, a prefixação da necessitação a todos os elementos do conjunto $\Gamma$.
    \end{notation}

    Definição abaixo segundo~\cite{Troelstra}.

    \begin{definition}\label{m-axioms}
        A axiomatização do sistema modal consiste no conjunto de esquemas de axiomas $\mathcal{A}=\set{\mathbf{A}_i\mid i\in[1,8]\vee i=\neg}\cup\set{\mathbf{B_1},\mathbf{B_2},\mathbf{B_3}}$ e no conjunto de regras $\mathcal{R}=\set{\mathbf{R_1},\mathbf{R_2}}$, definidos abaixo:
        \begin{alignat}{3}
            &\mathbf{A_1}\quad&&\alpha\to\beta\to\alpha\label{MA1}\tag*{}\\
            &\mathbf{A_2}\quad&&(\alpha\to\beta\to\gamma)\to(\alpha\to\beta)\to(\alpha\to\gamma)\label{MA2}\tag*{}\\
            &\mathbf{A_3}\quad&&\alpha\to\beta\to\alpha\wedge\beta\label{MA3}\tag*{}\\
            &\mathbf{A_4}\quad&&\alpha\wedge\beta\to\alpha\label{MA4}\tag*{}\\
            &\mathbf{A_5}\quad&&\alpha\wedge\beta\to\beta\label{MA5}\tag*{}\\
            &\mathbf{A_6}\quad&&\alpha\to\alpha\vee\beta\label{MA6}\tag*{}\\
            &\mathbf{A_7}\quad&&\beta\to\alpha\vee\beta\label{MA7}\tag*{}\\
            &\mathbf{A_8}\quad&&(\alpha\to\gamma)\to(\beta\to\gamma)\to(\alpha\vee\beta\to\gamma)\label{MA8}\tag*{}\\
            &\mathbf{A_\neg}\quad&&\neg\neg\alpha\to\alpha\label{MANEG}\tag*{}\\
            &\mathbf{B_1}\quad&&\nec(\alpha\to\beta)\to\nec\alpha\to\nec\beta\label{MB1}\tag*{}\\
            &\mathbf{B_2}\quad&&\nec\alpha\to\alpha\label{MB2}\tag*{}\\
            &\mathbf{B_3}\quad&&\nec\alpha\to\nec\nec\alpha\label{MB3}\tag*{}\\
            &\mathbf{R_E}\quad&&\text{Se }\Gamma\entails\alpha\text{, então }\Gamma\cup\Delta\entails\alpha\\
            &\mathbf{R_1}\quad&&\text{Se }\Gamma\entails\alpha\text{ e }\Gamma\entails\alpha\to\beta\text{, então }\Gamma\entails\beta\label{detachment}\tag*{}\\
            &\mathbf{R_2}\quad&&\text{Se }\entails\alpha\text{, então }\Gamma\entails\nec\alpha\text{.}\tag*{\qed}\label{necessitation} 
        \end{alignat}   
    \end{definition}

    \babireski{Falar aqui sobre como a axiomatização consiste nos esquemas clássicos mais os esquemas modais}. 

    Assim como feito para o sistema intuicionista, nomearemos os esquemas e regras acima de modo a facilitar a comunicação.
    Aos axiomas e regras que correspondem aos axiomas e regras intuicionistas receberão os mesmos nomes. Ademais, chamaremos $\mathbf{B_1}$ de axiomas da normalidade, $\mathbf{B_2}$ de axiomas da reflexividade e $\mathbf{B_3}$ de axiomas da transitividade.\footnote{Em analogia às condições relacionais impostas nos enquadramentos.} Nomearemos $\mathbf{A_\neg}$ como chamaremos de axiomas da eliminação da negação e a $\mathbf{R_2}$ como regra da necessitação.

    A definição das regras de dedução em relação a conjuntos de sentenças baseia-se tanto em~\cite{Troelstra} como em~\cite{Hakli}. Ao decorrer do texto, ocasionalmente chamaremos $\mathbf{R_1}$ de \emph{regra da separação} e $\mathbf{R_2}$ de \emph{regra da necessitação}. A definição da regra da necessitação deve ser cuidadosa de modo a permitir a prova do metateorema da dedução, feita futuramente neste trabalho. Neste sentido, restringimos a aplicação desta regra apenas a teoremas.\footnote{Para uma discussão mais aprofudada, ver~\cite{Hakli}.}

        \section{Efeitos}

Os \emph{efeitos computacionais}, ou simplesmente \emph{efeitos}, são todas as ações e interações efetuadas pelos computadores que vão além da simples computação.
Assim, uma função que computa a soma entre dois valores não apresenta nenhum efeito, enquanto uma função que computa a soma entre dois valores e imprime o resultado na tela produz um efeito ao fazer a impressão.
Alguns exemplos de efeitos são \emph{parcialidade}, \emph{continuações} e \emph{não-determinismo}.
Programas com efeitos podem mudar seus estados internos, bem como receber entradas externas.
Tais capacidades tornam a avaliação do comportamento do programa menos clara.

\vspace{.5\baselineskip}
Motivado pela busca de uma metalinguagem para modelar efeitos em linguagens de programação,~\cite{Moggi.1991} introduziu a linguagem $\lambda_c$.
Nesta linguagem, os termos-$\lambda$ são distinguidos entre valores de tipo $\alpha$ e computações $\mu\ \alpha$ de tipo $\alpha$ de modo que estas se comportam \emph{monadicamente}.
O construtor de tipos $\mu$ representa alguma \emph{noção de computação}, ou melhor dizendo, algum efeito.
Inspirado por este trabalho, \cite{Wadler.1993} sugere e defende o uso de \emph{mônadas} como estruturas dentro de linguagens funcionais como uma maneira de simular efeitos.

\vspace{.5\baselineskip}
Para os fins deste trabalho, consideraremos mônadas como um construtor de tipos $\mu$ munido das operações {\texttt{\textbf{lift}}}, {\texttt{\textbf{unit}}} e {\texttt{\textbf{bind}}}.\footnote{Estas operações devem respeitar certas condições, que não serão abordadas.}
A operação {\texttt{\textbf{lift}}} eleva funções a funções \emph{efeituosas}.\footnote{\emph{Effectful}.}
Do mesmo modo, a operação {\texttt{\textbf{unit}}} eleva um valor a um valor com efeitos e a operação {\texttt{\textbf{bind}}} encadeia efeitos.
Alternativamente, podemos usar a operação {\texttt{\textbf{join}}}, que une efeitos, em lugar da operação {\texttt{\textbf{bind}}}.
A seguir temos as assinaturas de tipo das operações citadas.

\vspace{.5\baselineskip}
\begin{tcolorbox}[enhanced jigsaw, breakable, sharp corners, colframe=black, colback=white, boxrule=0.5pt, left=1.5mm, right=1.5mm, top=1.5mm, bottom=1.5mm]
    \setmonofont{Fira Code}[Contextuals = Alternate]
    \texttt{\footnotesize\textbf{lift} : ∀α.∀β.∀μ.(α ⟶ β) ⟶ μα ⟶ μβ}\\
    \texttt{\footnotesize\textbf{unit} : ∀α.∀μ.α ⟶ μα}\\
    \texttt{\footnotesize\textbf{bind} : ∀α.∀β.∀μ.(α ⟶ μβ) ⟶ μα ⟶ μβ}\\
        \texttt{\footnotesize\textbf{join} : ∀α.∀μ.μμα ⟶ μα}
\end{tcolorbox}

\vspace{.5\baselineskip}
Conforme notado por~\cite{Benton+others.1998}, a linguagem $\lambda_c$ corresponde ao sistema \emph{laxo} --- doravante chamado de $\mathfrak{L}$ --- por meio da interpretação prova-programa.
Este sistema, que aumenta o sistema intuicionista $\mathfrak{I}$ com uma modalidade de \emph{laxidade}, foi inicialmente considerado por~\cite{Curry.1950, Curry.1952} e posteriormente redescoberto por~\cite{Fairtlough+Mendler.1997}.
Esta modalidade --- denotada $\bigcirc$ --- foi interpretada por estes como \emph{verdade com restrições}, motivo que justifica seu nome.
As regras que governam a modalidade laxa, conforme apresentadas abaixo, geram sentenças correspondentes aos tipos das funções {\texttt{\textbf{lift}}}, {\texttt{\textbf{unit}}} e {\texttt{\textbf{join}}}.
Do mesmo modo, podemos derivar a sentença que corresponde ao tipo da função {\texttt{\textbf{bind}}}.

\vspace{.5\baselineskip}
\begin{tcolorbox}[enhanced jigsaw, breakable, sharp corners, colframe=black, colback=white, boxrule=0.5pt, left=1.5mm, right=1.5mm, top=1.5mm, bottom=1.5mm]
\begin{definition}[$\mathcal{L}_\bigcirc$]\label{lax.language}
    A linguagem do sistema laxo, denotada $\mathcal{L}_\bigcirc$, pode ser induzida a partir da assinatura $\Sigma=\sequence{\mathcal{P},\mathcal{C}_\bigcirc}$, onde $\mathcal{C}_\bigcirc=\set{\bot^0,\bigcirc^1,\wedge^2,\vee^2,\to^2}$.
\end{definition}
\end{tcolorbox}

\begin{tcolorbox}[enhanced jigsaw, breakable, sharp corners, colframe=black, colback=white, boxrule=0.5pt, left=1.5mm, right=1.5mm, top=1.5mm, bottom=1.5mm]
\begin{definition}[$\entails_{\mathfrak{L}}$]
    Abaixo estão definidas as regras do sistema $\mathfrak{L}$.
\vspace{.5\baselineskip}
\begin{center}
    \footnotesize
    \AxiomC{}
    \RightLabel{\footnotesize$\mathbf{A_1}$}
    \UnaryInfC{$\Gamma\entails\alpha\to\beta\to\alpha$}
    \DisplayProof{}
    \quad
    \AxiomC{}
    \RightLabel{\footnotesize$\mathbf{A_2}$}
    \UnaryInfC{$\Gamma\entails(\alpha\to\beta\to\gamma)\to(\alpha\to\beta)\to\alpha\to\gamma$}
    \DisplayProof{}
\end{center}

\begin{center}
    \footnotesize
    \AxiomC{}
    \RightLabel{\footnotesize$\mathbf{A_3}$}
    \UnaryInfC{$\Gamma\entails\alpha\to\beta\to\alpha\wedge\beta$}
    \DisplayProof{}
    \quad
    \AxiomC{}
    \RightLabel{\footnotesize$\mathbf{A_4}$}
    \UnaryInfC{$\Gamma\entails\alpha\wedge\beta\to\alpha$}
    \DisplayProof{}
    \quad
    \AxiomC{}
    \RightLabel{\footnotesize$\mathbf{A_5}$}
    \UnaryInfC{$\Gamma\entails\alpha\wedge\beta\to\beta$}
    \DisplayProof{}
\end{center}

\begin{center}
    \footnotesize
    \AxiomC{}
    \RightLabel{\footnotesize$\mathbf{A_6}$}
    \UnaryInfC{$\Gamma\entails\alpha\to\alpha\vee\beta$}
    \DisplayProof{}
    \quad
    \AxiomC{}
    \RightLabel{\footnotesize$\mathbf{A_7}$}
    \UnaryInfC{$\Gamma\entails\beta\to\alpha\vee\beta$}
    \DisplayProof{}
    \quad
    \AxiomC{}
    \RightLabel{\footnotesize$\mathbf{A_8}$}
    \UnaryInfC{$\Gamma\entails(\alpha\to\gamma)\to(\beta\to\gamma)\to\alpha\vee\beta\to\gamma$}
    \DisplayProof{}
\end{center}

\begin{center}
    \footnotesize
    \AxiomC{}
    \RightLabel{\footnotesize$\mathbf{A_\bot}$}
    \UnaryInfC{$\Gamma\entails\neg\neg\alpha\to\alpha$}
    \DisplayProof{}
\end{center}

\begin{center}
    \footnotesize
    \AxiomC{}
    \RightLabel{\footnotesize$\mathbf{C_1}$}
    \UnaryInfC{$\Gamma\entails(\alpha\to\beta)\to\bigcirc\alpha\to\bigcirc\beta$}
    \DisplayProof{}
    \quad
    \AxiomC{}
    \RightLabel{\footnotesize$\mathbf{C_2}$}
    \UnaryInfC{$\Gamma\entails\alpha\to\bigcirc\alpha$}
    \DisplayProof{}
    \quad
    \AxiomC{}
    \RightLabel{\footnotesize$\mathbf{C_3}$}
    \UnaryInfC{$\Gamma\entails\bigcirc\bigcirc\alpha\to\bigcirc\alpha$}
    \DisplayProof{}
\end{center}

\begin{center}
    \footnotesize
    \AxiomC{\phantom{$\beta$}}
    \RightLabel{\footnotesize$\mathbf{R_1}$}
    \UnaryInfC{$\Gamma\cup\set{\alpha}\entails\alpha$}
    \DisplayProof{}
    \quad
    \AxiomC{$\Gamma\entails\alpha$}
    \AxiomC{$\Gamma\entails\alpha\to\beta$}
    \RightLabel{\footnotesize$\mathbf{R_2}$}
    \BinaryInfC{$\Gamma\entails\beta$}
    \DisplayProof{}
\end{center}
\end{definition}
\end{tcolorbox}

\vspace{.5\baselineskip}
Pode-se imergir sentenças do sistema $\mathfrak{L}$ em sentenças do sistema $\mathfrak{M}$ de modo que $\lax\alpha\coloneqq\pos\nec\alpha$, conforme definem~\cite{Pfenning+Davies.2001}.
Como estes apontam, isso permite com que usemos o sistema $\mathfrak{M}$ para razoar sobre o sistema $\mathfrak{L}$, uma vez que as regras deste podem ser derivadas naquele.
Outra imersão de interesse trata-se da tradução de~\cite{Fairtlough+Mendler.1997}, com sistema de origem o sistema $\mathfrak{L}$ e com sistema de destino um sistema bimodal $\langle\mathfrak{M},\mathfrak{M}\rangle$, que serve para reforçar ainda mais o caso da relação entre ambos os sistemas.

\vspace{.5\baselineskip}
A imersão apresentada acima sugere que possamos interpretar algumas das sentenças do sistema~$\mathfrak{M}$ como representando computações com efeitos.
De fato,~\cite{Pfenning+Davies.2001} alegam que $\nec\alpha$ intuitivamente representa valores de tipo $\alpha$ que sobrevivem a efeitos, enquanto $\pos\alpha$ representa computações que retornam valores de tipo $\alpha$.
Corrobora com esta visão o trabalho de~\cite{Kobayashi.1997}.
Notamos que, segundo~\cite{Zach+others.2019}, podem ser derivadas no sistema $\mathfrak{M}$ sentenças da forma $\alpha\to\pos\alpha$ e da forma $\pos\pos\alpha\to\pos\alpha$.
Do mesmo modo, podem ser derivadas sentenças da forma $\nec(\alpha\to\beta)\to\pos\alpha\to\pos\beta$.
Estas assemelham-se ao{\texttt{\textbf{lift}}}, {\texttt{\textbf{unit}}} e {\texttt{\textbf{join}}}, o que reforça o ponto apresentado aqui.

\vspace{.5\baselineskip}
Em computação, \emph{continuações} são um conceito descoberto e redescoberto diversas vezes~\citep{Reynolds.1993} que representa o restante de uma computação a partir de um determinado ponto.
O \emph{estilo de passagem por continuações} consiste num estilo de programação onde cada função, em vez de retornar um valor, recebe como argumento uma função com a sua continuação e a invoca ao fim de sua execução~\citep{Thielecke.1999}.
O estilo de passagem por continuações pode ser usado como representação dentro de compiladores~\citep{Appel.1991}, uma vez que essa inversão de controle --- ou seja, a passagem de continuações como argumento --- revela o fluxo do programa.
Uma vez que mônadas podem ser imersas em continuações e continuações podem ser imersas em mônadas \citep{Filinski.1994}, o sistema $\mathfrak{M}$ torna-se interessante do ponto de vista de compilação, motivo que justifica investigações acerca do uso de uma linguagem baseada em $\mathfrak{M}$ como representação durante a compilação.

        \section{Traduções}
    A primeira tradução do sistema intuicionista ao sistema modal foi proposta por~\cite{Goedel} motivado pela possibilidade de leitura da necessidade como uma modalidade de construtividade. Ou seja, por meio dessa tradução, a sentença $\nec \varphi$ poderia ser lida como \textit{$\varphi$ pode ser provada construtivamente} \citep{Troelstra}. Gödel alegou --- sem apresentar provas --- a correção fraca dessa tradução e conjeiturou sua completude fraca, posteriormente provadas por~\cite{McKinsey}. As as traduções apresentadas abaixo foram retiradas de~\cite{Troelstra}.

    \begin{definition}[$\bullet^\circ$]\label{translation.circle}
        Define-se a tradução $\bullet^\circ:\mathcal{L}\to\mathcal{L}_{\nec}$ do sistema intuicionista ao sistema modal $\mathbf{S4}$ indutivamente da seguinte maneira:
        \begin{align*}
            a^\circ&\mapsto a\\
            \bot^\circ&\mapsto\bot\\
            {(\varphi \wedge \psi)}^\circ & \mapsto \varphi^\circ \wedge \psi^\circ\displaybreak[0]\\
            {(\varphi \vee \psi)}^\circ   & \mapsto \nec \varphi^\circ \vee \nec \psi^\circ\displaybreak[0]\\
            {(\varphi \to \psi)}^\circ    & \mapsto \nec \varphi^\circ \to \psi^\circ
            \tag*{\qed} 
        \end{align*}
    \end{definition}
    
    \begin{definition}[$\bullet^\medsquare$]\label{translation.square}
        Define-se a tradução $\bullet^\medsquare:\mathcal{L}\to\mathcal{L}_{\nec}$ do sistema intuicionista ao sistema modal $\mathbf{S4}$ indutivamente da seguinte maneira:
        \begin{align*}
            a^\medsquare& \mapsto \nec a\displaybreak[0]\\
            \bot^\medsquare& \mapsto \bot\displaybreak[0]\\
            {(\varphi \wedge \psi)}^\medsquare & \mapsto \varphi^\medsquare \wedge \psi^\medsquare\displaybreak[0]\\
            {(\varphi \vee \psi)}^\medsquare & \mapsto \varphi^\medsquare \vee \psi^\medsquare\displaybreak[0]\\
            {(\varphi \to \psi)}^\medsquare & \mapsto \nec (\varphi^\medsquare \to \psi^\medsquare)
            \tag*{\qed} 
        \end{align*}
    \end{definition}
    
    Ambas as traduções providas são equivalentes, conforme demonstraremos futuramente.
    Ademais, faz-se interessante pontuar que as traduções $\bullet^\circ$ e $\bullet^\medsquare$ correspondem, respectivamente, às traduções $\bullet^\circ$ e $\bullet^*$ do sistema intuicionista ao sistema linear providas por~\cite{Girard}. A primeira tradução de Girard corresponde a uma ordem de avaliação por nome (\textit{call-by-name}) e a segunda a uma ordem de avaliação por valor (\textit{call-by-value}), conforme notam~\cite{Maraist}.

\section{Avaliação}
    Dada uma linguagem de programação, podem haver diferentes maneiras de aplicar reduções em expressões desta linguagem --- ou seja, diferentes maneiras de performar uma computação.
    A cada uma dessas maneiras, damos o nome \emph{ordem de avaliação}.
    Como notado na seção anterior, as imersões modais do sistema intuicionista assemelham-se muito às imersões lineares do sistema intuicionista.
    Como notam~\cite{Maraist}, correspondem a uma tradução a uma ordem de avaliação por nome e a uma ordem de avaliação por valor.
    Neste sentido, pode-se dizer que, do mesmo modo, as imersões modais correspondem a estas duas ordens de avaliação, que são duais entre si (Wadler).
    Assim, esta seção busca fundamentar estas duas ordens de avaliação.

    Para tanto, consideremos expressões-lambda $e$, que consistem em valores $v$ ou em aplicações $e\ e$ de uma expressão a outra expressão.
    Os valores $v$, por sua vez, dividem-se em letras $x$ ou em funções $\lambda x.e$ que recebem $x$ e retornam uma expressão $e$.
    As letras pertencem a um conjunto de letras $\mathcal{V}$.
    Consideraremos duas beta-reduções diferentes, uma na ordem de avaliação por nome e outra por valor.


    \chapter{Propriedades}
        Uma vez definidos os conceitos precisos para o desenvolvimento deste trabalho, aqui apresentaremos diversas provas que lhes dizem respeito.
        Notadamente, serão provados metateoremas acerca do sistema $\mathfrak{L}$, bem como serão demonstradas que as traduções são corretas e equivalem.
        Os metateoremas provados visam simplificar provas futuras.
        Nas provas abaixo, adotaremos a convenção de que deduções denotadas $\Gamma\vdash\alpha$ valem para ambos os sistemas em apreciação neste trabalho.
        Ou seja, são provas que não dependem do operador de necessidade e tampouco das regras associadas a ele ou ao operador de contradição.
        Como definido, $\Gamma\vdash_\mathfrak{B}\alpha$ representa uma dedução no sistema $\mathfrak{B}$, enquanto $\Gamma\vdash_\mathfrak{L}\alpha$ uma dedução no sistema $\mathfrak{L}$.

        \section{Derivações}
    Nesta seção apresentaremos alguns lemas e teoremas para os sistemas modais que permitirão simplificar muito as provas apresentadas no decorrer deste trabalho.
    Primeiramente, provaremos que, dada uma sentença qualquer, esta sempre implica a si mesma. A este lema daremos o nome de identidade\footnote{Em analogia ao combinador $\mathbf{I}$.} e, em seguida, usaremo-no para a prova da regra da dedução.

    \begin{lemma}\label{identity}
        $\entails\alpha\to\alpha$.
        \begin{proof}
            Pode ser provado pela seguinte sucessão de dedução:
            \footnotesize
            \begin{fitch}
                \fb\vdash\alpha\to\alpha\to\alpha&\hyperref[MA1]{$\mathbf{A_1}$}\\
                \fa\vdash\alpha\to(\alpha\to\alpha)\to\alpha&\hyperref[MA1]{$\mathbf{A_1}$}\\
                \fa\vdash(\alpha\to(\alpha\to\alpha)\to\alpha)\to(\alpha\to\alpha\to\alpha)\to\alpha\to\alpha&\hyperref[MA2]{$\mathbf{A_2}$}\\
                \fa\vdash(\alpha\to\alpha\to\alpha)\to\alpha\to\alpha&$\hyperref[detachment]{\mathbf{R_2}}\;\set{2,3}$\\
                \fa\vdash\alpha\to\alpha&$\hyperref[detachment]{\mathbf{R_2}}\;\set{1,4}$.
            \end{fitch}
            \normalsize
            Estando assim demonstrada a proposição.
        \end{proof}
    \end{lemma}

    Tendo-se provado o lema da identidade, agora provaremos a regra da dedução para os sistemas modais com base na prova apresentada por~\cite{Hakli}.
    Pequenas alterações foram feitas de modo a garantir a adequação da prova com a axiomatização provida na Definição~\ref{m-axioms}.

    \begin{theorem}[Dedução]\label{deduction}
        $\text{Se }\Gamma\cup\set{\alpha}\vdash\beta\text{, então }\Gamma\vdash\alpha\to\beta$.

        \begin{proof}
            Prova por indução forte sobre o tamanho da sucessão de dedução.\footnote{Nota-se que, para a indução forte, não se faz preciso provar nenhuma base \citep{Velleman}.}
            Assim, suponhamos que o teorema da dedução valha para qualquer sucessão de dedução de tamanho $n<k$.
            Demonstraremos analisando-se os casos e valendo-se da suposição acima --- doravante chamada $\mathbf{H}$ --- o passo de indução, ou seja, que o teorema da dedução vale para sucessões de dedução de tamanho $n=k$.

            \begin{case}
                \textsc{Caso 1.}
                Se a linha derradeira da sucessão de dedução que prova $\Gamma\cup\set{\alpha}\vdash\beta$ tenha sido a evocação de alguma premissa, sabe-se que $\beta\in\Gamma\cup\set{\alpha}$.
                Deste modo, existem dois casos a serem analisados.
            \end{case}

            \begin{subcase}
                \textsc{Caso 1.1.}
                Se a linha derradeira da sucessão de dedução que prova $\Gamma\cup\set{\alpha}\vdash\beta$ tenha sido a evocação de alguma premissa do conjunto $\Gamma$, sabe-se que $\beta\in\Gamma$. Deste modo, podemos demonstrar que $\Gamma\vdash\alpha\to\beta$ pela seguinte sucessão de dedução:

                \footnotesize
                \begin{fitch}
                    \fb\Gamma\vdash\beta&$\hyperref[premisse]{\mathbf{R_1}}$\\
                    \fa\Gamma\vdash\beta\to\alpha\to\beta&$\hyperref[MA1]{\mathbf{A_1}}$\\
                    \fa\Gamma\vdash\alpha\to\beta&$\hyperref[detachment]{\mathbf{R_2}}\;\set{1,2}$.
                \end{fitch}
                \normalsize
            \end{subcase}

            \begin{subcase}
                \textsc{Caso 1.2.}
                Se a linha derradeira da sucessão de dedução que prova $\Gamma\cup\set{\alpha}\vdash\beta$ tenha sido a evocação da premissa $\alpha$, sabe-se que $\beta=\alpha$.
                Deste modo, basta demonstrar que $\Gamma\vdash\alpha\to\alpha$, que consiste num enfraquecimento do lema \refer{identity}{L}.
            \end{subcase}

            \begin{case}
                \textsc{Caso 2.}
                Se a linha derradeira da sucessão de dedução que prova $\Gamma\cup\set{\alpha}\vdash\beta$ tenha sido a evocação de algum axioma, sabe-se que existe algum esquema $\mathbf{A_\beta}\in\mathcal{A}$ que instancia $\beta$.
                Deste modo, podemos demonstrar que $\Gamma\vdash\alpha\to\beta$ pela seguinte sucessão de dedução:

                \footnotesize
                \begin{fitch}
                    \fb\Gamma\vdash\beta&$\mathbf{A_\beta}$\\
                    \fa\Gamma\vdash\beta\to\alpha\to\beta&$\hyperref[MA1]{\mathbf{A_1}}$\\
                    \fa\Gamma\vdash\alpha\to\beta&$\hyperref[detachment]{\mathbf{R_2}}\;\set{1,2}$.
                \end{fitch}
                \normalsize
            \end{case}

            \begin{case}
                \textsc{Caso 3.}
                Se a linha derradeira da sucessão de dedução que prova $\Gamma\cup\set{\alpha}\vdash\beta$ tenha sido gerada pela aplicação da regra da necessitação a uma linha anterior $\mathbf{H_1}$, sabe-se que $\beta=\nec\varphi$ e que $\mathbf{H_1}={\entails\varphi}$.
                Deste modo, podemos demonstrar que $\Gamma\vdash\alpha\to\nec\varphi$ pela seguinte sucessão de dedução:

                \footnotesize
                \begin{fitch}
                    \fb\vdash\varphi&$\mathbf{H_1}$\\
                    \fa\Gamma\vdash\nec\varphi&$\hyperref[necessitation]{\mathbf{R_3}}\;\set{1}$\\
                    \fa\Gamma\vdash\nec\varphi\to\alpha\to\nec\varphi&$\hyperref[MA1]{\mathbf{A_1}}$\\
                    \fa\Gamma\vdash\alpha\to\nec\varphi&$\hyperref[detachment]{\mathbf{R_2}}\;\set{2,3}$.
                \end{fitch}
                \normalsize
            \end{case}

            \begin{case}
                \textsc{Caso 4.} Seja a sentença $\varphi_n=\beta$ gerada pela aplicação da regra do \emph{modus ponens} a duas sentenças $\varphi_i$ e $\varphi_j$ com $i<j<n$. Assumiremos, sem perda de generalidade, que $\varphi_j=\varphi_i\to\varphi_n$.
                Assim, a partir de $\mathbf{H}$ temos que $\mathbf{H_1}=\Gamma\entails\alpha\to\varphi_i$ e que $\mathbf{H_2}=\Gamma\entails\alpha\to\varphi_i\to\varphi_n$.
                Deste modo, podemos demonstrar que $\Gamma\vdash\alpha\to\nec\varphi$ pela seguinte sucessão de dedução:

                \footnotesize
                \begin{fitch}
                    \fb\Gamma\entails\alpha\to\varphi_j&$\mathbf{H_1}$\\
                    \fa\Gamma\entails\alpha\to\varphi_j\to\beta&$\mathbf{H_2}$\\
                    \fa\Gamma\entails(\alpha\to\varphi_j\to\beta)\to(\alpha\to\varphi_j)\to(\alpha\to\beta)&$\hyperref[MA2]{\mathbf{A_2}}$\\
                    \fa\Gamma\entails(\alpha\to\varphi_j)\to(\alpha\to\beta)&$\hyperref[detachment]{\mathbf{R_2}}\;\set{2,3}$\\
                    \fa\Gamma\entails\alpha\to\beta&$\hyperref[detachment]{\mathbf{R_2}}\;\set{1,4}$.
                \end{fitch}
                \normalsize
            \end{case}
            Uma vez provada a propriedade para todos os casos do passo de indução, provamos que o teorema da dedução vale para o sistema $\mathbf{S4}$.
        \end{proof}
    \end{theorem}

    \begin{theorem}[Enfraquecimento]\label{weakening}
        Se $\Gamma\subseteq\Delta$ e $\Gamma\vdash\alpha$, então $\Delta\vdash\alpha$.
        \begin{proof}
            Prova por indução forte sobre o tamanho da sucessão de dedução.
            Assim, suponhamos que o teorema do enfraquecimento valha para qualquer sucessão de dedução de tamanho $n<k$.
            Demonstraremos analisando-se os casos e valendo-se da suposição acima --- doravante chamada $\mathbf{H}$ --- o passo de indução, ou seja, que o teorema do enfraquecimento vale para sucessões de dedução de tamanho $n=k$.

            \begin{case}
                \textsc{Caso 1.}
                Se a linha derradeira da sucessão de dedução que prova $\Gamma\vdash\alpha$ tenha sido a invocação de alguma premissa $\alpha$, sabe-se que $\alpha\in\Gamma$.
                Como $\Gamma\subseteq\Delta$, sabe-se que $\alpha\in\Delta$.
                Deste modo, pode-se provar $\Delta\vdash\alpha$ pela invocação da mesma premissa $\alpha$.
            \end{case}

            \begin{case}
                \textsc{Caso 2.}
                Se a linha derradeira da sucessão de dedução que prova $\Gamma\vdash\alpha$ tenha sido a invocação de algum axioma, sabe-se que existe algum esquema $\mathbf{A_\alpha}\in\mathcal{A}$ que instancia $\alpha$.
                Deste modo, podemos demonstrar que $\Delta\vdash\alpha$ pela invocação do mesmo axioma $\mathbf{A_\alpha}$.
            \end{case}

            \begin{case}
                \textsc{Caso 3.}
                Se a linha derradeira da sucessão de dedução que prova $\Gamma\vdash\alpha$ tenha sido gerada pela aplicação da regra da necessitação a uma linha anterior, sabe-se que $\alpha=\nec\varphi$ e que ${\entails\varphi}$. Como pode-se provar $\varphi$ sem o uso de premissas, podemos aplicar a regra da necessitação a ${\entails\varphi}$ de modo a provar ${\Delta\entails\nec\varphi}$.
            \end{case}

            \begin{case}
                \textsc{Caso 4.} Seja a sentença $\varphi_n=\alpha$ gerada pela aplicação da regra do \emph{modus ponens} a duas sentenças $\varphi_i$ e $\varphi_j$ com $i<j<n$. Assumiremos, sem perda de generalidade, que $\varphi_j=\varphi_i\to\varphi_n$.
                Assim, a partir de $\mathbf{H}$ temos que $\mathbf{H_1}=\Delta\entails\varphi_i$ e que $\mathbf{H_2}=\Delta\entails\varphi_i\to\varphi_n$.
                Deste modo, podemos demonstrar que $\Delta\vdash\alpha$ pela aplicação da regra do \emph{modus ponens} a $\mathbf{H_1}$ e $\mathbf{H_2}$.
            \end{case}
            Uma vez provada a propriedade para todos os casos do passo de indução, provamos que o teorema do enfraquecimento vale para o sistema $\mathbf{S4}$.
        \end{proof}
    \end{theorem}

    Tendo-se provado o teorema da dedução, provaremos o teorema da generalização da regra da necessitação, conforme sugerido por~\cite{Troelstra}.
    Como apresentado abaixo, este teorema afirma que, caso possamos deduzir alguma sentença $\alpha$ a partir de um conjunto necessariamente verdadeiro de premissas, podemos concluir a necessidade desta sentença $\alpha$.

    \begin{theorem}[Generalização da necessitação]\label{gen-nec}
        Se $\nec\Gamma\entails\alpha$, então $\nec\Gamma\entails\nec\alpha$.
        \begin{proof}
            Prova por indução fraca sobre o tamanho $n$ do conjunto $\Gamma$ \citep{Troelstra}. A prova consiste em dois casos: um para a base da indução e outro para o passo da indução. No que segue, denotaremos $\mathbf{H_1}=\nec\Gamma\entails\alpha$.
            \begin{case}
                \textsc{Caso 1.} Para a base, consideraremos que $\Gamma=\varnothing$.
                Assim, sabemos que o conjunto possui tamanho nulo e que $\entails\alpha$. Portanto, sabe-se que existe uma sucessão de dedução $\sequence{\varphi_i\mid 0\leq i\leq n}$ com $\varphi_n=\alpha$
                Deste modo, pode-se demonstrar que $\entails\nec\alpha$ trivialmente pela aplicação da regra da necessitação \hyperref[necessitation]{$\mathbf{R_3}$} sobre a sentença $\varphi_n$.
            \end{case}
            \begin{case}
                \textsc{Caso 2.} 
                Para o passo, suponhamos que a generalização da regra da necessitação valha para qualquer conjunto $\Gamma$ de tamanho $n=k$.
                Demonstraremos, valendo-se da suposição acima --- doravante chamada $\mathbf{H_2}$ --- e pela sucessão de dedução apresentada abaixo, que a generalização da regra da necessitação vale para conjuntos $\Gamma$ de tamanho $n=k+1$.
                \footnotesize
                \begin{fitch}
                    \fb\nec\Gamma\cup\set{\nec\alpha}\entails\beta&$\mathbf{H_1}$\\
                    \fa\nec\Gamma\entails\nec\alpha\to\beta&$\hyperref[deduction]{\mathbf{T_1}}\;\set{1}$\\
                    \fa\nec\Gamma\entails\nec(\nec\alpha\to\beta)&$\mathbf{H_2}\;\set{2}$\\
                    \fa\nec\Gamma\entails\nec(\nec\alpha\to\beta)\to\nec\nec\alpha\to\nec\beta&$\hyperref[MB1]{\mathbf{B_1}}$\\
                    \fa\nec\Gamma\entails\nec\nec\alpha\to\nec\beta&$\hyperref[detachment]{\mathbf{R_2}}\;\set{3,4}$\\
                    \fa\nec\Gamma\entails\nec\alpha\to\nec\nec\alpha&$\hyperref[MB3]{\mathbf{B_3}}$\\
                    \fa\nec\Gamma\entails(\nec\alpha\to\nec\nec\alpha)\to(\nec\nec\alpha\to\nec\beta)\to\nec\alpha\to\nec\beta&$\hyperref[MA2]{\mathbf{A_2}}$\\
                    \fa\nec\Gamma\entails(\nec\nec\alpha\to\nec\beta)\to\nec\alpha\to\nec\beta&$\hyperref[detachment]{\mathbf{R_2}}\;\set{6,7}$\\
                    \fa\nec\Gamma\entails\nec\alpha\to\nec\beta&$\hyperref[detachment]{\mathbf{R_1}}\;\set{5,8}$\\
                    \fa\nec\Gamma\cup\set{\nec\alpha}\entails\nec\alpha&$\hyperref[premisse]{\mathbf{R_1}}$\\
                    \fa\nec\Gamma\cup\set{\nec\alpha}\entails\nec\alpha\to\nec\beta&$\hyperref[weakening]{\mathbf{T_2}}\;\set{9}$\\
                    \fa\nec\Gamma\cup\set{\nec\alpha}\entails\nec\beta&$\hyperref[detachment]{\mathbf{R_2}}\;\set{10,11}$.
                \end{fitch}
            \end{case}
            \vspace{.5\baselineskip}
            Tendo-se provado a base e o passo de indução, podemos concluir que generalização da regra da necessitação vale, ou seja, que se $\nec\Gamma\entails\alpha$, então $\nec\Gamma\entails\nec\alpha$.
        \end{proof}
    \end{theorem}

    Uma vez provada a generalização da regra da necessitação, a prova da regra da dedução estrita --- conforme descrito por~\cite{Barcan, Marcus} --- torna-se trivial, como pode ser visto abaixo. Esta regra afirma que, dada uma dedução de $\beta$ partindo de um conjunto de premissas necessariamente verdadeiras e uma premissa $\alpha$, podemos deduzir $\nec(\alpha\to\beta)$ a partir desse conjunto de premissas necessariamente verdadeiras. Isso nos permite simplificar as provas de correção das traduções, uma vez que uma das traduções apresentadas mapeia implicações materiais do sistema intuicionista em implicações estritas.

    \begin{theorem}[Dedução estrita]\label{strictdeduction}
        $\text{Se }\nec\Gamma\cup\set{\alpha}\entails\beta\text{, então }\nec\Gamma\entails\nec(\alpha\to\beta)$.
        \begin{proof}
            Pode ser provado pela seguinte sucessão de dedução:
            \footnotesize
            \begin{fitch}
                \fb\nec\Gamma\cup\set{\alpha}\entails\beta&$\mathbf{H_1}$\\
                \fa\nec\Gamma\entails\alpha\to\beta&\refer{deduction}{T}$\;\set{1}$\\
                \fa\nec\Gamma\entails\nec(\alpha\to\beta)&\refer{gen-nec}{T}$\;\set{2}$
            \end{fitch}
            \normalsize
            Estando assim demonstrada a proposição.
        \end{proof}
    \end{theorem}

    Agora, provaremos a aplicação da regra do \emph{modus ponens} a uma implicação estrita. Essa regra afirma que, dada uma prova de $\alpha$ e uma prova de $\nec(\alpha\to\beta)$ a partir de um conjunto de premissas, sabe-se que deve haver alguma prova de $\beta$ a partir desse mesmo conjunto de premissas.

    \begin{theorem}[Separação estrita]\label{strictsep}
        Se $\Gamma\entails\alpha$ e $\Gamma\entails\nec(\alpha\to\beta)$, então $\Gamma\entails\beta$.
        \begin{proof}
            Pode ser provado pela seguinte sucessão de dedução:
            \footnotesize
            \begin{fitch}
                \fb\Gamma\entails\alpha&$\mathbf{H_1}$\\
                \fa\Gamma\entails\nec(\alpha\to\beta)&$\mathbf{H_2}$\\
                \fa\Gamma\entails\nec(\alpha\to\beta)\to\alpha\to\beta&$\hyperref[MB2]{\mathbf{B_2}}$\\
                \fa\Gamma\entails\alpha\to\beta&$\hyperref[detachment]{\mathbf{R_2}}\;\set{2,3}$\\
                \fa\Gamma\entails\beta&$\hyperref[detachment]{\mathbf{R_2}}\;\set{1,4}$
            \end{fitch}
            \normalsize
            Estando assim demonstrada a proposição.
        \end{proof}
    \end{theorem}

    Abaixo, demonstraremos que, caso haja uma prova de alguma sentença $\gamma$ a partir de um par de premissas, sabe-se que deve haver alguma prova desta mesma sentença $\gamma$ a partir da conjunção deste par de premissas.

    \begin{theorem}\label{conjunctiondeduction}
        Se $\set{\alpha,\beta}\entails\gamma$, então $\set{\alpha\wedge\beta}\entails\gamma$.
        \begin{proof}
            Seja $\mathbf{H_1}=\set{\alpha,\beta}\entails\gamma$. A proposição pode ser provada pela seguinte sucessão de dedução:
            \footnotesize
            \begin{fitch}
                \fb\set{\alpha,\beta}\entails\gamma&$\mathbf{H_1}$\\
                \fa\set{\alpha}\entails\beta\to\gamma&$\hyperref[deduction]{\mathbf{T_1}}\;\set{1}$\\
                \fa\entails\alpha\to\beta\to\gamma&$\hyperref[deduction]{\mathbf{T_1}}\;\set{2}$\\
                \fa\set{\alpha\wedge\beta}\entails\alpha\wedge\beta&$\hyperref[premisse]{\mathbf{R_1}}$\\
                \fa\set{\alpha\wedge\beta}\entails\alpha\wedge\beta\to\alpha&$\hyperref[MA4]{\mathbf{A_4}}$\\
                \fa\set{\alpha\wedge\beta}\entails\alpha&$\hyperref[detachment]{\mathbf{R_2}}\;\set{4,5}$\\
                \fa\set{\alpha\wedge\beta}\entails\alpha\wedge\beta\to\beta&$\hyperref[MA5]{\mathbf{A_5}}$\\
                \fa\set{\alpha\wedge\beta}\entails\beta&$\hyperref[detachment]{\mathbf{R_2}}\;\set{4,7}$\\
                \fa\set{\alpha\wedge\beta}\entails\alpha\to\beta\to\gamma&$\hyperref[weakening]{\mathbf{T_2}}\;\set{3}$\\
                \fa\set{\alpha\wedge\beta}\entails\beta\to\gamma&$\hyperref[detachment]{\mathbf{R_2}}\;\set{6,9}$\\
                \fa\set{\alpha\wedge\beta}\entails\gamma&$\hyperref[detachment]{\mathbf{R_2}}\;\set{8,10}$
            \end{fitch}
            \normalsize
            Estando assim demonstrada a proposição.
        \end{proof}
    \end{theorem}

    Analogamente ao teorema anterior, demonstraremos que, caso haja uma prova de $\gamma$ partindo-se da premissa $\alpha$ e uma prova de $\gamma$ partindo-se da premissa $\beta$, então sabe-se que deve haver uma prova de $\gamma$ partindo-se da premissa $\alpha\vee\beta$.

    Os lemas 2 a 13 abaixo serão demonstrados a fim de diminuir o tamanho das provas futuras acerca do isomorfismo entre as traduções e a correção da tradução \emph{call-by-value}.

    \begin{lemma}\label{explosion}
        $\entails\bot\to\alpha$.
        \begin{proof}
            Pode ser provado pela seguinte sucessão de dedução:
            \footnotesize
            \begin{fitch}
                \fb\set{\bot}\entails\bot&$\hyperref[premisse]{\mathbf{R_1}}$\\
                \fa\set{\bot}\entails\bot\to(\alpha\to\bot)\to\bot&$\hyperref[MA1]{\mathbf{A_1}}$\\
                \fa\set{\bot}\entails\neg\neg\alpha&$\hyperref[detachment]{\mathbf{R_2}}\;\set{1,2}$\\
                \fa\set{\bot}\entails\neg\neg\alpha\to\alpha&$\hyperref[MANEG]{\mathbf{A_\neg}}$\\
                \fa\set{\bot}\entails\alpha&$\hyperref[detachment]{\mathbf{R_2}}\;\set{3,4}$\\
                \fa\entails\bot\to\alpha&$\hyperref[deduction]{\mathbf{T_1}}\;\set{5}$.
            \end{fitch}
            \normalsize
            Estando assim demonstrada a proposição.
        \end{proof}
    \end{lemma}

    \begin{lemma}\label{contrapositive}
        $\entails(\alpha\to\beta)\to(\neg\beta\to\neg\alpha)$.
        \begin{proof}
            Pode ser provado pela seguinte sucessão de dedução:
            \footnotesize
            \begin{fitch}
                \fb\set{\alpha\to\beta,\neg\beta}\entails\beta\to\bot&$\hyperref[premisse]{\mathbf{R_1}}$\\
                \fa\set{\alpha\to\beta,\neg\beta}\entails(\beta\to\bot)\to\alpha\to(\beta\to\bot)&\hyperref[MA1]{$\mathbf{A_1}$}\\
                \fa\set{\alpha\to\beta,\neg\beta}\entails\alpha\to\beta\to\bot&$\hyperref[detachment]{\mathbf{R_2}}\;\set{1,2}$\\
                \fa\set{\alpha\to\beta,\neg\beta}\entails(\alpha\to\beta\to\bot)\to(\alpha\to\beta)\to(\alpha\to\bot)&\hyperref[MA2]{$\mathbf{A_2}$}\\
                \fa\set{\alpha\to\beta,\neg\beta}\entails\alpha\to\beta&$\hyperref[premisse]{\mathbf{R_1}}$\\
                \fa\set{\alpha\to\beta,\neg\beta}\entails(\alpha\to\beta)\to(\alpha\to\bot)&$\hyperref[detachment]{\mathbf{R_2}}\;\set{3,4}$\\
                \fa\set{\alpha\to\beta,\neg\beta}\entails\neg\alpha&$\hyperref[detachment]{\mathbf{R_2}}\;\set{5,6}$\\
                \fa\set{\alpha\to\beta}\entails\neg\beta\to\neg\alpha&\refer{deduction}{T}$\;\set{7}$\\
                \fa\entails(\alpha\to\beta)\to(\neg\beta\to\neg\alpha)&\refer{deduction}{T}$\;\set{8}$.
            \end{fitch}
            \normalsize
            Estando assim demonstrada a proposição.
        \end{proof}
    \end{lemma}

    \begin{lemma}\label{and-intro}
        $\vdash(\alpha\to\beta)\to(\alpha\to\gamma)\to\alpha\to\beta\wedge\gamma$.
        \begin{proof}
            Pode ser provado pela seguinte sucessão de dedução:
            \footnotesize
            \begin{fitch}
                \fb\set{\alpha\to\beta,\alpha\to\gamma,\alpha}\vdash\alpha&$\hyperref[premisse]{\mathbf{R_1}}$\\
                \fa\set{\alpha\to\beta,\alpha\to\gamma,\alpha}\vdash\alpha\to\beta&$\hyperref[premisse]{\mathbf{R_1}}$\\
                \fa\set{\alpha\to\beta,\alpha\to\gamma,\alpha}\vdash\beta&$\hyperref[detachment]{\mathbf{R_2}}\;\set{1, 2}$\\
                \fa\set{\alpha\to\beta,\alpha\to\gamma,\alpha}\vdash\alpha\to\gamma&$\hyperref[premisse]{\mathbf{R_1}}$\\
                \fa\set{\alpha\to\beta,\alpha\to\gamma,\alpha}\vdash\gamma&$\hyperref[detachment]{\mathbf{R_2}}\;\set{1, 4}$\\
                \fa\set{\alpha\to\beta,\alpha\to\gamma,\alpha}\vdash\beta\to\gamma\to\beta\wedge\gamma&\hyperref[MA3]{$\mathbf{A_3}$}\\
                \fa\set{\alpha\to\beta,\alpha\to\gamma,\alpha}\vdash\gamma\to\beta\wedge\gamma&$\hyperref[detachment]{\mathbf{R_2}}\;\set{3, 6}$\\
                \fa\set{\alpha\to\beta,\alpha\to\gamma,\alpha}\vdash\beta\wedge\gamma&$\hyperref[detachment]{\mathbf{R_2}}\;\set{5, 7}$\\
                \fa\set{\alpha\to\beta,\alpha\to\gamma}\vdash\alpha\to\beta\wedge\gamma&\refer{deduction}{T}$\;\set{8}$\\
                \fa\set{\alpha\to\beta}\vdash(\alpha\to\gamma)\to\alpha\to\beta\wedge\gamma&\refer{deduction}{T}$\;\set{9}$\\
                \fa\vdash(\alpha\to\beta)\to(\alpha\to\gamma)\to\alpha\to\beta\wedge\gamma&\refer{deduction}{T}$\;\set{10}$.
            \end{fitch}
            \normalsize
            Estando assim demonstrada a proposição.
        \end{proof}
    \end{lemma}

    \begin{lemma}\label{nec-distr}
        $\nec\Gamma\vdash\nec(\alpha\wedge\beta)$ se e somente se $\nec\Gamma\vdash\nec\alpha\wedge\nec\beta$.
        \begin{proof}
            \begin{case}
                \textsc{Caso 1.} Pode ser provado pela seguinte sucessão de dedução:
                \footnotesize
                \begin{fitch}
                    \fb\nec\Gamma\entails\nec(\alpha\wedge\beta)&$\mathbf{H_1}$\\
                    \fa\nec\Gamma\entails\nec(\alpha\wedge\beta)\to\alpha\wedge\beta&\hyperref[MB2]{$\mathbf{B_2}$}\\
                    \fa\nec\Gamma\entails\alpha\wedge\beta&$\hyperref[detachment]{\mathbf{R_2}}\;\set{1, 2}$\\
                    \fa\nec\Gamma\entails\alpha\wedge\beta\to\alpha&\hyperref[MA4]{$\mathbf{A_4}$}\\
                    \fa\nec\Gamma\entails\alpha&$\hyperref[detachment]{\mathbf{R_2}}\;\set{3, 4}$\\
                    \fa\nec\Gamma\entails\nec\alpha&\refer{gen-nec}{T}$\;\set{5}$\\
                    \fa\nec\Gamma\entails\alpha\wedge\beta\to\beta&\hyperref[MA5]{$\mathbf{A_5}$}\\
                    \fa\nec\Gamma\entails\beta&$\hyperref[detachment]{\mathbf{R_2}}\;\set{3, 7}$\\
                    \fa\nec\Gamma\entails\nec\beta&\refer{gen-nec}{T}$\;\set{8}$\\
                    \fa\nec\Gamma\entails\nec\alpha\to\nec\beta\to\nec\alpha\wedge\nec\beta&\hyperref[MA3]{$\mathbf{A_3}$}\\
                    \fa\nec\Gamma\entails\nec\beta\to\nec\alpha\wedge\nec\beta&$\hyperref[detachment]{\mathbf{R_2}}\;\set{6, 10}$\\
                    \fa\nec\Gamma\entails\nec\alpha\wedge\nec\beta&$\hyperref[detachment]{\mathbf{R_2}}\;\set{8, 11}$\\
                \end{fitch}
                \normalsize
            \end{case}
            \begin{case}
                \textsc{Caso 2.} Pode ser provado pela seguinte sucessão de dedução:
                \footnotesize
                \begin{fitch}
                    \fb\nec\Gamma\vdash\nec\alpha\wedge\nec\beta&$\mathbf{H_1}$\\
                    \fa\nec\Gamma\vdash\nec\alpha\wedge\nec\beta\to\nec\alpha&\hyperref[MA4]{$\mathbf{A_4}$}\\
                    \fa\nec\Gamma\vdash\nec\alpha&$\hyperref[detachment]{\mathbf{R_2}}\;\set{1, 2}$\\
                    \fa\nec\Gamma\vdash\nec\alpha\to\alpha&\hyperref[MB2]{$\mathbf{B_2}$}\\
                    \fa\nec\Gamma\vdash\alpha&$\hyperref[detachment]{\mathbf{R_2}}\;\set{3, 4}$\\
                    \fa\nec\Gamma\vdash\nec\alpha\wedge\nec\beta\to\nec\beta&\hyperref[MA5]{$\mathbf{A_5}$}\\
                    \fa\nec\Gamma\vdash\nec\beta&$\hyperref[detachment]{\mathbf{R_2}}\;\set{1, 6}$\\
                    \fa\nec\Gamma\vdash\nec\beta\to\beta&\hyperref[MB2]{$\mathbf{B_2}$}\\
                    \fa\nec\Gamma\vdash\beta&$\hyperref[detachment]{\mathbf{R_2}}\;\set{7, 8}$\\
                    \fa\nec\Gamma\vdash\alpha\to\beta\to\alpha\wedge\beta&\hyperref[MA3]{$\mathbf{A_3}$}\\
                    \fa\nec\Gamma\vdash\beta\to\alpha\wedge\beta&$\hyperref[detachment]{\mathbf{R_2}}\;\set{5, 10}$\\
                    \fa\nec\Gamma\vdash\alpha\wedge\beta&$\hyperref[detachment]{\mathbf{R_2}}\;\set{9, 11}$\\
                    \fa\nec\Gamma\vdash\nec(\alpha\wedge\beta)&\refer{gen-nec}{T}$\;\set{12}$\\
                \end{fitch}
                \normalsize
            \end{case}
        \end{proof}
    \end{lemma}

    \begin{lemma}
        $\vdash\nec(\alpha\to\beta)\to\nec\alpha\to\beta$.
        \begin{proof}
            Pode ser provado pela seguinte sucessão de dedução:
            \footnotesize
            \begin{fitch}
                \fb\set{\nec(\alpha\to\beta),\nec\alpha}\entails\nec\alpha&$\hyperref[premisse]{\mathbf{R_1}}$\\
                \fa\set{\nec(\alpha\to\beta),\nec\alpha}\entails\nec\alpha\to\alpha&$\hyperref[MB2]{\mathbf{B_2}}$\\
                \fa\set{\nec(\alpha\to\beta),\nec\alpha}\entails\alpha&$\hyperref[detachment]{\mathbf{R_2}}\;\set{1,2}$\\
                \fa\set{\nec(\alpha\to\beta),\nec\alpha}\entails\nec(\alpha\to\beta)&$\hyperref[premisse]{\mathbf{R_1}}$\\
                \fa\set{\nec(\alpha\to\beta),\nec\alpha}\entails\nec(\alpha\to\beta)\to\alpha\to\beta&$\hyperref[MB2]{\mathbf{B_2}}$\\
                \fa\set{\nec(\alpha\to\beta),\nec\alpha}\entails\alpha\to\beta&$\hyperref[detachment]{\mathbf{R_2}}\;\set{4,5}$\\
                \fa\set{\nec(\alpha\to\beta),\nec\alpha}\entails\beta&$\hyperref[detachment]{\mathbf{R_2}}\;\set{3,6}$\\
                \fa\set{\nec(\alpha\to\beta)}\entails\nec\alpha\to\beta&$\hyperref[deduction]{\mathbf{T_1}}\;\set{7}$\\
                \fa\entails\nec(\alpha\to\beta)\to\nec\alpha\to\beta&$\hyperref[deduction]{\mathbf{T_1}}\;\set{8}$.
            \end{fitch}
            \normalsize
            Estando assim demonstrada a proposição.
        \end{proof}
    \end{lemma}

    \begin{lemma}\label{comp}
        $\entails(\alpha\to\beta)\to(\beta\to\gamma)\to\alpha\to\gamma$
        \begin{proof}
            Pode ser provado pela seguinte sucessão de dedução:
            \footnotesize
            \begin{fitch}
                \fb\set{\alpha\to\beta,\beta\to\gamma,\alpha}\entails\alpha&$\hyperref[premisse]{\mathbf{R_1}}$\\
                \fa\set{\alpha\to\beta,\beta\to\gamma,\alpha}\entails\alpha\to\beta&$\hyperref[premisse]{\mathbf{R_1}}$\\
                \fa\set{\alpha\to\beta,\beta\to\gamma,\alpha}\entails\beta&$\hyperref[detachment]{\mathbf{R_2}}\;\set{1,2}$\\
                \fa\set{\alpha\to\beta,\beta\to\gamma,\alpha}\entails\beta\to\gamma&$\hyperref[premisse]{\mathbf{R_1}}$\\
                \fa\set{\alpha\to\beta,\beta\to\gamma,\alpha}\entails\gamma&$\hyperref[detachment]{\mathbf{R_2}}\;\set{3,4}$\\
                \fa\set{\alpha\to\beta,\beta\to\gamma}\entails\alpha\to\gamma&$\hyperref[deduction]{\mathbf{T_{\getrefnumber{deduction}}}}\;\set{5}$\\
                \fa\set{\alpha\to\beta}\entails(\beta\to\gamma)\to\alpha\to\gamma&$\hyperref[deduction]{\mathbf{T_{\getrefnumber{deduction}}}}\;\set{6}$\\
                \fa\entails(\alpha\to\beta)\to(\beta\to\gamma)\to\alpha\to\gamma&$\hyperref[deduction]{\mathbf{T_{\getrefnumber{deduction}}}}\;\set{7}$\\
            \end{fitch}
            \normalsize
            Estando assim demonstrada a proposição.
        \end{proof}
    \end{lemma}

    \begin{lemma}\label{neg-intro}
        $\entails\alpha\to\neg\neg\alpha$
        \begin{proof}
            Pode ser provado pela seguinte sucessão de dedução:
            \footnotesize
            \begin{fitch}
                \fb\set{\alpha,\neg\alpha}\entails\alpha&$\hyperref[premisse]{\mathbf{R_1}}$\\
                \fa\set{\alpha,\neg\alpha}\entails\alpha\to\bot&$\hyperref[premisse]{\mathbf{R_1}}$\\
                \fa\set{\alpha,\neg\alpha}\entails\bot&$\hyperref[detachment]{\mathbf{R_2}}\;\set{1,2}$\\
                \fa\set{\alpha}\entails\neg\neg\alpha&$\hyperref[deduction]{\mathbf{T_{\getrefnumber{deduction}}}}\;\set{3}$\\
                \fa\entails\alpha\to\neg\neg\alpha&$\hyperref[deduction]{\mathbf{T_{\getrefnumber{deduction}}}}\;\set{4}$\\
            \end{fitch}
            \normalsize
            Estando assim demonstrada a proposição.
        \end{proof}
    \end{lemma}

    \begin{lemma}\label{or-left}
        $\entails(\alpha\to\beta)\to\alpha\to\beta\vee\gamma$.
        \begin{proof}
            Pode ser provado pela seguinte sucessão de dedução:
            \footnotesize 
            \begin{fitch}
                \fb\set{\alpha\to\beta,\alpha}\entails\alpha&$\hyperref[premisse]{\mathbf{R_1}}$\\
                \fa\set{\alpha\to\beta,\alpha}\entails\alpha\to\beta&$\hyperref[premisse]{\mathbf{R_1}}$\\
                \fa\set{\alpha\to\beta,\alpha}\entails\beta&$\hyperref[detachment]{\mathbf{R_2}}\;\set{1,2}$\\
                \fa\set{\alpha\to\beta,\alpha}\entails\beta\to\beta\vee\gamma&$\hyperref[MA6]{\mathbf{A_6}}$\\
                \fa\set{\alpha\to\beta,\alpha}\entails\beta\vee\gamma&$\hyperref[detachment]{\mathbf{R_2}}\;\set{3,4}$\\
                \fa\set{\alpha\to\beta}\entails\alpha\to\beta\vee\gamma&$\hyperref[deduction]{\mathbf{T_{\getrefnumber{deduction}}}}\;\set{5}$\\
                \fa\entails(\alpha\to\beta)\to\alpha\to\beta\vee\gamma&$\hyperref[deduction]{\mathbf{T_{\getrefnumber{deduction}}}}\;\set{6}$\\
            \end{fitch}
            \normalsize
            Estando assim demonstrada a proposição.
        \end{proof}
    \end{lemma}

    \begin{lemma}\label{or-right}
        $\entails(\alpha\to\beta)\to\alpha\to\gamma\vee\beta$.
        \begin{proof}
            Pode ser provado pela seguinte sucessão de dedução:
            \footnotesize 
            \begin{fitch}
                \fb\set{\alpha\to\beta,\alpha}\entails\alpha&$\hyperref[premisse]{\mathbf{R_1}}$\\
                \fa\set{\alpha\to\beta,\alpha}\entails\alpha\to\beta&$\hyperref[premisse]{\mathbf{R_1}}$\\
                \fa\set{\alpha\to\beta,\alpha}\entails\beta&$\hyperref[detachment]{\mathbf{R_2}}\;\set{1,2}$\\
                \fa\set{\alpha\to\beta,\alpha}\entails\beta\to\gamma\vee\beta&$\hyperref[MA7]{\mathbf{A_7}}$\\
                \fa\set{\alpha\to\beta,\alpha}\entails\gamma\vee\beta&$\hyperref[detachment]{\mathbf{R_2}}\;\set{3,4}$\\
                \fa\set{\alpha\to\beta}\entails\alpha\to\gamma\vee\beta&$\hyperref[deduction]{\mathbf{T_{\getrefnumber{deduction}}}}\;\set{5}$\\
                \fa\entails(\alpha\to\beta)\to\alpha\to\gamma\vee\beta&$\hyperref[deduction]{\mathbf{T_{\getrefnumber{deduction}}}}\;\set{6}$\\
            \end{fitch}
            \normalsize
            Estando assim demonstrada a proposição.
        \end{proof}
    \end{lemma}

    \begin{lemma}\label{or-subst}
        $\entails(\alpha\to\gamma)\to(\beta\to\delta)\to\alpha\vee\beta\to\gamma\vee\delta$.
        \begin{proof}
            Pode ser provado pela seguinte sucessão de dedução:
            \footnotesize 
            \begin{fitch}
                \fb\set{\alpha\to\gamma,\beta\to\delta,\alpha\vee\beta}\entails\alpha\to\gamma&$\hyperref[premisse]{\mathbf{R_1}}$\\
                \fa\set{\alpha\to\gamma,\beta\to\delta,\alpha\vee\beta}\entails(\alpha\to\gamma)\to\alpha\to\gamma\vee\delta&$\hyperref[or-left]{\mathbf{L_{\getrefnumber{or-left}}}}$\\
                \fa\set{\alpha\to\gamma,\beta\to\delta,\alpha\vee\beta}\entails\alpha\to\gamma\vee\delta&$\hyperref[detachment]{\mathbf{R_2}}\;\set{1,2}$\\
                \fa\set{\alpha\to\gamma,\beta\to\delta,\alpha\vee\beta}\entails\beta\to\delta&$\hyperref[premisse]{\mathbf{R_1}}$\\
                \fa\set{\alpha\to\gamma,\beta\to\delta,\alpha\vee\beta}\entails(\beta\to\delta)\to\beta\to\gamma\vee\delta&$\hyperref[or-right]{\mathbf{L_{\getrefnumber{or-right}}}}$\\
                \fa\set{\alpha\to\gamma,\beta\to\delta,\alpha\vee\beta}\entails\beta\to\gamma\vee\delta&$\hyperref[detachment]{\mathbf{R_2}}\;\set{4,5}$\\
                \fa\set{\alpha\to\gamma,\beta\to\delta,\alpha\vee\beta}\entails\alpha\vee\beta&$\hyperref[premisse]{\mathbf{R_1}}$\\
                \fa\set{\alpha\to\gamma,\beta\to\delta,\alpha\vee\beta}\entails(\alpha\to\gamma\vee\delta)\to(\beta\to\gamma\vee\delta)\to\alpha\vee\beta\to\gamma\vee\delta&$\hyperref[MA8]{\mathbf{A_8}}$\\
                \fa\set{\alpha\to\gamma,\beta\to\delta,\alpha\vee\beta}\entails(\beta\to\gamma\vee\delta)\to\alpha\vee\beta\to\gamma\vee\delta&$\hyperref[detachment]{\mathbf{R_2}}\;\set{3,8}$\\
                \fa\set{\alpha\to\gamma,\beta\to\delta,\alpha\vee\beta}\entails\alpha\vee\beta\to\gamma\vee\delta&$\hyperref[detachment]{\mathbf{R_2}}\;\set{6,9}$\\
                \fa\set{\alpha\to\gamma,\beta\to\delta,\alpha\vee\beta}\entails\gamma\vee\delta&$\hyperref[detachment]{\mathbf{R_2}}\;\set{7,10}$\\
                \fa\set{\alpha\to\gamma,\beta\to\delta}\entails\alpha\vee\beta\to\gamma\vee\delta&$\hyperref[deduction]{\mathbf{T_{\getrefnumber{deduction}}}}\;\set{11}$\\
                \fa\set{\alpha\to\gamma}\entails(\beta\to\delta)\to\alpha\vee\beta\to\gamma\vee\delta&$\hyperref[deduction]{\mathbf{T_{\getrefnumber{deduction}}}}\;\set{12}$\\
                \fa\entails(\alpha\to\gamma)\to(\beta\to\delta)\to\alpha\vee\beta\to\gamma\vee\delta&$\hyperref[deduction]{\mathbf{T_{\getrefnumber{deduction}}}}\;\set{13}$\\
            \end{fitch}
            \normalsize
            Estando assim demonstrada a proposição.
        \end{proof}
    \end{lemma}

    \begin{lemma}\label{or-distr}
        $\nec\Gamma\entails\nec\alpha\vee\nec\beta$ se e somente se $\nec\Gamma\entails\nec(\alpha\vee\beta)$.
        \begin{proof}
            \begin{case}
                \textsc{Caso 1.} Pode ser provado pela seguinte sucessão de dedução:
                \footnotesize
                \begin{fitch}
                    \fb\nec\Gamma\entails\nec\alpha\to\alpha&$\hyperref[MB2]{\mathbf{B_2}}$\\
                    \fa\nec\Gamma\entails\nec\beta\to\beta&$\hyperref[MB2]{\mathbf{B_2}}$\\
                    \fa\nec\Gamma\entails\nec\alpha\vee\nec\beta&$\mathbf{H_1}$\\
                    \fa\nec\Gamma\entails(\nec\alpha\to\alpha)\to(\nec\beta\to\beta)\to\nec\alpha\vee\nec\beta\to\alpha\vee\beta&\refer{or-subst}{L}\\
                    \fa\nec\Gamma\entails(\nec\beta\to\beta)\to\nec\alpha\vee\nec\beta\to\alpha\vee\beta&$\hyperref[detachment]{\mathbf{R_2}}\;\set{1,4}$\\
                    \fa\nec\Gamma\entails\nec\alpha\vee\nec\beta\to\alpha\vee\beta&$\hyperref[detachment]{\mathbf{R_2}}\;\set{2,5}$\\
                    \fa\nec\Gamma\entails\alpha\vee\beta&$\hyperref[detachment]{\mathbf{R_2}}\;\set{3,6}$\\
                    \fa\nec\Gamma\entails\nec(\alpha\vee\beta)&\refer{gen-nec}{T}$\;\set{7}$\\
                \end{fitch}
                \normalsize
            \end{case}
            \begin{case}
                \textsc{Caso 2.} Pode ser provado pela seguinte sucessão de dedução:
                \footnotesize
                \begin{fitch}
                    \fb\nec\Gamma\vdash\nec\alpha\wedge\nec\beta&$\mathbf{H_1}$\\
                    \fa\nec\Gamma\vdash\nec\alpha\wedge\nec\beta\to\nec\alpha&\hyperref[MA4]{$\mathbf{A_4}$}\\
                    \fa\nec\Gamma\vdash\nec\alpha&$\hyperref[detachment]{\mathbf{R_2}}\;\set{1, 2}$\\
                    \fa\nec\Gamma\vdash\nec\alpha\to\alpha&\hyperref[MB2]{$\mathbf{B_2}$}\\
                    \fa\nec\Gamma\vdash\alpha&$\hyperref[detachment]{\mathbf{R_2}}\;\set{3, 4}$\\
                    \fa\nec\Gamma\vdash\nec\alpha\wedge\nec\beta\to\nec\beta&\hyperref[MA5]{$\mathbf{A_5}$}\\
                    \fa\nec\Gamma\vdash\nec\beta&$\hyperref[detachment]{\mathbf{R_2}}\;\set{1, 6}$\\
                    \fa\nec\Gamma\vdash\nec\beta\to\beta&\hyperref[MB2]{$\mathbf{B_2}$}\\
                    \fa\nec\Gamma\vdash\beta&$\hyperref[detachment]{\mathbf{R_2}}\;\set{7, 8}$\\
                    \fa\nec\Gamma\vdash\alpha\to\beta\to\alpha\wedge\beta&\hyperref[MA3]{$\mathbf{A_3}$}\\
                    \fa\nec\Gamma\vdash\beta\to\alpha\wedge\beta&$\hyperref[detachment]{\mathbf{R_2}}\;\set{5, 10}$\\
                    \fa\nec\Gamma\vdash\alpha\wedge\beta&$\hyperref[detachment]{\mathbf{R_2}}\;\set{9, 11}$\\
                    \fa\nec\Gamma\vdash\nec(\alpha\wedge\beta)&\refer{gen-nec}{T}$\;\set{12}$\\
                \end{fitch}
                \normalsize
            \end{case}
        \end{proof}
    \end{lemma}

        \section{Interderivabilidade}

Conforme afirmado anteriormente, ambas as traduções apresentadas gozam de propriedade da interderivação.
Ou seja, a derivação de uma sentença traduzida por uma das traduções implica que se pode derivar esta mesma sentença traduzida pela outra tradução, e vice-versa.
Esta seção busca demonstrar essa interderivabilidade de duas maneiras: tanto como uma bi-implicação dentro do sistema $\mathfrak{L}$ quanto como uma bi-implicação na metalinguagem.
Para tanto, precisaremos demonstrar uma quantidade de lemas.
Nomearemos o primeiro deles \emph{explosão}, conforme abaixo.

\vspace{\baselineskip}
\begin{tcolorbox}[enhanced jigsaw, breakable, sharp corners, colframe=black, colback=white, boxrule=0.5pt, left=1.5mm, right=1.5mm, top=1.5mm, bottom=1.5mm]
\begin{lemma}[Explosão]\label{explosion}
    $\Gamma\entails_\mathfrak{L}\bot\to\alpha$.
    \begin{proof}
        Pode ser demonstrado pela dedução que segue.

        \vspace{0.5\baselineskip}
        \footnotesize
        \setlength{\rowskip}{0.5\baselineskip}
        \begin{xltabular}{\textwidth}{r | X l l}
            \scriptsize{\phantom{0}1}\phantom{ } & $\ \Gamma\set{\bot}\entails\bot$                              & $\hyperref[modal.rule.1]{\mathbf{R_1}}$\phantom{1} & \phantom{$\set{00,00}$}\\[\rowskip]
            \scriptsize{\phantom{0}2}\phantom{ } & $\ \Gamma\set{\bot}\entails\bot\to(\alpha\to\bot)\to\bot$     & $\hyperref[modal.axiom.1]{\mathbf{A_1}}$           & \\[\rowskip]
            \scriptsize{\phantom{0}3}\phantom{ } & $\ \Gamma\set{\bot}\entails(\alpha\to\bot)\to\bot$            & $\hyperref[modal.rule.2]{\mathbf{R_2}}$            & $\set{1,2}$\\[\rowskip]
            \scriptsize{\phantom{0}4}\phantom{ } & $\ \Gamma\set{\bot}\entails((\alpha\to\bot)\to\bot)\to\alpha$ & $\hyperref[modal.axiom.negation]{\mathbf{A_\neg}}$ & \\[\rowskip]
            \scriptsize{\phantom{0}5}\phantom{ } & $\ \Gamma\set{\bot}\entails\alpha$                            & $\hyperref[modal.rule.2]{\mathbf{R_2}}$            & $\set{3,4}$\\[\rowskip]
            \scriptsize{\phantom{0}6}\phantom{ } & $\ \Gamma\entails\bot\to\alpha$                               & \refer{deduction}{T}                               & $\set{5}$\\[\rowskip]
        \end{xltabular}
        \normalsize

        \vspace{0.5\baselineskip}
        Estando assim demonstrada a proposição.
    \end{proof}
\end{lemma}
\end{tcolorbox}

\vspace{.5\baselineskip}
Em seguida, demonstraremos um lema que combina duas implicações com uma conjunção dos antecedentes de modo a inferir uma conjunção dos consequentes.
Para tanto, foram usados os axiomas de introdução e eliminação da negação em conjunto com a regra da separação.
A demontração deste lema busca simplificar a prova de interderivabilidade, uma vez que a faremos por indução sobre a profundidade da sentença, sendo um dos seus casos a conjunção.

\vspace{\baselineskip}
\begin{tcolorbox}[enhanced jigsaw, breakable, sharp corners, colframe=black, colback=white, boxrule=0.5pt, left=1.5mm, right=1.5mm, top=1.5mm, bottom=1.5mm]
    \begin{lemma}\label{conjunction.exchange}
        Se $\Gamma\entails_\mathfrak{L}\alpha\to\gamma$ e $\Gamma\entails_\mathfrak{L}\beta\to\delta$, então $\Gamma\entails_\mathfrak{L}\alpha\wedge\beta\to\gamma\wedge\delta$.
        \begin{proof}
        Pode ser demonstrado pela dedução que segue.

        \vspace{0.5\baselineskip}
        \footnotesize
        \setlength{\rowskip}{0.5\baselineskip}
        \begin{xltabular}{\textwidth}{r | X l l}
            \scriptsize{\phantom{0}1}\phantom{ } & $\ \Gamma\cup\set{\alpha\wedge\beta}\vdash\alpha\wedge\beta$                     & $\hyperref[modal.rule.1]{\mathbf{R_1}}$\phantom{1} & \phantom{$\set{00,00}$}\\[\rowskip]
            \scriptsize{\phantom{0}2}\phantom{ } & $\ \Gamma\cup\set{\alpha\wedge\beta}\vdash\alpha\wedge\beta\to\alpha$            & $\hyperref[modal.axiom.4]{\mathbf{A_4}}$           & \\[\rowskip]
            \scriptsize{\phantom{0}3}\phantom{ } & $\ \Gamma\cup\set{\alpha\wedge\beta}\vdash\alpha$                                & $\hyperref[modal.rule.2]{\mathbf{R_2}}$            & $\set{1,2}$\\[\rowskip]
            \scriptsize{\phantom{0}4}\phantom{ } & $\ \Gamma\cup\set{\alpha\wedge\beta}\vdash\alpha\wedge\beta\to\beta$             & $\hyperref[modal.axiom.4]{\mathbf{A_4}}$           & \\[\rowskip]
            \scriptsize{\phantom{0}5}\phantom{ } & $\ \Gamma\cup\set{\alpha\wedge\beta}\vdash\beta$                                 & $\hyperref[modal.rule.2]{\mathbf{R_2}}$            & $\set{1,4}$\\[\rowskip]
            \scriptsize{\phantom{0}6}\phantom{ } & $\ \Gamma\cup\set{\alpha\wedge\beta}\vdash\alpha\to\gamma$                       & $\mathbf{H_1}$                                     & \\[\rowskip]
            \scriptsize{\phantom{0}7}\phantom{ } & $\ \Gamma\cup\set{\alpha\wedge\beta}\vdash\beta\to\delta$                        & $\mathbf{H_2}$                                     & \\[\rowskip]
            \scriptsize{\phantom{0}8}\phantom{ } & $\ \Gamma\cup\set{\alpha\wedge\beta}\vdash\gamma$                                & $\hyperref[modal.rule.2]{\mathbf{R_2}}$            & $\set{3,6}$\\[\rowskip]
            \scriptsize{\phantom{0}9}\phantom{ } & $\ \Gamma\cup\set{\alpha\wedge\beta}\vdash\delta$                                & $\hyperref[modal.rule.2]{\mathbf{R_2}}$            & $\set{5,7}$\\[\rowskip]
            \scriptsize{10}\phantom{ }           & $\ \Gamma\cup\set{\alpha\wedge\beta}\vdash\gamma\to\delta\to\gamma\wedge\delta$  & $\hyperref[modal.axiom.3]{\mathbf{A_4}}$           & \\[\rowskip]
            \scriptsize{11}\phantom{ }           & $\ \Gamma\cup\set{\alpha\wedge\beta}\vdash\delta\to\gamma\wedge\delta$           & $\hyperref[modal.rule.2]{\mathbf{R_2}}$            & $\set{8,10}$\\[\rowskip]
            \scriptsize{12}\phantom{ }           & $\ \Gamma\cup\set{\alpha\wedge\beta}\vdash\gamma\wedge\delta$                    & $\hyperref[modal.rule.2]{\mathbf{R_2}}$            & $\set{9,11}$\\[\rowskip]
            \scriptsize{13}\phantom{ }           & $\ \Gamma\vdash\alpha\wedge\beta\to\gamma\wedge\delta$                           & \refer{deduction}{T}                               & $\set{12}$
        \end{xltabular}
        \normalsize

        \vspace{0.5\baselineskip}
        Estando assim demonstrada a proposição.
        \end{proof}
    \end{lemma}
\end{tcolorbox}

\vspace{.5\baselineskip}
Agora, demonstraremos a \emph{distribução da necessidade sobre a conjunção}.
Este lema afirma que, caso tenhamos necessidade de uma conjunção, então temos a conjunção da necessidade dos conjuntos.
Assim como no lema anterior, foram usados os axiomas de introdução e eliminação da negação em conjunto com a regra da separação.
Neste lema, entretanto, o teorema da generalização da necessitação e o teorema do enfraqueciemnto desempenham um funções importantes.
Usaremos esta asseção para a demonstração da interderivabilidade.
Mais adiante, demonstraremos a proposição conversa deste lema

\vspace{\baselineskip}
\begin{tcolorbox}[enhanced jigsaw, breakable, sharp corners, colframe=black, colback=white, boxrule=0.5pt, left=1.5mm, right=1.5mm, top=1.5mm, bottom=1.5mm]
    \begin{lemma}\label{necessity.conjunction.distribution}
        $\Gamma\entails_\mathfrak{L}\nec(\alpha\wedge\beta)\to\nec\alpha\wedge\nec\beta$.
        \begin{proof}
        Pode ser demonstrado pela dedução que segue.

        \vspace{0.5\baselineskip}
        \footnotesize
        \setlength{\rowskip}{0.5\baselineskip}
        \begin{xltabular}{\textwidth}{r | X l l}
            \scriptsize{\phantom{0}1}\phantom{ } & $\ \set{\nec(\alpha\wedge\beta)}\vdash\nec(\alpha\wedge\beta)$                            & $\hyperref[modal.rule.1]{\mathbf{R_1}}$        & \phantom{$\set{00,00}$}\\[\rowskip]
            \scriptsize{\phantom{0}2}\phantom{ } & $\ \set{\nec(\alpha\wedge\beta)}\vdash\nec(\alpha\wedge\beta)\to\alpha\wedge\beta$        & $\hyperref[modal.axiom.modal.2]{\mathbf{B_2}}$ & \\[\rowskip]
            \scriptsize{\phantom{0}3}\phantom{ } & $\ \set{\nec(\alpha\wedge\beta)}\vdash\alpha\wedge\beta$                                  & $\hyperref[modal.rule.2]{\mathbf{R_2}}$        & $\set{1,2}$\\[\rowskip]
            \scriptsize{\phantom{0}4}\phantom{ } & $\ \set{\nec(\alpha\wedge\beta)}\vdash\alpha\wedge\beta\to\alpha$                         & $\hyperref[modal.axiom.4]{\mathbf{A_4}}$       & \\[\rowskip]
            \scriptsize{\phantom{0}5}\phantom{ } & $\ \set{\nec(\alpha\wedge\beta)}\vdash\alpha$                                             & $\hyperref[modal.rule.2]{\mathbf{R_2}}$        & $\set{3,4}$\\[\rowskip]
            \scriptsize{\phantom{0}6}\phantom{ } & $\ \set{\nec(\alpha\wedge\beta)}\vdash\nec\alpha$                                         & \refer{generalization}{T}\phantom{1}           & $\set{5}$\\[\rowskip]
            \scriptsize{\phantom{0}7}\phantom{ } & $\ \set{\nec(\alpha\wedge\beta)}\vdash\alpha\wedge\beta\to\beta$                          & $\hyperref[modal.axiom.5]{\mathbf{A_5}}$       & \\[\rowskip]
            \scriptsize{\phantom{0}8}\phantom{ } & $\ \set{\nec(\alpha\wedge\beta)}\vdash\beta$                                              & $\hyperref[modal.rule.2]{\mathbf{R_2}}$        & $\set{3,7}$\\[\rowskip]
            \scriptsize{\phantom{0}9}\phantom{ } & $\ \set{\nec(\alpha\wedge\beta)}\vdash\nec\beta$                                          & \refer{generalization}{T}                      & $\set{8}$\\[\rowskip]
            \scriptsize{10}\phantom{ }           & $\ \set{\nec(\alpha\wedge\beta)}\vdash\nec\alpha\to\nec\beta\to\nec\alpha\wedge\nec\beta$ & $\hyperref[modal.axiom.3]{\mathbf{A_3}}$       & \\[\rowskip]
            \scriptsize{11}\phantom{ }           & $\ \set{\nec(\alpha\wedge\beta)}\vdash\nec\beta\to\nec\alpha\wedge\nec\beta$              & $\hyperref[modal.rule.2]{\mathbf{R_2}}$        & $\set{6,10}$\\[\rowskip]
            \scriptsize{12}\phantom{ }           & $\ \set{\nec(\alpha\wedge\beta)}\vdash\nec\alpha\wedge\nec\beta$                          & $\hyperref[modal.rule.2]{\mathbf{R_2}}$        & $\set{9,11}$\\[\rowskip]
            \scriptsize{13}\phantom{ }           & $\ \vdash\nec(\alpha\wedge\beta)\to\nec\alpha\wedge\nec\beta$                             & \refer{deduction}{T}                           & $\set{12}$\\[\rowskip]
            \scriptsize{14}\phantom{ }           & $\ \Gamma\vdash\nec(\alpha\wedge\beta)\to\nec\alpha\wedge\nec\beta$                       & \refer{weakening}{T}                           & $\set{13}$
        \end{xltabular}
        \normalsize

        \vspace{0.5\baselineskip}
        Estando assim demonstrada a proposição.
        \end{proof}
    \end{lemma}
\end{tcolorbox}

\vspace{.5\baselineskip}
Consideremos agora um novo lema.
Este lema afirma que, dada uma implicação dupla, podemos juntar as sentenças antecedentes destas duas implicações em uma conjunção antecedente de uma implicação apenas.
Na computação, este corresponte ao tipo da função de \emph{descurrificação}.
Aqui entretanto referiremos a ele como \emph{importação}.
Usaremos esta asseção para a demonstração da do lema que o segue.

\vspace{\baselineskip}
\begin{tcolorbox}[enhanced jigsaw, breakable, sharp corners, colframe=black, colback=white, boxrule=0.5pt, left=1.5mm, right=1.5mm, top=1.5mm, bottom=1.5mm]
    \begin{lemma}[Importação]\label{importation}
        Se $\Gamma\entails_\mathfrak{L}\alpha\to\beta\to\gamma$ então $\Gamma\entails_\mathfrak{L}\alpha\wedge\beta\to\gamma$.
        \begin{proof}
        Pode ser demonstrado pela dedução que segue.

        \vspace{0.5\baselineskip}
        \footnotesize
        \setlength{\rowskip}{0.5\baselineskip}
        \begin{xltabular}{\textwidth}{r | X l l}
            \scriptsize{\phantom{0}1}\phantom{ } & $\ \Gamma\vdash\alpha\to\beta\to\gamma$                               & $\mathbf{H_1}$\phantom{1}                & \phantom{$\set{00,00}$}\\[\rowskip]
            \scriptsize{\phantom{0}2}\phantom{ } & $\ \Gamma\cup\set{\alpha\wedge\beta}\vdash\alpha\wedge\beta$          & $\hyperref[modal.rule.1]{\mathbf{R_1}}$  & \\[\rowskip]
            \scriptsize{\phantom{0}3}\phantom{ } & $\ \Gamma\cup\set{\alpha\wedge\beta}\vdash\alpha\wedge\beta\to\alpha$ & $\hyperref[modal.axiom.4]{\mathbf{A_4}}$ & \\[\rowskip]
            \scriptsize{\phantom{0}4}\phantom{ } & $\ \Gamma\cup\set{\alpha\wedge\beta}\vdash\alpha$                     & $\hyperref[modal.rule.2]{\mathbf{R_2}}$  & $\set{2,3}$\\[\rowskip]
            \scriptsize{\phantom{0}5}\phantom{ } & $\ \Gamma\cup\set{\alpha\wedge\beta}\vdash\alpha\wedge\beta\to\beta$  & $\hyperref[modal.axiom.5]{\mathbf{A_5}}$ & \\[\rowskip]
            \scriptsize{\phantom{0}6}\phantom{ } & $\ \Gamma\cup\set{\alpha\wedge\beta}\vdash\beta$                      & $\hyperref[modal.rule.2]{\mathbf{R_2}}$  & $\set{2,5}$\\[\rowskip]
            \scriptsize{\phantom{0}7}\phantom{ } & $\ \Gamma\cup\set{\alpha\wedge\beta}\vdash\alpha\to\beta\to\gamma$    & \refer{weakening}{T}                     & $\set{1}$\\[\rowskip]
            \scriptsize{\phantom{0}8}\phantom{ } & $\ \Gamma\cup\set{\alpha\wedge\beta}\vdash\beta\to\gamma$             & $\hyperref[modal.rule.2]{\mathbf{R_2}}$  & $\set{4,7}$\\[\rowskip]
            \scriptsize{\phantom{0}9}\phantom{ } & $\ \Gamma\cup\set{\alpha\wedge\beta}\vdash\gamma$                     & $\hyperref[modal.rule.2]{\mathbf{R_2}}$  & $\set{6,8}$\\[\rowskip]
            \scriptsize{10}\phantom{ }           & $\ \Gamma\vdash\alpha\wedge\beta\to\gamma$                            & \refer{deduction}{T}                     & $\set{9}$
        \end{xltabular}
        \normalsize

        \vspace{0.5\baselineskip}
        Estando assim demonstrada a proposição.
        \end{proof}
    \end{lemma}
\end{tcolorbox}

\vspace{.5\baselineskip}
A demonstração do lema da importação nos permite demonstrar o lema da \emph{agregação da necessidade sobre a conjunção}, que se trata da implicação conversa ao lema da distribuição da necessidade sobre a conjunção.
Este lema afirma que, caso tenhamos a conjunção de duas necessidades, temos a necessidade dos conjuntos.
O uso da importação se mostra importante porque este nos permite assumir um conjunto de sentenças necessariamente verdadeiras, anuindo o uso do teorema da generalização da necessitação.
Esse conjunto pode então ser transformado em uma implicação por meio do teorema da dedução e em uma conjunção por meio do lema da importação.

\vspace{\baselineskip}
\begin{tcolorbox}[enhanced jigsaw, breakable, sharp corners, colframe=black, colback=white, boxrule=0.5pt, left=1.5mm, right=1.5mm, top=1.5mm, bottom=1.5mm]
    \begin{lemma}\label{necessity.conjunction.undistribution}
        $\Gamma\entails_\mathfrak{L}\nec\alpha\wedge\nec\beta\to\nec(\alpha\wedge\beta)$.
        \begin{proof}
        Pode ser demonstrado pela dedução que segue.

        \vspace{0.5\baselineskip}
        \footnotesize
        \setlength{\rowskip}{0.5\baselineskip}
        \begin{xltabular}{\textwidth}{r | X l l}
            \scriptsize{\phantom{0}1}\phantom{ } & $\ \set{\nec\alpha,\nec\beta}\vdash\nec\alpha$                         & $\hyperref[modal.rule.1]{\mathbf{R_1}}$        & \phantom{$\set{00,00}$}\\[\rowskip]
            \scriptsize{\phantom{0}2}\phantom{ } & $\ \set{\nec\alpha,\nec\beta}\vdash\nec\alpha\to\alpha$                & $\hyperref[modal.axiom.modal.2]{\mathbf{B_2}}$ & \\[\rowskip]
            \scriptsize{\phantom{0}3}\phantom{ } & $\ \set{\nec\alpha,\nec\beta}\vdash\alpha$                             & $\hyperref[modal.rule.2]{\mathbf{R_2}}$        & $\set{1,2}$\\[\rowskip]
            \scriptsize{\phantom{0}4}\phantom{ } & $\ \set{\nec\alpha,\nec\beta}\vdash\nec\beta$                          & $\hyperref[modal.rule.1]{\mathbf{R_1}}$        & \\[\rowskip]
            \scriptsize{\phantom{0}5}\phantom{ } & $\ \set{\nec\alpha,\nec\beta}\vdash\nec\beta\to\beta$                  & $\hyperref[modal.axiom.modal.2]{\mathbf{B_2}}$ & \\[\rowskip]
            \scriptsize{\phantom{0}6}\phantom{ } & $\ \set{\nec\alpha,\nec\beta}\vdash\beta$                              & $\hyperref[modal.rule.2]{\mathbf{R_2}}$        & $\set{4,5}$\\[\rowskip]
            \scriptsize{\phantom{0}7}\phantom{ } & $\ \set{\nec\alpha,\nec\beta}\vdash\alpha\to\beta\to\alpha\wedge\beta$ & $\hyperref[modal.axiom.3]{\mathbf{A_3}}$       & \\[\rowskip]
            \scriptsize{\phantom{0}8}\phantom{ } & $\ \set{\nec\alpha,\nec\beta}\vdash\beta\to\alpha\wedge\beta$          & $\hyperref[modal.rule.2]{\mathbf{R_2}}$        & $\set{3,7}$\\[\rowskip]
            \scriptsize{\phantom{0}9}\phantom{ } & $\ \set{\nec\alpha,\nec\beta}\vdash\alpha\wedge\beta$                  & $\hyperref[modal.rule.2]{\mathbf{R_2}}$        & $\set{6,8}$\\[\rowskip]
            \scriptsize{10}\phantom{ }           & $\ \set{\nec\alpha,\nec\beta}\vdash\nec(\alpha\wedge\beta)$            & \refer{generalization}{T}\phantom{1}           & $\set{9}$\\[\rowskip]
            \scriptsize{11}\phantom{ }           & $\ \set{\nec\alpha}\vdash\nec\beta\to\nec(\alpha\wedge\beta)$          & \refer{deduction}{T}                           & $\set{10}$\\[\rowskip]
            \scriptsize{12}\phantom{ }           & $\ \vdash\nec\alpha\to\nec\beta\to\nec(\alpha\wedge\beta)$             & \refer{deduction}{T}                           & $\set{11}$\\[\rowskip]
            \scriptsize{13}\phantom{ }           & $\ \vdash\nec\alpha\wedge\nec\beta\to\nec(\alpha\wedge\beta)$          & \refer{importation}{L}                         & $\set{12}$\\[\rowskip]
            \scriptsize{14}\phantom{ }           & $\ \Gamma\vdash\nec\alpha\wedge\nec\beta\to\nec(\alpha\wedge\beta)$    & \refer{weakening}{T}                           & $\set{13}$
        \end{xltabular}
        \normalsize

        \vspace{0.5\baselineskip}
        Estando assim demonstrada a proposição.
        \end{proof}
    \end{lemma}
\end{tcolorbox}

\vspace{.5\baselineskip}
Demonstraremos agora a versão disjuntiva do lema \refer{conjunction.exchange}{L}.
Este lema combina duas implicações com uma disjunção dos antecedentes de modo a inferir uma disjunção dos consequentes.
Para tanto, foram usados os axiomas de introdução e eliminação da disjunução em conjunto com a regra da separação.
Assim como para a conjunção, a demontração deste lema busca simplificar a prova de interderivabilidade, sendo a disjunção um de seus casos da indução.

\vspace{\baselineskip}
\begin{tcolorbox}[enhanced jigsaw, breakable, sharp corners, colframe=black, colback=white, boxrule=0.5pt, left=1.5mm, right=1.5mm, top=1.5mm, bottom=1.5mm]
    \begin{lemma}\label{disjunction.exchange}
        Se $\Gamma\entails_\mathfrak{L}\alpha\to\gamma$ e $\Gamma\entails_\mathfrak{L}\beta\to\delta$, então $\Gamma\entails_\mathfrak{L}\alpha\vee\beta\to\gamma\vee\delta$.
        \begin{proof}
        Pode ser demonstrado pela dedução que segue.

        \vspace{0.5\baselineskip}
        \footnotesize
        \setlength{\rowskip}{0.5\baselineskip}
        \begin{xltabular}{\textwidth}{r | X l l}
            \scriptsize{\phantom{0}1}\phantom{ } & $\ \Gamma\cup\set{\alpha\vee\beta,\alpha\to\gamma,\alpha}\vdash\alpha$                                                                  & $\hyperref[modal.rule.1]{\mathbf{R_1}}$\phantom{1} & \phantom{$\set{00,00}$}\\[\rowskip]
            \scriptsize{\phantom{0}2}\phantom{ } & $\ \Gamma\cup\set{\alpha\vee\beta,\alpha\to\gamma,\alpha}\vdash\alpha\to\gamma$                                                         & $\hyperref[modal.rule.1]{\mathbf{R_1}}$            & \\[\rowskip]
            \scriptsize{\phantom{0}3}\phantom{ } & $\ \Gamma\cup\set{\alpha\vee\beta,\alpha\to\gamma,\alpha}\vdash\gamma$                                                                  & $\hyperref[modal.rule.2]{\mathbf{R_2}}$            & $\set{1,2}$\\[\rowskip]
            \scriptsize{\phantom{0}4}\phantom{ } & $\ \Gamma\cup\set{\alpha\vee\beta,\alpha\to\gamma,\alpha}\vdash\gamma\to\gamma\vee\delta$                                               & $\hyperref[modal.axiom.6]{\mathbf{A_6}}$           & \\[\rowskip]
            \scriptsize{\phantom{0}5}\phantom{ } & $\ \Gamma\cup\set{\alpha\vee\beta,\alpha\to\gamma,\alpha}\vdash\gamma\vee\delta$                                                        & $\hyperref[modal.rule.2]{\mathbf{R_2}}$            & $\set{3,4}$\\[\rowskip]
            \scriptsize{\phantom{0}6}\phantom{ } & $\ \Gamma\cup\set{\alpha\vee\beta,\beta\to\delta,\beta}\vdash\beta$                                                                     & $\hyperref[modal.rule.1]{\mathbf{R_1}}$            & \\[\rowskip]
            \scriptsize{\phantom{0}7}\phantom{ } & $\ \Gamma\cup\set{\alpha\vee\beta,\beta\to\delta,\beta}\vdash\beta\to\delta$                                                            & $\hyperref[modal.rule.1]{\mathbf{R_1}}$            & \\[\rowskip]
            \scriptsize{\phantom{0}8}\phantom{ } & $\ \Gamma\cup\set{\alpha\vee\beta,\beta\to\delta,\beta}\vdash\delta$                                                                    & $\hyperref[modal.rule.2]{\mathbf{R_2}}$            & $\set{6,7}$\\[\rowskip]
            \scriptsize{\phantom{0}9}\phantom{ } & $\ \Gamma\cup\set{\alpha\vee\beta,\beta\to\delta,\beta}\vdash\delta\to\gamma\vee\delta$                                                 & $\hyperref[modal.axiom.6]{\mathbf{A_7}}$           & \\[\rowskip]
            \scriptsize{10}\phantom{ }           & $\ \Gamma\cup\set{\alpha\vee\beta,\beta\to\delta,\beta}\vdash\gamma\vee\delta$                                                          & $\hyperref[modal.rule.2]{\mathbf{R_2}}$            & $\set{8,9}$\\[\rowskip]
            \scriptsize{11}\phantom{ }           & $\ \Gamma\cup\set{\alpha\vee\beta,\alpha\to\gamma}\vdash\alpha\to\gamma\vee\delta$                                                      & \refer{deduction}{T}                               & $\set{4}$\\[\rowskip]
            \scriptsize{12}\phantom{ }           & $\ \Gamma\cup\set{\alpha\vee\beta,\beta\to\delta}\vdash\beta\to\gamma\vee\delta$                                                        & \refer{deduction}{T}                               & $\set{10}$\\[\rowskip]
            \scriptsize{13}\phantom{ }           & $\ \Gamma\cup\set{\alpha\vee\beta}\vdash\alpha\to\gamma$                                                                                & $\mathbf{H_1}$                                     & \\[\rowskip]
            \scriptsize{14}\phantom{ }           & $\ \Gamma\cup\set{\alpha\vee\beta}\vdash(\alpha\to\gamma)\to\alpha\to\gamma\vee\delta$                                                  & \refer{deduction}{T}                               & $\set{11}$\\[\rowskip]
            \scriptsize{15}\phantom{ }           & $\ \Gamma\cup\set{\alpha\vee\beta}\vdash\alpha\to\gamma\vee\delta$                                                                      & $\hyperref[modal.rule.2]{\mathbf{R_2}}$            & $\set{13,14}$\\[\rowskip]
            \scriptsize{16}\phantom{ }           & $\ \Gamma\cup\set{\alpha\vee\beta}\vdash\beta\to\delta$                                                                                 & $\mathbf{H_2}$                                     & \\[\rowskip]
            \scriptsize{17}\phantom{ }           & $\ \Gamma\cup\set{\alpha\vee\beta}\vdash(\beta\to\delta)\to\beta\to\gamma\vee\delta$                                                    & \refer{deduction}{T}                               & $\set{12}$\\[\rowskip]
            \scriptsize{18}\phantom{ }           & $\ \Gamma\cup\set{\alpha\vee\beta}\vdash\beta\to\gamma\vee\delta$                                                                       & $\hyperref[modal.rule.2]{\mathbf{R_2}}$            & $\set{16,17}$\\[\rowskip]
            \scriptsize{19}\phantom{ }           & $\ \Gamma\cup\set{\alpha\vee\beta}\vdash\alpha\vee\beta$                                                                                & $\hyperref[modal.rule.1]{\mathbf{R_1}}$            & \\[\rowskip]
            \scriptsize{20}\phantom{ }           & $\ \Gamma\cup\set{\alpha\vee\beta}\vdash(\alpha\to\gamma\vee\delta)\to(\beta\to\gamma\vee\delta)\to\alpha\vee\beta\to\gamma\vee\delta$  & $\hyperref[modal.axiom.8]{\mathbf{A_8}}$           & \\[\rowskip]
            \scriptsize{21}\phantom{ }           & $\ \Gamma\cup\set{\alpha\vee\beta}\vdash(\beta\to\gamma\vee\delta)\to\alpha\vee\beta\to\gamma\vee\delta$                                & $\hyperref[modal.rule.2]{\mathbf{R_2}}$            & $\set{15,20}$\\[\rowskip]
            \scriptsize{22}\phantom{ }           & $\ \Gamma\cup\set{\alpha\vee\beta}\vdash\alpha\vee\beta\to\gamma\vee\delta$                                                             & $\hyperref[modal.rule.2]{\mathbf{R_2}}$            & $\set{18,21}$\\[\rowskip]
            \scriptsize{23}\phantom{ }           & $\ \Gamma\cup\set{\alpha\vee\beta}\vdash\gamma\vee\delta$                                                                               & $\hyperref[modal.rule.2]{\mathbf{R_2}}$            & $\set{19,22}$\\[\rowskip]
            \scriptsize{24}\phantom{ }           & $\ \Gamma\vdash\alpha\vee\beta\to\gamma\vee\delta$                                                                                      & \refer{deduction}{T}                               & $\set{23}$
        \end{xltabular}
        \normalsize

        \vspace{0.5\baselineskip}
        Estando assim demonstrada a proposição.
        \end{proof}
    \end{lemma}
\end{tcolorbox}

\vspace{.5\baselineskip}
Por fim, consideremos o lema derradeiro desta seção.
Demonstraremos o lema da \emph{agregação da necessidade sobre a disjunção}, a versão disjuntiva do lema da agregação da necessidade sobre a conjunção apresentada acima.
Este lema afirma que, caso tenhamos a disjunção de duas necessidades, temos a necessidade dos disjuntos.
De maneira semelhante à sua contraparte conjuntiva, este lema faz uso do teorema da dedução e da generalização da necessitação.

\vspace{\baselineskip}
\begin{tcolorbox}[enhanced jigsaw, breakable, sharp corners, colframe=black, colback=white, boxrule=0.5pt, left=1.5mm, right=1.5mm, top=1.5mm, bottom=1.5mm]
    \begin{lemma}\label{necessity.disjunction.undistribution}
        $\Gamma\entails_\mathfrak{L}\nec\alpha\vee\nec\beta\to\nec(\alpha\vee\beta)$.
        \begin{proof}
        Pode ser demonstrado pela dedução que segue.

        \vspace{0.5\baselineskip}
        \footnotesize
        \setlength{\rowskip}{0.5\baselineskip}
        \begin{xltabular}{\textwidth}{r | X l l}
            \scriptsize{\phantom{0}1}\phantom{ } & $\ \set{\nec\alpha}\vdash\nec\alpha$                                                                                                   & $\hyperref[modal.rule.1]{\mathbf{R_1}}$        & \phantom{$\set{00,00}$}\\[\rowskip]
            \scriptsize{\phantom{0}2}\phantom{ } & $\ \set{\nec\alpha}\vdash\nec\alpha\to\alpha$                                                                                          & $\hyperref[modal.axiom.modal.2]{\mathbf{B_2}}$ & \\[\rowskip]
            \scriptsize{\phantom{0}3}\phantom{ } & $\ \set{\nec\alpha}\vdash\alpha$                                                                                                       & $\hyperref[modal.rule.2]{\mathbf{R_2}}$        & $\set{1,2}$\\[\rowskip]
            \scriptsize{\phantom{0}4}\phantom{ } & $\ \set{\nec\alpha}\vdash\alpha\to\alpha\vee\beta$                                                                                     & $\hyperref[modal.axiom.6]{\mathbf{A_6}}$       & \\[\rowskip]
            \scriptsize{\phantom{0}5}\phantom{ } & $\ \set{\nec\alpha}\vdash\alpha\vee\beta$                                                                                              & $\hyperref[modal.rule.2]{\mathbf{R_2}}$        & $\set{3,4}$\\[\rowskip]
            \scriptsize{\phantom{0}6}\phantom{ } & $\ \set{\nec\alpha}\vdash\nec(\alpha\vee\beta)$                                                                                        & \refer{generalization}{T}                      & $\set{5}$\\[\rowskip]
            \scriptsize{\phantom{0}7}\phantom{ } & $\ \set{\nec\beta}\vdash\nec\beta$                                                                                                     & $\hyperref[modal.rule.1]{\mathbf{R_1}}$        & \\[\rowskip]
            \scriptsize{\phantom{0}8}\phantom{ } & $\ \set{\nec\beta}\vdash\nec\beta \to \beta$                                                                                           & $\hyperref[modal.axiom.modal.2]{\mathbf{B_2}}$ & \\[\rowskip]
            \scriptsize{\phantom{0}9}\phantom{ } & $\ \set{\nec\beta}\vdash\beta$                                                                                                         & $\hyperref[modal.rule.2]{\mathbf{R_2}}$        & $\set{7,8}$\\[\rowskip]
            \scriptsize{10}\phantom{ }           & $\ \set{\nec\beta}\vdash\beta\to\alpha\vee\beta$                                                                                       & $\hyperref[modal.axiom.7]{\mathbf{A_7}}$       & \\[\rowskip]
            \scriptsize{11}\phantom{ }           & $\ \set{\nec\beta}\vdash\alpha\vee\beta$                                                                                               & $\hyperref[modal.rule.2]{\mathbf{R_2}}$        & $\set{9,10}$\\[\rowskip]
            \scriptsize{12}\phantom{ }           & $\ \set{\nec\beta}\vdash\nec(\alpha\vee\beta)$                                                                                         & \refer{generalization}{T}                      & $\set{11}$\\[\rowskip]
            \scriptsize{13}\phantom{ }           & $\ \vdash\nec\alpha\to\nec(\alpha\vee\beta)$                                                                                           & \refer{deduction}{T}                           & $\set{6}$\\[\rowskip]
            \scriptsize{14}\phantom{ }           & $\ \vdash\nec\beta\to\nec(\alpha\vee\beta)$                                                                                            & \refer{deduction}{T}                           & $\set{12}$\\[\rowskip]
            \scriptsize{15}\phantom{ }           & $\ \vdash(\nec\alpha\to\nec(\alpha\vee\beta))\to(\nec\beta\to\nec(\alpha\vee\beta))\to\nec\alpha\vee\nec\beta\to\nec(\alpha\vee\beta)$ & $\hyperref[modal.axiom.8]{\mathbf{A_8}}$       & \\[\rowskip]
            \scriptsize{16}\phantom{ }           & $\ \vdash(\nec\beta\to\nec(\alpha\vee\beta))\to\nec\alpha\vee\nec\beta\to\nec(\alpha\vee\beta)$                                        & $\hyperref[modal.rule.2]{\mathbf{R_2}}$        & $\set{13,15}$\\[\rowskip]
            \scriptsize{17}\phantom{ }           & $\ \vdash\nec\alpha\vee\nec\beta\to\nec(\alpha\vee\beta)$                                                                              & $\hyperref[modal.rule.2]{\mathbf{R_2}}$        & $\set{14,16}$\\[\rowskip]
            \scriptsize{18}\phantom{ }           & $\ \Gamma\vdash\nec\alpha\vee\nec\beta\to\nec(\alpha\vee\beta)$                                                                        & \refer{weakening}{T}\phantom{1}                & $\set{17}$
        \end{xltabular}
        \normalsize

        \vspace{0.5\baselineskip}
        Estando assim demonstrada a proposição.
        \end{proof}
    \end{lemma}
\end{tcolorbox}

\vspace{.5\baselineskip}
Demonstrados os lemas anteriores, pode-se demonstrar o teorema da interderivabilidade como uma bi-implicação dentro do sistema de destino.
Para tanto, induziremos sobre a profundidade da sentença intuicionista traduzida, conforme~\cite{Troelstra}.
A bi-implicação neste teorema o torna a demonstração com as maiores sucessões de dedução neste trabalho, em virtude da necessidade do uso abundante dos axiomas da introdução e eliminação da conjunção.

\vspace{\baselineskip}
\begin{tcolorbox}[enhanced jigsaw, breakable, sharp corners, colframe=black, colback=white, boxrule=0.5pt, left=1.5mm, right=1.5mm, top=1.5mm, bottom=1.5mm]
\begin{theorem}\label{biimplication}
    $\Gamma\entails_\mathfrak{L}\nec\alpha^\circ\leftrightarrow\alpha^\medsquare$.

    \begin{proof}
        Prova por indução forte sobre a profundidade da sentença \citep{Troelstra}.
        Seja $n\in\mathbb{N}^+$ a profundidade da sentença $\alpha\in\mathcal{L}$.
        Suponhamos que a asserção valha para qualquer sentença de profundidade menor que $n$ e nomeemos esta suposição $\mathbf{H}$.
        Devemos considerar cinco casos: a letra, a contradição, a conjunção, a disjunção e a implicação.

        \vspace{.5\baselineskip}
        \textsc{Caso 1.}
        Se a sentença $\alpha$ for uma proposição $a\in\mathcal{P}$, sabe-se que $\nec a^\circ=\nec a$ e que $a^\medsquare=\nec a$ pelas definições das traduções.
        Deste modo, tanto a ida quanto a volta da bi-implicação possuem a forma $\nec a\to\nec a$ e podem ser provadas pelo lema \hyperref[identity]{$\mathbf{L_\getrefnumber{identity}}$}.
        Ambas as implicações posteriormente podem ser unidas em uma bi-implicação por meio das regras \hyperref[MA3]{$\mathbf{A_3}$} e \hyperref[detachment]{$\mathbf{R_2}$}.

        \vspace{.5\baselineskip}
        \textsc{Caso 2.}
        Se a sentença $\alpha$ for a constante $\bot$, sabe-se que $\nec\bot^\circ=\nec\bot$ e que $\bot^\medsquare=\bot$ pelas definições das traduções.
        Deste modo, a ida $\nec\bot\to\bot$ da bi-implicação constitui um axioma gerado pela regra \hyperref[MB2]{$\mathbf{B_2}$}.
        A volta $\bot\to\nec\bot$ da bi-implicação pode ser provada pelo lema \hyperref[explosion]{$\mathbf{L_3}$}.
        Ambas as implicações posteriormente podem ser unidas em uma bi-implicação por meio das regras \hyperref[MA3]{$\mathbf{A_3}$} e \hyperref[detachment]{$\mathbf{R_2}$}.

        \vspace{.5\baselineskip}
        \textsc{Caso 3.}
        Seja a sentença $\alpha$ a conjunção de duas sentenças ${\alpha_1}$ e ${\alpha_2}$.
        Sabe-se que $\nec{({\alpha_1}\wedge{\alpha_2})}^\circ=\nec({\alpha_1}^\circ\wedge{\alpha_2}^\circ)$ e que ${({\alpha_1}\wedge{\alpha_2})}^\medsquare={\alpha_1}^\medsquare\wedge{\alpha_2}^\medsquare$ pelas definições das traduções.
        A partir de $\mathbf{H}$, temos que $\Gamma\entails\nec{\alpha_1}^\circ\leftrightarrow{\alpha_1}^\medsquare$ e que $\Gamma\entails\nec{\alpha_2}^\circ\leftrightarrow{\alpha_2}^\medsquare$, ditos $\mathbf{H_1}$ e $\mathbf{H_2}$.
        Pode-se demonstrar $\Gamma\entails\nec\alpha^\circ\leftrightarrow\alpha^\medsquare$ por meio da dedução que segue.

        \vspace{.5\baselineskip}
        \footnotesize
        \setlength{\rowskip}{.5\baselineskip}
        \begin{xltabular}{\textwidth}{r | X l l}
            \scriptsize{\phantom{0}1}\phantom{ } & $\ \Gamma\vdash\nec{\alpha_1}^\circ\leftrightarrow{\alpha_1}^\medsquare$                                                                                                                                                      & $\mathbf{H_1}$\phantom{1}                & \phantom{$\set{00,00}$}\\[\rowskip]
            \scriptsize{\phantom{0}2}\phantom{ } & $\ \Gamma\vdash\nec{\alpha_2}^\circ\leftrightarrow{\alpha_2}^\medsquare$                                                                                                                                                      & $\mathbf{H_2}$                           & \\[\rowskip]
            \scriptsize{\phantom{0}3}\phantom{ } & $\ \Gamma\vdash(\nec{\alpha_1}^\circ\leftrightarrow{\alpha_1}^\medsquare)\to\nec{\alpha_1}^\circ\to{\alpha_1}^\medsquare$                                                                                                     & $\hyperref[modal.axiom.4]{\mathbf{A_4}}$ & \\[\rowskip]
            \scriptsize{\phantom{0}4}\phantom{ } & $\ \Gamma\vdash(\nec{\alpha_2}^\circ\leftrightarrow{\alpha_2}^\medsquare)\to\nec{\alpha_2}^\circ\to{\alpha_2}^\medsquare$                                                                                                     & $\hyperref[modal.axiom.4]{\mathbf{A_4}}$ & \\[\rowskip]
            \scriptsize{\phantom{0}5}\phantom{ } & $\ \Gamma\vdash(\nec{\alpha_1}^\circ\leftrightarrow{\alpha_1}^\medsquare)\to{\alpha_1}^\medsquare \to \nec{\alpha_1}^\circ$                                                                                                   & $\hyperref[modal.axiom.5]{\mathbf{A_5}}$ & \\[\rowskip]
            \scriptsize{\phantom{0}6}\phantom{ } & $\ \Gamma\vdash(\nec{\alpha_2}^\circ\leftrightarrow{\alpha_2}^\medsquare)\to{\alpha_2}^\medsquare \to \nec{\alpha_2}^\circ$                                                                                                   & $\hyperref[modal.axiom.5]{\mathbf{A_5}}$ & \\[\rowskip]
            \scriptsize{\phantom{0}7}\phantom{ } & $\ \Gamma\vdash\nec{\alpha_1}^\circ\to{\alpha_1}^\medsquare$                                                                                                                                                                  & $\hyperref[modal.rule.2]{\mathbf{R_2}}$  & $\set{1,3}$\\[\rowskip]
            \scriptsize{\phantom{0}8}\phantom{ } & $\ \Gamma\vdash\nec{\alpha_2}^\circ\to{\alpha_2}^\medsquare$                                                                                                                                                                  & $\hyperref[modal.rule.2]{\mathbf{R_2}}$  & $\set{2,4}$\\[\rowskip]
            \scriptsize{\phantom{0}9}\phantom{ } & $\ \Gamma\vdash{\alpha_1}^\medsquare\to\nec{\alpha_1}^\circ$                                                                                                                                                                  & $\hyperref[modal.rule.2]{\mathbf{R_2}}$  & $\set{1,5}$\\[\rowskip]
            \scriptsize{10}\phantom{ }           & $\ \Gamma\vdash{\alpha_2}^\medsquare\to\nec{\alpha_2}^\circ$                                                                                                                                                                  & $\hyperref[modal.rule.2]{\mathbf{R_2}}$  & $\set{2,6}$\\[\rowskip]
            \scriptsize{11}\phantom{ }           & $\ \Gamma\cup\set{\nec({\alpha_1}^\circ\wedge{\alpha_2}^\circ)} \vdash \nec{\alpha_1}^\circ \to {\alpha_1}^\medsquare$                                                                                                        & \refer{weakening}{T}                     & $\set{7}$\\[\rowskip]
            \scriptsize{12}\phantom{ }           & $\ \Gamma\cup\set{\nec({\alpha_1}^\circ\wedge{\alpha_2}^\circ)} \vdash \nec{\alpha_2}^\circ \to {\alpha_2}^\medsquare$                                                                                                        & \refer{weakening}{T}                     & $\set{8}$\\[\rowskip]
            \scriptsize{13}\phantom{ }           & $\ \Gamma\cup\set{\nec({\alpha_1}^\circ\wedge{\alpha_2}^\circ)} \vdash \nec({\alpha_1}^\circ \wedge {\alpha_2}^\circ)$                                                                                                        & $\hyperref[modal.rule.1]{\mathbf{R_1}}$  & \\[\rowskip]
            \scriptsize{14}\phantom{ }           & $\ \Gamma\cup\set{\nec({\alpha_1}^\circ\wedge{\alpha_2}^\circ)} \vdash \nec({\alpha_1}^\circ \wedge {\alpha_2}^\circ) \to \nec{\alpha_1}^\circ \wedge \nec{\alpha_2}^\circ$                                                   & \refer{necessity.conjunction.distribution}{L} & \\[\rowskip]
            \scriptsize{15}\phantom{ }           & $\ \Gamma\cup\set{\nec({\alpha_1}^\circ\wedge{\alpha_2}^\circ)} \vdash \nec{\alpha_1}^\circ \wedge \nec{\alpha_2}^\circ$                                                                                                      & $\hyperref[modal.rule.2]{\mathbf{R_2}}$ & $\set{13,14}$\\[\rowskip]
            \scriptsize{16}\phantom{ }           & $\ \Gamma\cup\set{\nec({\alpha_1}^\circ\wedge{\alpha_2}^\circ)} \vdash \nec{\alpha_1}^\circ \wedge \nec{\alpha_2}^\circ \to {\alpha_1}^\medsquare \wedge {\alpha_2}^\medsquare$                                               & \refer{conjunction.exchange}{L} & $\set{11,12}$\\[\rowskip]
            \scriptsize{17}\phantom{ }           & $\ \Gamma\cup\set{\nec({\alpha_1}^\circ\wedge{\alpha_2}^\circ)} \vdash {\alpha_1}^\medsquare \wedge {\alpha_2}^\medsquare$                                                                                                    & $\hyperref[modal.rule.2]{\mathbf{R_2}}$ & $\set{15,16}$\\[\rowskip]
            \scriptsize{18}\phantom{ }           & $\ \Gamma\cup\set{{\alpha_1}^\medsquare\wedge{\alpha_2}^\medsquare} \vdash {\alpha_1}^\medsquare \to \nec{\alpha_1}^\circ$                                                                                                    & \refer{weakening}{T} & $\set{9}$\\[\rowskip]
            \scriptsize{19}\phantom{ }           & $\ \Gamma\cup\set{{\alpha_1}^\medsquare\wedge{\alpha_2}^\medsquare} \vdash {\alpha_2}^\medsquare \to \nec{\alpha_2}^\circ$                                                                                                    & \refer{weakening}{T} & $\set{10}$\\[\rowskip]
            \scriptsize{20}\phantom{ }           & $\ \Gamma\cup\set{{\alpha_1}^\medsquare\wedge{\alpha_2}^\medsquare} \vdash {\alpha_1}^\medsquare \wedge {\alpha_2}^\medsquare$                                                                                                & $\hyperref[modal.rule.1]{\mathbf{R_1}}$        & \\[\rowskip]
            \scriptsize{21}\phantom{ }           & $\ \Gamma\cup\set{{\alpha_1}^\medsquare\wedge{\alpha_2}^\medsquare} \vdash {\alpha_1}^\medsquare \wedge {\alpha_2}^\medsquare \to \nec{\alpha_1}^\circ \wedge \nec{\alpha_2}^\circ$                                           & \refer{conjunction.exchange}{L} & $\set{18,19}$\\[\rowskip]
            \scriptsize{22}\phantom{ }           & $\ \Gamma\cup\set{{\alpha_1}^\medsquare\wedge{\alpha_2}^\medsquare} \vdash \nec{\alpha_1}^\circ \wedge \nec{\alpha_2}^\circ$                                                                                                  & $\hyperref[modal.rule.2]{\mathbf{R_2}}$ & $\set{20,21}$\\[\rowskip]
            \scriptsize{23}\phantom{ }           & $\ \Gamma\cup\set{{\alpha_1}^\medsquare\wedge{\alpha_2}^\medsquare} \vdash \nec{\alpha_1}^\circ \wedge \nec{\alpha_2}^\circ \to \nec({\alpha_1}^\circ \wedge {\alpha_2}^\circ)$                                               & \refer{necessity.conjunction.undistribution}{L} & \\[\rowskip]
            \scriptsize{24}\phantom{ }           & $\ \Gamma\cup\set{{\alpha_1}^\medsquare\wedge{\alpha_2}^\medsquare} \vdash \nec({\alpha_1}^\circ \wedge {\alpha_2}^\circ)$                                                                                                    & $\hyperref[modal.rule.2]{\mathbf{R_2}}$ & $\set{22,23}$\\[\rowskip]
            \scriptsize{25}\phantom{ }           & $\ \Gamma\vdash\nec({\alpha_1}^\circ \wedge {\alpha_2}^\circ) \to {\alpha_1}^\medsquare \wedge {\alpha_2}^\medsquare$                                                                                                         & \refer{deduction}{T} & $\set{17}$\\[\rowskip]
            \scriptsize{26}\phantom{ }           & $\ \Gamma\vdash{\alpha_1}^\medsquare \wedge {\alpha_2}^\medsquare \to \nec({\alpha_1}^\circ \wedge {\alpha_2}^\circ)$                                                                                                         & \refer{deduction}{T} & $\set{24}$\\[\rowskip]
            \scriptsize{27}\phantom{ }           & $\ \Gamma\vdash(\nec\alpha^\circ \to \alpha^\medsquare) \to (\alpha^\medsquare \to \nec\alpha^\circ) \to (\nec({\alpha_1}^\circ \wedge {\alpha_2}^\circ) \leftrightarrow {\alpha_1}^\medsquare \wedge {\alpha_2}^\medsquare)$ & $\hyperref[modal.axiom.4]{\mathbf{A_4}}$ & \\[\rowskip]
            \scriptsize{28}\phantom{ }           & $\ \Gamma\vdash(\alpha^\medsquare \to \nec\alpha^\circ) \to (\nec({\alpha_1}^\circ \wedge {\alpha_2}^\circ) \leftrightarrow {\alpha_1}^\medsquare \wedge {\alpha_2}^\medsquare)$                                              & $\hyperref[modal.rule.2]{\mathbf{R_2}}$ & $\set{25,27}$\\[\rowskip]
            \scriptsize{29}\phantom{ }           & $\ \Gamma\vdash\nec({\alpha_1}^\circ \wedge {\alpha_2}^\circ) \leftrightarrow {\alpha_1}^\medsquare \wedge {\alpha_2}^\medsquare$                                                                                             & $\hyperref[modal.rule.2]{\mathbf{R_2}}$ & $\set{26,28}$
        \end{xltabular}
        \normalsize

        \vspace{.5\baselineskip}
        \textsc{Caso 4.}
        Seja a sentença $\alpha$ a disjunção de duas sentenças ${\alpha_1}$ e ${\alpha_2}$.
        Sabe-se que $\nec{({\alpha_1}\vee\psi)}^\circ=\nec(\nec{\alpha_1}^\circ\vee\nec\psi^\circ)$ e que ${({\alpha_1}\vee\psi)}^\medsquare={\alpha_1}^\medsquare\vee\psi^\medsquare$ pelas definições das traduções.
        A partir de $\mathbf{H}$, temos que $\Gamma\entails\nec{\alpha_1}^\circ\leftrightarrow{\alpha_1}^\medsquare$ e que $\Gamma\entails\nec{\alpha_2}^\circ\leftrightarrow{\alpha_2}^\medsquare$, ditos $\mathbf{H_1}$ e $\mathbf{H_2}$.
        Pode-se demonstrar $\Gamma\entails\nec\alpha^\circ\leftrightarrow\alpha^\medsquare$ por meio da dedução que segue.

        \vspace{.5\baselineskip}
        \footnotesize
        \setlength{\rowskip}{.5\baselineskip}
        \begin{xltabular}{\textwidth}{r | X l l}
            \scriptsize{\phantom{0}1}\phantom{ } & $\ \Gamma \vdash \nec{\alpha_1}^\circ \leftrightarrow {\alpha_1}^\medsquare$ & $\mathbf{H_1}$ & \\[\rowskip]
            \scriptsize{\phantom{0}2}\phantom{ } & $\ \Gamma \vdash \nec{\alpha_2}^\circ \leftrightarrow {\alpha_2}^\medsquare$ & $\mathbf{H_2}$ & \\[\rowskip]
            \scriptsize{\phantom{0}3}\phantom{ } & $\ \Gamma \vdash (\nec{\alpha_1}^\circ \leftrightarrow {\alpha_1}^\medsquare) \to \nec{\alpha_1}^\circ \to {\alpha_1}^\medsquare$ & $\hyperref[modal.axiom.4]{\mathbf{A_4}}$ & \\[\rowskip]
            \scriptsize{\phantom{0}4}\phantom{ } & $\ \Gamma \vdash (\nec{\alpha_2}^\circ \leftrightarrow {\alpha_2}^\medsquare) \to \nec{\alpha_2}^\circ \to {\alpha_2}^\medsquare$ & $\hyperref[modal.axiom.4]{\mathbf{A_4}}$ & \\[\rowskip]
            \scriptsize{\phantom{0}5}\phantom{ } & $\ \Gamma \vdash (\nec{\alpha_1}^\circ \leftrightarrow {\alpha_1}^\medsquare) \to {\alpha_1}^\medsquare \to \nec{\alpha_1}^\circ$ & $\hyperref[modal.axiom.5]{\mathbf{A_5}}$ & \\[\rowskip]
            \scriptsize{\phantom{0}6}\phantom{ } & $\ \Gamma \vdash (\nec{\alpha_2}^\circ \leftrightarrow {\alpha_2}^\medsquare) \to {\alpha_2}^\medsquare \to \nec{\alpha_2}^\circ$ & $\hyperref[modal.axiom.5]{\mathbf{A_5}}$ & \\[\rowskip]
            \scriptsize{\phantom{0}7}\phantom{ } & $\ \Gamma \vdash \nec{\alpha_1}^\circ \to {\alpha_1}^\medsquare$ & $\hyperref[modal.rule.2]{\mathbf{R_2}}$ & $\set{1,3}$\\[\rowskip]
            \scriptsize{\phantom{0}8}\phantom{ } & $\ \Gamma \vdash \nec{\alpha_2}^\circ \to {\alpha_2}^\medsquare$ & $\hyperref[modal.rule.2]{\mathbf{R_2}}$ & $\set{2,4}$\\[\rowskip]
            \scriptsize{\phantom{0}9}\phantom{ } & $\ \Gamma \vdash {\alpha_1}^\medsquare \to \nec{\alpha_1}^\circ$ & $\hyperref[modal.rule.2]{\mathbf{R_2}}$ & $\set{1,5}$\\[\rowskip]
            \scriptsize{10}\phantom{ } & $\ \Gamma \vdash {\alpha_2}^\medsquare \to \nec{\alpha_2}^\circ$ & $\hyperref[modal.rule.2]{\mathbf{R_2}}$ & $\set{2,6}$\\[\rowskip]
            \scriptsize{11}\phantom{ } & $\ \Gamma \cup \set{\nec(\nec{\alpha_1}^\circ \vee \nec{\alpha_2}^\circ)} \vdash \nec{\alpha_1}^\circ \to {\alpha_1}^\medsquare$ & \refer{weakening}{T} & $\set{7}$\\[\rowskip]
            \scriptsize{12}\phantom{ } & $\ \Gamma \cup \set{\nec(\nec{\alpha_1}^\circ \vee \nec{\alpha_2}^\circ)} \vdash \nec{\alpha_2}^\circ \to {\alpha_2}^\medsquare$ & \refer{weakening}{T} & $\set{8}$\\[\rowskip]
            \scriptsize{13}\phantom{ } & $\ \Gamma \cup \set{\nec(\nec{\alpha_1}^\circ \vee \nec{\alpha_2}^\circ)} \vdash \nec(\nec{\alpha_1}^\circ \vee \nec{\alpha_2}^\circ)$ & $\hyperref[modal.rule.1]{\mathbf{R_1}}$ & \\[\rowskip]
            \scriptsize{14}\phantom{ } & $\ \Gamma \cup \set{\nec(\nec{\alpha_1}^\circ \vee \nec{\alpha_2}^\circ)} \vdash \nec(\nec{\alpha_1}^\circ \vee \nec{\alpha_2}^\circ) \to \nec{\alpha_1}^\circ \vee \nec{\alpha_2}^\circ$ & $\hyperref[modal.axiom.modal.2]{\mathbf{B_2}}$ & \\[\rowskip]
            \scriptsize{15}\phantom{ } & $\ \Gamma \cup \set{\nec(\nec{\alpha_1}^\circ \vee \nec{\alpha_2}^\circ)} \vdash \nec{\alpha_1}^\circ \vee \nec{\alpha_2}^\circ$ & $\hyperref[modal.rule.2]{\mathbf{R_2}}$ & $\set{13,14}$\\[\rowskip]
            \scriptsize{16}\phantom{ } & $\ \Gamma \cup \set{\nec(\nec{\alpha_1}^\circ \vee \nec{\alpha_2}^\circ)} \vdash \nec{\alpha_1}^\circ \vee \nec{\alpha_2}^\circ \to {\alpha_1}^\medsquare \vee {\alpha_2}^\medsquare$ & \refer{disjunction.exchange}{L} & $\set{11,12}$\\[\rowskip]
            \scriptsize{17}\phantom{ } & $\ \Gamma \cup \set{\nec(\nec{\alpha_1}^\circ \vee \nec{\alpha_2}^\circ)} \vdash {\alpha_1}^\medsquare \vee {\alpha_2}^\medsquare$ & $\hyperref[modal.rule.2]{\mathbf{R_2}}$ & $\set{15,16}$\\[\rowskip]
            \scriptsize{19}\phantom{ } & $\ \Gamma \cup \set{{\alpha_1}^\medsquare \vee {\alpha_2}^\medsquare} \vdash {\alpha_1}^\medsquare \to \nec{\alpha_1}^\circ$ & \refer{weakening}{T} & $\set{9}$\\[\rowskip]
            \scriptsize{20}\phantom{ } & $\ \Gamma \cup \set{{\alpha_1}^\medsquare \vee {\alpha_2}^\medsquare} \vdash {\alpha_2}^\medsquare \to \nec{\alpha_2}^\circ$ & \refer{weakening}{T} & $\set{10}$\\[\rowskip]
            \scriptsize{21}\phantom{ } & $\ \Gamma \cup \set{{\alpha_1}^\medsquare \vee {\alpha_2}^\medsquare} \vdash \nec{\alpha_1}^\circ \to \nec\nec{\alpha_1}^\circ$ & $\hyperref[modal.axiom.modal.3]{\mathbf{B_3}}$ & \\[\rowskip]
            \scriptsize{22}\phantom{ } & $\ \Gamma \cup \set{{\alpha_1}^\medsquare \vee {\alpha_2}^\medsquare} \vdash \nec{\alpha_2}^\circ \to \nec\nec{\alpha_2}^\circ$ & $\hyperref[modal.axiom.modal.3]{\mathbf{B_3}}$ & \\[\rowskip]
            \scriptsize{23}\phantom{ } & $\ \Gamma \cup \set{{\alpha_1}^\medsquare \vee {\alpha_2}^\medsquare} \vdash {\alpha_1}^\medsquare \to \nec\nec{\alpha_1}^\circ$ & \refer{composition}{L} & $\set{18,20}$\\[\rowskip]
            \scriptsize{24}\phantom{ } & $\ \Gamma \cup \set{{\alpha_1}^\medsquare \vee {\alpha_2}^\medsquare} \vdash {\alpha_2}^\medsquare \to \nec\nec{\alpha_2}^\circ$ & \refer{composition}{L} & $\set{19,21}$\\[\rowskip]
            \scriptsize{25}\phantom{ } & $\ \Gamma \cup \set{{\alpha_1}^\medsquare \vee {\alpha_2}^\medsquare} \vdash {\alpha_1}^\medsquare \vee {\alpha_2}^\medsquare$ & $\hyperref[modal.rule.1]{\mathbf{R_1}}$ & \\[\rowskip]
            \scriptsize{26}\phantom{ } & $\ \Gamma \cup \set{{\alpha_1}^\medsquare \vee {\alpha_2}^\medsquare} \vdash {\alpha_1}^\medsquare \vee {\alpha_2}^\medsquare \to \nec\nec{\alpha_1}^\circ \vee \nec\nec{\alpha_2}^\circ$ & \refer{disjunction.exchange}{L} & $\set{22,23}$\\[\rowskip]
            \scriptsize{27}\phantom{ } & $\ \Gamma \cup \set{{\alpha_1}^\medsquare \vee {\alpha_2}^\medsquare} \vdash \nec\nec{\alpha_1}^\circ \vee \nec\nec{\alpha_2}^\circ$ & $\hyperref[modal.rule.2]{\mathbf{R_2}}$ & $\set{24,25}$\\[\rowskip]
            \scriptsize{28}\phantom{ } & $\ \Gamma \cup \set{{\alpha_1}^\medsquare \vee {\alpha_2}^\medsquare} \vdash \nec\nec{\alpha_1}^\circ \vee \nec\nec{\alpha_2}^\circ \to \nec(\nec{\alpha_1}^\circ \vee \nec{\alpha_2}^\circ)$ & \refer{necessity.disjunction.undistribution}{L} & \\[\rowskip]
            \scriptsize{29}\phantom{ } & $\ \Gamma \cup \set{{\alpha_1}^\medsquare \vee {\alpha_2}^\medsquare} \vdash \nec(\nec{\alpha_1}^\circ \vee \nec{\alpha_2}^\circ)$ & $\hyperref[modal.rule.2]{\mathbf{R_2}}$ & $\set{26,27}$\\[\rowskip]
            \scriptsize{30}\phantom{ } & $\ \Gamma \vdash \nec(\nec{\alpha_1}^\circ \vee \nec{\alpha_2}^\circ) \to {\alpha_1}^\medsquare \vee {\alpha_2}^\medsquare$ & \refer{deduction}{T} & $\set{17}$\\[\rowskip]
            \scriptsize{31}\phantom{ } & $\ \Gamma \vdash {\alpha_1}^\medsquare \vee {\alpha_2}^\medsquare \to \nec(\nec{\alpha_1}^\circ \vee \nec{\alpha_2}^\circ)$ & \refer{deduction}{T} & $\set{28}$\\[\rowskip]
            \scriptsize{32}\phantom{ } & $\ \Gamma \vdash (\nec\alpha^\circ \to \alpha^\medsquare) \to (\alpha^\medsquare \to \nec\alpha^\circ) \to (\nec(\nec{\alpha_1}^\circ \vee \nec{\alpha_2}^\circ) \leftrightarrow {\alpha_1}^\medsquare \vee {\alpha_2}^\medsquare)$ & $\hyperref[modal.axiom.4]{\mathbf{A_4}}$ & \\[\rowskip]
            \scriptsize{33}\phantom{ } & $\ \Gamma \vdash (\alpha^\medsquare \to \nec\alpha^\circ) \to (\nec(\nec{\alpha_1}^\circ \vee \nec{\alpha_2}^\circ) \leftrightarrow {\alpha_1}^\medsquare \vee {\alpha_2}^\medsquare)$ & $\hyperref[modal.rule.2]{\mathbf{R_2}}$ & $\set{29,31}$\\[\rowskip]
            \scriptsize{34}\phantom{ } & $\ \Gamma \vdash \nec(\nec{\alpha_1}^\circ \vee \nec{\alpha_2}^\circ) \leftrightarrow {\alpha_1}^\medsquare \vee {\alpha_2}^\medsquare$ & $\hyperref[modal.rule.2]{\mathbf{R_2}}$ & $\set{30,32}$\
        \end{xltabular}
        \normalsize

        % \vspace{.5\baselineskip}
        %     \textsc{Caso 5.}
        %     Seja a sentença $\alpha$ a implicação de duas sentenças ${\alpha_1}$ e ${\alpha_2}$.
        %     Sabe-se que $\nec{({\alpha_1}\to{\alpha_2})}^\circ=\nec(\nec{\alpha_1}^\circ\to{\alpha_2}^\circ)$ e que ${({\alpha_1}\to{\alpha_2})}^\medsquare=\nec({\alpha_1}^\medsquare\to{\alpha_2}^\medsquare)$ pelas definições das traduções.
        %     A partir de $\mathbf{H}$, temos que $\Gamma\entails\nec{\alpha_1}^\circ\leftrightarrow{\alpha_1}^\medsquare$ e que $\Gamma\entails\nec{\alpha_2}^\circ\leftrightarrow{\alpha_2}^\medsquare$, ditos $\mathbf{H_1}$ e $\mathbf{H_2}$.
        %     Pode-se demonstrar $\Gamma\entails\nec\alpha^\circ\leftrightarrow\alpha^\medsquare$ por meio da dedução que segue.

        %         \footnotesize
        %         \begin{fitch}
        %             \fb\set{\nec(\nec{\alpha_1}^\circ\to{\alpha_2}^\circ),{\alpha_1}^\medsquare}\entails{\alpha_1}^\medsquare&$\hyperref[premisse]{\mathbf{R_1}}$\\
        %             \fa\set{\nec(\nec{\alpha_1}^\circ\to{\alpha_2}^\circ),{\alpha_1}^\medsquare}\entails{\alpha_1}^\medsquare\to\nec{\alpha_1}^\circ&$\mathbf{H_1}$\\
        %             \fa\set{\nec(\nec{\alpha_1}^\circ\to{\alpha_2}^\circ),\nec{\alpha_1}^\circ}\entails\nec{\alpha_1}^\circ&$\hyperref[premisse]{\mathbf{R_1}}$\\
        %             \fa\set{\nec(\nec{\alpha_1}^\circ\to{\alpha_2}^\circ),\nec{\alpha_1}^\circ}\entails\nec(\nec{\alpha_1}^\circ\to{\alpha_2}^\circ)&$\hyperref[premisse]{\mathbf{R_1}}$\\
        %             \fa\set{\nec(\nec{\alpha_1}^\circ\to{\alpha_2}^\circ),\nec{\alpha_1}^\circ}\entails\nec(\nec{\alpha_1}^\circ\to{\alpha_2}^\circ)\to\nec{\alpha_1}^\circ\to{\alpha_2}^\circ&\hyperref[MB2]{${\mathbf{B_2}}$}\\
        %             \fa\set{\nec(\nec{\alpha_1}^\circ\to{\alpha_2}^\circ),\nec{\alpha_1}^\circ}\entails\nec{\alpha_1}^\circ\to{\alpha_2}^\circ&$\hyperref[detachment]{\mathbf{R_2}}\;\set{4,5}$\\
        %             \fa\set{\nec(\nec{\alpha_1}^\circ\to{\alpha_2}^\circ),\nec{\alpha_1}^\circ}\entails{\alpha_2}^\circ&$\hyperref[detachment]{\mathbf{R_2}}\;\set{3,6}$\\
        %             \fa\set{\nec(\nec{\alpha_1}^\circ\to{\alpha_2}^\circ),\nec{\alpha_1}^\circ}\entails\nec{\alpha_2}^\circ&$\hyperref[gen-nec]{\mathbf{T_\getrefnumber{gen-nec}}}\;\set{7}$\\
        %             \fa\set{\nec(\nec{\alpha_1}^\circ\to{\alpha_2}^\circ)}\entails\nec{\alpha_1}^\circ\to\nec{\alpha_2}^\circ&$\hyperref[deduction]{\mathbf{T_\getrefnumber{deduction}}}\;\set{8}$\\
        %             \fa\set{\nec(\nec{\alpha_1}^\circ\to{\alpha_2}^\circ),{\alpha_1}^\medsquare}\entails\nec{\alpha_1}^\circ\to\nec{\alpha_2}^\circ&$\hyperref[weakening]{\mathbf{T_\getrefnumber{weakening}}}\;\set{9}$\\
        %             \fa\set{\nec(\nec{\alpha_1}^\circ\to{\alpha_2}^\circ),{\alpha_1}^\medsquare}\entails{\alpha_1}^\medsquare\to\nec{\alpha_2}^\circ&\refer{comp}{L}$\;\set{2,10}$\\
        %             \fa\set{\nec(\nec{\alpha_1}^\circ\to{\alpha_2}^\circ),{\alpha_1}^\medsquare}\entails\nec{\alpha_2}^\circ\to{\alpha_2}^\medsquare&$\mathbf{H_2}$\\
        %             \fa\set{\nec(\nec{\alpha_1}^\circ\to{\alpha_2}^\circ),{\alpha_1}^\medsquare}\entails{\alpha_1}^\medsquare\to{\alpha_2}^\medsquare&\refer{comp}{L}$\;\set{11,12}$\\
        %             \fa\set{\nec(\nec{\alpha_1}^\circ\to{\alpha_2}^\circ),{\alpha_1}^\medsquare}\entails{\alpha_2}^\medsquare&$\hyperref[detachment]{\mathbf{R_2}}\;\set{1,13}$\\
        %             \fa\set{\nec(\nec{\alpha_1}^\circ\to{\alpha_2}^\circ)}\entails\nec({\alpha_1}^\medsquare\to{\alpha_2}^\medsquare)&$\hyperref[gen-nec]{\mathbf{T_{\getrefnumber{gen-nec}}}}\;\set{14}$\\
        %             \fa\entails\nec(\nec{\alpha_1}^\circ\to{\alpha_2}^\circ)\to\nec({\alpha_1}^\medsquare\to{\alpha_2}^\medsquare)&$\hyperref[deduction]{\mathbf{T_\getrefnumber{deduction}}}\;\set{15}$
        %         \end{fitch}

        %         \footnotesize
        %         \begin{fitch}
        %             \fb\set{\nec({\alpha_1}^\medsquare\to{\alpha_2}^\medsquare),\nec{\alpha_1}^\circ}\entails\nec{\alpha_1}^\circ&$\hyperref[premisse]{\mathbf{R_1}}$\\
        %             \fa\set{\nec({\alpha_1}^\medsquare\to{\alpha_2}^\medsquare),\nec{\alpha_1}^\circ}\entails\nec{\alpha_1}^\circ\to{\alpha_1}^\medsquare&$\mathbf{H_1}$\\
        %             \fa\set{\nec({\alpha_1}^\medsquare\to{\alpha_2}^\medsquare),\nec{\alpha_1}^\circ}\entails\nec({\alpha_1}^\medsquare\to{\alpha_2}^\medsquare)&$\hyperref[premisse]{\mathbf{R_1}}$\\
        %             \fa\set{\nec({\alpha_1}^\medsquare\to{\alpha_2}^\medsquare),\nec{\alpha_1}^\circ}\entails\nec({\alpha_1}^\medsquare\to{\alpha_2}^\medsquare)\to{\alpha_1}^\medsquare\to{\alpha_2}^\medsquare&\hyperref[MB2]{${\mathbf{B_2}}$}\\
        %             \fa\set{\nec({\alpha_1}^\medsquare\to{\alpha_2}^\medsquare),\nec{\alpha_1}^\circ}\entails{\alpha_1}^\medsquare\to{\alpha_2}^\medsquare&$\hyperref[detachment]{\mathbf{R_2}}\;\set{2,3}$\\
        %             \fa\set{\nec({\alpha_1}^\medsquare\to{\alpha_2}^\medsquare),\nec{\alpha_1}^\circ}\entails\nec{\alpha_1}^\circ\to{\alpha_2}^\medsquare&\refer{comp}{L}$\;\set{2,5}$\\
        %             \fa\set{\nec({\alpha_1}^\medsquare\to{\alpha_2}^\medsquare),\nec{\alpha_1}^\circ}\entails{\alpha_2}^\medsquare&$\hyperref[detachment]{\mathbf{R_2}}\;\set{1,6}$\\
        %             \fa\set{\nec({\alpha_1}^\medsquare\to{\alpha_2}^\medsquare),\nec{\alpha_1}^\circ}\entails{\alpha_2}^\medsquare\to\nec{\alpha_2}^\circ&$\mathbf{H_2}$\\
        %             \fa\set{\nec({\alpha_1}^\medsquare\to{\alpha_2}^\medsquare),\nec{\alpha_1}^\circ}\entails\nec{\alpha_2}^\circ&$\hyperref[detachment]{\mathbf{R_2}}\;\set{7,8}$\\
        %             \fa\set{\nec({\alpha_1}^\medsquare\to{\alpha_2}^\medsquare),\nec{\alpha_1}^\circ}\entails\nec{\alpha_2}^\circ\to{\alpha_2}^\circ&\hyperref[MB2]{${\mathbf{B_2}}$}\\
        %             \fa\set{\nec({\alpha_1}^\medsquare\to{\alpha_2}^\medsquare),\nec{\alpha_1}^\circ}\entails{\alpha_2}^\circ&$\hyperref[detachment]{\mathbf{R_2}}\;\set{9,10}$\\
        %             \fa\set{\nec({\alpha_1}^\medsquare\to{\alpha_2}^\medsquare)}\entails\nec(\nec{\alpha_1}^\circ\to{\alpha_2}^\circ)&$\hyperref[strictdeduction]{\mathbf{T_\getrefnumber{strictdeduction}}}\;\set{11}$\\
        %             \fa\entails\nec({\alpha_1}^\medsquare\to{\alpha_2}^\medsquare)\to\nec(\nec{\alpha_1}^\circ\to{\alpha_2}^\circ)&$\hyperref[deduction]{\mathbf{T_\getrefnumber{deduction}}}\;\set{12}$
        %         \end{fitch}

        % \vspace{.5\baselineskip}
        % Estando assim demonstrada a proposição.
    \end{proof}
\end{theorem}
\end{tcolorbox}

\begin{tcolorbox}[enhanced jigsaw, breakable, sharp corners, colframe=black, colback=white, boxrule=0.5pt, left=1.5mm, right=1.5mm, top=1.5mm, bottom=1.5mm]
\begin{theorem}[Interderivabilidade]\label{interderivability}
    $\nec\Gamma^\circ\entails_\mathfrak{L}\alpha^\circ$ se e somente se $\Gamma^\nec\entails_\mathfrak{L}\alpha^\nec$.
\end{theorem}
\end{tcolorbox}

        \section{Correção}

    Neste sessão, apresentaremos uma prova de correção para a $\nec$-tradução. Antes disso, entretanto, apresentaremos uma prova de $\entails\alpha^\medsquare\to\nec\alpha^\medsquare$, que usaremos como lema para a prova da correção conforme sugerido por~\cite{Troelstra}. Provaremos o lema por indução sobre a profundidade da sentença e a correção por indução sobre o tamanho da prova.

    \begin{lemma}\label{square-nec}
        $\entails\alpha^\medsquare\to\nec\alpha^\medsquare$.

        \begin{proof}
            Prova por indução forte sobre a profundidade de $\alpha\in\mathcal{L}_\mathbf{I}$ \citep{Troelstra}.
            Assim, suponhamos que a proposição valha para qualquer sentença $\alpha$ de profundidade $n<k$.
            Demonstraremos analisando-se os casos e valendo-se da suposição acima --- doravante chamada $\mathbf{H}$ --- o passo de indução, ou seja, que a proposição vale para qualquer $\alpha$ de profundidade $n=k$.

            \begin{case}
                \textsc{Caso 1.}
                Se a sentença $\alpha$ for uma proposição $a\in\mathcal{P}$, sabe-se que $a^\medsquare=\nec a$ pela definição da tradução.
                Deste modo, $\nec{a}\to\nec\nec{a}$ constitui um axioma gerado pelo esquema \hyperref[MB3]{$\mathbf{B_3}$} --- sendo assim $\entails\nec{a}\to\nec\nec{a}$ provado trivialmente.
            \end{case}

            \begin{case}
                \textsc{Caso 2.}
                Se a sentença $\alpha$ for a constante $\bot$, sabe-se que $\bot^\medsquare=\bot$ pela definição da tradução.
                Deste modo, devemos provar que $\entails\bot\to\nec\bot$, o que consiste num caso particular da explosão provada pelo lema \hyperref[explosion]{$\mathbf{L_2}$}.
            \end{case}

            \begin{case}
                \textsc{Caso 3.}
                Se a sentença $\alpha$ for o resultado da conjunção de duas outras sentenças $\varphi$ e $\psi$, sabe-se que ${(\varphi\wedge\psi)}^\medsquare=\varphi^\medsquare\wedge\psi^\medsquare$ pela definição da tradução.
                A partir de $\mathbf{H}$, temos que $\mathbf{H_1}={\entails\varphi^\medsquare\to\nec\varphi^\medsquare}$ e que $\mathbf{H_2}={\entails\psi^\medsquare\to\nec\psi^\medsquare}$ por meio dos esquemas da eliminação da conjunção e da aplicação da regra do \emph{modus ponens}.
                Valendo-se do listado acima em conjunto com alguns lemas, pode-se provar que $\entails\varphi^\medsquare\wedge\psi^\medsquare\to\nec(\varphi^\medsquare\wedge\psi^\medsquare)$ pela seguinte sucessão de dedução:

                \footnotesize
                \begin{fitch}
                    \fb\set{\varphi^\medsquare\wedge\psi^\medsquare}\proves\varphi^\medsquare\wedge\psi^\medsquare&$\hyperref[premisse]{\mathbf{R_1}}$\\
                    \fa\set{\varphi^\medsquare\wedge\psi^\medsquare}\proves\varphi^\medsquare\wedge\psi^\medsquare\to\varphi^\medsquare&\hyperref[MA4]{${\mathbf{A_4}}$}\\
                    \fa\set{\varphi^\medsquare\wedge\psi^\medsquare}\proves\varphi^\medsquare&$\hyperref[detachment]{\mathbf{R_2}}\;\set{1,2}$\\
                    \fa\set{\varphi^\medsquare\wedge\psi^\medsquare}\proves\varphi^\medsquare\to\nec\varphi^\medsquare&$\mathbf{H_1}$\\
                    \fa\set{\varphi^\medsquare\wedge\psi^\medsquare}\proves\nec\varphi^\medsquare&$\hyperref[detachment]{\mathbf{R_2}}\;\set{3,4}$\\
                    \fa\set{\varphi^\medsquare\wedge\psi^\medsquare}\proves\varphi^\medsquare\wedge\psi^\medsquare\to\psi^\medsquare&\hyperref[MA5]{${\mathbf{A_5}}$}\\
                    \fa\set{\varphi^\medsquare\wedge\psi^\medsquare}\proves\psi^\medsquare&$\hyperref[detachment]{\mathbf{R_2}}\;\set{1,6}$\\
                    \fa\set{\varphi^\medsquare\wedge\psi^\medsquare}\proves\psi^\medsquare\to\nec\psi^\medsquare&$\mathbf{H_2}$\\
                    \fa\set{\varphi^\medsquare\wedge\psi^\medsquare}\proves\nec\psi^\medsquare&$\hyperref[detachment]{\mathbf{R_2}}\;\set{7,8}$\\
                    \fa\set{\varphi^\medsquare\wedge\psi^\medsquare}\proves\nec\varphi^\medsquare\to\nec\psi^\medsquare\to\nec\varphi^\medsquare\wedge\nec\psi^\medsquare&\hyperref[MA3]{${\mathbf{A_3}}$}\\
                    \fa\set{\varphi^\medsquare\wedge\psi^\medsquare}\proves\nec\psi^\medsquare\to\nec\varphi^\medsquare\wedge\nec\psi^\medsquare&$\hyperref[detachment]{\mathbf{R_2}}\;\set{5,10}$\\
                    \fa\set{\varphi^\medsquare\wedge\psi^\medsquare}\proves\nec\varphi^\medsquare\wedge\nec\psi^\medsquare&$\hyperref[detachment]{\mathbf{R_2}}\;\set{9,11}$\\
                    \fa\proves\varphi^\medsquare\wedge\psi^\medsquare\to\nec\varphi^\medsquare\wedge\nec\psi^\medsquare&$\hyperref[deduction]{\mathbf{T_\getrefnumber{deduction}}}\;\set{12}$\\
                    \fa\proves\nec\varphi^\medsquare\wedge\nec\psi^\medsquare\to\nec(\varphi^\medsquare\wedge\psi^\medsquare)&\refer{nec-distr}{L}\\
                    \fa\proves\varphi^\medsquare\wedge\psi^\medsquare\to\nec(\varphi^\medsquare\wedge\psi^\medsquare)&\refer{comp}{L}$\;\set{13, 14}$\\
                \end{fitch}
            \end{case}

            \begin{case}
                \textsc{Caso 4.}
                Se a sentença $\alpha$ for o resultado da disjunção de duas outras sentenças $\varphi$ e $\psi$, sabe-se que ${(\varphi\vee\psi)}^\medsquare=\varphi^\medsquare\vee\psi^\medsquare$ pela definição da tradução.
                A partir de $\mathbf{H}$, temos que $\mathbf{H_1}={\entails\varphi^\medsquare\to\nec\varphi^\medsquare}$ e que $\mathbf{H_2}={\entails\psi^\medsquare\to\nec\psi^\medsquare}$ por meio dos esquemas da eliminação da conjunção e da aplicação da regra do \emph{modus ponens}.
                Valendo-se do listado acima em conjunto com alguns lemas, pode-se provar que $\entails\varphi^\medsquare\vee\psi^\medsquare\to\nec(\varphi^\medsquare\vee\psi^\medsquare)$ pela seguinte sucessão de dedução:

                \footnotesize
                \begin{fitch}
                    \fb\set{\varphi^\medsquare\vee\psi^\medsquare}\entails\varphi^\medsquare\to\nec\varphi^\medsquare&$\mathbf{H_1}$\\
                    \fa\set{\varphi^\medsquare\vee\psi^\medsquare}\entails\psi^\medsquare\to\nec\psi^\medsquare&$\mathbf{H_2}$\\
                    \fa\set{\varphi^\medsquare\vee\psi^\medsquare}\entails\varphi^\medsquare\vee\psi^\medsquare&$\hyperref[premisse]{\mathbf{R_1}}$\\
                    \fa\set{\varphi^\medsquare\vee\psi^\medsquare}\entails(\varphi^\medsquare\to\nec\varphi^\medsquare)\to(\psi^\medsquare\to\nec\psi^\medsquare)\to\varphi^\medsquare\vee\psi^\medsquare\to\nec\varphi^\medsquare\vee\nec\psi^\medsquare&\refer{or-subst}{L}\\
                    \fa\set{\varphi^\medsquare\vee\psi^\medsquare}\entails(\psi^\medsquare\to\nec\psi^\medsquare)\to\varphi^\medsquare\vee\psi^\medsquare\to\nec\varphi^\medsquare\vee\nec\psi^\medsquare&$\hyperref[detachment]{\mathbf{R_1}}\;\set{1,4}$\\
                    \fa\set{\varphi^\medsquare\vee\psi^\medsquare}\entails\varphi^\medsquare\vee\psi^\medsquare\to\nec\varphi^\medsquare\vee\nec\psi^\medsquare&$\hyperref[detachment]{\mathbf{R_2}}\;\set{2,5}$\\
                    \fa\set{\varphi^\medsquare\vee\psi^\medsquare}\entails\nec\varphi^\medsquare\vee\nec\psi^\medsquare&$\hyperref[detachment]{\mathbf{R_2}}\;\set{3,6}$\\
                    \fa\set{\varphi^\medsquare\vee\psi^\medsquare}\entails\nec(\varphi^\medsquare\vee\psi^\medsquare)&\refer{or-distr}{L}$\;\set{7}$\\
                    \fa\entails\varphi^\medsquare\vee\psi^\medsquare\to\nec(\varphi^\medsquare\vee\psi^\medsquare)&$\hyperref[deduction]{\mathbf{T_\getrefnumber{deduction}}}\;\set{8}$
                \end{fitch}
            \end{case}

            \begin{case}
                \textsc{Caso 5.}
                Se a sentença $\alpha$ for o resultado da implicação de uma sentença $\varphi$ a uma sentença $\psi$, sabe-se que ${(\varphi\to\psi)}^\medsquare=\nec(\varphi^\medsquare\to\psi^\medsquare)$ pela definição da tradução.
                Deste modo, $\nec(\varphi^\medsquare\to\psi^\medsquare)\to\nec\nec(\varphi^\medsquare\to\psi^\medsquare)$ constitui um axioma gerado pelo esquema \hyperref[MB3]{$\mathbf{B_3}$} --- sendo assim $\entails\nec(\varphi^\medsquare\to\psi^\medsquare)\to\nec\nec(\varphi^\medsquare\to\psi^\medsquare)$ provado trivialmente.
            \end{case}
            \vspace{.5\baselineskip}
            Tendo-se provado todos os casos do passo de indução, podemos concluir que a propriedade vale, ou seja, que $\entails\alpha^\medsquare\to\nec\alpha^\medsquare$.
        \end{proof}
    \end{lemma}

    Uma vez provado o lema podemos, por fim, provar a correção da $\nec$-tradução.

    \begin{theorem}\label{square-correctness}
        Se $\Gamma\entails\alpha$, então $\Gamma^\medsquare\entails\alpha^\medsquare$.
    \end{theorem}

    \begin{proof}
        Prova por indução forte sobre o tamanho da sucessão de dedução.
        Assim, suponhamos que a tradução seja correta para qualquer sucessão dedução de tamanho $n < k$.
        Demonstraremos, analisando-se os casos, que a correção da tradução vale para sucessões de dedução de tamanho $n = k$.

        \begin{case}
            \textsc{Caso 1.}
            Seja a linha derradeira da sucessão de dedução que prova $\Gamma\entails\alpha$ gerada pela invocação de alguma premissa.
            Sabe-se que $\alpha\in\Gamma$ e, portanto, que $\alpha^\medsquare\in\Gamma^\medsquare$.
            Desde modo, pode-se demonstrar que $\Gamma^\medsquare\entails\alpha^\medsquare$ trivialmente pela evocação da premissa $\alpha^\medsquare$.
        \end{case}

        \begin{case}
            \textsc{Caso 2.}
            Seja a linha derradeira da sucessão de dedução que prova $\Gamma\entails\alpha$ gerada pela invocação de algum axioma.
            Sabe-se que existe algum esquema $\mathbf{A}\in\mathcal{A}$ que gera $\alpha$. 
            Deste modo, analisaremos os casos e demonstraremos que se pode derivar $\Gamma^\medsquare\entails\alpha^\medsquare$ para cada esquema $\mathbf{A}\in\mathcal{A}$.
            Nos casos abaixo, usaremos ocasionalmente a implicação estrita de modo a diminuir o espaço ocupado pelas provas.
        \end{case}

            \begin{subcase}
                \textsc{Caso 2.1.} Se a linha derradeira da sucessão de dedução que prova $\Gamma\entails_\mathbf{I}\alpha$ tenha sido a evocação de algum axioma gerado pelo esquema $\hyperref[IA1]{\mathbf{A_1}}$, sabemos que $\alpha=\varphi\to\psi\to\varphi$ e que $\alpha^\medsquare=\nec(\varphi^\medsquare\to\nec(\psi^\medsquare\to\varphi^\medsquare))$. Deste modo, podemos provar que $\Gamma^\medsquare\entails_\mathbf{S4}\nec(\varphi^\medsquare\to\nec(\psi^\medsquare\to\varphi^\medsquare))$ pela seguinte sucessão de dedução:

                \footnotesize
                \begin{fitch}
                    \fb\entails\varphi^\medsquare\to\nec\varphi^\medsquare&\refer{square-nec}{L}\\
                    \fa\entails\varphi^\medsquare\to\psi^\medsquare\to\varphi^\medsquare&\hyperref[MA1]{${\mathbf{A_1}}$}\\
                    \fa\entails\nec(\varphi^\medsquare\to\psi^\medsquare\to\varphi^\medsquare)&$\hyperref[necessitation]{\mathbf{R_3}}\;\set{2}$\\
                    \fa\entails\nec(\varphi^\medsquare\to\psi^\medsquare\to\varphi^\medsquare)\to\nec\varphi^\medsquare\to\nec(\psi^\medsquare\to\varphi^\medsquare)&\hyperref[MB1]{${\mathbf{B_1}}$}\\
                    \fa\entails\nec\varphi^\medsquare\to\nec(\psi^\medsquare\to\varphi^\medsquare)&$\hyperref[detachment]{\mathbf{R_2}}\;\set{3,4}$\\
                    \fa\entails(\varphi^\medsquare\to\nec\varphi^\medsquare)\to(\nec\varphi^\medsquare\to\nec(\psi^\medsquare\to\varphi^\medsquare))\to\varphi^\medsquare\to\nec(\psi^\medsquare\to\varphi^\medsquare)&\refer{comp}{L}\\
                    \fa\entails(\nec\varphi^\medsquare\to\nec(\psi^\medsquare\to\varphi^\medsquare))\to\varphi^\medsquare\to\nec(\psi^\medsquare\to\varphi^\medsquare)&$\hyperref[detachment]{\mathbf{R_2}}\;\set{1,6}$\\
                    \fa\entails\varphi^\medsquare\to\nec(\psi^\medsquare\to\varphi^\medsquare)&$\hyperref[detachment]{\mathbf{R_2}}\;\set{5,7}$\\
                    \fa\Gamma^\medsquare\entails\nec(\varphi^\medsquare\to\nec(\psi^\medsquare\to\varphi^\medsquare))&$\hyperref[necessitation]{\mathbf{R_3}}\;\set{8}$\\
                \end{fitch}
            \end{subcase}

            \begin{subcase}
                \textsc{Caso 2.2.} Se a linha derradeira da sucessão de dedução que prova $\Gamma\entails_\mathbf{I}\alpha$ tenha sido a evocação de algum axioma gerado pelo esquema $\hyperref[IA2]{\mathbf{A_2}}$, sabemos que $\alpha=(\varphi\to\psi\to\chi)\to(\varphi\to\psi)\to\varphi\to\chi$ e que $\alpha^\medsquare=(\varphi^\medsquare\strictif\psi^\medsquare\strictif\chi^\medsquare)\strictif(\varphi^\medsquare\strictif\psi^\medsquare)\strictif\varphi^\medsquare\strictif\chi^\medsquare$. Deste modo, podemos provar que $\Gamma^\medsquare\entails_\mathbf{S4}(\varphi^\medsquare\strictif\psi^\medsquare\strictif\chi^\medsquare)\strictif(\varphi^\medsquare\strictif\psi^\medsquare)\strictif\varphi^\medsquare\strictif\chi^\medsquare$ pela seguinte sucessão de dedução:

                \footnotesize
                \begin{fitch}
                    \fb\set{\varphi^\medsquare\strictif\psi^\medsquare\strictif\chi^\medsquare,\varphi^\medsquare\strictif\psi^\medsquare,\varphi^\medsquare}\entails\varphi^\medsquare&$\hyperref[premisse]{\mathbf{R_1}}$\\
                    \fa\set{\varphi^\medsquare\strictif\psi^\medsquare\strictif\chi^\medsquare,\varphi^\medsquare\strictif\psi^\medsquare,\varphi^\medsquare}\entails\varphi^\medsquare\strictif\psi^\medsquare&$\hyperref[premisse]{\mathbf{R_1}}$\\
                    \fa\set{\varphi^\medsquare\strictif\psi^\medsquare\strictif\chi^\medsquare,\varphi^\medsquare\strictif\psi^\medsquare,\varphi^\medsquare}\entails\psi^\medsquare&$\hyperref[strictsep]{\mathbf{T_\getrefnumber{strictsep}}}\;\set{1,2}$\\
                    \fa\set{\varphi^\medsquare\strictif\psi^\medsquare\strictif\chi^\medsquare,\varphi^\medsquare\strictif\psi^\medsquare,\varphi^\medsquare}\entails\varphi^\medsquare\strictif\psi^\medsquare\strictif\chi^\medsquare&$\hyperref[premisse]{\mathbf{R_1}}$\\
                    \fa\set{\varphi^\medsquare\strictif\psi^\medsquare\strictif\chi^\medsquare,\varphi^\medsquare\strictif\psi^\medsquare,\varphi^\medsquare}\entails\psi^\medsquare\strictif\chi^\medsquare&$\hyperref[strictsep]{\mathbf{T_\getrefnumber{strictsep}}}\;\set{1,4}$\\
                    \fa\set{\varphi^\medsquare\strictif\psi^\medsquare\strictif\chi^\medsquare,\varphi^\medsquare\strictif\psi^\medsquare,\varphi^\medsquare}\entails\chi^\medsquare&$\hyperref[strictsep]{\mathbf{T_\getrefnumber{strictsep}}}\;\set{3,5}$\\
                    \fa\set{\varphi^\medsquare\strictif\psi^\medsquare\strictif\chi^\medsquare,\varphi^\medsquare\strictif\psi^\medsquare}\entails\varphi^\medsquare\strictif\chi^\medsquare&$\hyperref[strictdeduction]{\mathbf{T_\getrefnumber{strictdeduction}}}\;\set{6}$\\
                    \fa\set{\varphi^\medsquare\strictif\psi^\medsquare\strictif\chi^\medsquare}\entails(\varphi^\medsquare\strictif\psi^\medsquare)\strictif\varphi^\medsquare\strictif\chi^\medsquare&$\hyperref[strictdeduction]{\mathbf{T_\getrefnumber{strictdeduction}}}\;\set{7}$\\
                    \fa\entails(\varphi^\medsquare\strictif\psi^\medsquare\strictif\chi^\medsquare)\to(\varphi^\medsquare\strictif\psi^\medsquare)\strictif\varphi^\medsquare\strictif\chi^\medsquare&$\hyperref[deduction]{\mathbf{T_\getrefnumber{deduction}}}\;\set{8}$\\
                    \fa\Gamma^\medsquare\entails(\varphi^\medsquare\strictif\psi^\medsquare\strictif\chi^\medsquare)\strictif(\varphi^\medsquare\strictif\psi^\medsquare)\strictif\varphi^\medsquare\strictif\chi^\medsquare&$\hyperref[necessitation]{\mathbf{R_3}}\;\set{9}$\\
                \end{fitch}
            \end{subcase}

            \begin{subcase}
                \textsc{Caso 2.3.} Se a linha derradeira da sucessão de dedução que prova $\Gamma\entails_\mathbf{I}\alpha$ tenha sido a evocação de algum axioma gerado pelo esquema $\hyperref[IA3]{\mathbf{A_3}}$, sabemos que $\alpha=\varphi\to\psi\to\varphi\wedge\psi$ e que $\alpha^\medsquare=\nec(\varphi^\medsquare\to\nec(\psi^\medsquare\to\varphi^\medsquare\wedge\psi^\medsquare))$. Deste modo, podemos provar que $\Gamma^\medsquare\entails_\mathbf{S4}\nec(\varphi^\medsquare\to\nec(\psi^\medsquare\to\varphi^\medsquare\wedge\psi^\medsquare))$ pela seguinte sucessão de dedução:

                \footnotesize
                \begin{fitch}
                    \fb\entails\varphi^\medsquare\to\nec\varphi^\medsquare&\refer{square-nec}{L}\\
                    \fa\entails\varphi^\medsquare\to\psi^\medsquare\to\varphi^\medsquare\wedge\psi^\medsquare&\hyperref[MA3]{${\mathbf{A_3}}$}\\
                    \fa\entails\nec(\varphi^\medsquare\to\psi^\medsquare\to\varphi^\medsquare\wedge\psi^\medsquare)&$\hyperref[necessitation]{\mathbf{R_3}}\;\set{2}$\\
                    \fa\entails\nec(\varphi^\medsquare\to\psi^\medsquare\to\varphi^\medsquare\wedge\psi^\medsquare)\to\nec\varphi^\medsquare\to\nec(\psi^\medsquare\to\varphi^\medsquare\wedge\psi^\medsquare)&\hyperref[MB1]{${\mathbf{B_1}}$}\\
                    \fa\entails\nec\varphi^\medsquare\to\nec(\psi^\medsquare\to\varphi^\medsquare\wedge\psi^\medsquare)&$\hyperref[detachment]{\mathbf{R_2}}\;\set{3,4}$\\
                    \fa\entails(\varphi^\medsquare\to\nec\varphi^\medsquare)\to(\nec\varphi^\medsquare\to\psi^\medsquare\strictif\varphi^\medsquare\wedge\psi^\medsquare)\to\varphi^\medsquare\to\psi^\medsquare\strictif\varphi^\medsquare\wedge\psi^\medsquare&\refer{comp}{L}\\
                    \fa\entails(\nec\varphi^\medsquare\to\nec(\psi^\medsquare\to\varphi^\medsquare\wedge\psi^\medsquare))\to\varphi^\medsquare\to\nec(\psi^\medsquare\to\varphi^\medsquare\wedge\psi^\medsquare)&$\hyperref[detachment]{\mathbf{R_2}}\;\set{1,6}$\\
                    \fa\entails\varphi^\medsquare\to\nec(\psi^\medsquare\to\varphi^\medsquare\wedge\psi^\medsquare)&$\hyperref[detachment]{\mathbf{R_2}}\;\set{5,7}$\\
                    \fa\Gamma^\medsquare\entails\nec(\varphi^\medsquare\to\nec(\psi^\medsquare\to\varphi^\medsquare\wedge\psi^\medsquare))&$\hyperref[necessitation]{\mathbf{R_3}}\;\set{8}$\\
                \end{fitch} 
            \end{subcase}

            \begin{subcase}
                \textsc{Caso 2.4.} Se a linha derradeira da sucessão de dedução que prova $\Gamma\entails_\mathbf{I}\alpha$ tenha sido a evocação de algum axioma gerado pelo esquema $\hyperref[IA4]{\mathbf{A_4}}$, sabemos que $\alpha=\varphi\wedge\psi\to\varphi$ e que $\alpha^\medsquare=\nec(\varphi^\medsquare\wedge\psi^\medsquare\to\varphi^\medsquare)$. Deste modo, podemos provar que $\Gamma^\medsquare\entails_\mathbf{S4}\nec(\varphi^\medsquare\wedge\psi^\medsquare\to\varphi^\medsquare)$ pela seguinte sucessão de dedução:

                \footnotesize
                \begin{fitch}
                    \fa\entails\varphi^\medsquare\wedge\psi^\medsquare\to\varphi^\medsquare&$\hyperref[MA4]{\mathbf{A_4}}$\\
                    \fa\Gamma^\medsquare\entails\nec(\varphi^\medsquare\wedge\psi^\medsquare\to\varphi^\medsquare)&$\hyperref[necessitation]{\mathbf{R_3}}\;\set{1}$
                \end{fitch}
            \end{subcase}

            \begin{subcase}
                \textsc{Caso 2.5.} Se a linha derradeira da sucessão de dedução que prova $\Gamma\entails_\mathbf{I}\alpha$ tenha sido a evocação de algum axioma gerado pelo esquema $\hyperref[IA5]{\mathbf{A_5}}$, sabemos que $\alpha=\varphi\wedge\psi\to\psi$ e que $\alpha^\medsquare=\nec(\varphi^\medsquare\wedge\psi^\medsquare\to\psi^\medsquare)$. Deste modo, podemos provar que $\Gamma^\medsquare\entails_\mathbf{S4}\nec(\varphi^\medsquare\wedge\psi^\medsquare\to\psi^\medsquare)$ pela seguinte sucessão de dedução:

                \footnotesize
                \begin{fitch}
                    \fb\entails\varphi^\medsquare\wedge\psi^\medsquare\to\psi^\medsquare&$\hyperref[MA5]{\mathbf{A_5}}$\\
                    \fa\Gamma^\medsquare\entails\nec(\varphi^\medsquare\wedge\psi^\medsquare\to\psi^\medsquare)&$\hyperref[necessitation]{\mathbf{R_3}}\;\set{1}$
                \end{fitch}
            \end{subcase}

            \begin{subcase}
                \textsc{Caso 2.6.} Se a linha derradeira da sucessão de dedução que prova $\Gamma\entails_\mathbf{I}\alpha$ tenha sido a evocação de algum axioma gerado pelo esquema $\hyperref[IA6]{\mathbf{A_6}}$, sabemos que $\alpha=\varphi\to\varphi\vee\psi$ e que $\alpha^\medsquare=\nec(\varphi^\medsquare\to\varphi^\medsquare\vee\psi^\medsquare)$. Deste modo, podemos provar que $\Gamma^\medsquare\entails_\mathbf{S4}\nec(\varphi^\medsquare\to\varphi^\medsquare\vee\psi^\medsquare)$ pela seguinte sucessão de dedução:

                \footnotesize
                \begin{fitch}
                    \fb\entails\varphi^\medsquare\to\varphi^\medsquare\vee\psi^\medsquare&$\hyperref[MA6]{\mathbf{A_6}}$\\
                    \fa\Gamma^\medsquare\entails\nec(\varphi^\medsquare\to\varphi^\medsquare\vee\psi^\medsquare)&$\hyperref[necessitation]{\mathbf{R_3}}\;\set{1}$
                \end{fitch}
            \end{subcase}

            \begin{subcase}
                \textsc{Caso 2.7.} Se a linha derradeira da sucessão de dedução que prova $\Gamma\entails_\mathbf{I}\alpha$ tenha sido a evocação de algum axioma gerado pelo esquema $\hyperref[IA7]{\mathbf{A_7}}$, sabemos que $\alpha=\psi\to\varphi\vee\psi$ e que $\alpha^\medsquare=\nec(\psi^\medsquare\to\varphi^\medsquare\vee\psi^\medsquare)$. Deste modo, podemos provar que $\Gamma^\medsquare\entails_\mathbf{S4}\nec(\psi^\medsquare\to\varphi^\medsquare\vee\psi^\medsquare)$ pela seguinte sucessão de dedução:

                \footnotesize
                \begin{fitch}
                    \fb\entails\psi^\medsquare\to\varphi^\medsquare\vee\psi^\medsquare&$\hyperref[MA7]{\mathbf{A_7}}$\\
                    \fa\Gamma^\medsquare\entails\nec(\psi^\medsquare\to\varphi^\medsquare\vee\psi^\medsquare)&$\hyperref[necessitation]{\mathbf{R_3}}\;\set{1}$
                \end{fitch}
            \end{subcase}

            \begin{subcase}
                \textsc{Caso 2.8.} Se a linha derradeira da sucessão de dedução que prova $\Gamma\entails_\mathbf{I}\alpha$ tenha sido a evocação de algum axioma gerado pelo esquema $\hyperref[IA8]{\mathbf{A_8}}$, sabemos que $\alpha=(\varphi\to\chi)\to(\psi\to\chi)\to\varphi\vee\psi\to\chi$ e que $\alpha^\medsquare=(\varphi^\medsquare\strictif\chi^\medsquare)\strictif(\psi^\medsquare\strictif\chi^\medsquare)\strictif\varphi^\medsquare\vee\psi^\medsquare\strictif\chi^\medsquare$. Deste modo, podemos provar que $\Gamma^\medsquare\entails_\mathbf{S4}(\varphi^\medsquare\strictif\chi^\medsquare)\strictif(\psi^\medsquare\strictif\chi^\medsquare)\strictif\varphi^\medsquare\vee\psi^\medsquare\strictif\chi^\medsquare$ pela seguinte sucessão de dedução:

                \footnotesize
                \begin{fitch}
                    \fb\set{\varphi^\medsquare\strictif\chi^\medsquare,\psi^\medsquare\strictif\chi^\medsquare,\varphi^\medsquare\vee\psi^\medsquare}\entails\varphi^\medsquare\strictif\chi^\medsquare&$\hyperref[premisse]{\mathbf{R_1}}$\\
                    \fa\set{\varphi^\medsquare\strictif\chi^\medsquare,\psi^\medsquare\strictif\chi^\medsquare,\varphi^\medsquare\vee\psi^\medsquare}\entails(\varphi^\medsquare\strictif\chi^\medsquare)\to\varphi^\medsquare\to\chi^\medsquare&\hyperref[MB2]{${\mathbf{B_2}}$}\\
                    \fa\set{\varphi^\medsquare\strictif\chi^\medsquare,\psi^\medsquare\strictif\chi^\medsquare,\varphi^\medsquare\vee\psi^\medsquare}\entails\varphi^\medsquare\to\chi^\medsquare&$\hyperref[detachment]{\mathbf{R_2}}\;\set{1,2}$\\
                    \fa\set{\varphi^\medsquare\strictif\chi^\medsquare,\psi^\medsquare\strictif\chi^\medsquare,\varphi^\medsquare\vee\psi^\medsquare}\entails\psi^\medsquare\strictif\chi^\medsquare&$\hyperref[premisse]{\mathbf{R_1}}$\\
                    \fa\set{\varphi^\medsquare\strictif\chi^\medsquare,\psi^\medsquare\strictif\chi^\medsquare,\varphi^\medsquare\vee\psi^\medsquare}\entails(\psi^\medsquare\strictif\chi^\medsquare)\to\psi^\medsquare\to\chi^\medsquare&\hyperref[MB2]{${\mathbf{B_2}}$}\\
                    \fa\set{\varphi^\medsquare\strictif\chi^\medsquare,\psi^\medsquare\strictif\chi^\medsquare,\varphi^\medsquare\vee\psi^\medsquare}\entails\psi^\medsquare\to\chi^\medsquare&$\hyperref[detachment]{\mathbf{R_2}}\;\set{4,5}$\\
                    \fa\set{\varphi^\medsquare\strictif\chi^\medsquare,\psi^\medsquare\strictif\chi^\medsquare,\varphi^\medsquare\vee\psi^\medsquare}\entails\varphi^\medsquare\vee\psi^\medsquare&$\hyperref[premisse]{\mathbf{R_1}}$\\
                    \fa\set{\varphi^\medsquare\strictif\chi^\medsquare,\psi^\medsquare\strictif\chi^\medsquare,\varphi^\medsquare\vee\psi^\medsquare}\entails(\varphi^\medsquare\to\chi^\medsquare)\to(\psi^\medsquare\to\chi^\medsquare)\to\varphi^\medsquare\vee\psi^\medsquare\to\chi^\medsquare&\hyperref[MA8]{${\mathbf{A_8}}$}\\
                    \fa\set{\varphi^\medsquare\strictif\chi^\medsquare,\psi^\medsquare\strictif\chi^\medsquare,\varphi^\medsquare\vee\psi^\medsquare}\entails(\psi^\medsquare\to\chi^\medsquare)\to\varphi^\medsquare\vee\psi^\medsquare\to\chi^\medsquare&$\hyperref[detachment]{\mathbf{R_2}}\;\set{3,8}$\\
                    \fa\set{\varphi^\medsquare\strictif\chi^\medsquare,\psi^\medsquare\strictif\chi^\medsquare,\varphi^\medsquare\vee\psi^\medsquare}\entails\varphi^\medsquare\vee\psi^\medsquare\to\chi^\medsquare&$\hyperref[detachment]{\mathbf{R_2}}\;\set{6,9}$\\
                    \fa\set{\varphi^\medsquare\strictif\chi^\medsquare,\psi^\medsquare\strictif\chi^\medsquare,\varphi^\medsquare\vee\psi^\medsquare}\entails\chi^\medsquare&$\hyperref[detachment]{\mathbf{R_2}}\;\set{7,10}$\\
                    \fa\set{\varphi^\medsquare\strictif\chi^\medsquare,\psi^\medsquare\strictif\chi^\medsquare}\entails\varphi^\medsquare\vee\psi^\medsquare\strictif\chi^\medsquare&$\hyperref[strictdeduction]{\mathbf{T_\getrefnumber{strictdeduction}}}\;\set{11}$\\
                    \fa\set{\varphi^\medsquare\strictif\chi^\medsquare}\entails(\psi^\medsquare\strictif\chi^\medsquare)\strictif\varphi^\medsquare\vee\psi^\medsquare\strictif\chi^\medsquare&$\hyperref[strictdeduction]{\mathbf{T_\getrefnumber{strictdeduction}}}\;\set{12}$\\
                    \fa\entails(\varphi^\medsquare\strictif\chi^\medsquare)\to(\psi^\medsquare\strictif\chi^\medsquare)\strictif\varphi^\medsquare\vee\psi^\medsquare\strictif\chi^\medsquare&$\hyperref[deduction]{\mathbf{T_\getrefnumber{deduction}}}\;\set{13}$\\
                    \fa\Gamma^\medsquare\entails(\varphi^\medsquare\strictif\chi^\medsquare)\strictif(\psi^\medsquare\strictif\chi^\medsquare)\strictif\varphi^\medsquare\vee\psi^\medsquare\strictif\chi^\medsquare&$\hyperref[necessitation]{\mathbf{R_3}}\;\set{14}$
                \end{fitch}
            \end{subcase}

            \begin{subcase}
                \textsc{Caso 2.9.} Se a linha derradeira da sucessão de dedução que prova $\Gamma\entails_\mathbf{I}\alpha$ tenha sido a evocação de algum axioma gerado pelo esquema $\mathbf{A_{\bot}}$, sabemos que $\alpha=\bot\to\varphi$ e que $\alpha^\medsquare=\nec(\bot\to\varphi^\medsquare)$. Deste modo, podemos provar que $\Gamma^\medsquare\entails_\mathbf{S4}\nec(\bot\to\varphi^\medsquare)$ pela seguinte sucessão de dedução:

                \footnotesize
                \begin{fitch}
                    \fb\entails\bot\to\varphi^\medsquare&\refer{explosion}{L}\\
                    \fa\Gamma^\medsquare\entails\nec(\bot\to\varphi^\medsquare)&$\hyperref[necessitation]{\mathbf{R_3}}\;\set{1}$
                \end{fitch}
            \end{subcase}

        \begin{case}
            \textsc{Caso 3.}
            Se a linha derradeira da sucessão de dedução que prova $\Gamma\entails_\mathbf{I}\alpha$ tenha sido gerada pela aplicação da regra do \emph{modus ponens} a duas sentenças $\varphi_i$ e $\varphi_j$ com $i<j<n$ pode-se assumir, sem perda de generalidade, que $\varphi_j=\varphi_i\to\alpha$.
            Assim, a partir de $\mathbf{H}$ temos que $\mathbf{H_1}=\Gamma^\medsquare\entails_\mathbf{S4}\varphi_i^\medsquare$ e que $\mathbf{H_2}=\Gamma^\medsquare\entails_\mathbf{S4}\nec(\varphi_i^\medsquare\to\alpha^\medsquare)$.
            Deste modo, podemos demonstrar que $\Gamma^\medsquare\entails_\mathbf{S4}\alpha^\medsquare$ pela seguinte sucessão de dedução:

            \footnotesize
            \begin{fitch}
                \fb\varphi_i^\medsquare&$\mathbf{H_2}$\\
                \fa\nec(\varphi_i^\medsquare\to\alpha^\medsquare)&$\mathbf{H_1}$\\
                \fa\nec(\varphi_i^\medsquare\to\alpha^\medsquare)\to\varphi_i^\medsquare\to\alpha^\medsquare&\hyperref[MB2]{${\mathbf{B_2}}$}\\
                \fa\varphi_i^\medsquare\to\alpha^\medsquare&$\hyperref[detachment]{\mathbf{R_2}}\;\set{2, 3}$\\
                \fa\alpha^\medsquare&$\hyperref[detachment]{\mathbf{R_2}}\;\set{1, 4}$
            \end{fitch}
        \end{case}
        \vspace{.5\baselineskip}
        Tendo-se provado todos os casos do passo de indução, podemos concluir que a correção da $\nec$-tradução, ou seja, que se $\Gamma\entails_\mathbf{I}\alpha$, então $\Gamma^\medsquare\entails_\mathbf{S4}\alpha^\medsquare$.
    \end{proof}

        \chapter{Implementação} 

A biblioteca desenvolvida por \cite{Silveira} apresenta algumas diferenças menores de definições quando comparada ao que foi formalizado anteriormente neste trabalho.
Tais diferenças serão abordadas em detalhes nesta seção.
Primeiramente, trataremos das diferenças na linguagem dos sistemas modais.
A linguagem da biblioteca de \cite{Silveira} permite sistemas multimodais, ou seja, sistemas com uma quantidade qualquer de modalidades.
Distinguiremos estas modalidades por um valor $i\in\mathbb{N}$.
Ademais, a biblioteca se \cite{Silveira} trata a negação como uma operação primitiva -- confrontando a contradição, usada neste trabalho --, bem como trata a possibilidade do mesmo modo.
Pequenas modificações nas definições deste trabalho precisaram ser feitas para dar conta de tais diferenças.
Tais modificações serão assomadas quando relevantes.

\begin{definition}[$\mathcal{L}_{\nec}$]
    A linguagem dos sistemas multimodais com $n\in\mathbb{N}^+$ modalidades distinguidas por $i\in\mathbb{N}^+$ pode ser induzida a partir da assinatura $\Sigma_{\nec}=\sequence{\mathcal{P},\mathcal{C}_{\nec}}$, onde $\mathcal{C}_{\nec}=\set{\neg^1,\nec^1_i,\pos^1_i,\wedge^2,\vee^2,\to^2\mid i\leq n}$.
\end{definition}

Outra diferença entre as definições deste trabalho e as definições de \cite{Silveira} consistem no conjunto regras de deduções, ligeiramente maior na biblioteca, conforme mostrado abaixo.
Isso acontece pela adição da regra $\mathbf{A_9}$ -- que pode ser derivada usando as regras apresentadas neste trabalho -- e pela regra $\mathbf{B_4}$ -- neste trabalho tratada como uma definição.
Ainda, as regras $\mathbf{B_1}$, $\mathbf{B_2}$ e $\mathbf{B_3}$ precisaram ser quantificadas universalmente sobre $i\in\mathbb{N}^+$.
Todas as definições e teoremas deste trabalho serão quantificados analogamente, de modo a encaixar-se nas definições da biblioteca.

\begin{definition}[$\vdash_{\mathfrak{L}}$]
    Abaixo estão definidas as regras do sistema multimodal $\mathfrak{L}$.
\hfill\break
\begin{center}
    \footnotesize
    \AxiomC{}
    \RightLabel{\footnotesize$\mathbf{A_1}$}
    \UnaryInfC{$\Gamma\vdash\alpha\to\beta\to\alpha$}
    \DisplayProof
    \quad
    \AxiomC{}
    \RightLabel{\footnotesize$\mathbf{A_2}$}
    \UnaryInfC{$\Gamma\vdash(\alpha\to\beta\to\gamma)\to(\alpha\to\beta)\to\alpha\to\gamma$}
    \DisplayProof
\end{center}

\begin{center}
    \footnotesize
    \AxiomC{}
    \RightLabel{\footnotesize$\mathbf{A_3}$}
    \UnaryInfC{$\Gamma\vdash\alpha\to\beta\to\alpha\wedge\beta$}
    \DisplayProof
    \quad
    \AxiomC{}
    \RightLabel{\footnotesize$\mathbf{A_4}$}
    \UnaryInfC{$\Gamma\vdash\alpha\wedge\beta\to\alpha$}
    \DisplayProof
    \quad
    \AxiomC{}
    \RightLabel{\footnotesize$\mathbf{A_5}$}
    \UnaryInfC{$\Gamma\vdash\alpha\wedge\beta\to\beta$}
    \DisplayProof
\end{center}

\begin{center}
    \footnotesize
    \AxiomC{}
    \RightLabel{\footnotesize$\mathbf{A_6}$}
    \UnaryInfC{$\Gamma\vdash\alpha\to\alpha\vee\beta$}
    \DisplayProof
    \quad
    \AxiomC{}
    \RightLabel{\footnotesize$\mathbf{A_7}$}
    \UnaryInfC{$\Gamma\vdash\beta\to\alpha\vee\beta$}
    \DisplayProof
    \quad
    \AxiomC{}
    \RightLabel{\footnotesize$\mathbf{A_8}$}
    \UnaryInfC{$\Gamma\vdash(\alpha\to\gamma)\to(\beta\to\gamma)\to\alpha\vee\beta\to\gamma$}
    \DisplayProof
\end{center}

\begin{center}
    \footnotesize
    \AxiomC{}
    \RightLabel{\footnotesize$\mathbf{A_9}$}
    \UnaryInfC{$\Gamma\vdash(\neg\beta\to\neg\alpha)\to\alpha\to\beta$}
    \DisplayProof
    \quad
    \AxiomC{}
    \RightLabel{\footnotesize$\mathbf{A_\neg}$}
    \UnaryInfC{$\Gamma\vdash\neg\neg\alpha\to\alpha$}
    \DisplayProof
\end{center}

\begin{center}
    \footnotesize
    \AxiomC{}
    \RightLabel{\footnotesize$\mathbf{B_1}$}
    \UnaryInfC{$\Gamma\vdash\Box_i(\alpha\to\beta)\to\Box_i\alpha\to\Box_i\beta$}
    \DisplayProof
    \quad
    \AxiomC{}
    \RightLabel{\footnotesize$\mathbf{B_2}$}
    \UnaryInfC{$\Gamma\vdash\Box_i\alpha\to\alpha$}
    \DisplayProof
    \quad
    \AxiomC{}
    \RightLabel{\footnotesize$\mathbf{B_3}$}
    \UnaryInfC{$\Gamma\vdash\Box_i\alpha\to\Box_i\Box_i\alpha$}
    \DisplayProof
\end{center}

\begin{center}
    \footnotesize
    \AxiomC{}
    \RightLabel{\footnotesize$\mathbf{B_4}$}
    \UnaryInfC{$\Gamma\vdash\pos_i\alpha\leftrightarrow\neg\Box_i\neg\alpha$}
    \DisplayProof
\end{center}

\begin{center}
    \footnotesize
    \AxiomC{$\alpha\in\Gamma$}
    \RightLabel{\footnotesize$\mathbf{R_1}$}
    \UnaryInfC{$\Gamma\vdash\alpha$}
    \DisplayProof
    \quad
    \AxiomC{$\Gamma\vdash\alpha$}
    \AxiomC{$\Gamma\vdash\alpha\to\beta$}
    \RightLabel{\footnotesize$\mathbf{R_2}$}
    \BinaryInfC{$\Gamma\vdash\beta$}
    \DisplayProof
    \quad
    \AxiomC{$\vdash\alpha$}
    \RightLabel{\footnotesize$\mathbf{R_3}$}
    \UnaryInfC{$\Gamma\vdash\Box_i\alpha$}
    \DisplayProof
\end{center}
\end{definition}
\hfill\break
Uma vez ditas as diferenças, partiremos para a definição da linguagem intuicionista.
Para tanto, definiu-se um tipo indutivo com cinco construtores, conforme apresentado abaixo.
O construtor \texttt{\footnotesize\textbf{Contradiction}} representa a contradição $\bot$.
O construtor \texttt{\footnotesize\textbf{Proposition}} eleva um valor $n\in\mathbb{N}$ a uma sentença intuicionista que representa uma letra proposicional.
Por fim, os construtores \texttt{\footnotesize\textbf{Conjunction}}, \texttt{\footnotesize\textbf{Disjunction}} e \texttt{\footnotesize\textbf{Implication}} criam conjunções, disjunções e implicações --- nesta ordem --- a partir de sentenças menores.
Notações foram definidas de modo a replicar a notação usada neste trabalho, com letras propositionais sendo introduzidas por um operador \texttt{\footnotesize\textbf{\#}}.
Deste modo, podemos representar a sentença intuicionista $a\wedge b\to a\vee b$ como \texttt{\footnotesize\textbf{\#1 ∧ \#2 → \#1 ∨ \#2}}, para $a,b\in\mathcal{P}$.
\\
\begin{mdframed}
\noindent
\texttt
{\noindent\footnotesize\textbf{Inductive} Propositional : \textbf{Set} :=\\
| Contradiction : Propositional\\
| Proposition\ \ \ : nat -> Propositional\\
| Conjunction\ \ \ : Propositional -> Propositional -> Propositional\\
| Disjunction\ \ \ : Propositional -> Propositional -> Propositional\\
| Implication\ \ \ : Propositional -> Propositional -> Propositional.\\
\\
Notation "\ ⊥\ "\ := Contradiction.
}
\end{mdframed}
\hfill\break
Similarmente, uma dedução intuicionista foi definida como um tipo indutivo, desta vez com onze construtores.
Cada um desses construtores representa uma das regras de dedução apresentadas anteriormente.
Assim como feito com a linguagem, foi definida uma notação de modo a facilitar a escrita de asserções.
Cabe destacar que o conjunto de assunções $\Gamma$ foi representada pela sua função indicadora $\mathbf{I}_\Gamma:\mathcal{L}\to\{0,1\}$, em que $\mathbf{I}_\Gamma(\alpha)=1$ se e somente se $\alpha\in\Gamma$.
Na implementação oferecida abaixo, isso foi feito por meio de uma função do tipo \texttt{\footnotesize\textbf{Sentence}} ao tipo \texttt{\footnotesize\textbf{Prop}}, o tipo das proposições.
\hfill\break
\begin{mdframed}
\noindent
\texttt
{\noindent\footnotesize\textbf{Definition} Theory := Propositional -> Prop.\\
\\
\textbf{Reserved} \textbf{Notation} "Γ ⊢ α"\ (at level 110).\\
\\
\textbf{Inductive} Deduction : Theory -> Propositional -> \textbf{Prop} :=\\
| A₁ : \textbf{forall} Γ α β,\ \ \ Γ ⊢ α → β → α\\
| A₂ : \textbf{forall} Γ α β γ, Γ ⊢ (α → β → γ) → (α → β) → α → γ\\
| A₃ : \textbf{forall} Γ α β,\ \ \ Γ ⊢ α → β → α ∧ β\\
| A₄ : \textbf{forall} Γ α β,\ \ \ Γ ⊢ α ∧ β → α\\
| A₅ : \textbf{forall} Γ α β,\ \ \ Γ ⊢ α ∧ β → β\\
| A₆ : \textbf{forall} Γ α β,\ \ \ Γ ⊢ α → α ∨ β\\
| A₇ : \textbf{forall} Γ α β,\ \ \ Γ ⊢ β → α ∨ β\\
| A₈ : \textbf{forall} Γ α β γ, Γ ⊢ (α → γ) → (β → γ) → α ∨ β → γ\\
| A₉ : \textbf{forall} Γ α,\ \ \ \ \ Γ ⊢ ⊥ → α\\
| R₁ : \textbf{forall} Γ α,\ \ \ \ \ Γ ∈ α -> Γ ⊢ α\\
| R₂ : \textbf{forall} Γ α β,\ \ \ Γ ⊢ α → β -> Γ ⊢ α -> Γ ⊢ β\\
\textbf{where} "Γ ⊢ α" := (Deduction Γ α).\\
\\
\textbf{Notation} "Γ ⊢ α" := (Deduction Γ α) (at level 110).
}
\end{mdframed}
\hfill\break
Para ilustração do uso do tipo indutivo \texttt{\footnotesize\textbf{Deduction}}, abaixo fornecemos a prova da identidade $\Gamma\vdash\alpha\to\alpha$.
A prova tem uma relação de um para um com a prova apresentada anteriormente, tirando-se o fato desta ser feita de baixo para cima, da conclusão em direção ao assumido.
Isso acontece porque, via de regra, transformamos a meta de prova em vez das assunções das provas na aplicação.
Nada impediria, entretanto, que o oposto fosse feito.
Abaixo, usamos o comando \texttt{\footnotesize\textbf{apply}} para a aplicar regras de dedução à meta de prova.
Ao fazermos isso, estamos dizendo que esta conclusão foi gerada a partir da aplicação da dita regra.
\hfill\break
\begin{mdframed}
\noindent
\texttt
{\noindent\footnotesize\textbf{Lemma} identity : \textbf{forall} Γ α, Γ ⊢ α → α.\\
\textbf{Proof}.\\
\phantom{\ \ \ \ }\textbf{intros} Γ α.\\
\phantom{\ \ \ \ }\textbf{apply} R₂ \textbf{with} (α → α → α).\\
\phantom{\ \ \ \ }+ \textbf{apply} R₂ \textbf{with} (α → (α → α) → α).\\
\phantom{\ \ \ \ }\phantom{\ \ \ \ }* \textbf{apply} A₂.\\
\phantom{\ \ \ \ }\phantom{\ \ \ \ }* \textbf{apply} A₁.\\
\phantom{\ \ \ \ }+ \textbf{apply} A₁.\\
\textbf{Qed}.
}
\end{mdframed}
\hfill\break
Uma vez definido o sistema intuicionista, podemos definir as funções de tradução do sistema intuicionista ao sistema modal $\mathbf{S4}$, definido por \cite{Silveira}.
A tradução $\bullet^\nec$ e a tradução $\bullet^\circ$, agora chamadas \texttt{\footnotesize\textbf{square}} e \texttt{\footnotesize\textbf{circle}}, foram definidas com apenas duas alterações.
Primeiramente, ambas as funções agora recebem um argumento a mais, que identifica a modalidade usada, uma vez que estamos usando uma biblioteca para sistema multimodais.
Em segundo lugar, a contradição $\bot$ passou a ser mapeada para $a\wedge\neg a$ para algum $a\in\mathcal{P}$, uma vez que esta biblioteca não traz a contradição como um operador primitivo.
\hfill\break
\begin{mdframed}
\noindent
\texttt
{\noindent\footnotesize\textbf{Fixpoint} square (α : Propositional) (i : Index) : Multimodal :=\\
\textbf{match} α \textbf{with}\\
| ⊥\ \ \ \ \ => \#1 ∧ ¬\#1\\
| \#a\ \ \ \ => [i] \#a\\
| α ∧ β => (square α i) ∧ (square β i)\\
| α ∨ β => (square α i) ∨ (square β i)\\
| α → β => [i]((square α i) → (square β i))\\
\textbf{end}.\\
\\
\textbf{Fixpoint} circle (α : Propositional) (i : Index) : Multimodal :=\\
\textbf{match} α \textbf{with}\\
| ⊥\ \ \ \ \ => \#1 ∧ ¬\#1\\
| \#a\ \ \ \ => \#a\\
| α ∧ β => (circle α i) ∧ (circle β i)\\
| α ∨ β => [i](circle α i) ∨ [i](circle β i)\\
| α → β => [i](circle α i) → (circle β i)\\
\textbf{end}.
}
\end{mdframed}
\hfill\break
Faz-se preciso definir novos construtores de tipos para representar conjuntos cujas sentenças são resultados da aplicação de alguma função ou operação, que foram chamados de \texttt{\footnotesize\textbf{Imboxed}}, \texttt{\footnotesize\textbf{Squared}} e \texttt{\footnotesize\textbf{Circled}}.
Cada um desses possui apenas um construtor, nomeados \texttt{\footnotesize\textbf{Imboxing}}, \texttt{\footnotesize\textbf{Squaring}} e \texttt{\footnotesize\textbf{Circling}}, nesta ordem.
\texttt{\footnotesize\textbf{Imboxing}} recebe como argumento um conjunto de sentenças multimodais de tipo \texttt{\footnotesize\textbf{Multimodal -> Prop}} e prefixa a necessidade a cada uma destas sentenças, ou seja, representa o conjunto $\nec\Gamma$. \texttt{\footnotesize\textbf{Squaring}} e \texttt{\footnotesize\textbf{Circling}} recebem como argumento conjuntos de sentenças intuicionistas de tipo \texttt{\footnotesize\textbf{Propositional -> Prop}} e aplicam os seus elementos às funções de tradução \texttt{\footnotesize\textbf{square}} e \texttt{\footnotesize\textbf{circled}}, ou seja, representam os conjuntos $\Gamma^\nec$ e $\Gamma^\circ$.
\hfill\break
\begin{mdframed}
\noindent
\texttt
{\noindent\footnotesize\textbf{Inductive} Imboxed Γ i : Multimodal -> \textbf{Prop} :=\\
| Imboxing : \textbf{forall} α , Γ α  -> Imboxed Γ i ([i]α).
\\
\\
\textbf{Inductive} Squared Γ i : Multimodal -> \textbf{Prop} :=\\
| Squaring : \textbf{forall} α , Γ α  -> Squared Γ i (square α i).
\\
\\
\textbf{Inductive} Circled Γ i : Multimodal -> \textbf{Prop} :=\\
| Circling : \textbf{forall} α , Γ α  -> Circled Γ i (circle α i).
}
\end{mdframed}
\hfill\break
Agora que as definições precisas para o desenvolvimento desta formalização foram feitos, partiremos pra uma enumeração das principais asserções provadas, com considerações pontuais.
Não serão apresentadas as provas das asserções por questões de brevidade.
As provas, entretanto, são paralelas às provas apresentadas anteriormente, a não ser quando destacado o oposto.
\hfill\break
\begin{mdframed}
\texttt{\noindent\footnotesize\textbf{Theorem} generalization : \textbf{forall} Γ α i,\\Imboxed Γ i ⊢ α -> Imboxed Γ i ⊢ [i]α.}
\end{mdframed}

\begin{mdframed}
\texttt{\noindent\footnotesize\textbf{Theorem} stability : \textbf{forall} Γ α i,\\Γ ⊢ square α i → [i](square α i).}
\end{mdframed}

\begin{mdframed}
\texttt{\noindent\footnotesize\textbf{Theorem} biimplication : \textbf{forall} Γ α i,\\Γ ⊢ [i](circle α i) ↔ square α i.}
\end{mdframed}

\begin{mdframed}
\texttt{\noindent\footnotesize\textbf{Theorem} equivalence : \textbf{forall} Γ α i,\\Imboxed (Circled Γ i) i ⊢ circle α i <-> Squared Γ i ⊢ square α i.}
\end{mdframed}

\begin{mdframed}
\texttt{\noindent\footnotesize\textbf{Theorem} soundness : \textbf{forall} Γ α i,\\Γ ⊢ α -> Squared Γ i ⊢ square α i.}
\end{mdframed}

\begin{mdframed}
\texttt{\noindent\footnotesize\textbf{Theorem} soundness : \textbf{forall} Γ α i,\\Γ ⊢ α -> Imboxed (Squared Γ i) i ⊢ circle α i.}
\end{mdframed}

        \chapter{Conclusão}
    Os sistemas modais possuem atributos interessantes para a representação de efeitos computacionais, sobretudo o sistema $\mathfrak{M}$ definido neste trabalho.
    Uma linguagem baseada nesse sistema, podendo representar efeitos monadicamente, imerge e pode ser imersas em continuações~\citep{Filinski.1994}.
    O estilo de passagem por continuações trata-se de uma das diversas representações usadas em compiladores.
    Deste modo, este sistema possui interesse no ponto de vista de compilação.
    Motivados por esta relação formalizamos duas traduções do sistema intuicionista $\mathfrak{I}$ ao sistema $\mathfrak{M}$ encontradas em~\cite{Troelstra+Schwichtenberg.2000} no provador de teoremas \emph{Rocq}.
    Esperamos que estes desenvolvimentos fundamentem investigações futuras acerca do uso de uma linguagem baseada no sistema $\mathfrak{M}$ durante a compilação.

    \vspace{.5\baselineskip}
    Apresentamos neste trabalho apresentou noções gerais sobre sistemas formais e sobre traduções entre tais sistemas.
    Com base nisso, foram definidos os sistemas intuicionista e modais, bem como duas traduções da linguagem do primeiro à linguagem do segundo.
    Foram apresentadas demonstrações da correção de ambas as traduções e de sua interderivabilidade.
    Para tanto, foram usados um conjunto de lemas e teoremas.
    Todas as demonstrações foram formalizadas no provador de teoremas \emph{Rocq}.
    Deste modo, cumpriram-se as metas definidas na introdução.

    \vspace{.5\baselineskip}
    Sugerimos diversos trabalhos futuros. Primeiramente, a demonstração da completude das traduções apresentadas neste trabalho e sua formalização assistida por computador.
    Em segundo lugar, sugerimos que o mesmo seja feito para a tradução do sistema laxo $\mathfrak{L}$ ao sistema $\mathfrak{M}$, conforme apresentada por~\cite{Pfenning+Davies.2001}. 
    Outra tradução de interesse trata-se da tradução de~\cite{Fairtlough+Mendler.1997} do sistema $\mathfrak{L}$ a um sistema bimodal $\langle\mathfrak{M},\mathfrak{M}\rangle$, uma vez que os desenvolvimentos de~\cite{Nunes+Roggia+Torrens.2025} permitem a fusões entre sistemas modais.
    Por fim, sugerimos investigações acerca do uso de uma de linguagem de representação baseada no sistema $\mathfrak{M}$ para uso em compiladores.

    \vspace{.5\baselineskip}
    Para as demonstrações de completude, sugerimos que estas deixem de se basear nas demonstrações de~\cite{Troelstra+Schwichtenberg.2000} e passem a se basear na demonstração de~\cite{Flagg+Friedman.1986}.
    Este abandono deve-se ao uso de propriedades de sequentes na demonstração, que deveriam ser acomodadas ao sistema de dedução usado neste trabalho.
    A sugestão desta demonstração, que se baseia na definição de uma contratradução a uma das traduções abordadas neste trabalho, acontece por esta ser feita construtivamente.
    Com isso, os autores reduzem o problema de demonstrar a completude da tradução ao problema de demonstrar a correção da contratradução, feito por indução sobre o tamanho da demonstração.

    \vspace{.5\baselineskip}
    Este trabalho foi apresentado como \emph{poster} durante a \emph{XXII Brazilian Logic Conference}, ocorrido entre os dias doze e dezesseis de maio do corrente ano em Serra Negra --- São Paulo, e foi parcialmente apoiado pela \emph{Fundação de Amparo à Pesquisa e Inovação do Estado de Santa Catarina} --- Fapesc.


    \bibliographystyle{bibliography}
    \bibliography{bibliography}
\end{document}
