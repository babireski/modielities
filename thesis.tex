\documentclass{report}

\usepackage{amsthm}
\usepackage{amsmath}
\usepackage{amssymb}
\usepackage{epigraph}
\usepackage{mathtools}
\usepackage{styles/modalities}
\usepackage{ragged2e}
\usepackage{hyperref}
\usepackage{proof}
\usepackage[round]{natbib}
\usepackage[french, brazil]{babel}
\usepackage{styles/fitch}
\usepackage{styles/cases}
\usepackage{styles/functions}

\usepackage[T1]{fontenc}

\usepackage[a4paper, margin=1.5in]{geometry}

\hypersetup{
    pdftitle={Uma formalização da interpretação modal do sistema intuicionista},
    pdfauthor={Elian Babireski},
    colorlinks=true,
    linkcolor=blue,
    citecolor=blue,
    filecolor=blue,
    urlcolor=blue
}

\usepackage[
    left = \flqq{},% 
    right = \frqq{},% 
    leftsub = \flq{},% 
    rightsub = \frq{} %
]{dirtytalk}

\newcommand{\entails}{\mathrel{\vdash}}
\newcommand{\point}{\mathpunct{.}}

\newtheorem*{notation}{Notação}
\newtheorem{definition}{Definição}
\newtheorem{lemma}{Lema}
\newtheorem{theorem}{Teorema}

\usepackage{titlesec}

\titleformat{\chapter}[block]
  {\normalfont\huge\bfseries}{\thechapter.}{1em}{\Huge}
\titlespacing*{\chapter}{0pt}{-19pt}{20pt}

\usepackage{setspace}
\onehalfspacing{}

\begin{document}
    \title{Uma formalização da interpretação modal do sistema intuicionista}
    \author{Elian Babireski}
    \date{2024}

    \maketitle

    \begin{abstract}
        Resumo aqui.
    \end{abstract}

    \tableofcontents

    \setlength\epigraphwidth{.5\textwidth}
\setlength\epigraphrule{0pt}

\vspace*{\fill}
\epigraph{\justifying\say{\say{Oh, you can't help that,} said the Cat: \say{we're all mad here. I'm mad. You're mad.} \say{How do you know I'm mad?} said Alice. \say{You must be,} said the Cat, \say{or you wouldn't have come here.}}}{--- Lewis Carroll, \textit{Alice in Wonderland}}
    \chapter{Introdução}

As lógicas modais consistem em um conjunto de extensões da lógica clássica que contam com a adição de um ou mais operadores, chamados modalidades, que qualificam sentenças. No caso do sistema \textbf{S4}, são adicionadas as modalidades de necessidade ($\nec$) e possibilidade ($\pos$) em conjunto à regra da necessitação\footnote{Se $\vdash A$ então $\vdash \nec A$} e os axiomas $\mathbf{K}\text{: } \nec(A \to B) \to \nec A \to \nec B$, $\mathbf{T}\text{: } \nec A \to A$ e $\text{\textbf{4}: } \nec A \to \nec \nec A$ \citep{Troelstra}. Ademais, pode-se derivar nesse sistema, por meio da dualidade entre as modalidades\footnote{$\pos A \equiv \neg \nec \neg A $}, sentenças duais aos axiomas \textbf{T} e \textbf{4}, sendo elas $\mathbf{T}_\meddiamond \text{: } A \to \pos A$ e $\mathbf{4}_\meddiamond \text{: } \pos \pos A \to \pos A$, respectivamente~\cite{Zach}.

As mônadas ganharam destaque na área de linguagens de programação desde que~\cite{Moggi} formalizou uma metalinguagem que faz uso dessas estruturas para modelar noções de computação --- como parcialidade, não-determinismo, exceções e continuações --- de uma maneira puramente funcional. Pode-se notar uma grande semelhança entre as sentenças $\mathbf{T}_\meddiamond$ e $\mathbf{4}_\meddiamond$ e as transformações naturais monádicas $\mathbf{\eta:} 1_C \to T$ e $\mathbf{\mu:} T^2 \to T$, respectivamente. Nesse sentido,~\cite{Pfenning} demonstraram que se pode traduzir essa metalinguagem para o sistema \textbf{S4} da lógica modal, pelo qual se torna interessante analisar esse sistema como uma linguagem de programação sob a ótica do isomorfismo de Curry-Howard.

~\cite{Troelstra} apresentam duas traduções equivalentes da lógica intuicionista para o sistema \textbf{S4} da lógica modal, sendo um deles correspondente a uma abordagem \textit{call-by-name} e outra a um abordagem \textit{call-by-value}. Tais traduções possuem grande similaridade com as traduções da lógica intuicionista para a lógica linear definidas por~\cite{Girard}. Essas traduções equivalem à tradução por negação dupla que, por sua vez, equivalem a traduções \textit{continuation-passing style} (CPS) em compiladores por meio do isomorfismo de Curry-Howard~\citep{Reynolds}, o que torna esse tema interessante no ponto de vista de compilação.

Durante grande parte da história, provas lógicas e matemáticas eram validadas manualmente pela comunidade acadêmica, o que muitas vezes --- a depender do tamanho e complexidade da prova --- se mostrava ser um trabalho complexo e sujeito a erros. Hoje em dia, exitem \textit{softwares} chamados assistentes de provas que permitem verificar --- graças ao isomorfismo de Curry-Howard --- a corretude de provas~\citep{Chlipala}. O assistente de provas que será usado neste trabalho é o \textsc{coq}, que utiliza o cálculo de construções indutivas e um conjunto axiomático pequeno para permitir a escrita de provas simples e intuitivas~\citep{Coq}.

    \section{Objetivos}
    Este trabalho consiste numa continuação do desenvolvimento da biblioteca de formalização de sistemas modais normais iniciado por~\cite{Silveira} e posteriormente expandida de forma a permitir a fusão de sistemas modais por~\cite{Nunes}. Nele, formalizaremos as traduções do sistema intuicionista ao sistema modal $\mathbf{S4}$ no asssitente de provas \textsc{coq} e provaremos suas propriedades. Uma formalização de traduções entre sistemas de dedução similar a nossa foi feita por~\cite{Sehnem}, neste caso tendo como alvo o sistema linear de~\cite{Girard}. Todas as formalizações citadas acima deram-se no assistente de provas \textsc{coq}, o mesmo assistente usado neste trabalho. Como objetivos específicos, listamos:

    \begin{itemize}
        \item Fornecer uma introdução ao conceito de sistemas de dedução;
        \item Fornecer uma introdução ao conceito de traduções entre sistemas;
        \item Fornecer uma introdução ao sistema intuicionista;
        \item Fornecer uma introdução aos sistemas modais, em especial o $\mathbf{S4}$;
        \item Apresentar as traduções do sistema intuicionista ao sistema $\mathbf{S4}$;
        \item Provar manualmente a correção das traduções providas bem como outras propriedades pertinentes;
        \item Formalizar as provas no provador de teoremas interativo \textsc{coq}.
    \end{itemize}

    \section{Estruturação}
    Estruturaremos este trabalho em cinco partes. A parte \textbf{(1)} trata-se desta introdução. A parte \textbf{(2)} consiste numa fundamentação de conceitos basilares ao desenvolvimento deste trabalho, notadamente os conceitos de \emph{sistemas de dedução}, \emph{traduções} e \emph{provadores de teoremas}. A parte \textbf{(3)} apresenta as definições dos sistemas e traduções relevantes a este trabalho. Na parte \textbf{(4)} são provadas todas as propriedades abarcadas no escopo deste trabalho. Por fim, a parte \textbf{(5)} compreende considerações parciais acerca do desenvolvido até o momento.
    \chapter{Fundamentação}

Nesta parte do trabalho, serão apresentadas definições gerais que fundamentarão as definições mais estritas que serão apresentadas futuramente. Notadamente, fundamentaremos as noções de sistemas e traduções. Ademais, discorreremos acerca da noção de provadores, que serão usados para certificar as provas apresentadas posteriormente. Antes disso, entretanto, introduziremos duas notações que serão usadas copiosamente, uma para o conjunto das partes e outra para sucessões.

\begin{notation}
    Seja $A$ um conjunto, $\wp(A)$ denota o conjunto $\set{X\mid X\subseteq A}$.
\end{notation}

\begin{notation}
    Seja $i\in\mathbb{N}^+$ e $n\in\mathbb{N}$, $\sequence{a_i\mid i\leq n}$ denota uma sucessão de $n$ elementos de modo que o elemento $a_i$ encontra-se na posição $i$.
\end{notation}

\section{Sistemas}

Sistemas de dedução buscam formalizar e sistematizar o processo de razoamento. Estudos acerca disso datam da antiguidade, dentre os quais destaca-se~\cite{Aristotle}. Considera-se que os estudos modernos neste campo foram, dentre outras pessoas, fundados por~\cite{Frege} e continuados por~\cite{Russel-A,Russel-B,Russel-C}. Estas investigações --- bem como outras --- levaram ao desenvolvimento do sistema hoje tido como padrão. Posteriormente a isso, viu-se o surgimento de diversos sistemas não-padrões, fato que --- conforme~\cite{Beziau-B} --- justifica uma conceituação de sistema de dedução, que apresentaremos nesta seção.

Ainda segundo~\cite{Beziau-B}, os primeiros desenvolvimentos neste sentido foram feitos por~\cite{Tarski}, que define o conceito de dedução com base num operador de fecho $C\mathrel{:}\wp(\mathcal{L})\to\wp(\mathcal{L})$, sendo $\mathcal{L}$ um conjunto qualquer. Neste trabalho entretanto usaremos a definição proposta por~\cite{Beziau} baseada numa relação de dedução ${\vdash}\subseteq\wp(\mathcal{L})\times\mathcal{L}$, uma vez que, por sua simplicidade, não traz elementos irrelevantes aos intuitos deste. Cabe destacar, conforme apontam~\cite{Font}, que ambas as definições são equivalentes\footnote{Destaca-se, entretanto, que a definição de~\cite{Tarski} requer a satisfação de postulados não requeridos por~\cite{Beziau}, sendo portanto menos geralista.}, uma vez que $\Gamma\entails\alpha$ se e somente se $\alpha\in C(\Gamma)$.

\begin{definition}[Sistema]
    Um sistema de dedução consiste num par $\mathbf{L} = \sequence{\mathcal{L}, \vdash}$, onde $\mathcal{L}$ consiste em um conjunto e ${\vdash}\subseteq\wp(\mathcal{L})\times\mathcal{L}$ em uma relação sobre o produto cartesiano do conjunto das partes de $\mathcal{L}$ e o conjunto $\mathcal{L}$, sem demais condições.
    \qed{}
\end{definition}

Conforme~\cite{Beziau} aponta, a qualidade e quantidade dos elementos de um sistema $\mathbf{L}=\sequence{\mathcal{L}, \vdash}$ não são especificados, portanto sendo esta uma definição de grande generalidade. Neste sentido, com base no escopo deste trabalho, restringiremos a definição do conjunto $\mathcal{L}$ --- dito \emph{linguagem} --- a linguagens proposicionais. Os elementos destas, aos quais daremos o nome de \emph{sentenças}, notabilizam-se por serem formadas por \emph{letras} --- que consistem em proposições indivisas --- e \emph{operadores} --- que podem gerar proposições maiores a partir de proposições menores. Ao par formado por letras e operadores daremos o nome \emph{assinatura}, conforme abaixo.

\begin{definition}[Assinatura]
    Uma assinatura proposicional consiste num par $\Sigma=\sequence{\mathcal{P},\mathcal{C}}$, onde $\mathcal{P}$ consiste num conjunto letras e $\mathcal{C}=\bigcup\set{\mathcal{C}_i\mid i\in\mathbb{N}}$ num conjunto de operadores de modo que $\mathalpha{\bullet}\in\mathcal{C}_n$ se e somente se $\mathalpha{\bullet}$ possuir aridade $n$.
    \qed{}
\end{definition}

\begin{notation}
    Seja $\mathcal{C}$ um conjunto de operadores, $\mathalpha{\bullet}^n$ denota um operador $\mathalpha{\bullet}\in\mathcal{C}_n$.
\end{notation}

Podemos interpretar os conjuntos $\mathcal{P}$ e $\mathcal{C}$ de uma assinatura $\Sigma=\sequence{\mathcal{P},\mathcal{C}}$ como contrutores de sentenças.
Neste sentido, o conjunto $\mathcal{C}_0$ assemelha-se mais ao conjunto $\mathcal{P}$, uma vez que seus elementos --- ditos \emph{constantes} --- não geram sentenças maiores partindo de sentenças menores.
Nota-se que uma assinatura constitui um elemento suficiente para definirmos indutivamente a linguagem de um sistema, conforme definido abaixo de maneira similar a~\cite{Franks}.
Por fim, destacamos que, para todos os sistemas apresentados neste trabalho, usaremos o conjunto de letras $\mathcal{P}=\set{p_i\mid i\in\mathbb{N}}$ e letras romanas em caixa-baixa para representar seus elementos.

\begin{definition}[Linguagem]
    Seja $\Sigma=\sequence{\mathcal{P},\mathcal{C}}$ uma assinatura proposicional. Uma linguagem proposicional $\mathcal{L}$ induzida a partir de $\Sigma$ consiste no menor conjunto de sentenças bem-formadas induzido a partir das seguintes regras:
    \begin{enumerate}[label=\textbf{\emph{(\alph*)}}, left=\parindent]
        \item$\mathcal{P}\subseteq\mathcal{L}$
        \item\text{Se }$\mathalpha{\bullet}\in\mathcal{C}_n\text{ e }\set{\varphi_i\mid i\leq n}\subseteq\mathcal{L}\text{, então }\bullet\sequence{\varphi_i\mid i\leq n}\in\mathcal{L}$.\qed{}
    \end{enumerate}
\end{definition}

Neste trabalho, representaremos sentenças por letras gregas em caixa-baixa e conjuntos de sentenças por letras gregas em caixa-alta.\footnote{Desconsiderando-se o $\Sigma$, usado para representar assinaturas.}
Ademais, impõe-se definir a noção de profundidade de uma sentença. Esta noção, em termos simples, consiste no comprimento do maior ramo da construção da dada sentença. A definição provida abaixo consiste numa generalização para quaisquer aridades da definição dada por~\cite{Troelstra}. Usaremos essa definição futuramente para fazer demonstrações por meio provas indutivas sobre esta propriedade.

\begin{definition}[Profundidade]
    Seja $\mathbf{L} = \sequence{\mathcal{L}, \vdash}$ um sistema com linguagem induzida a partir de uma assinatura $\Sigma=\sequence{\mathcal{P},\mathcal{C}}$. Considerando-se uma proposição $a\in\mathcal{P}$, um operador ${\bullet}\in\mathcal{C}$ e uma aridade $n>0$, definimos a profundidade $|\alpha|$ de uma sentença $\alpha\in\mathcal{L}$ indutivamente da seguinte maneira:
    \begin{align*}
        |a|&\coloneqq 0\\
        |{\bullet^0}|&\coloneqq 0\\
        |{\bullet^n\sequence{\varphi_i\mid i\leq n}}|&\coloneqq\max\set{|\varphi_i|\mid i\leq n}+1.
        \tag*{\qed} 
    \end{align*}
\end{definition}

Com isso, encerram-se as definições relacionadas a linguagens de sistemas de dedução. Agora, apresentaremos definições relacionadas a relações de dedução, que gozam da mesma generalidade dada a liguagens. Deste modo, a relação $\mathalpha{\vdash}$ pode ser tanto uma relação de \emph{derivação} --- definida sintaticamente --- quanto uma relação de \emph{satisfação}\footnote{Sendo esta denotada por $\mathalpha{\vDash}$.} --- definida semanticamente. Neste trabalho, serão abordados apenas sistemas definidos sobre relações de derivação. Cabe destacar, entretanto, que nada na definição de tradução impede que esta seja feita sobre relações de satisfação, conforme veremos com mais detalhes futuramente.

Neste trabalho, definiremos a relações de dedução baseada em axiomatizações, ou seja, em conjuntos de \emph{axiomas} --- sentenças postuladas como verdadeiras --- e conjuntos de \emph{regras de dedução} --- que permitem derivar mais sentenças verdadeiras caso certas condições sejam satisfeitas. Axiomatizações consistem numa abordagem hilbertiana de dedução que, segundo~\cite{Troelstra}, distinguem-se por conter um conjunto reduzido de regras de dedução que nunca descartam premissas. Ainda baseando-se em~\cite{Troelstra} e em contraste a~\cite{Frege} e~\cite{Hilbert-A, Hilbert-B}, preferiremos esquemas de axiomas a axiomas individuais de modo a eliminarmos a necessidade de instanciações.

\begin{definition}[Axiomatização]
    Seja $\mathbf{L}=\sequence{\mathcal{L},\vdash}$ um sistema. Uma axiomatização para o sistema $\mathbf{L}$ consiste num par $\mathcal{H}=\sequence{\mathcal{A},\mathcal{R}}$, sendo $\mathcal{A}$ um conjunto de esquemas de axiomas e $\mathcal{R}$ um conjunto de regras de dedução.
\end{definition}


Neste trabalho, consideraremos axiomatizações definidas em relação a conjuntos de premissas $\Gamma\subseteq\mathcal{L}$. 
Por este motivo, todas as axiomatizações apresentadas futuramente neste trabalho conterão implicitamente a \emph{regra da premissa} $\mathbf{P}$ --- sendo esta regra definida como se $\alpha\in\Gamma$, então $\Gamma\entails\alpha$.
Similarmente, faremos o mesmo para a \emph{regra do enfraquecimento} $\mathbf{E}$ --- definida aqui como se $\Gamma\entails\alpha$, então $\Gamma\cup\Delta\entails\alpha$ ---, tendo em vista que todos os sistemas apresentados neste trabaho são estruturais. Assim, tendo-se claro o conceito de axiomatização, podemos finalmente o conceito de dedução.

\begin{definition}[Dedução]
    Seja um sistema $\mathbf{L} = \sequence{\mathcal{L},{\vdash}}$ com uma relação de dedução definida sobre uma axiomatização $\mathcal{H} = \sequence{\mathcal{A},\mathcal{R}}$ e  seja um conjunto de sentenças $\Gamma\cup\set{\alpha}\subseteq\mathcal{L}$.
    A dedução $\Gamma\vdash\alpha$ vale se e somente se houver sucessão de sentenças $\sequence{\varphi_i\in\mathcal{L}\mid i\leq n}$ de modo que $\varphi_n=\alpha$ e que cada sentença $\varphi_i$ ou tenha sido gerada ou por algum esquema $\mathbf{A}\in\mathcal{A}$ ou pela aplicação de alguma regra $\mathbf{R}\in\mathcal{R}$ a sentenças anteriores.
    \qed{}
\end{definition}

\section{Traduções}

Traduções entre sistemas consistem em funções que mapeiam sentenças de um sistema a sentenças de outro, garantindo certas propriedades. As propriedades a serem garantidas variam e ainda são discutidas na literatura, deixando a definição exata de tradução --- assim como houve com a definição de sistema --- varie de acordo com a predileção e as necessidades de cada autor. Nesta seção, serão abordadas historicamente noções de tradução entre sistemas, bem como serão definidos e nomeados os conceitos de tradução que serão usados no restante deste trabalho.

\begin{definition}[Condições]
    Chamaremos a condição $\varnothing\entails_\mathbf{A}\alpha$ implica em $\varnothing\entails_\mathbf{B}\alpha^*$ de correção fraca e a condição $\varnothing\entails_\mathbf{B}\alpha^*$ implica em $\varnothing\entails_\mathbf{A}\alpha$ de completude fraca. Analogamente, considerando-se dedução com premissas, chamaremos a condição $\Gamma\entails_\mathbf{A}\alpha$ implica em $\Gamma^*\entails_\mathbf{B}\alpha^*$ de correção forte e a condição $\Gamma^*\entails_\mathbf{B}\alpha^*$ implica em $\Gamma\entails_\mathbf{A}\alpha$ de completude forte.
\end{definition}

Historicamente, autores usaram diferentes combinações das condições apresentadas acima e, em certos casos, outras. Neste trabalho, adotaremos uma noção forte de tradução que requer tanto a correção forte quanto a completude forte, conforme~\cite{Coniglio}. Definiremos, ainda, uma notação que nos permite aplicar sucintamente a tradução a todos os elementos de um conjunto.

\begin{definition}[Tradução] 
    Uma sentença $\varphi$ de um sistema $\mathbf{A} = \langle \mathcal{L}_\mathbf{A}, \vdash_\mathbf{A}\rangle$ pode ser traduzida a uma sentença $\varphi^*$ em um sistema $\mathbf{B} = \langle \mathcal{L}_\mathbf{B}, \vdash_\mathbf{B} \rangle$ caso exista uma função $\bullet^* : \mathcal{L}_\mathbf{A} \to \mathcal{L}_\mathbf{B}$ que garanta que $\Gamma \vdash_\mathbf{A} \varphi \Leftrightarrow \Gamma^* \vdash_\mathbf{B} \varphi^*$.
    \qed{}
\end{definition}

\begin{notation}
    Seja $\Gamma\in\wp(\mathcal{L}_\mathbf{A})$ um conjunto de sentenças bem-formadas e $\bullet^*\mathrel{:}\mathcal{L}_\mathbf{A}\to\mathcal{L}_\mathbf{B}$ uma tradução. $\Gamma^*$ denota o conjunto $\set{\alpha^*\mid\alpha\in\Gamma}\in\wp(\mathcal{L}_\mathbf{B})$, ou seja, a aplicação da tradução a todos os elementos do conjunto $\Gamma$.
    \qed{}
\end{notation}

A primeira tradução entre dois sistemas conhecida na literatura foi definida por~\cite{Kolmogorov} como uma maneira de demonstrar que o uso da \emph{lei do terceiro excluso}\footnote{Definido como $\entails\alpha\vee\neg\alpha$.} não leva a contradições. Essa definição consiste basicamente em prefixar uma dupla negação a cada elemento da construção de uma dada sentença, motivo pelo qual chamaremos essa tradução de \emph{tradução de negação dupla} \citep{Coniglio}. Essa mesma tradução foi também descoberta independentemente por Gödel e por Getzen. Curiosamente, essa tradução mostra-se relevante para o escopo deste trabalho, uma vez que consiste na contraparte da passagem por continuações segundo a interpretação prova-programa.

\begin{definition}[$\bullet^\neg$] Define-se a tradução $\bullet^\neg:\mathcal{L}_\mathbf{C}\to\mathcal{L}_\mathbf{I}$ do sistema clássico ao sistema intuicionista indutivamente da seguinte maneira:
    \begin{align*}
        p^\neg&\coloneqq\neg\neg p\\
        \bot^\neg&\coloneqq\bot\\
        {(\varphi\wedge\psi)}^\neg&\coloneqq\neg\neg(\varphi^\neg \wedge \psi^\neg)\\
        {(\varphi\vee\psi)}^\neg&\coloneqq\neg\neg (\varphi^\neg \vee \psi^\neg)\\
        {(\varphi\to\psi)}^\neg&\coloneqq\neg\neg (\varphi^\neg \to \psi^\neg)
        \tag*{\qed} 
    \end{align*}
\end{definition}

\section{Provadores}

A primeira prova de destaque a ser realizada com grande uso de computadores foi a do teorema das quatro cores\footnote{Que afirma que \emph{qualquer mapa planar tem uma quatro-coloração}.}, feita por~\cite{Appel}, motivado pela grande quantidade de casos a serem analisados.
Conforme~\cite{Wilson} afirma, esta prova foi por uns recebida com entusiasmo e por outros, devido ao uso de computadores, com cetistismo e desapontamento.
Dentre aqueles que compartilharam destas visões opositoras, destaca-se~\cite{Tymoczko}.
Ainda segundo~\cite{Wilson}, o teorema tornou-se mais aceito com o passar do tempo e foi, posteriormente, formalizado em um provador de teoremas por~\cite{Gonthier}.

Provadores de teoremas consistem em programas de computador que verificam a validade de teoremas. Dentre estes, podemos destacar as classes dos provadores \emph{automados} e dos provadores \emph{interativos}. Os primeiros buscam provar teoremas de maneira que requeira a menor quantidade de intervenção humana, enquanto os segundos --- que ganharam destaque depois das limitações dos primeiros ficarem evidentes --- delegam-se a verificar rigorosamente provas desenvolvidas por humanos em sua linguagem~(\babireski{Citação}). Formalizaremos das provas apresentadas neste trabalho no provador de teoremas interativo \textsc{coq}.


    \chapter{Sistemas}
        \section{Intuicionismo}
    \begin{definition}[$\mathcal{L}_\mathbf{I}$]
        A linguagem do sistema intuicionista, denotada $\mathcal{L}_\mathbf{I}$, pode ser induzida a partir da assinatura $\Sigma_\mathbf{I}=\sequence{\mathcal{P},\mathcal{C}_\mathbf{I}}$, onde $\mathcal{C}_\mathbf{I}=\set{\bot^0,\wedge^2,\vee^2,\to^2}$.
        \qed{}
    \end{definition}

    \begin{notation}
        Serão usadas as seguintes abreviações:
        \begin{align*}
            \top&\coloneqq\bot\to\bot\\
            \neg\alpha&\coloneqq\alpha\to\bot\\
            \alpha\leftrightarrow\beta&\coloneqq(\alpha\to\beta)\wedge(\beta\to\alpha)
        \end{align*}
    \end{notation}

    \begin{definition}
        A axiomatização do sistema intuicionista consiste no conjunto de esquemas de axiomas $\mathcal{A}=\set{\mathbf{A}_i\mid i\in[1,8]\vee i=\bot}$ e no conjunto de regras $\mathcal{R}=\set{\mathbf{R_1}}$, definidos abaixo:
        \begin{alignat*}{3}
            & \mathbf{A_1}\quad && \alpha\to\beta\to\alpha \\
            & \mathbf{A_2}\quad && (\alpha\to\beta\to\gamma)\to(\alpha\to\beta)\to(\alpha\to\gamma) \\
            & \mathbf{A_3}\quad && \alpha\to\beta\to\alpha\wedge\beta \\
            & \mathbf{A_4}\quad && \alpha\wedge\beta\to\alpha \\
            & \mathbf{A_5}\quad && \alpha\wedge\beta\to\beta \\
            & \mathbf{A_6}\quad && \alpha\to\alpha\vee\beta \\
            & \mathbf{A_7}\quad && \beta\to\alpha\vee\beta \\
            & \mathbf{A_8}\quad && (\alpha\to\gamma)\to(\beta\to\gamma)\to(\alpha\vee\beta\to\gamma) \\
            & \mathbf{A_\bot}\quad && \bot\to\alpha \\
            & \mathbf{R_1}\quad && \text{Se }\Gamma\vdash\alpha\text{ e }\Gamma\vdash\alpha\to\beta\text{, então }\Gamma\vdash\beta\text{.} & \tag*{\qed}
        \end{alignat*}   
    \end{definition}

    Daremos nomes aos esquemas e regras acima de modo a facilitar a comunicação no decorrer deste trabalho. Chamaremos o esquema $\mathbf{A_1}$ de esquema da constante e o esquema $\mathbf{A_1}$ de esquema da aplicação.\footnote{Em analogia aos combinadores $\mathbf{K}$ e $\mathbf{S}$.} Ao esquema $\mathbf{A_3}$ daremos o nome de introdução da conjunção, enquanto os esquemas $\mathbf{A_4}$ e $\mathbf{A_5}$ serão chamados de eliminação da conjunção. Analogamente, os esquemas $\mathbf{A_6}$ e $\mathbf{A_7}$ serão chamados de introdução da disjunção, enquanto ao esquema $\mathbf{A_8}$ chamaremos de eliminação da disjunção.
    Por fim, chamaremos $\mathbf{A_\bot}$ de esquema da explosão e $\mathbf{R_1}$ de regra da separação ou \emph{modus ponens}.

        \section{Modal}
    \babireski{\cite{Blackburn} traz uma visão da evolução dos sistemas modais.}

    \begin{definition}[$\mathcal{L}_\mathbf{M}$]
        A linguagem dos sistemas modais, denotada $\mathcal{L}_\mathbf{M}$, consiste no menor conjunto induzido a partir das seguintes regras:
        \begin{align*}
            &\bot\in\mathcal{L}_\mathbf{M} \\
            &\mathcal{P}\subseteq\mathcal{L}_\mathbf{M} \\
            &\alpha\in\mathcal{L}_\mathbf{M}\Rightarrow\nec\alpha\in\mathcal{L}_\mathbf{M} \\
            &\alpha,\beta\in\mathcal{L}_\mathbf{M}\Rightarrow\alpha\circ\beta\in\mathcal{L}_\mathbf{M}\text{, para }\circ\in\set{\wedge,\vee,\to}\text{.}
            \tag*{\qed}
        \end{align*}
    \end{definition}

    \begin{notation}
        Serão usadas as seguintes abreviações:
        \begin{align*}
            \top&\coloneqq\bot\to\bot\\
            \neg\alpha&\coloneqq\alpha\to\bot\\
            \pos\alpha&\coloneqq\neg\nec\neg\alpha\\
            \alpha\fishhook\beta&\coloneqq\nec(\alpha\to\beta)\\
            \alpha\leftrightarrow\beta&\coloneqq(\alpha\to\beta)\wedge(\beta\to\alpha)
        \end{align*}
    \end{notation}

    \begin{notation}
        Seja $\Gamma\in\wp(\mathcal{L}_\mathbf{M})$ um conjunto de sentenças bem-formadas.
        $\nec\Gamma$ denota o conjunto $\set{\nec\alpha\mid\alpha\in\Gamma}\in\wp(\mathcal{L}_\mathbf{M})$, ou seja, a prefixação da negação a todos os elementos do conjunto.
    \end{notation}

    \begin{definition}\label{m-axioms}
        A axiomatização do sistema modal consiste no conjunto de esquemas de axiomas $\mathcal{A}=\set{\mathbf{A}_i\mid i\in[1,8]\vee i=\neg}\cup\set{\mathbf{B_1},\mathbf{B_2},\mathbf{B_3}}$ e no conjunto de regras $\mathcal{R}=\set{\mathbf{R_1},\mathbf{R_2}}$, definidos abaixo:
        \begin{alignat}{3}
            &\mathbf{A_1}\quad&&\alpha\to\beta\to\alpha\label{MA1}\tag*{}\\
            &\mathbf{A_2}\quad&&(\alpha\to\beta\to\gamma)\to(\alpha\to\beta)\to(\alpha\to\gamma)\label{MA2}\tag*{}\\
            &\mathbf{A_3}\quad&&\alpha\to\beta\to\alpha\wedge\beta\label{MA3}\tag*{}\\
            &\mathbf{A_4}\quad&&\alpha\wedge\beta\to\alpha\label{MA4}\tag*{}\\
            &\mathbf{A_5}\quad&&\alpha\wedge\beta\to\beta\label{MA5}\tag*{}\\
            &\mathbf{A_6}\quad&&\alpha\to\alpha\vee\beta\label{MA6}\tag*{}\\
            &\mathbf{A_7}\quad&&\beta\to\alpha\vee\beta\label{MA7}\tag*{}\\
            &\mathbf{A_8}\quad&&(\alpha\to\gamma)\to(\beta\to\gamma)\to(\alpha\vee\beta\to\gamma)\label{MA8}\tag*{}\\
            &\mathbf{A_\neg}\quad&&\neg\neg\alpha\to\alpha\label{MANEG}\tag*{}\\
            &\mathbf{B_1}\quad&&\nec(\alpha\to\beta)\to\nec\alpha\to\nec\beta\label{MB1}\tag*{}\\
            &\mathbf{B_2}\quad&&\nec\alpha\to\alpha\label{MB2}\tag*{}\\
            &\mathbf{B_3}\quad&&\nec\alpha\to\nec\nec\alpha\label{MB3}\tag*{}\\
            &\mathbf{R_1}\quad&&\text{Se }\Gamma\entails\alpha\text{ e }\Gamma\entails\alpha\to\beta\text{, então }\Gamma\entails\beta\label{detachment}\tag*{}\\
            &\mathbf{R_2}\quad&&\text{Se }\entails\alpha\text{, então }\Gamma\entails\nec\alpha\text{.}\tag*{\qed}\label{necessitation} 
        \end{alignat}   
    \end{definition}

    Chamaremos $\mathbf{R_1}$ de \emph{regra da separação} e $\mathbf{R_2}$ de \emph{regra da necessitação}.

        \section{Metateoremas}
    Nesta seção apresentaremos alguns teoremas para os sistemas modais que permitirão simplificar muito as provas apresentadas no decorrer deste trabalho.
    Primeiramente, provaremos a regra da dedução \hyperref[deduction]{$\mathbf{T_1}$} com base na prova apresentada por~\cite{Hakli}.
    Pequenas alterações na prova foram feitas para garantir a adequação com a axiomatização provida na definição \refer{m-axioms}{D}.

    \begin{theorem}\label{deduction}
        $\text{Se }\Gamma\cup\set{\alpha}\vdash\beta\text{, então }\Gamma\vdash\alpha\to\beta$.

        \begin{proof}
            Prova por indução forte sobre o tamanho da sucessão de dedução\footnote{Note que, para a indução forte, não se faz preciso provar nenhuma base \citep{Velleman}.}.
            Assim, suponhamos que o teorema da dedução valha para qualquer sucessão dedução de tamanho $n<k$.
            Demonstraremos, analisando-se os casos, que o teorema da dedução vale para sucessões de dedução de tamanho $n=k+1$.

            \begin{case}
                \textsc{Caso 1.}
                Se a linha derradeira da sucessão de dedução que prova $\Gamma\cup\set{\alpha}\vdash\beta$ tenha sido a evocação de alguma premissa, sabe-se que $\beta\in\Gamma\cup\set{\alpha}$.
                Deste modo, existem outros dois casos a serem analisados.
            \end{case}

            \begin{subcase}
                \textsc{Caso 1.1.}
                Se a linha derradeira da sucessão de dedução que prova $\Gamma\cup\set{\alpha}\vdash\beta$ tenha sido a evocação de alguma premissa, sabe-se que existe alguma premissa $\mathbf{P_\beta}\in\Gamma$ igual a $\beta$. Deste modo, podemos demonstrar que $\Gamma\vdash\alpha\to\beta$ pela seguinte sucessão de dedução:

                \begin{fitch}
                    \fa\Gamma\vdash\beta&$\mathbf{P_\beta}$\\
                    \fa\Gamma\vdash\beta\to\alpha\to\beta&$\hyperref[MA1]{\mathbf{A_1}}$\\
                    \fa\Gamma\vdash\alpha\to\beta&$\hyperref[detachment]{\mathbf{R_1}}\;\set{1,2}$.
                \end{fitch}
            \end{subcase}

            \begin{subcase}
                \textsc{Caso 1.2.}
                Se a linha derradeira da sucessão de dedução que prova $\Gamma\cup\set{\alpha}\vdash\beta$ tenha sido a evocação da premissa $\alpha$, sabe-se que $\beta=\alpha$.
                Deste modo, basta demonstrar que $\Gamma\vdash\alpha\to\alpha$ o que consiste num enfraquecimento do lema \refer{identity}{L}.
            \end{subcase}

            \begin{case}
                \textsc{Caso 2.}
                Se a linha derradeira da sucessão de dedução que prova $\Gamma\cup\set{\alpha}\vdash\beta$ tenha sido a evocação de algum axioma, sabe-se que existe algum esquema $\mathbf{A_\beta}\in\mathcal{A}$ que gera $\beta$.
                Deste modo, podemos demonstrar que $\Gamma\vdash\alpha\to\beta$ pela seguinte sucessão de dedução:

                \begin{fitch}
                    \fa\Gamma\vdash\beta&$\mathbf{A_\beta}$\\
                    \fa\Gamma\vdash\beta\to\alpha\to\beta&$\hyperref[MA1]{\mathbf{A_1}}$\\
                    \fa\Gamma\vdash\alpha\to\beta&$\hyperref[detachment]{\mathbf{R_1}}\;\set{1,2}$.
                \end{fitch}
            \end{case}

            \begin{case}
                \textsc{Caso 3.}
                Se a linha derradeira da sucessão de dedução que prova $\Gamma\cup\set{\alpha}\vdash\beta$ tenha sido gerada pela aplicação da regra da necessitação a uma linha anterior, sabe-se que $\beta=\nec\varphi$ e que $\mathbf{H_1}=\varphi$.
                Deste modo, podemos demonstrar que $\Gamma\vdash\alpha\to\nec\varphi$ pela seguinte sucessão de dedução:

                \begin{fitch}
                    \fa\vdash\varphi&$\mathbf{H_1}$\\
                    \fa\Gamma\vdash\nec\varphi&$\hyperref[necessitation]{\mathbf{R_2}}\;\set{1}$\\
                    \fa\Gamma\vdash\nec\varphi\to\alpha\to\nec\varphi&$\hyperref[MA1]{\mathbf{A_1}}$\\
                    \fa\Gamma\vdash\alpha\to\nec\varphi&$\hyperref[detachment]{\mathbf{R_1}}\;\set{2,3}$.
                \end{fitch}
            \end{case}

            \begin{case}
                \textsc{Caso 4.} Seja a sentença $\varphi_n=\beta$ gerada pela aplicação da regra da separação a duas sentenças $\varphi_i$ e $\varphi_j$ com $i<j<n$. Assumiremos, sem perda de generalidade, que $\varphi_j=\varphi_i\to\varphi_n$.
                Assim, pela premissa da indução temos que $\mathbf{H_1}=\alpha\to\varphi_i$ e que $\mathbf{H_2}=\alpha\to\varphi_i\to\varphi_n$.
                Deste modo, podemos demonstrar que $\Gamma\vdash\alpha\to\nec\varphi$ pela seguinte sucessão de dedução:

                \begin{fitch}
                    \fa\Gamma\entails\alpha\to\varphi_j&$\mathbf{H_1}$\\
                    \fa\Gamma\entails\alpha\to\varphi_j\to\beta&$\mathbf{H_2}$\\
                    \fa\Gamma\entails(\alpha\to\varphi_j\to\beta)\to(\alpha\to\varphi_j)\to(\alpha\to\beta)&$\hyperref[MA2]{\mathbf{A_2}}$\\
                    \fa\Gamma\entails(\alpha\to\varphi_j)\to(\alpha\to\beta)&$\hyperref[detachment]{\mathbf{R_1}}\;\set{2,3}$\\
                    \fa\alpha\to\beta&$\hyperref[detachment]{\mathbf{R_1}}\;\set{1,4}$.
                \end{fitch}
            \end{case}
        \end{proof}
    \end{theorem}

    \begin{theorem}\label{gen-nec}
        Se $\nec\Gamma\entails\alpha$, então $\nec\Gamma\entails\nec\alpha$.

        \begin{proof}
            Prova por indução fraca sobre o tamanho $n$ do conjunto $\nec\Gamma$ \citep{Troelstra}. A prova consiste em dois casos: um para a base da indução e outro para o passo da indução.

            \begin{case}
                \textsc{Caso 1.} Para a base, consideraremos que $\nec\Gamma=\varnothing$ --- ou seja, que possui tamanho $n=0$.
                Assim, sabemos que $\nec\Gamma\entails\alpha$ e, portanto, que existe uma sucessão de dedução $\sequence{\varphi_i\mid 0\leq i\leq n}$ com $\varphi_n=\alpha$
                Pode-se demonstrar que $\entails\nec\alpha$ trivialmente pela aplicação da regra da necessitação \hyperref[necessitation]{$\mathbf{R_2}$} sobre a sentença $\varphi_n$.
            \end{case}

            \begin{case}
                \textsc{Caso 2.} 
                Para o passo, suponhamos que a generalização da regra da necessitação valha para qualquer conjunto $\nec\Gamma$ de tamanho $n=k$.
                Demonstraremos, pela sucessão de dedução apresentada abaixo, que a generalização da regra da necessitação vale para conjuntos $\nec\Gamma$ de tamanho $n=k+1$.

                \begin{fitch}
                    \fa\nec\Gamma\cup\set{\nec\alpha}\entails\beta\\
                    \fa\nec\Gamma\entails\nec\alpha\to\beta\\
                    \fa\nec\Gamma\entails\nec(\nec\alpha\to\beta)\\
                    \fa\nec\Gamma\entails\nec(\nec\alpha\to\beta)\to\nec\nec\alpha\to\nec\beta\\
                    \fa\nec\Gamma\entails\nec\nec\alpha\to\nec\beta\\
                    \fa\nec\Gamma\entails\nec\alpha\to\nec\nec\alpha\\
                    \fa\nec\Gamma\entails(\nec\alpha\to\nec\nec\alpha)\to(\nec\nec\alpha\to\nec\beta)\to\nec\alpha\to\nec\beta\\
                    \fa\nec\Gamma\entails(\nec\nec\alpha\to\nec\beta)\to\nec\alpha\to\nec\beta\\
                    \fa\nec\Gamma\entails\nec\alpha\to\nec\beta\\
                    \fa\nec\Gamma\cup\set{\nec\alpha}\entails\nec\alpha\\
                    \fa\nec\Gamma\cup\set{\nec\alpha}\entails\nec\alpha\to\nec\beta\\
                    \fa\nec\Gamma\cup\set{\nec\alpha}\entails\nec\beta\\
                \end{fitch}
            \end{case}
        \end{proof}
    \end{theorem}

    Uma vez provada a generalização da regra da implicação, a prova da regra da dedução estrita --- conforme descrito por~\cite{Barcan, Marcus} --- torna-se trivial, como pode ser visto abaixo. Esta regra derivada permite simplificar as provas de corretude das traduções, uma vez que uma das traduções que serão apresentadas mapeia implicações materiais do sistema intuicionista em implicações estritas.

    \begin{theorem}\label{strictdeduction}
        $\text{Se }\nec\Gamma\cup\set{\alpha}\entails\beta\text{, então }\nec\Gamma\entails\nec(\alpha\to\beta)$.

        \begin{proof}
            Pode ser provado pela seguinte sucessão de dedução:

            \begin{fitch}
                \fa\nec\Gamma\cup\set{\alpha}\entails\beta&$\mathbf{H_1}$\\
                \fa\nec\Gamma\entails\alpha\to\beta&\refer{deduction}{T}$\;\set{1}$\\
                \fa\nec\Gamma\entails\nec(\alpha\to\beta)&\refer{gen-nec}{T}$\;\set{2}$.\qedhere
            \end{fitch}
        \end{proof}
    \end{theorem}

        \section{Derivações}
    Nesta seção serão apresentadas diversas derivações feitas a partir da axiomatização do sistema modal que servirão de lemas para a prova de teoremas futuros neste trabalho.

    \begin{lemma}\label{identity}
        $\entails\alpha\to\alpha$.
        \begin{proof}
            Pode ser provado pela seguinte sucessão de dedução:
        
            \begin{fitch}
                \fb\vdash\alpha\to\alpha\to\alpha&\hyperref[MA1]{$\mathbf{A_1}$}\\
                \fa\vdash\alpha\to(\alpha\to\alpha)\to\alpha&\hyperref[MA1]{$\mathbf{A_1}$}\\
                \fa\vdash(\alpha\to(\alpha\to\alpha)\to\alpha)\to(\alpha\to\alpha\to\alpha)\to\alpha\to\alpha&\hyperref[MA2]{$\mathbf{A_2}$}\\
                \fa\vdash(\alpha\to\alpha\to\alpha)\to\alpha\to\alpha&$\hyperref[detachment]{\mathbf{R_1}}\;\set{2,3}$\\
                \fa\vdash\alpha\to\alpha&$\hyperref[detachment]{\mathbf{R_1}}\;\set{1,4}$.
            \end{fitch}

            % \begin{fitch}
            %     \fa\set{a}\entails\alpha&$\mathbf{P_1}$\\
            %     \fa\entails\alpha\to\alpha&\refer{deduction}{T}$\;\set{1}$.
            % \end{fitch}
            \vspace*{-18pt-0.7em}
            \qedhere
        \end{proof}
    \end{lemma}

    \begin{lemma}\label{explosion}
        $\entails\bot\to\alpha$
        \begin{proof}
            Pode ser provado pela seguinte sucessão de dedução:
        
            \begin{fitch}
                \fb\set{\bot}\entails\bot&$\mathbf{P_1}$\\
                \fa\set{\bot}\entails\bot\to(\alpha\to\bot)\to\bot&$\hyperref[MA1]{\mathbf{A_1}}$\\
                \fa\set{\bot}\entails\neg\neg\alpha&$\hyperref[detachment]{\mathbf{R_1}}\;\set{1,2}$\\
                \fa\set{\bot}\entails\neg\neg\alpha\to\alpha&$\hyperref[MANEG]{\mathbf{A_\neg}}$\\
                \fa\set{\bot}\entails\alpha&$\hyperref[detachment]{\mathbf{R_1}}\;\set{3,4}$\\
                \fa\entails\bot\to\alpha&$\hyperref[deduction]{\mathbf{T_1}}\;\set{5}$.
            \end{fitch}
            \vspace*{-18pt-0.7em}
            \qedhere
        \end{proof}
    \end{lemma}

    \begin{lemma}\label{contrapositive}
        $\entails(\alpha\to\beta)\to(\neg\beta\to\neg\alpha)$

        \begin{proof}
            Pode ser provado pela seguinte sucessão de dedução:
        
            \begin{fitch}
                \fb\set{\alpha\to\beta,\neg\beta}\entails\beta\to\bot&$\mathbf{P_2}$\\
                \fa\set{\alpha\to\beta,\neg\beta}\entails(\beta\to\bot)\to\alpha\to(\beta\to\bot)&\hyperref[MA1]{$\mathbf{A_1}$}\\
                \fa\set{\alpha\to\beta,\neg\beta}\entails\alpha\to\beta\to\bot&$\hyperref[detachment]{\mathbf{R_1}}\;\set{1,2}$\\
                \fa\set{\alpha\to\beta,\neg\beta}\entails(\alpha\to\beta\to\bot)\to(\alpha\to\beta)\to(\alpha\to\bot)&\hyperref[MA2]{$\mathbf{A_2}$}\\
                \fa\set{\alpha\to\beta,\neg\beta}\entails\alpha\to\beta&$\mathbf{P_1}$\\
                \fa\set{\alpha\to\beta,\neg\beta}\entails(\alpha\to\beta)\to(\alpha\to\bot)&$\hyperref[detachment]{\mathbf{R_1}}\;\set{3,4}$\\
                \fa\set{\alpha\to\beta,\neg\beta}\entails\neg\alpha&$\hyperref[detachment]{\mathbf{R_1}}\;\set{5,6}$\\
                \fa\set{\alpha\to\beta}\entails\neg\beta\to\neg\alpha&\refer{deduction}{T}$\;\set{7}$\\
                \fa\entails(\alpha\to\beta)\to(\neg\beta\to\neg\alpha)&\refer{deduction}{T}$\;\set{8}$.
            \end{fitch}
            \vspace*{-18pt-0.7em}
            \qedhere
        \end{proof}
    \end{lemma}

    \begin{lemma}\label{and-intro}
        $\vdash(\alpha\to\beta)\to(\alpha\to\gamma)\to\alpha\to\beta\wedge\gamma$.

        \begin{proof}
            Pode ser provado pela seguinte sucessão de dedução:
            
            \begin{fitch}
                \fb\set{\alpha\to\beta,\alpha\to\gamma,\alpha}\vdash\alpha&$\mathbf{P_3}$\\
                \fa\set{\alpha\to\beta,\alpha\to\gamma,\alpha}\vdash\alpha\to\beta&$\mathbf{P_1}$\\
                \fa\set{\alpha\to\beta,\alpha\to\gamma,\alpha}\vdash\beta&$\hyperref[detachment]{\mathbf{R_1}}\;\set{1, 2}$\\
                \fa\set{\alpha\to\beta,\alpha\to\gamma,\alpha}\vdash\alpha\to\gamma&$\mathbf{P_2}$\\
                \fa\set{\alpha\to\beta,\alpha\to\gamma,\alpha}\vdash\gamma&$\hyperref[detachment]{\mathbf{R_1}}\;\set{1, 4}$\\
                \fa\set{\alpha\to\beta,\alpha\to\gamma,\alpha}\vdash\beta\to\gamma\to\beta\wedge\gamma&\hyperref[MA3]{$\mathbf{A_3}$}\\
                \fa\set{\alpha\to\beta,\alpha\to\gamma,\alpha}\vdash\gamma\to\beta\wedge\gamma&$\hyperref[detachment]{\mathbf{R_1}}\;\set{3, 6}$\\
                \fa\set{\alpha\to\beta,\alpha\to\gamma,\alpha}\vdash\beta\wedge\gamma&$\hyperref[detachment]{\mathbf{R_1}}\;\set{5, 7}$\\
                \fa\set{\alpha\to\beta,\alpha\to\gamma}\vdash\alpha\to\beta\wedge\gamma&\refer{deduction}{T}$\;\set{8}$\\
                \fa\set{\alpha\to\beta}\vdash(\alpha\to\gamma)\to\alpha\to\beta\wedge\gamma&\refer{deduction}{T}$\;\set{9}$\\
                \fa\vdash(\alpha\to\beta)\to(\alpha\to\gamma)\to\alpha\to\beta\wedge\gamma&\refer{deduction}{T}$\;\set{10}$.
            \end{fitch}
            \vspace*{-18pt-0.7em}
            \qedhere
        \end{proof}
    \end{lemma}

    \begin{lemma}
        $\vdash\nec(\alpha\wedge\beta)\to\nec\alpha\wedge\nec\beta$.

        \begin{proof}
            Pode ser provado pela seguinte sucessão de dedução:

            \begin{fitch}
                \fb\entails\alpha\wedge\beta\to\alpha&\hyperref[MA4]{$\mathbf{A_4}$}\\
                \fa\entails\nec(\alpha\wedge\beta\to\alpha)&$\hyperref[necessitation]{\mathbf{R_2}}\;\set{1}$\\
                \fa\entails\nec(\alpha\wedge\beta\to\alpha)\to(\nec(\alpha\wedge\beta)\to\nec\alpha)&\hyperref[MB1]{$\mathbf{B_1}$}\\
                \fa\entails\nec(\alpha\wedge\beta)\to\nec\alpha&$\hyperref[detachment]{\mathbf{R_1}}\;\set{2, 3}$\\
                \fa\entails\alpha\wedge\beta\to\beta&\hyperref[MA5]{$\mathbf{A_5}$}\\
                \fa\entails\nec(\alpha\wedge\beta\to\beta)&$\hyperref[necessitation]{\mathbf{R_2}}\;\set{5}$\\
                \fa\entails\nec(\alpha\wedge\beta\to\beta)\to(\nec(\alpha\wedge\beta)\to\nec\beta)&\hyperref[MB1]{$\mathbf{B_1}$}\\
                \fa\entails\nec(\alpha\wedge\beta)\to\nec\beta&$\hyperref[detachment]{\mathbf{R_1}}\;\set{6, 7}$\\
                \fa\entails(\nec(\alpha\wedge\beta)\to\nec\alpha)\to(\nec(\alpha\wedge\beta)\to\nec\beta)\to\nec(\alpha\wedge\beta)\to\nec\alpha\wedge\nec\beta&\refer{and-intro}{L}\\
                \fa\entails(\nec(\alpha\wedge\beta)\to\nec\beta)\to\nec(\alpha\wedge\beta)\to\nec\alpha\wedge\nec\beta&$\hyperref[detachment]{\mathbf{R_1}}\;\set{4, 9}$\\
                \fa\entails\nec(\alpha\wedge\beta)\to\nec\alpha\wedge\nec\beta&$\hyperref[detachment]{\mathbf{R_1}}\;\set{6,9}$\\
            \end{fitch}
            \vspace*{-18pt-0.7em}
            \qedhere
        \end{proof}
    \end{lemma}

    \begin{lemma}    
        $\vdash\nec\alpha\wedge\nec\beta\to\nec(\alpha\wedge\beta)$.

        \begin{proof}
            Pode ser provado pela seguinte sucessão de dedução:

            \begin{fitch}
                \fb\set{\nec\alpha\wedge\nec\beta}\vdash\alpha\to\beta\to\alpha\wedge\beta&\hyperref[MA3]{$\mathbf{A_3}$}\\
                \fa\set{\nec\alpha\wedge\nec\beta}\vdash\nec\alpha\wedge\nec\beta\to\nec\alpha&$\mathbf{A_4}$\\
                \fa\set{\nec\alpha\wedge\nec\beta}\vdash\nec\alpha\to\alpha\\
                \fa\set{\nec\alpha\wedge\nec\beta}\vdash\nec\alpha\wedge\nec\beta\to\nec\beta&$\mathbf{A_5}$\\
                \fa\set{\nec\alpha\wedge\nec\beta}\vdash\nec\alpha\to\beta\\
                \fa\set{\nec\alpha\wedge\nec\beta}\vdash\nec\alpha&$\mathbf{R_1}\;\set{1, 4}$\\
            \end{fitch}
            \vspace*{-18pt-0.7em}
            \qedhere
        \end{proof}
    \end{lemma}

    \begin{lemma}
        $\vdash\nec(\alpha\to\beta)\to\nec\alpha\to\beta$.

        \begin{proof}
            Pode ser provado pela seguinte sucessão de dedução:

            \begin{fitch}
                \fb\set{\nec(\alpha\to\beta),\nec\alpha}\entails\nec\alpha&$\mathbf{P_2}$\\
                \fa\set{\nec(\alpha\to\beta),\nec\alpha}\entails\nec\alpha\to\alpha&$\hyperref[MB2]{\mathbf{B_2}}$\\
                \fa\set{\nec(\alpha\to\beta),\nec\alpha}\entails\alpha&$\hyperref[detachment]{\mathbf{R_1}}\;\set{1,2}$\\
                \fa\set{\nec(\alpha\to\beta),\nec\alpha}\entails\nec(\alpha\to\beta)&$\mathbf{P_1}$\\
                \fa\set{\nec(\alpha\to\beta),\nec\alpha}\entails\nec(\alpha\to\beta)\to\alpha\to\beta&$\hyperref[MB2]{\mathbf{B_2}}$\\
                \fa\set{\nec(\alpha\to\beta),\nec\alpha}\entails\alpha\to\beta&$\hyperref[detachment]{\mathbf{R_1}}\;\set{4,5}$\\
                \fa\set{\nec(\alpha\to\beta),\nec\alpha}\entails\beta&$\hyperref[detachment]{\mathbf{R_1}}\;\set{3,6}$\\
                \fa\set{\nec(\alpha\to\beta)}\entails\nec\alpha\to\beta&$\hyperref[deduction]{\mathbf{T_1}}\;\set{7}$\\
                \fa\entails\nec(\alpha\to\beta)\to\nec\alpha\to\beta&$\hyperref[deduction]{\mathbf{T_1}}\;\set{8}$.
            \end{fitch}
            \vspace*{-18pt-0.7em}
            \qedhere
        \end{proof}
    \end{lemma}
        \section{Dualidades}

    \babireski{Ver~\cite{Zach} acerca dos axiomas duais e suas derivações.}

    % \begin{theorem}
    %     $\vdash\nec(\alpha\to\beta)\to\pos\alpha\to\pos\beta$.

    %     \begin{proof}
    %         Pode ser provado pela seguinte sucessão de dedução:

    %         \begin{fitch}
    %             \fa\set{\nec(\alpha\to\beta),\pos\alpha}\vdash\pos\beta\\
    %             \fa\set{\nec(\alpha\to\beta)}\vdash\pos\alpha\to\pos\beta\\
    %             \fa\vdash\nec(\alpha\to\beta)\to\pos\alpha\to\pos\beta\\
    %         \end{fitch}
    %         \vspace*{-18pt-0.7em}
    %         \qedhere
    %     \end{proof}
    % \end{theorem}

    \begin{theorem}
        $\vdash\alpha\to\pos\alpha$.
        \begin{proof}
            Pode ser provado pela seguinte sucessão de dedução:

            \begin{fitch}
                \fa\entails\nec\neg\alpha\to\neg\alpha&$\hyperref[MB2]{\mathbf{B_2}}$\\
                \fa\entails(\nec\neg\alpha\to\neg\alpha)\to\neg\neg\alpha\to\pos\alpha&$\hyperref[contrapositive]{\mathbf{L_3}}$\\
                \fa\entails\neg\neg\alpha\to\pos\alpha&$\hyperref[detachment]{\mathbf{R_1}}\;\set{1,2}$\\
                \fa\entails(\neg\neg\alpha\to\neg\nec\neg\alpha)\to\alpha\to(\neg\neg\alpha\to\pos\alpha)&$\hyperref[MA1]{\mathbf{A_1}}$\\
                \fa\entails\alpha\to\neg\neg\alpha\to\pos\alpha&$\hyperref[detachment]{\mathbf{R_1}}\;\set{3,4}$\\
                \fa\entails(\alpha\to\neg\neg\alpha\to\pos\alpha)\to(\alpha\to\neg\neg\alpha)\to(\alpha\to\pos\alpha)&$\hyperref[MA2]{\mathbf{A_2}}$\\
                \fa\entails\alpha\to\neg\neg\alpha&\babireski{Provar.}\\
                \fa\entails(\alpha\to\neg\neg\alpha)\to(\alpha\to\pos\alpha)&$\hyperref[detachment]{\mathbf{R_1}}\;\set{5,6}$\\
                \fa\entails\alpha\to\pos\alpha&$\hyperref[detachment]{\mathbf{R_1}}\;\set{7,8}$.
            \end{fitch}
            \vspace*{-18pt-0.7em}
            \qedhere
        \end{proof}
    \end{theorem}

    \begin{theorem}
        $\vdash\pos\pos\alpha\to\pos\alpha$.
        \begin{proof}
            Pode ser provado pela seguinte sucessão de dedução:

            \begin{fitch}
                
                \fa\entails\neg\nec\alpha\to\neg\nec\pos\alpha\\
                \fa\entails\pos\pos\alpha\to\pos\alpha\\
            \end{fitch}
            \vspace*{-18pt-0.7em}
            \qedhere
        \end{proof}
    \end{theorem}

    Apesar da similaridades com as transformações naturais, deve-se destacar que as noções de computação não podem ser interpretadas simplesmente como necessidade ou possibilidade, uma vez que apresenta propriedades presente em ambas as modalidades. Neste sentido, a modalidade de \emph{laxidade} --- que combina noções de necessidade e possibilidade --- mostra-se uma melhor representação de efeitos computacionais sobre a interpretação programa-prova.
    
    Ao sistema que comporta a modalidade de laxidade damos o nome de sistema laxo. Este sistema consiste numa extensão do sistema intuicionista com a adição da modalidade $\lax^1$ definida por~\cite{Fairtlough,Mendler}. A sentença $\lax\alpha$ --- lida como \emph{laxamente} $\alpha$ --- pode ser axiomatizada pelos seguintes esquemas:
    \begin{alignat*}{3}
        &\mathbf{C_1}\quad&&\alpha\to\lax\alpha\\
        &\mathbf{C_2}\quad&&\lax\lax\alpha\to\lax\alpha\\
        &\mathbf{C_3}\quad&&(\alpha\to\beta)\to\lax\alpha\to\lax\beta
    \end{alignat*}

    Esse sistema, entretanto, pode ser interpretado modalmente por meio da seguinte tradução \citep{Pfenning}:

    \begin{definition}[$\bullet^+$] A tradução $\bullet^+:\mathcal{L}_\mathbf{M}\to\mathcal{L}_\mathbf{L}$ do sistema $\mathbf{S_4}$ intuicionista ao sistema $\mathbf{L}$ pode ser definida indutivamente da seguinte maneira:
        \begin{align*}
            a^+&\coloneq a\\
            \bot^+&\coloneq\bot\\
            {(\lax\alpha)}^+&\coloneq\pos\nec\alpha^+\\
            {(\alpha\to\beta)}^+&\coloneq\nec\alpha^+\to\beta^+
            \tag*{\qed} 
        \end{align*}
    \end{definition}

    \chapter{Traduções}
        \section{Definições}
    A primeira tradução do sistema intuicionista ao sistema modal foi proposta por~\cite{Goedel} motivado pela possibilidade de leitura da necessidade como uma modalidade de construtividade. Ou seja, por meio dessa tradução, a sentença $\nec \varphi$ poderia ser lida como \textit{$\varphi$ pode ser provada construtivamente} \citep{Troelstra}. Gödel conjeiturou a corretude fraca dessa tradução, que foi posteriormente provada por~\cite{McKinsey} em conjunto com sua completude fraca.

    \begin{definition}[$\bullet^\circ$] Define-se a tradução $\bullet^\circ$ indutivamente da seguinte maneira:
        \begin{align*}
            p^\circ                     & \coloneqq p                                       \\
            \bot^\circ                  & \coloneqq \bot                                    \\
            {(\varphi \wedge \psi)}^\circ & \coloneqq \varphi^\circ \wedge \psi^\circ         \\
            {(\varphi \vee \psi)}^\circ   & \coloneqq \nec \varphi^\circ \vee \nec \psi^\circ \\
            {(\varphi \to \psi)}^\circ    & \coloneqq \nec \varphi^\circ \to \psi^\circ
            \tag*{\qed} 
        \end{align*}
    \end{definition}
    
    \begin{definition}[$\bullet^\nec$] Define-se a tradução $\bullet^\nec$ indutivamente da seguinte maneira:
        \begin{align*}
            p^\nec                     & \coloneqq \nec p                                        \\
            \bot^\nec                  & \coloneqq \bot                                          \\
            {(\varphi \wedge \psi)}^\nec & \coloneqq \varphi^\nec \wedge \psi^\nec     \\
            {(\varphi \vee \psi)}^\nec   & \coloneqq \varphi^\nec \vee \psi^\nec       \\
            {(\varphi \to \psi)}^\nec    & \coloneqq \nec (\varphi^\nec \to \psi^\nec)
            \tag*{\qed} 
        \end{align*}
    \end{definition}
    
    Faz-se interessante pontuar que as traduções $\bullet^\circ$ e $\bullet^\nec$ correspondem, respectivamente, às traduções $\bullet^\circ$ e $\bullet^*$ do sistema intuicionista ao sistema linear providas por~\cite{Girard}, sendo as primeiras correspondentes a uma ordem de avaliação por nome (\textit{call-by-name}) e as segundas a uma ordem de avaliação por valor (\textit{call-by-value}). 
    Ademais, as duas traduções providas são equivalentes, conforme demonstrado pelo teorema $\mathbf{T_2}$.

    \begin{theorem}
        $\entails\nec\alpha^\circ\leftrightarrow\alpha^\nec$.

        \begin{proof}
            Prova por indução forte sobre a profundidade de $\alpha$.
            Assim, suponhamos que as traduções equivalham para qualquer $\alpha$ de profundidade $n<k$.
            Demonstraremos, analisando-se os casos, que as traduções equivalem para qualquer $\alpha$ de profundidade $n=k$.
    
            \begin{case}
                \textsc{Caso 1.}
                Se a sentença $\alpha$ for uma proposição $a$, sabe-se que $\nec a^\circ=\nec a$ e que $a^\nec =\nec a$ pelas definições das traduções.
                Deste modo, tanto a ida quanto a volta possuem a forma $\nec a\to\nec a$ e podem ser provadas pelo lema \hyperref[identity]{$\mathbf{L_\getrefnumber{identity}}$}.
                Ambas as implicações, então, podem ser unidas em uma bi-implicação por meio do esquema \hyperref[MA3]{$\mathbf{A_3}$}.
            \end{case}

            \begin{case}
                \textsc{Caso 2.}
                Se a sentença $\alpha$ for a constante $\bot$, sabe-se que $\nec\bot^\circ=\nec\bot$ e que $\bot^\nec=\bot$ pelas definições das traduções.
                Deste modo, a ida $\nec\bot\to\bot$ constitui um axioma gerado pelo esquema \hyperref[MB2]{$\mathbf{B_2}$} --- sendo assim provada trivialmente --- e a volta $\bot\to\nec\bot$ pode ser provada pelo lema \hyperref[explosion]{$\mathbf{L_2}$}.
                Ambas as implicações, então, podem ser unidas em uma bi-implicação por meio do esquema \hyperref[MA3]{$\mathbf{A_3}$}.
            \end{case}
    
            \begin{case}
                \textsc{Caso 3.}
                Se a sentença $\alpha$ for o resultado da conjunção de duas outras sentenças $\varphi$ e $\psi$, sabe-se que $\nec{(\varphi\wedge\psi)}^\circ=\nec(\varphi^\circ\wedge\psi^\circ)$ e que ${(\varphi\wedge\psi)}^\nec=\varphi^\nec\wedge\psi^\nec$ pelas definições das traduções.
                Separaremos a prova em dois casos: um para a ida $\nec(\varphi^\circ\wedge\psi^\circ)\to\varphi^\nec\wedge\psi^\nec$ e outro para a volta $\varphi^\nec\wedge\psi^\nec\to\nec(\varphi^\circ\wedge\psi^\circ)$. Ambas as implicações, então, podem ser unidas em uma bi-implicação por meio do esquema \hyperref[MA3]{$\mathbf{A_3}$}.
            \end{case}

                \begin{subcase}
                    \textsc{Caso 3.1.}
                    Pela premissa de indução, temos que $\mathbf{H_1}=\nec\varphi^\circ\leftrightarrow\varphi^\nec$ e que $\mathbf{H_2}=\nec\psi^\circ\leftrightarrow\psi^\nec$.
                    Valendo-se dessas sentenças e da regra da relocação em conjunto com alguns lemas, pode-se provar a sentença $\nec(\varphi^\circ\wedge\psi^\circ)\to\varphi^\nec\wedge\psi^\nec$ pela seguinte sucessão de dedução:

                    \begin{fitch}
                        \fa\set{\nec(\varphi^\circ\wedge\psi^\circ)}\proves\nec(\varphi^\circ\wedge\psi^\circ)&$\mathbf{P}$\\
                        \fa\set{\nec(\varphi^\circ\wedge\psi^\circ)}\proves\nec(\varphi^\circ\wedge\psi^\circ)\to\nec\varphi^\circ\wedge\nec\psi^\circ&$\mathbf{L_5}$\\
                        \fa\set{\nec(\varphi^\circ\wedge\psi^\circ)}\proves\nec\varphi^\circ\wedge\nec\psi^\circ&$\mathbf{R_1}\;\sequence{1,2}$\\
                        \fa\set{\nec(\varphi^\circ\wedge\psi^\circ)}\proves\nec\varphi^\circ\wedge\nec\psi^\circ\to\nec\varphi^\circ&$\mathbf{A_4}$\\
                        \fa\set{\nec(\varphi^\circ\wedge\psi^\circ)}\proves\nec\varphi^\circ&$\mathbf{R_1}\;\sequence{3,4}$\\
                        \fa\set{\nec(\varphi^\circ\wedge\psi^\circ)}\proves\nec\varphi^\circ\to\varphi^\nec&$\mathbf{H}$\\
                        \fa\set{\nec(\varphi^\circ\wedge\psi^\circ)}\proves\varphi^\nec&$\mathbf{R_1}\;\sequence{5,6}$\\
                        \fa\set{\nec(\varphi^\circ\wedge\psi^\circ)}\proves\nec\varphi^\circ\wedge\nec\psi^\circ\to\nec\psi^\circ&$\mathbf{A_4}$\\
                        \fa\set{\nec(\varphi^\circ\wedge\psi^\circ)}\proves\nec\psi^\circ&$\mathbf{R_1}\;\sequence{3,8}$\\
                        \fa\set{\nec(\varphi^\circ\wedge\psi^\circ)}\proves\nec\psi^\circ\to\psi^\nec&$\mathbf{H}$\\
                        \fa\set{\nec(\varphi^\circ\wedge\psi^\circ)}\proves\psi^\nec&$\mathbf{R_1}\;\sequence{9,10}$\\
                        \fa\set{\nec(\varphi^\circ\wedge\psi^\circ)}\proves\varphi^\nec\to\psi^\nec\to\varphi^\nec\wedge\psi^\nec&$\mathbf{A_3}$\\
                        \fa\set{\nec(\varphi^\circ\wedge\psi^\circ)}\proves\psi^\nec\to\varphi^\nec\wedge\psi^\nec&$\mathbf{R_1}\;\sequence{7,12}$\\
                        \fa\set{\nec(\varphi^\circ\wedge\psi^\circ)}\proves\varphi^\nec\wedge\psi^\nec&$\mathbf{R_1}\;\sequence{9,13}$\\
                        \fa\proves\nec(\varphi^\circ\wedge\psi^\circ)\to\varphi^\nec\wedge\psi^\nec&$\mathbf{T_1}\;\sequence{14}$\\
                    \end{fitch}
                \end{subcase}

                \begin{subcase}
                    \textsc{Caso 3.2.}
                    Pela premissa de indução, temos que $\mathbf{H_1}=\nec\varphi^\circ\leftrightarrow\varphi^\nec$ e que $\mathbf{H_2}=\nec\psi^\circ\leftrightarrow\psi^\nec$.
                    Valendo-se dessas sentenças e da regra da relocação em conjunto com alguns lemas, pode-se provar a sentença $\varphi^\nec\wedge\psi^\nec\to\nec(\varphi^\circ\wedge\psi^\circ)$ pela seguinte sucessão de dedução:

                    \begin{fitch}
                        \fa\set{\varphi^\nec\wedge\psi^\nec}\proves\varphi^\nec\wedge\psi^\nec&$\mathbf{P}$\\
                        \fa\set{\varphi^\nec\wedge\psi^\nec}\proves\varphi^\nec\wedge\psi^\nec\to\varphi^\nec&$\mathbf{P}$\\
                        \fa\set{\varphi^\nec\wedge\psi^\nec}\proves\varphi^\nec&$\mathbf{P}$\\
                        \fa\set{\varphi^\nec\wedge\psi^\nec}\proves\varphi^\nec\to\nec\varphi^\circ&$\mathbf{P}$\\
                        \fa\set{\varphi^\nec\wedge\psi^\nec}\proves\nec\varphi^\circ&$\mathbf{P}$\\
                        \fa\set{\varphi^\nec\wedge\psi^\nec}\proves\varphi^\nec\wedge\psi^\nec\to\psi^\nec&$\mathbf{P}$\\
                        \fa\set{\varphi^\nec\wedge\psi^\nec}\proves\psi^\nec&$\mathbf{P}$\\
                        \fa\set{\varphi^\nec\wedge\psi^\nec}\proves\psi^\nec\to\nec\psi^\circ&$\mathbf{P}$\\
                        \fa\set{\varphi^\nec\wedge\psi^\nec}\proves\nec\psi^\circ&$\mathbf{P}$\\
                        \fa\set{\varphi^\nec\wedge\psi^\nec}\proves\nec\varphi^\circ\to\nec\psi^\circ\to\nec\varphi^\circ\wedge\nec\psi^\circ&$\mathbf{P}$\\
                        \fa\set{\varphi^\nec\wedge\psi^\nec}\proves\nec\psi^\circ\to\nec\varphi^\circ\wedge\nec\psi^\circ&$\mathbf{P}$\\
                        \fa\set{\varphi^\nec\wedge\psi^\nec}\proves\nec\varphi^\circ\wedge\nec\psi^\circ&$\mathbf{P}$\\
                        \fa\set{\varphi^\nec\wedge\psi^\nec}\proves\nec\varphi^\circ\wedge\nec\psi^\circ\to\nec(\varphi^\circ\wedge\psi^\circ)&$\mathbf{P}$\\
                        \fa\set{\varphi^\nec\wedge\psi^\nec}\proves\nec(\varphi^\circ\wedge\psi^\circ)&$\mathbf{P}$\\
                        \fa\proves\varphi^\nec\wedge\psi^\nec\to\nec(\varphi^\circ\wedge\psi^\circ)&$\mathbf{P}$\\
                    \end{fitch}
                \end{subcase}

            \begin{case}
                \textsc{Caso 4.}
                Se a sentença $\alpha$ for o resultado da disjunção de duas outras sentenças $\varphi$ e $\psi$, sabe-se que $\nec{(\varphi\vee\psi)}^\circ=\nec(\nec\varphi^\circ\vee\nec\psi^\circ)$ e que ${(\varphi\vee\psi)}^\nec=\varphi^\nec\vee\psi^\nec$ pelas definições das traduções.
                Separaremos a prova em dois casos: um para a ida $\nec(\nec\varphi^\circ\vee\nec\psi^\circ)\to\varphi^\nec\vee\psi^\nec$ e outro para a volta $\varphi^\nec\vee\psi^\nec\to\nec(\nec\varphi^\circ\vee\nec\psi^\circ)$.
                Ambas as implicações, então, podem ser unidas em uma bi-implicação por meio do esquema \hyperref[MA3]{$\mathbf{A_3}$}.

                \begin{fitch}
                    \fa\set{\nec(\nec\varphi^\circ\vee\nec\psi^\circ)}\entails\nec(\nec\varphi^\circ\vee\nec\psi^\circ)\\
                    \fa\set{\nec(\nec\varphi^\circ\vee\nec\psi^\circ)}\entails\nec(\nec\varphi^\circ\vee\nec\psi^\circ)\to\nec\varphi^\circ\vee\nec\psi^\circ\\
                    \fa\set{\nec(\nec\varphi^\circ\vee\nec\psi^\circ)}\entails\nec\varphi^\circ\vee\nec\psi^\circ\\
                    \fa\set{\nec(\nec\varphi^\circ\vee\nec\psi^\circ)}\entails\varphi^\nec\vee\psi^\nec\\
                    \fa\entails\nec(\nec\varphi^\circ\vee\nec\psi^\circ)\to\varphi^\nec\vee\psi^\nec\\
                \end{fitch}

                \begin{fitch}
                    \fa\set{\varphi^\nec\vee\psi^\nec}\entails\varphi^\nec\vee\psi^\nec&$\mathbf{P}$\\
                    \fa\set{\varphi^\nec\vee\psi^\nec}\entails\varphi^\nec\leftrightarrow\nec\varphi^\circ&$\mathbf{H}$\\
                    \fa\set{\varphi^\nec\vee\psi^\nec}\entails\nec\varphi^\circ\vee\psi^\nec&\refer{replacement}{T}$\;\set{1,2}$\\
                    \fa\set{\varphi^\nec\vee\psi^\nec}\entails\psi^\nec\leftrightarrow\nec\psi^\circ&$\mathbf{H}$\\
                    \fa\set{\varphi^\nec\vee\psi^\nec}\entails\nec\varphi^\circ\vee\nec\psi^\circ&\refer{replacement}{T}$\;\set{3,4}$\\
                    \fa\set{\varphi^\nec\vee\psi^\nec}\entails\nec\varphi^\circ\leftrightarrow\nec\nec\varphi^\circ\\
                    \fa\set{\varphi^\nec\vee\psi^\nec}\entails\nec\nec\varphi^\circ\vee\nec\psi^\circ\\
                    \fa\set{\varphi^\nec\vee\psi^\nec}\entails\nec\psi^\circ\leftrightarrow\nec\nec\psi^\circ\\
                    \fa\set{\varphi^\nec\vee\psi^\nec}\entails\nec\nec\varphi^\circ\vee\nec\nec\psi^\circ\\
                    \fa\set{\varphi^\nec\vee\psi^\nec}\entails\nec\nec\varphi^\circ\vee\nec\nec\psi^\circ\to\nec(\nec\varphi^\circ\vee\nec\psi^\circ)\\
                    \fa\set{\varphi^\nec\vee\psi^\nec}\entails\nec(\nec\varphi^\circ\vee\nec\psi^\circ)\\
                    \fa\entails\varphi^\nec\vee\psi^\nec\to\nec(\nec\varphi^\circ\vee\nec\psi^\circ)\\
                \end{fitch}
            \end{case}

            \begin{case}
                \textsc{Caso 5.}
                Se a sentença $\alpha$ for o resultado da implicação de uma sentença $\varphi$ a uma sentença $\psi$, sabe-se que $\nec{(\varphi\to\psi)}^\circ=\nec(\nec\varphi^\circ\to\psi^\circ)$ e que ${(\varphi\to\psi)}^\nec=\nec(\varphi^\nec\to\psi^\nec)$ pelas definições das traduções.
                Separaremos a prova em dois casos: um para a ida $\nec(\nec\varphi^\circ\to\psi^\circ)\to\nec(\varphi^\nec\to\psi^\nec)$ e outro para a volta $\nec(\varphi^\nec\to\psi^\nec)\to\nec(\nec\varphi^\circ\to\psi^\circ)$.
                Ambas as implicações, então, podem ser unidas em uma bi-implicação por meio do esquema \hyperref[MA3]{$\mathbf{A_3}$}.

                \begin{fitch}
                    \fa\set{\nec(\nec\varphi^\circ\to\psi^\circ)}\entails\nec(\nec\varphi^\circ\to\psi^\circ)\\
                    \fa\set{\nec(\nec\varphi^\circ\to\psi^\circ)}\entails\nec\nec\varphi^\circ\to\nec\psi^\circ\\
                    \fa\set{\nec(\nec\varphi^\circ\to\psi^\circ)}\entails\nec\varphi^\nec\to\psi^\nec\\
                    \fa\set{\nec(\nec\varphi^\circ\to\psi^\circ)}\entails\nec(\varphi^\nec\to\psi^\nec)\\
                \end{fitch}

                \begin{fitch}
                    \fa\set{\nec(\varphi^\nec\to\psi^\nec)}\entails\nec(\varphi^\nec\to\psi^\nec)\\
                    \fa\set{\nec(\varphi^\nec\to\psi^\nec)}\entails\nec(\nec\varphi^\circ\to\psi^\nec)\\
                    \fa\set{\nec(\varphi^\nec\to\psi^\nec)}\entails\nec(\nec\varphi^\circ\to\nec\psi^\circ)\\
                    \fa\set{\nec(\varphi^\nec\to\psi^\nec)}\entails\nec(\nec\varphi^\circ\to\psi^\circ)\\
                    \fa\entails\nec(\varphi^\nec\to\psi^\nec)\to\nec(\nec\varphi^\circ\to\psi^\circ)\\
                \end{fitch}
            \end{case}
        \end{proof}
    \end{theorem}

        \section{Correção}
    \begin{theorem}
        $\text{Se }\Gamma\entails_\mathbf{I}\alpha\text{, então }\Gamma^\medsquare\entails_\mathbf{4}\alpha^\medsquare$.
    \end{theorem}

    \begin{proof}
        Prova por indução forte sobre o tamanho da sucessão de dedução.
        Assim, suponhamos que a tradução seja correta para qualquer sucessão dedução de tamanho $n<k$.
        Demonstraremos, analisando-se os casos, que o a correção da tradução vale para sucessões de dedução de tamanho $n=k+1$.

        \begin{case}
            \textsc{Caso 1.}
            Se a linha derradeira da sucessão de dedução que prova $\Gamma\entails\alpha$ tenha sido a evocação de alguma premissa, sabe-se que $\alpha\in\Gamma$ e, portanto, que $\alpha^\medsquare\in\Gamma^\medsquare$. Desde modo, pode-se demonstrar que $\Gamma^\medsquare\entails\alpha^\medsquare$ trivialmente pela evocação da premissa $\alpha^\medsquare$.
        \end{case}

        \begin{case}
            \textsc{Caso 2.}
            Se a linha derradeira da sucessão de dedução que prova $\Gamma\entails\alpha$ tenha sido a evocação de algum axioma, sabe-se que existe algum esquema $\mathbf{A_\alpha}\in\mathcal{A}$ que gera $\alpha$. Deste modo, devemos demonstrar que para cada esquema $\mathbf{A}\in\mathcal{A}$, pode-se derivar $\Gamma^\medsquare\entails_\mathbf{4}\mathbf{A}^\medsquare$.
        \end{case}

            \begin{subcase}
                \textsc{Caso 2.1} ($\mathbf{A_1}$).

                \begin{fitch}
                    \fa\Gamma^\medsquare\cup\set{\nec{}a,\nec{}b}\entails\nec{}a&$\mathbf{P_1}$\\
                    \fa\Gamma^\medsquare\cup\set{\nec{}a}\entails\nec{}b\strictif\nec{}a&\refer{strictdeduction}{T}$\;\set{1}$\\
                    \fa\Gamma^\medsquare\entails\nec{}a\strictif\nec{}b\strictif\nec{}a&\refer{strictdeduction}{T}$\;\set{2}$.
                \end{fitch}
            \end{subcase}

            \begin{subcase}
                \textsc{Caso 2.2} ($\mathbf{A_2}$).

                \begin{fitch}
                    \fa\Gamma^\medsquare\cup\set{\nec{a}\strictif\nec{b}\strictif\nec{c},\nec{a}\strictif\nec{b},\nec{a}}\entails\nec{}a\\
                    \fa\Gamma^\medsquare\cup\set{\nec{a}\strictif\nec{b}\strictif\nec{c},\nec{a}\strictif\nec{b},\nec{a}}\entails\nec{}a\strictif\nec{}b\\
                    \fa\Gamma^\medsquare\cup\set{\nec{a}\strictif\nec{b}\strictif\nec{c},\nec{a}\strictif\nec{b},\nec{a}}\entails\nec{}b\\
                    \fa\Gamma^\medsquare\cup\set{\nec{a}\strictif\nec{b}\strictif\nec{c},\nec{a}\strictif\nec{b},\nec{a}}\entails\nec{a}\strictif\nec{b}\strictif\nec{c}\\
                    \fa\Gamma^\medsquare\cup\set{\nec{a}\strictif\nec{b}\strictif\nec{c},\nec{a}\strictif\nec{b},\nec{a}}\entails\nec{b}\strictif\nec{c}\\
                    \fa\Gamma^\medsquare\cup\set{\nec{a}\strictif\nec{b}\strictif\nec{c},\nec{a}\strictif\nec{b},\nec{a}}\entails\nec{c}\\
                    \fa\Gamma^\medsquare\cup\set{\nec{a}\strictif\nec{b}\strictif\nec{c},\nec{a}\strictif\nec{b}}\entails\nec{a}\strictif\nec{c}\\
                    \fa\Gamma^\medsquare\cup\set{\nec{a}\strictif\nec{b}\strictif\nec{c}}\entails(\nec{a}\strictif\nec{b})\strictif\nec{a}\strictif\nec{c}\\
                    \fa\Gamma^\medsquare\entails(\nec{a}\strictif\nec{b}\strictif\nec{c})\strictif(\nec{a}\strictif\nec{b})\strictif\nec{a}\strictif\nec{c}\\

                \end{fitch}
            \end{subcase}

            \begin{subcase}
                \textsc{Caso 2.3} ($\mathbf{A_3}$).

                \begin{fitch}
                    \fa\Gamma^\medsquare\cup\set{\nec{a},\nec{b}}\entails\nec{a}\\
                    \fa\Gamma^\medsquare\cup\set{\nec{a},\nec{b}}\entails\nec{b}\\
                    \fa\Gamma^\medsquare\cup\set{\nec{a},\nec{b}}\entails\nec{a}\to\nec{b}\to\nec{a}\wedge\nec{b}\\
                    \fa\Gamma^\medsquare\cup\set{\nec{a},\nec{b}}\entails\nec{b}\to\nec{a}\wedge\nec{b}\\
                    \fa\Gamma^\medsquare\cup\set{\nec{a},\nec{b}}\entails\nec{a}\wedge\nec{b}\\
                    \fa\Gamma^\medsquare\cup\set{\nec{a},\nec{b}}\entails\nec{a}\wedge\nec{b}\\
                    \fa\Gamma^\medsquare\cup\set{\nec{a}}\entails\nec{b}\strictif\nec{a}\wedge\nec{b}\\
                    \fa\Gamma^\medsquare\entails\nec{a}\strictif\nec{b}\strictif\nec{a}\wedge\nec{b}\\
                \end{fitch} 
            \end{subcase}

            \begin{subcase}
                \textsc{Caso 2.4} ($\mathbf{A_4}$).

                \begin{fitch}
                    \fa\Gamma^\medsquare\entails\nec{a}\wedge\nec{b}\to\nec{a}&$\hyperref[MA4]{\mathbf{A_4}}$\\
                    \fa\Gamma^\medsquare\entails\nec{a}\wedge\nec{b}\strictif\nec{a}&$\hyperref[necessitation]{\mathbf{R_2}}\;\set{1}$.
                \end{fitch}
            \end{subcase}

            \begin{subcase}
                \textsc{Caso 2.5} ($\mathbf{A_5}$).

                \begin{fitch}
                    \fa\Gamma^\medsquare\entails\nec{a}\wedge\nec{b}\to\nec{b}&$\hyperref[MA5]{\mathbf{A_5}}$\\
                    \fa\Gamma^\medsquare\entails\nec{a}\wedge\nec{b}\strictif\nec{b}&$\hyperref[necessitation]{\mathbf{R_2}}\;\set{1}$.
                \end{fitch}
            \end{subcase}

            \begin{subcase}
                \textsc{Caso 2.6} ($\mathbf{A_6}$).

                \begin{fitch}
                    \fa\Gamma^\medsquare\entails\nec{a}\to\nec{a}\vee\nec{b}&$\hyperref[MA6]{\mathbf{A_6}}$\\
                    \fa\Gamma^\medsquare\entails\nec{a}\strictif\nec{a}\vee\nec{b}&$\hyperref[necessitation]{\mathbf{R_2}}\;\set{1}$.
                \end{fitch}
            \end{subcase}

            \begin{subcase}
                \textsc{Caso 2.7} ($\mathbf{A_7}$).

                \begin{fitch}
                    \fa\Gamma^\medsquare\entails\nec{b}\to\nec{a}\vee\nec{b}&$\hyperref[MA7]{\mathbf{A_7}}$\\
                    \fa\Gamma^\medsquare\entails\nec{b}\strictif\nec{a}\vee\nec{b}&$\hyperref[necessitation]{\mathbf{R_2}}\;\set{1}$.
                \end{fitch}
            \end{subcase}

            \begin{subcase}
                \textsc{Caso 2.8} ($\mathbf{A_8}$).
                
                \footnotesize
                \begin{fitch}
                    \fa\Gamma^\medsquare\cup\set{\nec{a}\strictif\nec{c},\nec{b}\strictif\nec{c},\nec{a}\vee\nec{b}}\entails\nec{a}\strictif\nec{c}\\
                    \fa\Gamma^\medsquare\cup\set{\nec{a}\strictif\nec{c},\nec{b}\strictif\nec{c},\nec{a}\vee\nec{b}}\entails(\nec{a}\strictif\nec{c})\to\nec{a}\to\nec{c}\\
                    \fa\Gamma^\medsquare\cup\set{\nec{a}\strictif\nec{c},\nec{b}\strictif\nec{c},\nec{a}\vee\nec{b}}\entails\nec{a}\to\nec{c}\\
                    \fa\Gamma^\medsquare\cup\set{\nec{a}\strictif\nec{c},\nec{b}\strictif\nec{c},\nec{a}\vee\nec{b}}\entails\nec{b}\strictif\nec{c}\\
                    \fa\Gamma^\medsquare\cup\set{\nec{a}\strictif\nec{c},\nec{b}\strictif\nec{c},\nec{a}\vee\nec{b}}\entails(\nec{b}\strictif\nec{c})\to\nec{b}\to\nec{c}\\
                    \fa\Gamma^\medsquare\cup\set{\nec{a}\strictif\nec{c},\nec{b}\strictif\nec{c},\nec{a}\vee\nec{b}}\entails\nec{b}\to\nec{c}\\
                    \fa\Gamma^\medsquare\cup\set{\nec{a}\strictif\nec{c},\nec{b}\strictif\nec{c},\nec{a}\vee\nec{b}}\entails\nec{a}\vee\nec{b}\\
                    \fa\Gamma^\medsquare\cup\set{\nec{a}\strictif\nec{c},\nec{b}\strictif\nec{c},\nec{a}\vee\nec{b}}\entails(\nec{a}\to\nec{c})\to(\nec{b}\to\nec{c})\to\nec{a}\vee\nec{b}\to\nec{c}\\
                    \fa\Gamma^\medsquare\cup\set{\nec{a}\strictif\nec{c},\nec{b}\strictif\nec{c},\nec{a}\vee\nec{b}}\entails(\nec{b}\to\nec{c})\to\nec{a}\vee\nec{b}\to\nec{c}\\
                    \fa\Gamma^\medsquare\cup\set{\nec{a}\strictif\nec{c},\nec{b}\strictif\nec{c},\nec{a}\vee\nec{b}}\entails\nec{a}\vee\nec{b}\to\nec{c}\\
                    \fa\Gamma^\medsquare\cup\set{\nec{a}\strictif\nec{c},\nec{b}\strictif\nec{c}}\entails\nec{a}\vee\nec{b}\strictif\nec{c}\\
                    \fa\Gamma^\medsquare\cup\set{\nec{a}\strictif\nec{c}}\entails(\nec{b}\strictif\nec{c})\strictif\nec{a}\vee\nec{b}\strictif\nec{c}\\
                    \fa\Gamma^\medsquare\entails(\nec{a}\strictif\nec{c})\strictif(\nec{b}\strictif\nec{c})\strictif\nec{a}\vee\nec{b}\strictif\nec{c}\\
                \end{fitch}
            \end{subcase}

            \begin{subcase}
                \textsc{Caso 2.9} ($\mathbf{A_\bot}$).

                \begin{fitch}
                    \fa\Gamma^\medsquare\entails\bot\to\nec{a}&\refer{explosion}{L}\\
                    \fa\Gamma^\medsquare\entails\bot\strictif\nec{a}&$\hyperref[necessitation]{\mathbf{R_2}}\;\set{1}$.\\
                \end{fitch}
            \end{subcase}

        \begin{case}
            \textsc{Caso 3.}
            Deve-se demonstrar que, se $\entails\nec(\alpha^\medsquare\to\beta^\medsquare)$ ($\mathbf{H_1}$) e $\entails\alpha^\medsquare$ ($\mathbf{H_2}$), então $\beta^\medsquare$.
            Isso pode ser feito pela seguinte sucessão de dedução:

            \begin{fitch}
                \fa\nec(\alpha^\medsquare\to\beta^\medsquare)\to\alpha^\medsquare\to\beta^\medsquare&$\mathbf{B_2}$\\
                \fa\nec(\alpha^\medsquare\to\beta^\medsquare)&$\mathbf{H_1}$\\
                \fa\alpha^\medsquare\to\beta^\medsquare&$\mathbf{R_1}\;\sequence{1, 2}$\\
                \fa\alpha^\medsquare&$\mathbf{H_2}$\\
                \fa\beta^\medsquare&$\mathbf{R_1}\;\sequence{3, 4}$.
            \end{fitch}
        \end{case}
    \end{proof}
        \section{Completude}
    \babireski{Não vai rolar de provar a completude como~\cite{Troelstra}. Vou precisar procurar outros artigos.}

    \bibliographystyle{plainnat}
    \bibliography{bibliography}
\end{document}
